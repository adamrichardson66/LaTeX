\documentclass[11pt,english,
handout
]{beamer}

%Preamble  
\input{/Users/Adam/Desktop/LBCC/MATH80/MATH80_Lesson_Plans/MATH80_Slides_Preamble.tex}

%Textbook: Essential Calculus - Early Transcendentals, 2nd edition - Stewart. ISBN: 978-1-133-11228-0



\begin{document}

%Slide titles are all contained in this file..
\ExecuteMetaData[/Users/Adam/Desktop/LBCC/MATH80/MATH80_Lesson_Plans/MATH80_Slide_Titles.tex]{1601}

%Global Title Slide Format is contained in the following file.
\input{/Users/Adam/Desktop/LBCC/MATH80/MATH80_Lesson_Plans/MATH80_Title_Slide_Format.tex}
\makebeamertitle













\begin{frame}{Here We Go!}

This chapter is where we put together almost any- and everything to use that we have learned in all previous math courses. It showcases the beautiful connections between mathematics and physics, and prepares you to study in a rigorous way the complex ideas that permeate our reality. The chapter builds upon the fundamental concepts learned in this section, so pay close attention. We will also prove several important theorems for edification and to examine the methodology of the proofs themselves because you may encounter such methods later.
\end{frame}







\begin{frame}[t]{Vector Fields}
\small

\begin{definition}
Let $D$ be a subset of $\MB{R}^2$. A \textbf{vector field on} $\MB{R}^2$ is a function $\mathbf{F}$ that assigns to each point $(x,y)$ in $D$ a two-dimensional vector $\mathbf{F}(x,y)$.
\end{definition}

\lspace


\begin{center}
\includegraphics[scale=0.3]{2dvector_field.png}
\end{center}
\end{frame}










\begin{frame}[t]{Vector Fields}
\small

\begin{definition}
Let $D$ be a subset of $\MB{R}^2$. A \textbf{vector field on} $\MB{R}^2$ is a function $\mathbf{F}$ that assigns to each point $(x,y)$ in $D$ a two-dimensional vector $\mathbf{F}(x,y)$.
\end{definition}


Note: this is identical to the definition of a \textbf{slope field} that you saw back in Calc II, only we didn't call it a vector field.

\lspace
\visible<2->{Since $\mathbf{F}(x,y)$ is a two-dimensional vector, we can write it in terms of its component functions $P$ and $Q$:

\[
\mathbf{F}(x,y)=P(x,y)\mathbf{i}+Q(x,y)\mathbf{j}\ospace \mathbf{F}=P\mathbf{i}+Q\mathbf{j}.
\]

where $P$ and $Q$ are \textbf{scalar fields}.}
\end{frame}

















\begin{frame}[t]{Vector Fields}
\small

\begin{definition}
Let $E$ be a subset of $\MB{R}^3$. A \textbf{vector field on $\MB{R}^3$} is a function $\mathbf{F}$ that assigns to each point $(x,y,z)$ in $E$ a three-dimensional vector $\mathbf{F}(x,y,z)$.\

\lspace
\begin{center}
\includegraphics[scale=0.3]{3dvector_field.png}
\end{center}
\end{definition}
\end{frame}









\begin{frame}[t]{Vector Fields}
\small
\begin{example}
Let $\mathbf{F}(x,y)=-y\mathbf{i}+x\mathbf{j}$. Sketch $\mathbf{F}$.\pause 

\lspace
Here we select points and draw arrows representative of the vector produced at that point.

\begin{center}
\includegraphics[scale=0.3]{vectorfield_ex1.png}
\end{center}
\end{example}
\end{frame}







\begin{frame}[t]{Vector Fields}
\small
\begin{example}
Let $\mathbf{F}(x,y)=-y\mathbf{i}+x\mathbf{j}$. Sketch $\mathbf{F}$.

\lspace
It appears that each vector is tangent to some circle centered at the origin. To check this, we can take the dot product of the position vector $\mbf{v}=\<x,y>=x\mathbf{i}+y\mathbf{j}$ and our function $\mathbf{F}$ evaluated at the point corresponding to that position vector, $\mathbf{F}(x,y)=\mbf{F}(\mbf{v})$:\pause 

\[
\mbf{v}\cdotr\mbf{F}(\mbf{v})=\<x,y>\cdotr\mathbf{F}(x,y)=(x\mathbf{i}+y\mathbf{j})\cdotr(-y\mathbf{i}+x\mathbf{j})=-xy+yx=0.
\]\pause 

So $\mathbf{F}(x,y)$ is orthogonal to the position vector $\<x,y>$. \pause Notice also that 

\[
|\mathbf{v}|=\sqrt{x^2+y^2}\aspace|\mathbf{F}(x,y)|=\sqrt{(-y)^2+x^2}=|\mathbf{v}|,
\]
so the magnitude of $\mathbf{F}$ is the same as the radius of the circle to which it is tangent.
\end{example}
\end{frame}












\begin{frame}[t]{Vector Fields}
\small
\begin{example}
Sketch the vector field on $\MB{R}^3$ given by $\mathbf{F}(x,y,z)=z\mathbf{k}$.\pause 


\begin{center}
\includegraphics[scale=0.23]{vectorfield_ex2.png}
\end{center}

The vector field doesn't depend on $x$ or $y$. It will scale the vector $\mathbf{k}$ by whatever value $z$ is, so the vector field is vertical rays that extend in length as distance is increased from the $xy$-plane and in direction of the sign of $z$.
\end{example}
\end{frame}









\begin{frame}[t]{Applications}

There are infinitely many applications of vector fields.\pause 

\lspace
\begin{example}
Imagine a fluid flowing steadily through a pipe and let $\mathbf{V}(x,y,z)$ be the velocity vector at a point $(x,y,z)$. Then $\mathbf{V}$ assigns a (three-dimensional) vector to each point in a certain domain $D\subset\MB{R}^3$. Thus $\mathbf{V}$ is a vector field.
\end{example}
\end{frame}








\begin{frame}[t]{Applications}
\lspace
\begin{example}
Newton's Law of Gravitation states that the magnitude of the gravitational force between two objects with masses $m_1$ and $m_2$ is given by

\[
|\mathbf{F}|=\frac{m_1m_2g}{r^2}
\]

where $r$ is the distance between the objects and $g$ is the gravitational constant. \pause Suppose the object of mass $m_2$ is located at the origin in $\MB{R}^3$, and let the position vector of the object with mass $m_1$ be $\mathbf{x}=\<x,y,z>$. Then $r=|\mathbf{x}|$ so $r^2=|\mathbf{x}|^2$.
\end{example}
\end{frame}









\begin{frame}[t]{Applications}
\lspace
\begin{example}

The gravitational force exerted on this object acts toward the origin, and a unit vector in that direction is $-\frac{\mathbf{x}}{|\mathbf{x}|}$. \pause Thus, the gravitational force acting on the object at $\mathbf{x}$ is

\begin{minipage}{0.5\textwidth}
\[
\mathbf{F}(\mathbf{x})=-\frac{gm_1m_2}{|\mathbf{x}|^3}\mathbf{x}=-\frac{gm_1m_2}{|\mathbf{x}|^2}\cdotr\frac{\mathbf{x}}{|\mbf{x}|}
\]
\end{minipage}%
\begin{minipage}{0.5\textwidth}
\centering
\includegraphics[scale=0.3]{vectorfield_ex3.png}
\end{minipage}
\end{example}
\end{frame}










\begin{frame}[t]{Applications}
\small
\begin{example}
Suppose an electric charge $Q$ is located at the origin. According to Coulomb's Law, the electric force $\mathbf{F}(\mathbf{x})$ exerted by this charge on a charge $q$ located at a point $(x,y,z)$ with position vector $\mathbf{x}=\<x,y,z>$ is
\[
\mathbf{F}(\mathbf{x})=\frac{\varepsilon qQ}{|\mathbf{x}|^3}\mathbf{x}=\frac{\ve qQ}{|\mbf{x}|^2}\cdotr\frac{\mbf{x}}{|\mbf{x}|}
\]
where $\varepsilon$ is a constant. \pause Notice the similarity between the last two formulas. \pause Both are examples of \textbf{force fields}.\pause 

\lspace
Sometimes physicists often consider the force per unit charge:
\[
\mathbf{E}(\mathbf{x})=\frac{1}{q}\mathbf{F}(\mathbf{x})=\frac{\varepsilon Q}{|\mathbf{x}|^3}\mathbf{x}.
\]
$\mathbf{E}$ is called the \textbf{electric field} of $Q$.
\end{example}
\end{frame}













\begin{frame}[t]{Gradient Fields}

Recall that if $f$ is a scalar function of two variables, then the gradient is

\[
\nabla f(x,y)=f_x(x,y)\mathbf{i}+f_y(x,y)\mathbf{j}=\<f_x,f_y>.
\]\pause

In other words, the gradient is actually a vector field! \pause Sometimes it is called the \textbf{gradient vector field}. If $f$ is a scalar function of three variables, then the gradient vector field is 

\[
\nabla f(x,y,z)=f_x(x,y,z)\mathbf{i}+f_y(x,y,z)\mathbf{j}+f_z(x,y,z)\mathbf{k}=\<f_x,f_y,f_z>.
\]
\end{frame}












\begin{frame}[t]{Gradient Fields}
\small
\begin{example}
Find the gradient vector field of $f(x,y)=x^2y-y^3$. Plot the gradient vector field together with a contour map of $f$.\pause 

\lspace
\begin{minipage}{0.5\textwidth}
The gradient vector field is 
\[
\nabla f(x,y)=2xy\mathbf{i}+(x^2-3y^2)\mathbf{j}.
\]
\end{minipage}%
\begin{minipage}{0.5\textwidth}
\centering
\includegraphics[scale=0.2]{vectorfield_ex4.png}
\end{minipage}

\lspace
Notice here that the gradient field consists of all vectors that are orthogonal to the level curves, i.e. all vectors tangent to the orthogonal trajectories!
\end{example}
\end{frame}













\begin{frame}[t]{Gradient Fields}
\small
\begin{definition}
A vector field $\mathbf{F}$ is called a \textbf{conservative vector field} if it is the gradient of some scalar function, i.e. there exists a function $f$ such that $\mathbf{F}=\nabla f$. \pause In this situation $f$ is called a \textbf{potential function} for $\mathbf{F}$. \pause Not all vector fields are conservative, but many that arise in applications are. We will see why this terminology is used in a later section.
\end{definition}\pause

\begin{center}
\uline{\textbf{Key Idea}}
\end{center}

Geometrically, if $\mathbf{F}$ is a conservative vector field, then we can find a surface described by $f$ in $\MB{R}^3$ such $\mathbf{F}$ is the gradient vector at any point on that surface, i.e. $\nabla f=\mathbf{F}$. Ponder this and try to imagine a vector field with a corresponding surface.
\end{frame}







\end{document}