\documentclass[11pt,english,
handout
]{beamer}

%Preamble  
\input{/Users/Adam/Desktop/LBCC/MATH80/MATH80_Lesson_Plans/MATH80_Slides_Preamble.tex}

%Textbook: Essential Calculus - Early Transcendentals, 2nd edition - Stewart. ISBN: 978-1-133-11228-0



\begin{document}

%Slide titles are all contained in this file..
\ExecuteMetaData[/Users/Adam/Desktop/LBCC/MATH80/MATH80_Lesson_Plans/MATH80_Slide_Titles.tex]{1602}

%Global Title Slide Format is contained in the following file.
\input{/Users/Adam/Desktop/LBCC/MATH80/MATH80_Lesson_Plans/MATH80_Title_Slide_Format.tex}
\makebeamertitle










\begin{frame}{Introduction}

In this section we define and develop an extremely important use of the integral: \textit{line integrals}. Line integrals are similar to single integrals except instead of integrating over an interval, we'll be integrating over a curve $C$. They were invented in the early 19th century to solve problems involving fluid flow, forces, electricity, and magnetism.

\lspace
\textbf{Note.} We will be studying two different kinds of line integrals: line integrals of scalar fields and line integrals of vector fields. These are conceptually different but they connect when working with \textit{conservative} vector fields.
\end{frame}







\begin{frame}[t]{Development - Line Integrals of Scalar Fields}
\small
Let $C$ be a smooth curve in the $xy$-plane defined by the parametric equations $x=x(t)$, $y=y(t)$ where $a\leq t\leq b$, or equivalently by the vector equation $\mathbf{r}(t)=x(t)\mathbf{i}+y(t)\mathbf{j}$. \visible<2->{We divide the parameter interval $[a,b]$ into $n$ subintervals $[t_{i-1},t_i]$ of equal length. Let $x_i=x(t_i)$ and $y_i=y(t_i)$.} \visible<2->{Then, the corresponding points $P_i(x_i,y_i)$ divide $C$ into $n$ subarcs with lengths $\Delta s_i$. 


\begin{center}
\includegraphics[scale=0.3]{line_integral1.png}
\end{center}}
\end{frame}












\begin{frame}[t]{Development - Line Integrals of Scalar Fields}
\small
Choose any point $P_i^*(x_i^*,y_i^*)$ in the $i$th subarc. Let $f$ be a scalar function of two variables whose domain includes the curve $C$. If we evaluate $f$ at $(x_i^*,y_i^*)$, multiply the result by $\Delta s_i$, and add all of these products up, we get an approximation of the area under the ``curtain'' formed by $C$ and its projection onto the $xy$-axis:
\[
\sum_{i=1}^nf(x_i^*,y_i^*)\,\Delta s_i.
\]
\begin{minipage}{0.5\textwidth}
\begin{center}
\includegraphics[scale=0.25]{line_integral1.png}
\end{center}
\end{minipage}%
\begin{minipage}{0.5\textwidth}
\begin{center}
\includegraphics[scale=0.22]{line_integral2.png}
\end{center}
\end{minipage}
\end{frame}










\begin{frame}[t]{Development - Line Integrals of Scalar Fields}
\small
Taking the limit as $n\rightarrow \infty$ we get...\pause 

\lspace
\begin{definition}
If a \textit{scalar field} $f$ is defined on a smooth curve $C$ given by the parametric equations $x=x(t)$, $y=y(t)$, $a\leq t\leq b$, then the \textbf{line integral of $f$ along $C$ (with respect to arc length)} is 

\[
\int_Cf(x,y)\,ds=\lom{n}{\infty}\sum_{i=1}^nf(x_i^*,y_i^*)\,\Delta s_i
\]

provided this limit exists.
\end{definition}\pause

\textbf{Recall:} Earlier, we found the arc length function 

\[
s(t)=\int_a^t|\mathbf{r}'(u)|\,du=\int_a^t\sqrt{\left(\dd{x}{u}\right)^2+\left(\dd{y}{u}\right)^2}\,du.
\]
\end{frame}









\begin{frame}[t]{Development - Line Integrals of Scalar Fields}
\small

Recall also that if we differentiate both sides of this equation we get

\begin{align*}
\dd{s}{t}&=|\mathbf{r}'(t)|=\sqrt{\left(\dd{x}{t}\right)^2+\left(\dd{y}{t}\right)^2},\text{ so}\\[2mm]
ds&=|\mbf{r}'(t)|\,dt=\sqrt{\left(\dd{x}{t}\right)^2+\left(\dd{y}{t}\right)^2}\,dt.
\end{align*}\pause 

Consequently, we can rewrite our formula for a \textbf{line integral} as



\[\boxed{
\int_Cf(x,y)\,ds=\int_a^bf(x(t)),y(t))\sqrt{\left(\dd{x}{t}\right)^2+\left(\dd{y}{t}\right)^2}\,dt=\int_a^bf(\mathbf{r}(t))|\mathbf{r}'(t)|\,dt.
}\]
\end{frame}








\begin{frame}[t]{Development - Line Integrals of Scalar Fields}
\small
\textbf{Note:} The line integral can also be interpreted as finding the ``weighted'' arc length, where the weight at any point on $\mathbf{r}(t)$ is given by the function $f$. \pause In fact, using the formula above, if we take $f\equiv 1$, i.e. set the weight equal to 1, then 

\[
\int_a^bf(\mathbf{r}(t))|\mathbf{r}'(t)|\,dt=\int_a^b1\cdot|\mathbf{r}'(t)|\,dt=\int_a^b|\mathbf{r}'(t)|\,dt,
\]

which is the usual length of the unweighted arc.

\lspace
\href{https://upload.wikimedia.org/wikipedia/commons/4/42/Line_integral_of_scalar_field.gif}{(Animation)}

\lspace
\textbf{Recall:} The value of the line integral does not depend on the parameterization of the curve.
\end{frame}






\begin{frame}[t]{Line Integrals of Scalar Fields}
\small
\begin{example}
Evaluate $\int_C2+x^2y\,ds$ where $C$ is the upper half of the unit circle.

\lspace
The parameterization of this curve is standard: $x=\cos t$, $y=\sin t$, with $0\leq t \leq \pi$. \pause Thus,

\begin{align*}
\int_C(2+x^2y)\,ds&=\int_0^\pi(2+\cos^2t\sin t)\sqrt{\left(\dd{x}{t}\right)^2+\left(\dd{y}{t}\right)^2}\,dt\\[2mm]
&=\int_0^\pi(2+\cos^2t\sin t)\sqrt{\sin^2t+\cos^2t}\,dt\\[2mm]
&=\int_0^\pi(2+\cos^2t\sin t)\cdot 1\,dt=\left[2t-\frac{\cos^3t}{3}\right]_0^\pi=2\pi+\frac{2}{3}.
\end{align*}
\end{example}

\end{frame}















\begin{frame}{Line Integrals of Scalar Fields}
\small
\textbf{Note:} If a curve $C$ is a \textbf{piecewise smooth curve}, i.e. $C$ can be decomposed into a finite union of smooth curves $C_1,C_2,\ldots,C_n$, then the line integral of the curve $C$ is the sum of the line integrals of the curves $C_i$:

\[
\boxed{\int_Cf(x,y)\,ds=\int_{C_1}f(x,y)\,ds+\int_{C_2}f(x,y)\,ds+\cdots+\int_{C_n}f(x,y)\,ds}
\]
\end{frame}














\begin{frame}[t]{Basic Physics Applications}
\small
Physical interpretations of the line integral depend on the physical interpretation of the function $f$. \pause For example, suppose $\rho(x,y)$ represents the \textit{linear} density at a point $(x,y)$ of a thin wire shaped like a curve $C$. \pause Then the mass of the curve from $P_{i-1}$ to $P_i$ is approximately $\rho(x_i^*,y_i^*)\,\Delta s_i$. \pause Extending this idea, we get that the \textbf{mass} $m$ of the wire is
\[
m=\lom{n}{\infty}\sum_{i=1}^n\rho(x_i^*,y_i^*)\,\Delta s_i=\int_C\rho(x,y)\,ds.
\]\pause

The \textbf{center of mass} of the wire with density function $\rho$ is the point $(\bar{x},\bar{y})$ where
\[
\bar{x}=\frac{1}{m}\int_Cx\rho(x,y)\,ds\aspace\bar{y}=\frac{1}{m}\int_Cy\rho(x,y)\,ds.
\]
\end{frame}











\begin{frame}[t]{Basic Physics Applications}
\small
\begin{example}
A wire takes the shape of a semicircle  $x^2+y^2=1$ with $y\geq 0$, and it is thicker near its base than near the top. Find the center of mass of the wire if the linear density at any point is proportional to its distance from the line $y=1$.\pause

\lspace
Our linear density function is $\rho(x,y)=k(1-y)$ where $k$ is a constant. \pause Then we have

\[
m=\int_Ck(1-y)\,ds=\int_0^\pi k(1-\sin t)\,dt=k\left[t+\cos t\right]_0^\pi=k(\pi-2).
\]
\end{example}
\end{frame}










\begin{frame}[t]{Basic Physics Applications}
\small
\begin{example}
A wire takes the shape of a semicircle  $x^2+y^2=1$ with $y\geq 0$, and it is thicker near its base than near the top. Find the center of mass of the wire if the linear density at any point is proportional to its distance from the line $y=1$.

\lspace
Thus,
\begin{align*}
\bar{y}&=\frac{1}{m}\int_Cy\rho(x,y)\,ds=\frac{1}{k(\pi-2)}\int_Cyk(1-y)\,ds\\[2mm]
&=\frac{1}{\pi-2}\int_0^\pi(\sin t-\sin^2t)\,dt=\frac{1}{\pi-2}\left[-\cos t-\frac{1}{2}t^2+\frac{1}{4}\sin2t\right]_0^\pi\\[2mm]
&=\frac{4-\pi}{2(\pi-2)}.
\end{align*}
\end{example}
\end{frame}










\begin{frame}[t]{Basic Physics Applications}
\small
\begin{example}
A wire takes the shape of a semicircle  $x^2+y^2=1$ with $y\geq 0$, and it is thicker near it base than near the top. Find the center of mass of the wire if the linear density at any point is proportional to its distance from the line $y=1$.

\lspace
Since the wire is symmetric about the $y$-axis, $\bar{x}=0$, and we have

\[
(\bar{x},\bar{y})=\left(0,\frac{4-\pi}{2(\pi-2)}\right)\approx(0,0.38).
\]
\end{example}
\end{frame}
















\begin{frame}[t]{Line Integrals with respect to $x$ and $y$}
\small
Two other very important line integrals arise:

\lspace
\begin{definition}
The \textbf{line integrals of $f$ along $C$ with respect to $x$ and $y$} are

\begin{align*}
\int_Cf(x,y)\,dx&=\lom{n}{\infty}\sum_{i=1}^nf(x_i^*,y_i^*)\,\Delta x_i\\[2mm]
\int_Cf(x,y)\,dy&=\lom{n}{\infty}\sum_{i=1}^nf(x_i^*,y_i^*)\,\Delta y_i
\end{align*}
\end{definition}
\end{frame}









\begin{frame}[t]{Line Integrals with respect to $x$ and $y$}
\small 
Since $\dd{x}{t}=x'(t)$ and $\dd{y}{t}=y'(t)$, we can write


\lspace
\[
\boxed{\int_Cf(x,y)\,dx=\int_a^bf(x(t),y(t))x'(t)\,dt}
\]
\[
\boxed{\int_Cf(x,y)\,dy=\int_a^bf(x(t),y(t))y'(t)\,dt}
\]
\end{frame}









\begin{frame}[t]{Line Integrals with respect to $x$ and $y$}
\small 

Roughly and geometrically speaking, this is like finding the net area under the projection of the curve $C$ onto the $xz$-plane and the $yz$ plane respectively.

\lspace
\begin{center}
\includegraphics[scale=0.25]{line_integral3.jpg}
\end{center}
\end{frame}
















\begin{frame}{Line Integrals with respect to $x$ and $y$}
\small 

\textbf{Notation.} As you might imagine, line integrals with respect to $x$ and $y$ often occur together. It is customary to abbreviate the sum of such integrals as follows.
\[
\int_CP(x,y)\,dx+\int_CQ(x,y)\,dy=\int_CP(x,y)\,dx+Q(x,y)\,dy
\]\pause 

\lspace
\textbf{Question.} Does $\displaystyle \int_CP(x,y)\,dx+Q(x,y)\,dy=\int_C[P(x,y)+Q(x,y)]\,ds$?\pause 

\lspace
Not in general! \pause This is equivalent to asking if it is always true that the sides of a triangle add up to the length of the hypotenuse. When would the above equation be true?? %This is only true when the vertices of the triangle are collinear, but in that case our curve must be parallel to the $xz$- or $yz$-planes.
\end{frame}








\begin{frame}{Line Integrals with respect to $x$ and $y$}

\textbf{Note:} Often times we need to do a line integral over an actual line segment. For this, it is handy to remember the following vector representation of a line segment:

\[
\boxed{\mathbf{r}(t)=(1-t)\mathbf{r}_0+t\mathbf{r}_1\pspace0\leq t\leq 1}
\]
\end{frame}









\begin{frame}[t]{Line Integrals with respect to $x$ and $y$}
\small 
\begin{example}
Evaluate $\displaystyle \int_Cy^2\,dx+x\,dy$ where $C=C_1\cup C_2$ where $C_1$ is the line segment from $(-5,-3)$ to $(0,2)$ and $C_2$ is the arc of the parabola $x=4-y^2$ from $(-5,-3)$ to $(0,2)$.\pause 

\begin{center}
\includegraphics[scale=0.35]{ex1.png}
\end{center}
\end{example}
\end{frame}











\begin{frame}[t]{Line Integrals with respect to $x$ and $y$}
\small 
\begin{example}
Evaluate $\displaystyle \int_Cy^2\,dx+x\,dy$ where $C=C_1\cup C_2$ where $C_1$ is the line segment from $(-5,-3)$ to $(0,2)$ and $C_2$ is the arc of the parabola $x=4-y^2$ from $(-5,-3)$ to $(0,2)$.

\lspace
Considering distances between the initial and terminal coordinates of the end points of the line segment yield the parameterization 
\[
x=5t-5\pspace y=5t-3\pspace 0\leq t\leq 1.
\]\pause
From these we get $dx=5\,dt$ and $dy=5\,dt$, so
\begin{align*}
\int_Cy^2\,dx+x\,dy&=\int_0^1 (5t-3)^2(5\,dt)+(5t-5)(5\,dt)\\[2mm]
&=5\int_0^1(25t^2-25t+4)\,dt=5\left[\frac{25t^3}{3}-\frac{25t^2}{2}+4t\right]_0^1=-\frac{5}{6}.
\end{align*}
\end{example}
\end{frame}








\begin{frame}[t]{Line Integrals with respect to $x$ and $y$}
\small 
\begin{example}
Evaluate $\displaystyle \int_Cy^2\,dx+x\,dy$ where $C=C_1\cup C_2$ where $C_1$ is the line segment from $(-5,-3)$ to $(0,2)$ and $C_2$ is the arc of the parabola $x=4-y^2$ from $(-5,-3)$ to $(0,2)$.

\lspace
The parabola is already expressed as a function of $y$, so take $y$ to be the parameter and write $C_2$ as
\[
x=4-y^2\pspace y=y\pspace -3\leq y\leq 2.
\]\pause 
Then $dx=-2y\,dy$ so
\begin{align*}
\int_Cy^2\,dx+x\,dy&=\int_{-3}^2y^2(-2y)\,dy+(4-y^2)\,dy=\int_{-3}^2(-2y^3-y^2+4)\,dy\\[2mm]
&=\left[-\frac{1}{2}y^4-\frac{1}{3}y^3+4y\right]_{-3}^2=40\frac{5}{6}.
\end{align*}
\end{example}
\end{frame}









\begin{frame}[t]{Line Integrals with respect to $x$ and $y$}
\small 
\begin{example}
Evaluate $\displaystyle \int_Cy^2\,dx+x\,dy$ where $C=C_1\cup C_2$ where $C_1$ is the line segment from $(-5,-3)$ to $(0,2)$ and $C_2$ is the arc of the parabola $x=4-y^2$ from $(-5,-3)$ to $(0,2)$.

\lspace
Therefore
\begin{align*}
\int_Cy^2\,dx+x\,dy&=\int_{C_1}y^2\,dx+x\,dy+\int_{C_2}y^2\,dx+x\,dy\\[2mm]
&=-\frac{5}{6}+40\frac{5}{6}=40.
\end{align*}
\end{example}
\end{frame}











\begin{frame}[t]{Line Integrals with respect to $x$ and $y$}
\small 
\textbf{Note:} We got different results depending on which curve we traversed, even though the curves had the same initial and terminal points. \pause In general, the value of a line integral depends not only on the endpoints alone, but the path chosen as well. In the next section we will see conditions under which the line integral is actually independent of the path.\pause

\lspace
\textbf{Note:} We made a choice of \textbf{orientation} in the previous example; we chose which point was initial and which point is terminal. \pause This effects the value of the line integral with respect to $x$ or $y$. \pause Let $-C$ be the curve $C$ but with opposite orientation. Then

\[
\int_{-C}f(x,y)\,dx=-\int_Cf(x,y)\,dx\aspace \int_{-C}f(x,y)\,dy=-\int_Cf(x,y)\,dy
\] 
\end{frame}











\begin{frame}[t]{Line Integrals with respect to $x$ and $y$}
\small 

This makes sense if you recall why 
\[
\int_a^bf(x)\,dx=-\int_b^af(x)\,dx.
\]\pause 

However, if we integrate a scalar function $f$ with respect to arc length, the value of the line integral does \textit{not} change when we reverse the orientation of $C$. (why?) In other words,

\[
\int_{-C}f(x,y)\,ds=\int_Cf(x,y)\,ds.
\]
\end{frame}



















\begin{frame}{Line Integrals in Space}
\small
For line integrals of functions of three variables and above, the definitions generalize as expected. \pause Suppose $C$ is a smooth curve given by the equations

\[
x=x(t)\pspace y=y(t)\pspace z=z(t)\pspace a\leq t\leq b
\]

or by a vector equation 
\[
\mathbf{r}(t)=x(t)\mathbf{i}+y(t)\mathbf{j}+z(t)\mathbf{k}.
\]
\end{frame}













\begin{frame}[t]{Line Integrals in Space}
\small

\begin{definition}
If $f$ is a scalar function of three variables and $f$ is continuous on some domain that contains $C$, then the \textbf{line integral of $f$ along $C$} with respect to arc length is 

\begin{align*}
\int_Cf(x,y,z)\,ds&=\lom{n}{\infty}\sum_{i=1}^nf(x_i^*,y_i^*,z_i^*)\,\Delta s_i\\[2mm]
&=\int_Cf(x(t),y(t),z(t))\sqrt{\left(\dd{x}{t}\right)^2+\left(\dd{y}{t}\right)^2+\left(\dd{z}{t}\right)^2}\,dt\\[2mm]
&=\int_Cf(\mathbf{r}(t))|\mathbf{r}'(t)|\,dt.
\end{align*}
\end{definition}
\end{frame}









\begin{frame}[t]{Line Integrals in Space}
\small

Line integrals with respect to $x$, $y$, and $z$ can be constructed in the analogous way although the geometric interpretation is much more difficult visualize. For example,

\[
\int_Cf(x,y,z)\,dz=\int_Cf(x(t),y(t),z(t))z'(t)\,dt\text{ and}
\]
\[
\int_CP(x,y,z)\,dx+Q(x,y,z)\,dy+R(x,y,z)\,dz
\]
\end{frame}

















\begin{frame}[t]{Line Integrals in Space}
\small
\begin{example}
Evaluate the integral $\displaystyle \int_Cy\sin z\,ds$ where $C$ is the circular helix given by $x=\cos t$, $y=\sin t$, $z=t$, where $0\leq t\leq 2\pi$.\pause

\lspace
\begin{align*}
\int_Cy\sin z\,ds&=\int_0^{2\pi}\sin^2t \sqrt{\left(\dd{x}{t}\right)^2+\left(\dd{y}{t}\right)^2+\left(\dd{z}{t}\right)^2}\,dt\\[2mm]
&=\int_0^{2\pi}\sin^2t \sqrt{\cos^2t+\sin^2t+1}\,dt\\[2mm]
&=\frac{\sqrt{2}}{2}\int_0^{2\pi}1-\cos2t\,dt=\frac{\sqrt{2}}{2}\left[t-\frac{1}{2}\sin2t\right]_0^{2\pi}=\sqrt{2}\pi.
\end{align*}
\end{example}
\end{frame}












\begin{frame}[t]{Development - Line Integrals of \textbf{Vector Fields}}

Now we begin our journey into vector calculus. \pause Earlier in this course, we saw that the work done by a constant force $\mathbf{F}$ in moving an object from point $P$ to point $Q$ in space is $W=\mathbf{F}\cdotr\mathbf{D}$ where $\mathbf{D}=\ovec{PQ}$ the displacement vector. \pause Now suppose 

\[
\mathbf{F}=P\mathbf{i}+Q\mathbf{j}+R\mathbf{k}
\] 

is a continuous force field on $\MB{R}^3$. \pause 

\lspace

Our goal is to compute the work done by this force field in moving an object along a space curve $C$ from $P$ to $Q$. 
\end{frame}






\begin{frame}[t]{Development - Line Integrals of Vector Fields}

\small
Divide up $C$ into subarcs $P_{i-1}P_i$ of length $\Delta s_i$ by dividing the parameter interval into $n$ equal-length subintervals. \visible<1->{Choose a point $P_i^*(x_i^*,y_i^*,z_i^*)$ in the $i$th subarc that corresponds to the parameter $t_i^*$ in the $i$th subinterval.} \visible<1->{If $\Delta s_i$ is small, then the object moves along $C$ approximately in the direction of $\mathbf{T}(t_i^*)$, the unit tangent vector at $P_i^*$.}

\visible<1->{
\begin{center}
\includegraphics[scale=0.3]{vect_lineint.png}
\end{center}}
\end{frame}















\begin{frame}[t]{Development - Line Integrals of Vector Fields}

\small
Consequently, the work done by the force field in moving the object from $P_{i-1}$ to $P_i$ is approximately

\[
W_i=\mathbf{F}(x_i^*,y_i^*,z_i^*)\cdotr[\Delta s_i\,\mathbf{T}(t_i^*)]= [\mathbf{F}(x_i^*,y_i^*,z_i^*)\cdotr\mathbf{T}(x_i^*,y_i^*,z_i^*)]\Delta s_i.
\]

\vspace{0.1mm}
\visible<1->{
\begin{center}
\includegraphics[scale=0.3]{vect_lineint.png}
\end{center}}
\end{frame}








\begin{frame}[t]{Development - Line Integrals of Vector Fields}
\small

Thus the total work done, $W$, is approximately 
\[
\sum_{i=1}^n [\mathbf{F}(x_i^*,y_i^*,z_i^*)\cdotr\mathbf{T}(x_i^*,y_i^*,z_i^*)]\Delta s_i,
\]
and precisely

\[
\boxed{W=\int_C\mathbf{F}(x,y,z)\cdotr\mathbf{T}(x,y,z)\,ds=\int_C \mathbf{F}\cdotr \mathbf{T}\,ds.}
\]

\lspace
This says that work done by the force field in pushing an object along $C$ is the line integral with respect to arc length of the tangential component of the force field along $C$.
\end{frame}














\begin{frame}[t]{Development - Line Integrals of Vector Fields}
\small
If $C$ is given by the vector equation 
\[
\mathbf{r}(t)=x(t)\mathbf{i}+y(t)\mathbf{j}+z(t)\mathbf{k},
\] 
then from our previous studies,

\[
\mathbf{T}(t)=\frac{\mathbf{r}'(t)}{|\mathbf{r}'(t)|}\pspace \dd{\mbf{r}}{t}=\mbf{r}'(t) \pspace ds=|\mbf{r}'(t)|\,dt.
\]\pause 

Therefore we can rewrite the work formula as

\[
W=\int_C\left[\mathbf{F}(\mathbf{r}(t))\cdotr\frac{\mathbf{r}'(t)}{|\mathbf{r}'(t)|}\right]|\mathbf{r}'(t)|\,dt= \int_C\mathbf{F}(\mathbf{r}(t))\cdotr\mathbf{r}'(t)\,dt=\int_C\mathbf{F}\cdotr\,d\mathbf{r}.
\]
\end{frame}




\begin{frame}[t]{Development - Line Integrals of Vector Fields}
\footnotesize
In summary,

\lspace
\begin{definition}
Let $\mathbf{F}$ be a continuous vector field defined on a smooth curve $C$ which is given by a vector equation $\mathbf{r}(t)$ with $a\leq t\leq b$. Then the \textbf{line integral of $\mathbf{F}$ along $C$} is

\lspace
\[
\boxed{W=\int_C\mathbf{F}\cdotr\mathbf{T}\,ds=\int_C\mathbf{F}(\mathbf{r}(t))\cdotr\mathbf{r}'(t)\,dt=\int_C\mathbf{F}\cdotr\,d\mathbf{r}}
\]
\end{definition}\pause
\lspace
This says is that the line integral over a curve in a force field adds up the curve-tangential component of the force acting on the curve at every point. \pause Considered another way, this gives us a \textit{measure} of how well the curve is aligned with the force field. \pause If $C$ is a closed curve, then the line integral gives us the \textbf{circulation} of the vector field around $C$, i.e. how much the vector field tends to circulate around the curve.

\end{frame}


















\begin{frame}[t]{Line Integrals of Vector Fields}
\small

\begin{example}
Find the work done by the force field $\mathbf{F}(x,y)=x^2\mathbf{i}-xy\mathbf{j}$ in moving a particle along the quarter circle $\mathbf{r}(t)=\cos t\mathbf{i}+\sin t\mathbf{j}$, $0\leq t \leq \frac{\pi}{2}$.


\begin{center}
\includegraphics[scale=0.4]{ex2.png}
\end{center}
\end{example}
\end{frame}













\begin{frame}[t]{Line Integrals of Vector Fields}
\small

\begin{example}
Find the work done by the force field $\mathbf{F}(x,y)=x^2\mathbf{i}-xy\mathbf{j}$ in moving a particle along the quarter circle $\mathbf{r}(t)=\cos t\mathbf{i}+\sin t\mathbf{j}$, $0\leq t \leq \frac{\pi}{2}$.

\lspace
\begin{minipage}{0.5\textwidth}
Since $x=\cos t$ and $y=\sin t$, we have 
\[
\mathbf{F}(\mathbf{r}(t))=\cos^2 t\mathbf{i}-\cos t\sin t \mathbf{j},
\]

\[
\mathbf{r}'(t)=-\sin t\mathbf{i}+\cos t\mathbf{j}.
\]
\end{minipage}%
\begin{minipage}{0.5\textwidth}
\centering
\includegraphics[scale=0.35]{ex2.png}
\end{minipage}
\end{example}
\end{frame}














\begin{frame}[t]{Line Integrals of Vector Fields}
\small

\begin{example}
Find the work done by the force field $\mathbf{F}(x,y)=x^2\mathbf{i}-xy\mathbf{j}$ in moving a particle along the quarter circle $\mathbf{r}(t)=\cos t\mathbf{i}+\sin t\mathbf{j}$, $0\leq t \leq \frac{\pi}{2}$.

\begin{align*}
\int_C\mathbf{F}\cdotr\,d\mathbf{r}&= \int_0^{\frac{\pi}{2}}\mathbf{F}(\mathbf{r}(t))\cdotr\mathbf{r}'(t)\,dt\\[2mm]
&=\int_0^{\frac{\pi}{2}}\<\cos^2t,-\cos t\sin t>\cdotr\<-\sin t,\cos t>\,dt\\[2mm]
&=\int_0^{\frac{\pi}{2}}-2\cos^2t\sin t\,dt=2\int_1^0u^2\,du\\[2mm]
&=\left.\frac{2}{3}u^3\right|_1^0=-\frac{2}{3}.
\end{align*}

\end{example}
\end{frame}













\begin{frame}[t]{Line Integrals of Vector Fields}
\small

\begin{example}
Find the work done by the force field $\mathbf{F}(x,y)=x^2\mathbf{i}-xy\mathbf{j}$ in moving a particle along the quarter circle $\mathbf{r}(t)=\cos t\mathbf{i}+\sin t\mathbf{j}$, $0\leq t \leq \frac{\pi}{2}$.

\lspace
\begin{minipage}{0.5\textwidth}
The sign of the result makes sense when we look at the picture again; the motion of a particle along the curve is against the current of the force field most of the time, so the field is doing negative work to help it along, i.e. it is pushing against the motion of the particle.
\end{minipage}%
\begin{minipage}{0.5\textwidth}
\centering
\includegraphics[scale=0.35]{ex2.png}
\end{minipage}

\end{example}
\end{frame}















\begin{frame}{Line Integrals of Vector Fields}
\small

\textbf{Note:} If we reverse the orientation of $C$, the sign of the integral will also change. This is because when the direction is reversed, $\mathbf{T}$ is replaced by $-\mathbf{T}$. In other words,

\[
\int_C\mathbf{F}\cdotr\mathbf{T}\,ds=\int_{-C}\mathbf{F}\cdotr(\mathbf{-T})\,ds=-\int_{-C}\mathbf{F}\cdotr\mathbf{T}\,ds.
\]
\end{frame}












\begin{frame}[t]{Line Integrals of Vector Fields}
\small
\begin{example}
Find $\displaystyle \int_C\mathbf{F}\cdot\,d\mathbf{r}$ where $\mathbf{F}(x,y,z)=xy\mathbf{i}+yz\mathbf{j}+zx\mathbf{k}$ and $C$ is the twisted cubic:
\[
x=t\pspace y=t^2\pspace z=t^3\pspace 0\leq t\leq 1
\]\pause


\begin{minipage}[t]{0.5\textwidth}
\begin{align*}
\mathbf{r}(t)&=t\mathbf{i}+t^2\mathbf{j}+t^3\mathbf{k}\\[5mm]
\mathbf{r}'(t)&=\mathbf{i}+2t\mathbf{j}+3t^2\mathbf{k}\\[5mm]
\mathbf{F}(\mathbf{r}(t))&=t^3\mathbf{i}+t^5\mathbf{j}+t^4\mathbf{k}
\end{align*}
\end{minipage}%
\begin{minipage}[t]{0.5\textwidth}
\begin{align*}
\int_C\mathbf{F}\cdot\,d\mathbf{r}&= \int_0^1\mathbf{F}(\mathbf{r}(t))\cdot\mathbf{r}'(t)\,dt\\[2mm]
&=\int_0^1t^3+5t^6,dt\\[2mm]
&=\left[\frac{1}{4}t^4+\frac{5}{7}t^7\right]_0^1=\frac{27}{28}.
\end{align*}
\end{minipage}
\end{example}
\end{frame}









\begin{frame}[t]{Line Integrals of Vector Fields}
\small

Now we note the profound connection between line integrals of vector fields and line integrals of scalar fields. Suppose $\mathbf{F}$ is a vector field given by 

\[
\mathbf{F}=P\mathbf{i}+Q\mathbf{j}+R\mathbf{k}.
\] \pause

Computing its line integral along $C$ reveals
\lspace 
{\scriptsize
\begin{align*}
\int_C\mathbf{F}\cdotr\,d\mathbf{r}&= \int_a^b\mathbf{F}(\mathbf{r}(t))\cdotr\mathbf{r}'(t)\,dt=\int_a^b (P\mathbf{i}+Q\mathbf{j}+R\mathbf{k})\cdotr(x'(t)\mathbf{i}+y'(t)\mathbf{j}+z'(t)\mathbf{k})\,dt\\[2mm]
&=\int_a^b\left[P(x(t),y(t),z(t))x'(t)+Q(x(t),y(t),z(t))y'(t)+R(x(t),y(t),z(t))z'(t)\right]\,dt\\[2mm]
&=\int_a^bP(x(t),y(t),z(t))x'(t)\,dt+Q(x(t),y(t),z(t))y'(t)\,dt+R(x(t),y(t),z(t))z'(t)\,dt\\[2mm]
&=\int_CP\,dx+Q\,dy+R\,dz.
\end{align*}
}
\end{frame}








\begin{frame}{Line Integrals of Vector Fields}
\small


\[
\int_C\mathbf{F}\cdotr\,d\mathbf{r}=\int_C\mathbf{F}\cdotr\mbf{T}\,ds=\int_CP\,dx+Q\,dy+R\,dz
\]

\lspace
\centering
\fbox{\parbox{0.9\textwidth}{The line integral of a vector field over a curve $C$ is equivalent to the line integral of the sum of the components with respect to $x$, $y$, and $z$.}}
\end{frame}












\end{document}