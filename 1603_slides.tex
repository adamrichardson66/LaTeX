\documentclass[11pt,english,
handout
]{beamer}

%Preamble  
\input{/Users/Adam/Desktop/LBCC/MATH80/MATH80_Lesson_Plans/MATH80_Slides_Preamble.tex}

%Textbook: Essential Calculus - Early Transcendentals, 2nd edition - Stewart. ISBN: 978-1-133-11228-0



\begin{document}

%Slide titles are all contained in this file..
\ExecuteMetaData[/Users/Adam/Desktop/LBCC/MATH80/MATH80_Lesson_Plans/MATH80_Slide_Titles.tex]{1603}

%Global Title Slide Format is contained in the following file.
\input{/Users/Adam/Desktop/LBCC/MATH80/MATH80_Lesson_Plans/MATH80_Title_Slide_Format.tex}
\makebeamertitle















\begin{frame}[t]{The Fundamental Theorem of Calculus}

Recall:

\lspace
\begin{theorem}[Fundamental Theorem of Calculus, Part 2]
Let $F'$ be continuous on $[a,b]$. Then $\displaystyle \int_a^bF'(x)\,dx=F(b)-F(a)$,

\lspace
i.e., the integral of a rate of change is the net change.
\end{theorem}\pause

\lspace
We adapt this important theorem to line integrals by interpreting the gradient $\nabla f$ of a function $f$ as a ``derivative'' of the function.
\end{frame}











\begin{frame}[t]{The Fundamental Theorem of Line Integrals}

\begin{theorem}[The Fundamental Theorem for Line Integrals]
Let $C$ be a smooth curve given by the vector function $\mathbf{r}(t)$, $a\leq t\leq b$. Let $f$ be a differentiable function of $n$ variables whose gradient vector $\nabla f$ is continuous on $C$. Then

\[
\int_C\nabla f\cdotr d\mathbf{r}=f(\mathbf{r}(b))-f(\mathbf{r}(a)).
\]
\end{theorem}\pause


Since $\mbf{F}=\nabla f$ is a conservative vector field by definition, this theorem says we can evaluate the integral of a conservative vector field just by knowing the value of $f$ at the endpoints of $C$. 
\end{frame}









\begin{frame}[t]{The Fundamental Theorem of Line Integrals}
\small

Let's prove the case where $f$ is a function of three variables.\pause 
\lspace
\begin{proof}
Let $C$ be a smooth curve and $f$ a differentiable function where $\nabla f$ is continuous on $C$. \pause Then
\begin{align*}
\int_C\nabla f\cdotr d\mathbf{r}&=\int_a^b\nabla f(\mathbf{r}(t))\cdotr\mathbf{r}'(t)\,dt\\[2mm]
&=\int_a^b\left(\pp{f}{x}\dd{x}{t}+\pp{f}{y}\dd{y}{t}+\pp{f}{z}\dd{z}{t}\right)\,dt\\[2mm]
&=\int_a^b\dd{}{t}f(\mathbf{r}(t))\,dt\\[2mm]
&=f(\mathbf{r}(b))-f(\mathbf{r}(a)).
\end{align*}\end{proof}
\end{frame}












%\begin{example}
%Find the work done by the gravitational field
%
%\[
%\mathbf{F}(\mathbf{x})=-\frac{m_1m_2g}{|\mathbf{x}|^3}\mathbf{x}.
%\]
%\end{example}



\begin{frame}[t]{Independence of Path}
\small
\begin{definition}
A piecewise-smooth curve in $\MB{R}^n$ is a \textbf{path}.
\end{definition}\pause 

\lspace
Suppose $C_1$ and $C_2$ are two paths that have the same initial point $A$ and terminal point $B$. \pause We saw before that, in general, 
\[
\int_{C_1}\mathbf{F}\cdotr \,d\mathbf{r}\neq\int_{C_2}\mathbf{F}\cdotr\,d\mathbf{r},
\]
but an important implication of FTL is that
\[
\int_{C_1}\nabla f\cdotr\,d\mathbf{r}=\int_{C_2}\nabla f\cdotr\,d\mathbf{r}
\]
whenever $\nabla f$ is continuous because we only need to evaluate the function at the endpoints of the curve. In other words, \textit{the line integral of a conservative vector field is independent of path.}
\end{frame}













\begin{frame}{Independence of Path}
\small
\begin{definition}
If $\mathbf{F}$ is a continuous vector field with domain $D$, we say the integral $\displaystyle \int_C\mathbf{F}\cdotr\,d\mathbf{r}$ is \textbf{path-independent} (or \textbf{independent of path}) if $\displaystyle \int_{C_1}\mathbf{F}\cdotr \,d\mathbf{r}=\int_{C_2}\mathbf{F}\cdotr\,d\mathbf{r}$ for any two paths $C_1$ and $C_2$ in $D$ that have the same initial points and the same terminal points.
\end{definition}\pause

\lspace
\textbf{Note:} The line integral of a \uline{conservative} vector field depends only on the initial and terminal point of a curve, so line integrals of conservative vector fields are path-independent. This makes sense if you remember the geometric interpretation from the end of the last section and the animation.
\end{frame}
 
 
 
 
 
 
 
 
 
 
 
 
 

 
\begin{frame}[t]{Independence of Path}
\small
 \begin{definition}
 A curve is called \textbf{closed} if its terminal point coincides with its initial point, i.e. $\mathbf{r}(b)=\mathbf{r}(a)$.
 \end{definition}\pause 
 
 \lspace
 \begin{theorem}
 $\displaystyle \int_C\mathbf{F}\cdotr\,d\mathbf{r}$ is path-independent in $D$ if and only if $\int_C\mathbf{F}\cdotr\,d\mathbf{r}=0$ for every closed path $C$ in $D$.
 \end{theorem}\pause
 \lspace
 
\begin{proofs}
First, suppose  $\displaystyle \int_C\mathbf{F}\cdotr\,d\mathbf{r}$ is path-independent in $D$ and $C$ is any closed path in $D$. \pause Choose any two points $A$ and $B$ on $C$. Then we can write $C=C_1\cup C_2$ where $C_1$ is the path from $A$ to $B$ and $C_2$ is the path from $B$ to $A$.
\end{proofs}
\end{frame}









\begin{frame}[t]{Independence of Path}
\small
\begin{proofs}
Then
\[
\int_C\mathbf{F}\cdotr\,d\mathbf{r}=\int_{C_1}\mathbf{F}\cdotr\,d\mathbf{r}+\int_{C_2}\mathbf{F}\cdotr\,d\mathbf{r}=\int_{C_1}\mathbf{F}\cdotr\,d\mathbf{r}-\int_{-C_2}\mathbf{F}\cdotr\,d\mathbf{r}=0
\]

by FTL and since the initial and terminal points of $C_1$ and $-C_2$ coincide respectively.

\lspace
Conversely, suppose that $\displaystyle \int_C\mathbf{F}\cdotr\,d\mathbf{r}=0$ for all closed paths $C$ in $D$. \pause Let $A$ and $B$ be any two points in $D$ and let $C_1$ and $C_2$ be two paths connecting $A$ and $B$. Define $C=C_1\cup -C_2$. \end{proofs}
\end{frame}










\begin{frame}[t]{Independence of Path}
\small
\begin{proof}
Then
\[
0=\int_C\mathbf{F}\cdotr\,d\mathbf{r}=\int_{C_1}\mathbf{F}\cdotr\,d\mathbf{r}+\int_{-C_2}\mathbf{F}\cdotr\,d\mathbf{r}=\int_{C_1}\mathbf{F}\cdotr\,d\mathbf{r}-\int_{C_2}\mathbf{F}\cdotr\,d\mathbf{r}.
\]\pause

Thus, $\displaystyle \int_{C_1}\mathbf{F}\cdotr\,d\mathbf{r}=\int_{C_2}\mathbf{F}\cdotr\,d\mathbf{r}$ and since $C_1$ and $C_2$ were chosen arbitrarily, this equality holds for any curves $C_1$ and $C_2$. In other words, $\int_C\mathbf{F}\cdotr\,d\mathbf{r}$ is path-independent.
\end{proof}
\end{frame}











\begin{frame}[t]{Independence of Path}
\small
\textbf{Note:} The line integral of any conservative vector field $\mathbf{F}$ is independent of path, so it follows that $\displaystyle \int_C\mathbf{F}\cdotr\,d\mathbf{r}=0$ for any closed curve $C$ over a conservative vector field $F$. \pause 

Physically this means that the net work done by a conservative force field as it moves an object around a closed path is 0, which makes sense because the displacement is 0.\pause

\lspace
\begin{definition}
A set $D$ is called \textbf{open} if for any point $P$ in $D$, there exists a disk with center $P$ that lies entirely inside $D$.
\end{definition}\pause

\lspace
\begin{definition}
A set $D$ is called \textbf{connected} if any two points in $D$ can be joined by a path that lies in $D$.
\end{definition}
\end{frame}












\begin{frame}[t]{Independence of Path}
\small
The next theorem gives us the other direction of implication of this result that we would hope for. We'll prove it in $\MB{R}^2$, it will take some work, and it can also  generalized to higher dimensions.\pause 

\lspace
\begin{theorem}
Suppose $\mathbf{F}$ is a vector field that is continuous on an open connected region $D$. If $\int_C\mathbf{F}\cdotr\,d\mathbf{r}$ is path-independent in $D$, then $\mathbf{F}$ is a conservative vector field on $D$, i.e. there exists a function $f$ such that $\nabla f=\mathbf{F}$.
\end{theorem}\pause 

\lspace
\begin{proofs}
Let $(a,b)$ be a fixed point in $D$. We claim the potential function we seek is

\[
f(x,y)=\int_{(a,b)}^{(x,y)}\mathbf{F}\cdotr\,d\mathbf{r}
\]

for any point $(x,y)$ in $D$. 
\end{proofs}
\end{frame}












\begin{frame}[t]{Independence of Path}
\small

\begin{proofs}
Since $\displaystyle \int_C\mathbf{F}\cdotr\,d\mathbf{r}$ is path-independent, it does not matter which path $C$ from $(a,b)$ to $(x,y)$ is used to evaluate $f(x,y)$ so it is a well-defined function. \pause Since $D$ is open, there exists a disk contained in $D$ with center $(x,y)$. Choose any point $(x_1,y)$ in the disk with $x_1<x$ and let $C$ consist of any path $C_1$ from $(a,b)$ to $(x_1,y)$ followed by the horizontal line segment $C_2$ from $(x_1,y)$ to $(x,y)$.

\vspace{0.5mm}
\begin{center}
\includegraphics[scale=0.35]{line_int_proof1.png}
\end{center}
\end{proofs}
\end{frame}








\begin{frame}[t]{Independence of Path}
\small

\begin{proofs}
Then

\[
f(x,y)=\int_{C_1}\mathbf{F}\cdotr\,d\mathbf{r}+\int_{C_2}\mathbf{F}\cdotr\,d\mathbf{r}=\int_{(a,b)}^{(x_1,y)}\mathbf{F}\cdotr\,d\mathbf{r}+\int_{C_2}\mathbf{F}\cdotr\,d\mathbf{r}.
\]

\lspace
\begin{center}
\includegraphics[scale=0.35]{line_int_proof1.png}
\end{center}
\end{proofs}
\end{frame}









\begin{frame}{Independence of Path}
\small

\begin{proofs}
The integral on the left is independent of $x$, thus,

\[
\pp{}{x}f(x,y)=\pp{}{x}\left(\int_{(a,b)}^{(x_1,y)}\mathbf{F}\cdotr\,d\mathbf{r}+\int_{C_2}\mathbf{F}\cdotr\,d\mathbf{r}\right)=0+\pp{}{x}\int_{C_2}\mathbf{F}\cdotr\,d\mathbf{r}.
\]\pause

\lspace
Write $\mathbf{F}=P\mathbf{i}+Q\mathbf{j}$ so

\[
\int_{C_2}\mathbf{F}\cdotr\,d\mathbf{r}=\int_{C_2}P\,dx+Q\,dy.
\]

\end{proofs}
\end{frame}










\begin{frame}{Independence of Path}
\small

\begin{proofs}
On $C_2$, $y$ is constant so $dy=0$. Using $t$ as our parameter with $x_1\leq t\leq x$, we have

\[
\pp{}{x}f(x,y)=\pp{}{x}\int_{C_2}P\,dx+Q\,dy=\pp{}{x}\int_{x_1}^xP(t,y)\,dt+0=P(x,y)
\]
\lspace

by FTC1.\pause

\lspace
We use a similar argument with a vertical line segment:
\end{proofs}
\end{frame}











\begin{frame}[t]{Independence of Path}
\small

\begin{proofs}
\begin{center}
\includegraphics[scale=0.3]{line_int_proof2.png}
\end{center}
\[
\pp{}{y}f(x,y)=\pp{}{y}\left(\int_{C_2}P\,dx+Q\,dy\right)=0+\pp{}{y}\int_{y_1}^yQ(x,t)\,dt=Q(x,y)
\]
\end{proofs}
\end{frame}







\begin{frame}{Independence of Path}
\small

\begin{proof}

Behold,

\[
\mathbf{F}=P\mathbf{i}+Q\mathbf{j}=\pp{f}{x}\mathbf{i}+\pp{f}{y}\mathbf{j}=\nabla f.
\]
\end{proof}\pause

\lspace
To recap: we just showed that if a line integral of a vector field is path independent, then that vector field must be conservative. This combined with the previous theorem says that if the line integral of a vector field over any closed curve is 0, then the vector field must be conservative. However, we still haven't developed a way to determine if a vector field is conservative given the vector field itself. 
\end{frame}










\begin{frame}{Conservative Vector Fields}
\small

\begin{theorem}
If
\[
\mathbf{F}(x,y)=P(x,y)\mathbf{i}+Q(x,y)\mathbf{j}
\]

\vspace{3mm}
is a conservative vector field, where $P$ and $Q$ have continuous first-order derivatives on a domain $D$, then throughout $D$ we have

\[
\pp{P}{y}=\pp{Q}{x}\pspace\text{i.e.}\pspace\pp{P}{y}-\pp{Q}{x}=0.
\]
\end{theorem}\pause 

\textbf{Note from the Future:} $\displaystyle \pp{P}{y}-\pp{Q}{x}$ is called the \textbf{curl} of $\mbf{F}$, and this theorem essentially says: if $\mbf{F}$ is conservative, then it doesn't push anything out of the $xy$-plane. \pause Start imagining an analogous result in higher dimensions! \pause The converse is also true, but under stricter conditions.
\end{frame}









\begin{frame}{Conservative Vector Fields}
\small

\begin{proof}
Let $\mathbf{F}=P\mathbf{i}+Q\mathbf{j}$ be a conservative vector field. Then there exists a function $f$ such that $\nabla f =\mathbf{F}$, i.e.

\[
P=\pp{f}{x}\aspace Q=\pp{f}{y}.
\]\pause 

By Clairaut's Theorem,

\[
\pp{P}{y}=\pp{}{y}\left(\pp{f}{x}\right)=\pp{^2f}{y\,\partial x}=\pp{^2f}{x\,\partial y}=\pp{}{x}\left(\pp{f}{y}\right)=\pp{Q}{x},\text{ so}
\]

\[
\pp{P}{y}-\pp{Q}{x}=0.
\]
\end{proof}


\end{frame}







\begin{frame}{Conservative Vector Fields}
\small

\begin{definition}
A \textbf{simple curve} is a curve that does not intersect itself anywhere between its endpoints.
\end{definition}\pause

\lspace
\begin{definition}
A \textbf{simply-connected region} in the plane is a connected region such that every simple closed curve in $D$ encloses only points that are in $D$. In other words, $D$ has no holes and it does not consist of two pieces.
\end{definition}

\end{frame}



\begin{frame}{Conservative Vector Fields}
\small

\begin{theorem}
Let $\mathbf{F}=P\mathbf{i}+Q\mathbf{j}$ be a vector field on an open simply-connected region $D$. Suppose that $P$ and $Q$ have continuous first-order partial derivatives and 
\[
\pp{P}{y}=\pp{Q}{x}\pspace\pspace\text{i.e.}\pspace\pp{P}{y}-\pp{Q}{x}=0\text{ throughout $D$.}
\]
Then $\mathbf{F}$ is conservative.
\end{theorem}

\lspace
Notice that the only new condition for us to get this characterization of conservative vector fields was that $D$ be simply-connected, i.e. have no ``holes''. \pause The shape and behavior of the region turns out to be extremely important and is studied in depth in graduate level complex analysis courses and beyond, like in my own research :-) \pause We will prove this theorem in the next section when we have a more powerful tool at hand.
\end{frame}








\begin{frame}[t]{Conservative Vector Fields}
\small
\begin{example}
Is the vector field $\mathbf{F}(x,y)=(x-y)\mathbf{i}+(x-2)\mathbf{j}$ a conservative vector field?\pause

\lspace
\begin{minipage}{0.5\textwidth}

\[
\pp{P}{y}=-1\aspace \pp{Q}{x}=1
\]

\lspace
so $\mathbf{F}$ is not conservative.
\vfill
\end{minipage}%
\begin{minipage}[c]{0.5\textwidth}
\centering
\includegraphics[scale=0.4]{nconex.png}
\end{minipage}
\end{example}
\end{frame}






















\begin{frame}[t]{Conservative Vector Fields}
\small
\begin{example}
Is the vector field $\mathbf{F}(x,y)=(x-y)\mathbf{i}+(x-2)\mathbf{j}$ a conservative vector field?

\lspace
\begin{minipage}{0.5\textwidth}
\small
Graphically, this vector field is not conservative because if we draw a closed path around the source in the field, the work produced by the field will be positive in pushing an object counterclockwise and negative in pushing it clockwise because it will (continuously) transfer energy to the object.

\lspace
This means $\int_C\mathbf{F}\cdotr\,d\mathbf{r}\neq0$.


\end{minipage}%
\begin{minipage}[c]{0.5\textwidth}
\centering
\includegraphics[scale=0.4]{nconex.png}
\end{minipage}
\end{example}
\end{frame}
























\begin{frame}[t]{Conservative Vector Fields}
\small

\begin{example}
Is $\mathbf{F}=(3+2xy)\mathbf{i}+(x^2-3y^2)\mathbf{j}$ conservative?\pause

\lspace
\begin{center}
\begin{minipage}{0.4\textwidth}
\[
\pp{P}{y}=2x=\pp{Q}{x},
\]

\lspace
and the domain of $\mathbf{F}$ is all of $\MB{R}^2$ which is simply connected so by our previous theorem, $\mathbf{F}$ is indeed conservative.
\end{minipage}\hspace{1cm}%
\begin{minipage}{0.5\textwidth}
\centering
\includegraphics[scale=0.35]{conex.png}
\end{minipage}
\end{center}
\end{example}
\end{frame}


















\begin{frame}[t]{Conservative Vector Fields}
\small

\begin{example}
Is $\mathbf{F}=(3+2xy)\mathbf{i}+(x^2-3y^2)\mathbf{j}$ conservative?

\lspace
\begin{center}
\begin{minipage}{0.4\textwidth}

In this example, we can draw closed curves where force positively/negatively/orthogonally aligns with the tangent of the object's path. In order for the object to move that way, it would have to transfer energy back to the field to maintain its trajectory, so the field conserves its energy.
\end{minipage}\hspace{1cm}%
\begin{minipage}{0.5\textwidth}
\centering
\includegraphics[scale=0.35]{conex.png}
\end{minipage}
\end{center}
\end{example}
\end{frame}






















\begin{frame}{Conservative Vector Fields}
\small

Now we can determine if a vector field $\mbf{F}$ is conservative, but how can we find the potential function $f$ such that 
\[
\nabla f=\mathbf{F}?
\]\pause 

\lspace
We will do an example with a three-dimensional vector field, but the process is equivalent for any $n$-dimensional vector field, $n\geq1$.
\end{frame}






%%%%%%%%% omitted examples




%\begin{example}$ $


%\begin{enumerate}
%\item If $\mathbf{F}(x,y)=(3+2xy)\mathbf{i}+(x^2-3y^2)\mathbf{j}$, find the potential function $\nabla f$.
%
%We saw a moment ago that this is indeed a conservative vector field. We are looking for functions $f_x$ and $f_y$ such that
%
%\[
%f_x(x,y)=2+2xy\,(*)\aspace f_y(x,y)=x^2-3y^2.\,(**)
%\]
%
%If we integrate $f_x$ above with respect to $x$, we get
%
%\[
%\pp{}{x}f_x(x,y)=f(x,y)=3x+x^2y+g(y).
%\]
%
%Here, $g(y)$ is a function of $y$ because any function of $y$ is constant with respect to $x$. Next, we differentiate both sides of this equation with respect to $y$:
%
%\[
%\pp{}{y}f(x,y)=\pp{}{y}[3x+x^2y+g(y)]=x^2+g'(y).
%\]
%
%Comparing that with $(**)$, it must be the case that $g'(y)=-3y^2$ which means $g(y)=-y^3+K$ where $K$ is some constant. Now that we know $g(y)$, we have
%
%\[
%f(x,y)=3x+x^2y-y^3+K.
%\]


%\item Evaluate the line integral $\int_C\mathbf{F}\cdotr\,d\mathbf{r}$ where $C$ is the curve given by $\mathbf{r}(t)=e^t\sin t\mathbf{i}+e^t\cos t\mathbf{j}$ where $0\leq t\leq \pi$.\
%
%Here we can use FTL, we just need to find the initial and terminal points of $\mathbf{r}(t)$. We have $\mathbf{r}(0)=(0,1)$ and $\mathbf{r}(\pi)=(0,-e^\pi)$. $K$ will get washed out in the calculation using FTL, so we can simply choose $K=0$ and dispense with it. Then
%
%\[
%\int_C\mathbf{F}\cdotr\,d\mathbf{r}=\int_C\nabla f\cdotr\,d\mathbf{r}=f(0,-e^\pi)-f(0,1)=e^{3\pi}+1.
%\]
%
%
%
%\end{enumerate}
%\end{example}


%%%%%%%%%%%






\begin{frame}[t]{Conservative Vector Fields}
\small

\begin{example}
If $\mathbf{F}(x,y,z)=y^2\mathbf{i}+(2xy+e^{3z})\mathbf{j}+3ye^{3z}\mathbf{k}$, find the potential function $f$ such that $\nabla f=\mathbf{F}$.\pause

\lspace
In order for such a function to exist, it must be the case that

\begin{align}
f_x(x,y,z)&=y^2 \tag{1}\\[2mm]
f_y(x,y,z)&=2xy+e^{3z} \tag{2}\\[2mm]
f_z(x,y,z)&=3ye^{3z}.\tag{3}
\end{align}\pause 

Integrating (1) with respect to \uline{$x$} yields $f(x,y,z)=xy^2+g(y,z)$. \pause Differentiating this with respect to $y$ gives
\[
f_y(x,y,z)=2xy+g_y(y,z).
\]
\end{example}
\end{frame}











\begin{frame}[t]{Conservative Vector Fields}
\small

\begin{example}
If $\mathbf{F}(x,y,z)=y^2\mathbf{i}+(2xy+e^{3z})\mathbf{j}+3ye^{3z}\mathbf{k}$, find the potential function $f$ such that $\nabla f=\mathbf{F}$.

\lspace
Comparing this to (2) reveals that $g_y(y,z)=e^{3z}$, so $g(y,z)=ye^{3z}+h(z)$. \pause Now we can write

\[
f(x,y,z)=xy^2+g(y,z)=xy^2+ye^{3z}+h(z).
\]\pause 

Lastly, differentiating with respect to $z$ reveals $f_z(x,y,z)=3ye^{3z}+h'(z)$. \pause Comparing to (3) reveals that $h'(z)=0$ so $h(z)=K$, a constant. \pause Thus,

\[
f(x,y,z)=xy^2+ye^{3z}+K.
\]
\end{example}
\end{frame}









\begin{frame}[t]{Conservative Vector Fields}
\small

\begin{example}
If $\mathbf{F}(x,y,z)=y^2\mathbf{i}+(2xy+e^{3z})\mathbf{j}+3ye^{3z}\mathbf{k}$, find the potential function $f$ such that $\nabla f=\mathbf{F}$.

\[
f(x,y,z)=xy^2+ye^{3z}+K
\]

\vspace{3mm}
Let's verify that this is correct by computing the partial derivatives:

\lspace
\begin{align*}
f_x(x,y,z)&=y^2\pspace\checkmark\\[2mm]
f_y(x,y,z)&=2xy+e^{3z}\pspace\checkmark\\[2mm]
f_z(x,y,z)&=3ye^{3z}\pspace\checkmark
\end{align*}
\end{example}
\end{frame}















\begin{frame}[t]{Conservation of Energy}
\small
You can can probably guess from from the names floating around that all of this strongly related to conservation of energy. We now make that connection explicit.\pause 

\lspace
Suppose $\mathbf{F}$ is a continuous force field that moves an object along a path $C$ given by $\mathbf{r}(t)$ where $a\leq t\leq b$, $\mathbf{r}(a)=A$ is the initial point, and $\mathbf{r}(b)=B$ is the terminal point. \pause According to Newton's Second Law of Motion, the force $\mathbf{F}(\mathbf{r}(t))$ at a point on $C$ is related to the acceleration $\mathbf{a}(t)=\mathbf{r}''(t)$ by the equation

\[
\mathbf{F}(\mathbf{r}(t))=m\mathbf{r}''(t).
\]
\end{frame}








\begin{frame}[t]{Conservation of Energy}
\small
We can reexpress the work done by the force on the object as follows.

\begin{align*}
W&=\int_C\mathbf{F}\cdotr\,d\mathbf{r}=\int_a^b\mathbf{F}(\mathbf{r}(t))\cdotr\mathbf{r}'(t)\,dt\\[2mm]
&=\int_a^bm\mathbf{r}''(t)\cdotr\mathbf{r}'(t)\,dt=\frac{m}{2}\int_a^b\dd{}{t}[\mathbf{r}'(t)\cdotr\mathbf{r}'(t)]\,dt\\[2mm]
&=\frac{m}{2}\int_a^b\dd{}{t}|\mathbf{r}'(t)|^2\,dt=\frac{m}{2}\left[|\mathbf{r}'(t)|^2\right]_a^b\\[2mm]
&=\frac{m}{2}\left(|\mathbf{r}'(b)|^2-|\mathbf{r}'(a)|\right)=\frac{m}{2}\left(|\mathbf{v}(b)|^2-|\mathbf{v}(a)|\right)\\[2mm]
&=\frac{1}{2}m|\mathbf{v}(b)|^2-\frac{1}{2}m|\mathbf{v}(a)|^2.
\end{align*}
\end{frame}









\begin{frame}{Conservation of Energy}
\small
\[
W=\frac{1}{2}m|\mathbf{v}(b)|^2-\frac{1}{2}m|\mathbf{v}(a)|^2.
\]

The quantity $\displaystyle \frac{1}{2}m|\mathbf{v}(t)|^2$ (half the mass times the square of the speed) is called the \textbf{kinetic energy} of the object. \pause We can therefore rewrite the work equation as

\[
W=K(B)-K(A)
\]

which says that the work done by the force field along $C$ is the change in the kinetic energy at the endpoints of $C$.
\end{frame}







\begin{frame}{Conservation of Energy}
\small
Now further assume that $\mathbf{F}$ is a conservative force field, i.e. $\mathbf{F}=\nabla f$. \pause The \textbf{potential energy} of an object at the point $(x,y,z)$ is defined as $P(x,y,z)=-f(x,y,z)$, so we can write $\mathbf{F}=-\nabla P$. (The sign is chosen to agree with the intuition that work done against a force field increases potential energy.) \pause Then by FTL,

\[
W=\int_C\mathbf{F}\cdotr\,d\mathbf{r}=-\int_C\nabla P\,d\mathbf{r}=-[P(\mathbf{r}(b))-P(\mathbf{r}(a))]=P(A)-P(B).
\]
\end{frame}











\begin{frame}{Conservation of Energy}
\small

Putting this equation together with the one we derived before, we have

\begin{align*}
K(B)-K(A)&=P(A)-P(B)\\[3mm]
&\Updownarrow\\[3mm]
\underbrace{P(A)+K(A)}_\text{total energy at $A$}&=\underbrace{P(B)+K(B)}_\text{total energy at $B$}
\end{align*}\pause

In other words, if an object moves from point $A$ to point $B$ under the influence of a conservative force field, the total energy remains constant. This is called the \textbf{Law of Conservation of Energy}, and it is the reason for the terminology ``conservative'' and ``potential''.
\end{frame}










\end{document}