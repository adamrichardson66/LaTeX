\documentclass[11pt,oneside,english]{amsart}
\usepackage[T1]{fontenc}
\usepackage{geometry}
\usepackage{parskip}
\geometry{verbose,tmargin=0.65in,bmargin=0.65in,lmargin=0.75in,rmargin=0.75in,headheight=0.75cm,headsep=1cm,footskip=1cm}
\setlength{\parskip}{7mm}
\usepackage{setspace}
\onehalfspacing
\pagenumbering{gobble}


\usepackage{bbm}
\usepackage{multicol}
\usepackage{graphicx}
\usepackage{adjustbox}
\usepackage{tikz}
\usetikzlibrary{cd}
\usepackage{pgfplots}
\usepackage{ulem}
\usepackage{adjustbox}
\usepackage{bm}
\usepackage{stmaryrd}
\usepackage{cancel}
\usepackage{mathtools}
\DeclarePairedDelimiter{\ceil}{\lceil}{\rceil}
\DeclarePairedDelimiter\floor{\lfloor}{\rfloor}
\usepackage{enumitem}
\setlist[enumerate,1]{label=\textbf{\arabic*.}}
\usepackage{color, colortbl}
\definecolor{Gray}{gray}{0.9}
\usepackage{babel}
\usepackage{mdframed}
\usepackage{esint}

\theoremstyle{definition}
\newtheorem{theorem}{Theorem}
\newtheorem{corollary}{Corollary}
\newtheorem*{example}{Example}
\newtheorem*{examples}{Examples}
\newtheorem*{definition}{Definition}
\newtheorem*{note}{Nota Bene}

\newcommand{\aspace}{\hspace{7mm}\text{and}\hspace{7mm}}
\newcommand{\ospace}{\hspace{7mm}\text{or}\hspace{7mm}}
\newcommand{\pspace}{\hspace{10mm}}
\newcommand{\lhe}{\stackrel{\text{L'H}}{=}}
\newcommand{\lom}[2]{\lim_{{#1}\rightarrow{#2}}}
\newcommand{\R}{\mathbb{R}}
\newcommand{\dd}[2]{\frac{d{#1}}{d{#2}}}
\newcommand{\pp}[2]{\frac{\partial{#1}}{\partial{#2}}}
\newcommand{\DD}[2]{\frac{\Delta{#1}}{\Delta{#2}}}
\newcommand{\ovec}[1]{\overrightarrow{#1}}
\newcommand{\mbf}[1]{\mathbf{#1}}

\def\<#1>{\mathinner{\langle#1\rangle}}

\makeatletter
\g@addto@macro\normalsize{%
  \setlength\belowdisplayshortskip{5mm}
}
\makeatother



%Textbook: Essential Calculus - Early Transcendentals, 2nd edition - Stewart. ISBN: 978-1-133-11228-0


\begin{document}
\vspace*{-1cm}
\title{16.6 - Parametric Surfaces and Their Areas}
\maketitle


Now we're going to extend our ability to describe surfaces to any general surface. More specifically, we are going to re-express everything we learned about surfaces in terms of parameters and parametric equations.

Recall that a space curve can be described by a vector function $\mathbf{r}(t)$ of a single parameter $t$. Similarly, we can describe a surface by a vector function $\mathbf{r}(u,v)$ of two parameters. 

\begin{definition}
We write

\[
\mathbf{r}(u,v)=x(u,v)\mathbf{i}+y(u,v)\mathbf{j}+z(u,v)\mathbf{k}
\]

and suppose it is defined on a region $D$ in the $uv$-plane. 

The set of all points $(x,y,z)$ in $\R^3$ such that

\[
x=x(u,v)\pspace y=y(u,v)\pspace z=z(u,v)
\]

and $(u,v)$ varies throughout $D$ is called a \textbf{parametric surface} $S$ and the equations above are called the \textbf{parametric equations} of the surface.

\begin{center}
\includegraphics[scale=0.5]{para_surf.png}
\end{center}
\end{definition}

\begin{example}
Identify and sketch the surface with vector equation 

\[
\mathbf{r}(u,v)=2\cos u\mathbf{i}+v\mathbf{j}+2\sin u\mathbf{k}.
\]

The parametric equations are

\[
x=2\cos u\pspace y=v\pspace z=2\sin u.
\]

There is clearly some circular behavior going on. Notice that $x^2+z^2=4\cos^2 u+4\sin^2 u=4$, so this is a cylinder with axis the $y$-axis and radius 2.
\end{example}

\textbf{Note.} We can also describe parts of surfaces by restricting the domains of our parameters $u$ and $v$.


\begin{note}
Given a parametric surface $S$ described by we are often interested in two families of curves related to $S$ called the \textbf{grid curves}. The grid curves are the curves on the surface generated when one of the variables is held fixed. These curves correspond to vertical and horizontal lines in the $uv$-plane. E.g. fix $u=u_0$. Then $\mathbf{r}(u_0,v)=\mathbf{r}_{u_0}(v)$ which is a vector function of a single variable, i.e. a space curve.

\begin{center}
\includegraphics[scale=0.5]{grid_curves.png}
\end{center}
\end{note}

\begin{example}

Graph $\mathbf{r}(u,v)=\<(2+\sin v)\cos u,(2+\sin v)\sin u,u+\cos v>$ Which grid curves have $u$ constant? (the circles) Which have $v$ constant? (the long paths)
\begin{center}
\includegraphics[scale=0.5]{spiral_pasta.png}
\end{center}
\end{example}


Now we find an equation given a surface. This problem is one we will see more in the later sections.

\pagebreak

\begin{example}
Find a vector function that represents the plane that passes through the point $P_0$ with position vector $\mathbf{r}_0$ and that contains two nonparallel vectors $\mathbf{a}$ and $\mathbf{b}$.

\begin{center}
\includegraphics[scale=0.5]{point_plane.png}
\end{center}

Picture the figure above. Any vector in the direction of $\mathbf{a}$ must have the form $u\mathbf{a}$, and any vector in the direction of $\mathbf{b}$ must have the form $v\mathbf{b}$ where $u,v$ vary. Pick another point $P$ on our plane. Then there exist $u,v$ such that $\ovec{P_0P}=u\mathbf{a}+v\mathbf{b}$. If $\mathbf{r}$ is the position vector of $P$, then

\[
\mathbf{r}=\ovec{OP_0}+\ovec{P_0P}=\mathbf{r_0}+u\mathbf{a}+b\mathbf{b}.
\]

We can thus write the vector equation of this plane as

\[
\mathbf{r}(u,v)=\mathbf{r}_0+u\mathbf{a}+b\mathbf{b}
\]

where $u,v$ are real numbers. If we write

\[
\mathbf{r}=\<x,y,z>\pspace \mathbf{r}_0=\<x_0,y_0,z_0>\pspace \mathbf{a}=\<a_1,a_2,a_3>\pspace \mathbf{b}=\<b_1,b_2,b_3>
\]

then we can glean the parametric equations:

\[
x=x_0+ua_1+vb_1\pspace y=y_0+ua_2+vb_2\pspace z=z_0+ua_3+vb_3
\]
\end{example}

\pagebreak

\begin{example}
Find parametric equations for the sphere.

This one should come as no real surprise; we can use the formulas for spherical coordinates. Setting $\rho=a$ we get

\[
x=a\sin\phi\cos\theta\pspace y=a\sin\phi\sin\theta \pspace z=a\cos\phi
\]

for the parametric equations and the vector equation is

\[
\mathbf{r}(\phi,\theta)=a\sin\phi\cos\theta\mathbf{i}+a\sin\phi\sin\theta\mathbf{j}+a\cos\phi\mathbf{k}
\]

\begin{minipage}{0.5\textwidth}
\begin{center}
\includegraphics[scale=0.5]{sphere_domain.png}
\end{center}
\end{minipage}%
\begin{minipage}{0.5\textwidth}
\begin{center}
\includegraphics[scale=0.5]{sphere_grid.png}
\end{center}
\end{minipage}

\end{example}

\vspace{6mm}
\begin{example}$ $

\begin{center}
\includegraphics[scale=0.5]{sphere3.png}
\end{center}

The left is graphed by solving for $z$ in $x^2+y^2+z^2=1$ and graphing the top and bottom separately. The right is graphed using the parametric equation.
\end{example}

\begin{note}
For general parametric surfaces, the grid curves are similar to lines of longitude and latitude.
\end{note}


\pagebreak

\begin{example}
Find a parametric representation for the surface $z=2\sqrt{x^2+y^2}$, i.e. the top half of the cone $z^2=4x^2+4y^2$.

Here we have two choices, we can use a cartesian parameterization or a polar parameterization. 

First, choose $x$ and $y$ to be parameters. Then our parametric equations are

\[
x=x\pspace y=y \pspace z=2\sqrt{x^2+y^2},
\]

so the vector equation is

\[
\mathbf{r}(x,y)=x\mathbf{i}+y\mathbf{j}+2\sqrt{x^2+y^2}\mathbf{k}.
\]

Instead, we could have expressed the surface in polar form. We have $z=2\sqrt{x^2+y^2}=2\sqrt{r^2}=2r$ and $x=r\cos\theta$, $y=r\sin\theta$ so we can write

\[
\mathbf{r}(r,\theta)=r\cos\theta\mathbf{i}+r\sin\theta\mathbf{j}+2r\mathbf{k}.
\]
\end{example}


\section*{Surfaces of Revolution}


Surfaces of revolution can also be described parametrically, and it's not difficult to see how. Let $S$ be the surface generated by rotating the curve $y=f(x)$ about the $x$-axis. For ease we assume $f(x)\geq 0$. 

\begin{center}
\includegraphics[scale=0.5]{revolution.png}
\end{center}


Then we can describe the surface with the parametric equations

\[
x=x \pspace y=f(x)\cos\theta\pspace z=f(x)\sin\theta,
\]

or 

\[
\mathbf{r}(x,\theta)=x\mathbf{i}+f(x)\cos\theta\mathbf{j}+f(x)\sin\theta\mathbf{k}
\]

\section*{Tangent Planes}

Let $\mathbf{r}(u,v)=x(u,v)\mathbf{i}+y(u,v)\mathbf{j}+z(u,v)\mathbf{k}$. We want to find the tangent plane to a surface $S$ traced out by this vector function at a point $P_0$ with position vector $\mathbf{r}(u_0,v_0)$.

Set $u=u_0$. Then $\mathbf{r}(u_0,v)$ is a vector function of a single parameter, and it defines a grid curve $C_1$ on $S$. The tangent vector to $C_1$ at $P_0$ is found by taking the derivative of $\mathbf{r}$ with respect to $v$:

\[
\mathbf{r}_v=\pp{x}{v}(u_0,v_0)\mathbf{i}+\pp{y}{v}(u_0v_0)\mathbf{j}+\pp{z}{v}(u_0,v_0)\mathbf{k}
\]

We can take a similar approach by fixing $v=v_0$ and computing

\[
\mathbf{r}_u=\pp{x}{u}(u_0,v_0)\mathbf{i}+\pp{y}{u}(u_0v_0)\mathbf{j}+\pp{z}{u}(u_0,v_0)\mathbf{k}
\]

Both of these vectors lie in the tangent plane so their cross product $\mathbf{r}_u\times\mathbf{r}_v$ is normal to the tangent plane, and can be used to describe the plane.

\begin{definition}
A surface $S$ is called \textbf{smooth} if $\mathbf{r}_u\times\mathbf{r}_v$ is not $\mathbf{0}$. In other words is has no ``corners''.
\end{definition}

\pagebreak

\begin{example}
Find the tangent plane to the surface with parametric equations $x=u^2$, $y=v^2$, $z=u+2v$ at the point $(1,1,3)$.

First we need the tangent vectors:

\[
\mathbf{r}_u=\pp{x}{u}\mathbf{i}+\pp{y}{u}\mathbf{j}+\pp{z}{u}=2u\mathbf{i}+\mathbf{k}
\]

\[
\mathbf{r}_v=\pp{x}{v}\mathbf{i}+\pp{yy}{v}\mathbf{j}+\pp{z}{v}=2v\mathbf{j}+2\mathbf{k}
\]

Now we compute the cross product:

\[
\mathbf{r}_u\times\mathbf{r}_v=\begin{vmatrix}\mathbf{i}&\mathbf{j}&\mathbf{k}\\ 2u & 0 & 1 \\0 & 2v & 2\end{vmatrix}=-2u\mathbf{i}-4u\mathbf{j}+4uv\mathbf{k}
\]

Now, notice the point $(1,1,3)$ corresponds to the parameter values $u=1$, $v=1$, so the normal vector there is

\[
-2(1)\mathbf{i}-4(1)\mathbf{j}+4(1)(1)\mathbf{k}=-2\mathbf{i}-4\mathbf{j}+4\mathbf{k}.
\]

Finally, an equation of the tangent plane is

\[
-2(x-1)-4(y-1)+4(z-3)=0\ospace x+2y-2z+3=0.
\]
\end{example}

\pagebreak

\section*{Surface Area}

Now we finally get to surface area of a general parameterized surface. For ease, start by considering a surface whose parameter domain $D$ is a rectangle that we subdivide into smaller subrectangles $R_{ij}$. Let's choose $(u_i^*,v_i^*)$ to be the lower left corner of the subrectangle $R_{ij}$. The image $S_{ij}$ of $R_{ij}$ under $\mathbf{r}$ is called a \textbf{patch} and has the point $P_{ij}$ with position vector $\mathbf{r}(u_i^*,v_i^*)$ as one of its corners. Write

\[
\mathbf{r}_u^*=\mathbf{r}_u(u_i^*,v_i^*)\aspace \mathbf{r}_v^*=\mathbf{r}_v(u_i^*,v_i^*)
\]


\begin{center}
\includegraphics[scale=0.6]{patch1.png}
\end{center}

The figure shows how two edges of the patch can be approximated by vectors $\Delta u\mathbf{r}_u^*$ and $\Delta v\mathbf{r}_v^*$. The resulting parallelogram gives us a way to approximate the area of the patch. The area of the parallelogram is

\begin{center}
\includegraphics[scale=0.6]{patch2.png}
\end{center}

\[
|(\Delta u\mathbf{r}_u^*)\times(\Delta v\mathbf{r}_v^*)|=|\mathbf{r}_u^*\times\mathbf{r}_v^*|\Delta u\,\Delta v
\]

so an approximation of the area of $S$ is

\[
A(S)\approx\sum_{i=1}^m\sum_{j=1}^n|\mathbf{r}_u^*\times\mathbf{r}_v^*|\Delta u\,\Delta v
\]

By now you can probably guess where this is headed...

\begin{definition}
If a smooth parametric surface $S$ is given by the equation

\[
\mathbf{r}(u,v)=x(u,v)\mathbf{i}+y(u,v)\mathbf{j}+z(u,v)\mathbf{k}\pspace (u,v)\in D
\]

and $S$ is covered just once as $(u,v)$ ranges throughout the parameter domain $D$, then the \textbf{surface area} of $S$ is

\[
A(S)=\iint_D|\mathbf{r}_u\times\mathbf{r}_v|\,dA
\]

where $\displaystyle \mathbf{r}_u=\pp{x}{u}\mathbf{i}+\pp{y}{u}\mathbf{j}+\pp{z}{u}\mathbf{k}\aspace \mathbf{r}_v=\pp{x}{v}\mathbf{i}+\pp{y}{v}\mathbf{j}+\pp{z}{v}\mathbf{k}$.

\end{definition}


\begin{example}
Find the surface area of the cone $z=\sqrt{x^2+y^2}$ where $-1\leq x,y\leq 1$.

Let's use a polar representation. We have $z=\sqrt{x^2+y^2}=\sqrt{r^2}=r$ and $x=r\cos\theta$, $y=r\sin\theta$. Since $-1,\leq x,y\leq 1$, $0\leq r\leq 1$ and $0\leq \theta\leq 2\pi$. The vector equation of the cone is

\[
\mathbf{r}(r,\theta)=r\cos\theta\mathbf{i}+r\sin\theta\mathbf{j}+r\mathbf{k}
\]

Our partial derivative vectors are

\[
\mathbf{r}_r=\<\cos\theta,\sin\theta,1>\aspace\mathbf{r}_\theta=\<-r\sin\theta,r\cos\theta,0>
\]

\begin{align*}
\mathbf{r}_u\times \mathbf{r}_v&=\begin{vmatrix}\mathbf{i}&\mathbf{j}&\mathbf{k}\\ \cos\theta & \sin\theta & 1\\ -r\sin\theta & r\cos\theta & 0\end{vmatrix}\\[2mm]
&=(0-r\cos\theta)\mathbf{i}-(0+r\sin\theta)\mathbf{j}+(r\cos^2\theta+r\sin^2\theta)\mathbf{k}\\[2mm]
&=-r\cos\theta\mathbf{i}-r\sin\theta\mathbf{j}+r\mathbf{k}
\end{align*}

\[
|\mathbf{r}_u\times\mathbf{r}_v|=\sqrt{r^2\cos^2\theta+r^2\sin^2\theta+r^2}=\sqrt{2}r.
\]



\begin{align*}
A(S)&=\iint_D|\mathbf{r}_u\times\mathbf{r}_v|\,dA\\[2mm]
&=\iint_D\sqrt{2}r\,dA\\[2mm]
&=\int_0^{2\pi}\int_0^1\sqrt{2}r^2\,dr\,d\theta\\[2mm]
&=\int_0^{2\pi}\,d\theta\int_0^1\sqrt{2}r^2\,dr\\[2mm]
&=2\pi\cdot\left.\frac{\sqrt{2}}{3}r^3\right|_0^1\\[2mm]
&=\frac{2\sqrt{2}\pi}{3}.
\end{align*}
\end{example}

\section*{Surface Area of a Graph of a Function of Two Variables}

Now we can easily derive the surface area formula that we skipped earlier in the course. Suppose we have a surface $S$ described by a function of two variables $z=f(x,y)$ where $(x,y)$ lies in $D$, and that $f$ has continuous partial derivatives. The parametric equations of the surface are

\[
x=x\pspace y=y\pspace z=f(x,y),\text{ so}
\]

\[
\mathbf{r}_x=\mathbf{i}+f_x\mathbf{k}\aspace \mathbf{r}_y=\mathbf{j}+f_y\mathbf{k},\text{ and}
\]

\[
\mathbf{r}_x\times\mathbf{r}_y=\begin{vmatrix}\mathbf{i}&\mathbf{j}&\mathbf{k}\\ 1& 0 & \pp{f}{x}\\ 0 & 1 & \pp{f}{y}\end{vmatrix}=-\pp{f}{x}\mathbf{i}-\pp{f}{y}\mathbf{j}+\mathbf{k}.
\]

Thus,

\[
|\mathbf{r}_x\times \mathbf{r}_y|=\sqrt{\left(\pp{f}{x}\right)^2+\left(\pp{f}{y}\right)^2+1}=\sqrt{1+\left(\pp{z}{x}\right)^2+\left(\pp{z}{y}\right)^2}.
\]

Therefore, 

\[\boxed{
A(S)=\iint_D\sqrt{1+\left(\pp{z}{x}\right)^2+\left(\pp{z}{y}\right)^2}\,dA}
\]

\begin{note}
Notice the similarity to the arc length formula from Calc II: $\displaystyle L=\int_a^b\sqrt{1+\left(\dd{y}{x}\right)^2}\,dx$. This is not a coincidence, it has to do with measures and how they change when the number of dimensions are increased.
\end{note}

\pagebreak

\textbf{Claim.} The surface area formula derived earlier agrees with the formula $\displaystyle A(S)=\int_a^b2\pi f(x)\sqrt{1+f'(x)^2}\,dx$ derived in Calc II.

\begin{proof}
Let $S$ be generated by rotating the curve $y=f(x)$ about the $x$-axis with $f(x)\geq 0$ and $a\leq x\leq b$. We can choose are parametric equations as


\[
x=x\pspace y=f(x)\cos\theta\pspace z=f(x)\sin\theta\pspace a\leq x\leq b\pspace 0\leq \theta\leq 2\pi
\]

The necessary tangent vectors are 

\[
\mathbf{r}_x=\mathbf{i}+f'(x)\cos\theta\mathbf{j}+f'(x)\sin\theta\mathbf{k}
\]
\[
\mathbf{r}_\theta=-f(x)\sin\theta\mathbf{i}+f(x)\cos\theta\mathbf{k}.
\]



\[
\text{Then }\mathbf{r}_x\times\mathbf{r}_\theta=\begin{vmatrix}\mathbf{i}&\mathbf{j}&\mathbf{k}\\ 1 & f'(x)\cos\theta & f'(x)\sin\theta\\ 0 & -f(x)\sin\theta & f(x)\cos\theta\end{vmatrix}=f(x)f'(x)\mathbf{i}-f(x)\cos\theta\mathbf{j}-f(x)\sin\theta\mathbf{k},\text{ so}
\]

\begin{align*}
|\mathbf{r}_x\times\mathbf{r}_\theta|&=\sqrt{f(x)^2f'(x)^2+f(x)^2\cos^2\theta+f(x)^2\sin^2\theta}\\[2mm]
&=\sqrt{f(x)^2[1+f'(x)^2]}\\[2mm]
&=f(x)\sqrt{1+f'(x)^2}.
\end{align*}

Therefore, our formula gives the area of the surface as

\begin{align*}
A(S)&=\iint_D|\mathbf{r}_x\times\mathbf{r}_\theta|\,dA\\[2mm]
&=\int_0^{2\pi}\int_a^bf(x)\sqrt{1+f'(x)^2}\,dx\,d\theta\\[2mm]
&=\int_a^b2\pi f(x)\sqrt{1+f'(x)^2}\,dx.
\end{align*}
\end{proof}



\end{document}