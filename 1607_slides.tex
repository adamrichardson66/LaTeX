\documentclass[11pt,english,
handout
]{beamer}

%Preamble  
\input{/Users/Adam/Desktop/LBCC/MATH80/MATH80_Lesson_Plans/MATH80_Slides_Preamble.tex}

%Textbook: Essential Calculus - Early Transcendentals, 2nd edition - Stewart. ISBN: 978-1-133-11228-0



\begin{document}

%Slide titles are all contained in this file..
\ExecuteMetaData[/Users/Adam/Desktop/LBCC/MATH80/MATH80_Lesson_Plans/MATH80_Slide_Titles.tex]{1607}

%Global Title Slide Format is contained in the following file.
\input{/Users/Adam/Desktop/LBCC/MATH80/MATH80_Lesson_Plans/MATH80_Title_Slide_Format.tex}
\makebeamertitle













\begin{frame}{Surface Integrals}


We learned how to compute line integrals before, and surface integrals are similar except they are taken over a surface instead of a space curve. Suppose $f$ is a scalar function of three variables whose domain region includes a surface $S$ in space. We are going to adapt our canonical measuring tool the integral, to allow us to integrate $f$ over this surface. We investigate how to do this for scalar functions and then vector functions.

\end{frame}











\begin{frame}[t]{Surface Integrals of Scalar Functions of Three Variables}
\small
\begin{minipage}{0.4\textwidth}
\centering
\includegraphics[scale=0.5]{patches1.png}
\end{minipage}%
\begin{minipage}{0.2\textwidth}
\[
\xrightarrow{\quad\mbf{r}\quad}
\]
\end{minipage}%
\begin{minipage}{0.4\textwidth}
\includegraphics[scale=0.35]{patches2.png}
\end{minipage}%

{\footnotesize
Suppose $S$ is determined by the vector equation
\[
\mathbf{r}(u,v)=x(u,v)\mathbf{i}+y(u,v)\mathbf{j}+z(u,v)\mathbf{k},\pspace (u,v)\in D.
\]
Assume $D$ is a rectangle and divide it into subrectangles $R_{ij}$ with dimensions $\Delta u$ and $\Delta v$. Then the surface $S$ is divided into corresponding patches $S_{ij}$. We evaluate $f$ at a point $P_{ij}^*$ in each patch, multiply by the area $\Delta S_{ij}$ and form the Riemann sum
\[
\sum_{i=1}^m\sum_{j=1}^nf(P_{ij}^*)\Delta S_{ij}.
\]}
\end{frame}














\begin{frame}[t]{Surface Integrals of Scalar Functions of Three Variables}
\small
Taking the limit, refining the partition, produces the \textbf{surface integral of $f$ over the surface $S$}:

\[
\iint_Sf(x,y,z)\,dS=\lom{m,n}{\infty}\sum_{i=1}^m\sum_{j=1}^nf(P_{ij}^*)\Delta S_{ij}.
\]\pause 

Notice the similarity to the definition of a line integral, as well as the definition of a double integral. \pause In order to evaluate the integral, we use approximating parallelograms in the tangent plane for each patch $\Delta S_{ij}$. For each patch, we have the approximation
\[
\Delta S_{ij}\approx|\mathbf{r}_u\times\mathbf{r}_v|\Delta u\Delta v
\]

where $\mathbf{r}_u$ and $\mathbf{r}_v$ are tangent vectors at a corner of the parallelogram. 
\end{frame}






\begin{frame}[t]{Surface Integrals of Scalar Functions of Three Variables}
\small
\[
\Delta S_{ij}\approx|\mathbf{r}_u\times\mathbf{r}_v|\Delta u\Delta v
\]

If the components of $\mathbf{r}_u$ and $\mathbf{r}_v$ are continuous, and the vectors are nonzero and nonparallel in the interior of $D$, it follows that \textbf{surface integral of $f$ over the surface $S$} is

\[
\boxed{\iint_Sf(x,y,z)\,dS=\iint_Df(\mathbf{r}(u,v))|\mathbf{r}_u\times\mathbf{r}_v|\,dA.}
\]

\lspace
Look familiar? \pause It's the change of variables formula! \pause Also note that
\[
\iint_S1\,dS=\iint_D|\mathbf{r}_u\times\mathbf{r}_v|\,dA=A(S).
\]
\end{frame}












\begin{frame}[t]{Surface Integrals of Scalar Functions of Three Variables}
\small
\begin{example}
Compute the surface integral $\iint_Sx^2\,dS$ where $S$ is the sphere of radius $a$.\pause 

\lspace
We will use the spherical parameterizations:
\[
x=a\sin\phi\cos\theta\pspace y=a\sin\phi\sin\theta\pspace z=a\cos\phi\pspace0\leq \theta\leq 2\pi
\]\pause

Then

\begin{align*}
\mathbf{r}_\phi\times\mathbf{r}_\theta&=\begin{vmatrix}\mathbf{i}&\mathbf{j}&\mathbf{k}\\[2mm] a\cos\phi\cos\theta & a\cos\phi\sin\theta & -a\sin\phi\\[2mm] a\sin\phi\sin\theta & a\sin\phi\cos\theta & 0\end{vmatrix}\\[2mm]
&=a^2\sin^2\phi\cos\theta\mathbf{i}+a^2\sin^2\phi\sin\theta\mathbf{j} +a^2\sin\phi\cos\phi\mathbf{k}.
\end{align*}
\end{example}
\end{frame}











\begin{frame}[t]{Surface Integrals of Scalar Functions of Three Variables}
\small
\begin{example}
Compute the surface integral $\iint_Sx^2\,dS$ where $S$ is the sphere of radius $a$.

\lspace
So, not surprisingly,

\begin{align*}
|\mathbf{r}_\phi\times\mathbf{r}_\theta|&=\sqrt{a^4\sin^4\phi\cos^2\theta+a^4\sin^4\phi\sin^2\theta+a^4\sin^2\phi\cos^2\phi}\\[2mm]
&=\sqrt{a^4\sin^4\phi+a^4\sin^2\phi\cos^2\phi}\\[2mm]
&=a^2\sqrt{\sin^2\phi}\\[2mm]
&=a^2\sin\phi.
\end{align*}
\end{example}
\end{frame}











\begin{frame}[t]{Surface Integrals of Scalar Functions of Three Variables}
\small
\begin{example}
Compute the surface integral $\iint_Sx^2\,dS$ where $S$ is the sphere of radius $a$.

\lspace
Thus,
\begin{align*}
\iint_Sx^2\,dS&=\iint_D(a\sin\phi\cos\theta)^2|\mathbf{r}_u\times\mathbf{r}_v|\,dA=a^4\int_0^{2\pi}\int_0^\pi\sin^2\phi\cos^2\theta\sin\phi\,d\phi\,d\theta\\[2mm]
&=a^4\int_0^{2\pi}\cos^2\theta\,d\theta\int_0^\pi\sin^3\theta\,d\theta\\[2mm]
&=a^4\int_0^{2\pi}\frac{1}{2}(1+\cos2\theta)\,d\theta\int_0^\pi(\sin\phi-\sin\phi\cos^2\phi)\,d\phi\\[2mm]
&=\frac{a^4}{2}\left[\theta+\frac{1}{2}\sin2\theta\right]_0^{2\pi}\left[-\cos\phi+\frac{1}{3}\cos^3\theta\right]_0^\pi=\frac{4\pi}{3}a^4.
\end{align*}
\end{example}
\end{frame}












\begin{frame}[t]{Quick Physics Applications}
\small
As you may suspect, surface integrals are ubiquitous in physics, and we will explore a few amazing ideas in the last two sections. For now, we briefly point out the respective analogous formulas to those we saw earlier a couple times. \pause Imagine a thin sheet of, say, graphene has the shape of a surface $S$ and the density of the graphene at the point $(x,y,z)$ is $\rho(x,y,z)$. \pause Then the total \textbf{mass} of the sheet is
\[
m=\iint_S\rho(x,y,z)\,dS
\]

and the \textbf{center of mass} is $(\bar{x},\bar{y},\bar{z})$ where
{\footnotesize
\[
\bar{x}=\frac{1}{m}\iint_Sx\rho(x,y,z)\,dS\pspace\bar{y}=\frac{1}{m}\iint_Sy\rho(x,y,z)\,dS\pspace\bar{z}=\frac{1}{m}\iint_Sz\rho(x,y,z)\,dS
\]}

Moments of inertia can also be defined and are in the textbook.
\end{frame}
















\begin{frame}{Surface Integrals of Graphs Two-Variable Functions}
\small
Any surface $S$ given in the Cartesian form $z=g(x,y)$ can be regarded as a parametric surface with parametric equations
\[
x=x\pspace y=y\pspace z=g(x,y),
\]
and so
\[
\mathbf{r}_x=\mathbf{i}+\pp{g}{x}\mathbf{k}\pspace \mathbf{r}_y=\mathbf{j}+\pp{g}{y}\mathbf{k}.
\]
Thus,
\[
\mathbf{r}_x\times\mathbf{r}_y=-\pp{g}{x}\mathbf{i}-\pp{g}{y}\mathbf{j}+\mathbf{k}\aspace |\mathbf{r}_x\times\mathbf{r}_y|=\sqrt{\left(\pp{z}{x}\right)^2+\left(\pp{z}{y}\right)^2+1}.
\]
\end{frame}






\begin{frame}[t]{Surface Integrals of Graphs Two-Variable Functions}
\small
The surface integral of a surface $z=g(x,y)$ is

\[
\boxed{\iint_Sf(x,y,z)\,dS=\iint_Df(x,y,g(x,y))\sqrt{\left(\pp{z}{x}\right)^2+\left(\pp{z}{y}\right)^2+1}\,dA}
\]\pause 

\lspace
\begin{corollary}
If $S$ is a piecewise-smooth surface, i.e. it is a finite union of surfaces $S_1,S_2,\ldots,S_n$ that only intersect on their boundary, then the surface integral over $S$ is the sum of the surface integrals over all $S_i$.
\end{corollary}
\end{frame}












\begin{frame}[t]{Surface Integrals of Graphs Two-Variable Functions}
\small
\begin{example}
Evaluate $\iint_Sz\,dS$ where $S$ is the surface pictured below.

\begin{center}
\includegraphics[scale=0.4]{slant_cylinder.png}
\end{center}
\end{example}
\end{frame}










\begin{frame}[t]{Surface Integrals of Graphs Two-Variable Functions}
\small
\begin{example}

\vspace{3mm}
\begin{minipage}[c]{0.5\textwidth}
For $S_1$, we use $\theta$ and $z$ as parameters and write 
\[
x=\cos\theta\pspace y=\sin\theta\pspace z=z
\]
where $0\leq\theta \leq 2\pi$ and 
\[
\pspace 0\leq z\leq 1+x=1+\cos\theta.
\]
Thus
\begin{align*}
\mathbf{r}_\theta\times\mathbf{r}_z&=\begin{vmatrix}\mathbf{i}&\mathbf{j}&\mathbf{k}\\[2mm] -\sin\theta & \cos\theta & 0\\[2mm] 0 & 0 & 1\end{vmatrix}\\[2mm]
&=\cos\theta\mathbf{i}+\sin\theta\mathbf{j}.
\end{align*}
\end{minipage}%
\begin{minipage}[c]{0.5\textwidth}
\begin{center}
\includegraphics[scale=0.3]{slant_cylinder.png}
\end{center}
\end{minipage}
\end{example}
\end{frame}











\begin{frame}[t]{Surface Integrals of Graphs Two-Variable Functions}
\small
\begin{example}

\vspace{3mm}
\begin{minipage}{0.5\textwidth}
Lastly, 
\[
|\mathbf{r}_\theta\times\mathbf{r}_z|=\sqrt{\cos^2\theta+\sin^2\theta}=1,\text{ so}
\]
\end{minipage}%
\begin{minipage}{0.5\textwidth}
\begin{center}
\includegraphics[scale=0.3]{slant_cylinder.png}
\end{center}
\end{minipage}
\end{example}
\end{frame}












\begin{frame}[t]{Surface Integrals of Graphs Two-Variable Functions}
\small
\begin{example}

\vspace{3mm}
\begin{minipage}{0.5\textwidth}
\footnotesize
\begin{align*}
\iint_{S_1}z\,dS&=\iint_Dz|\mathbf{r}_\theta\times\mathbf{r}_z|\,dA\\[2mm]
&=\int_0^{2\pi}\int_0^{1+\cos\theta}z\,dz\,d\theta\\[2mm]
&=\int_0^{2\pi}\frac{1}{2}(1+\cos\theta)^2\,d\theta\\[2mm]
&=\frac{1}{2}\int_0^{2\pi}\left[1+2\cos\theta+\frac{1}{2}(1+\cos2\theta)\right]\,d\theta\\[2mm]
&=\frac{1}{2}\left[\frac{3}{2}\theta+2\sin\theta+\frac{1}{4}\sin2\theta\right]_0^{2\pi}\\[2mm]
&=\frac{3\pi}{2}.
\end{align*}
\end{minipage}%
\begin{minipage}{0.5\textwidth}
\begin{center}
\includegraphics[scale=0.29]{slant_cylinder.png}
\end{center}
\end{minipage}
\end{example}
\end{frame}
















\begin{frame}[t]{Surface Integrals of Graphs Two-Variable Functions}
\small
\begin{example}

Since $S_2$ lies in the $xy$-plane, $\displaystyle \iint_{S_2}z\,dS=\iint_{S_2}0\,dS=0$.\pause

\lspace
For $S_3$, $z=1+x$, and we can use the formula we derived above and convert to polar coordinates:

{\scriptsize
\begin{align*}
\iint_{S_3}z\,dS&=\iint_D(1+x)\sqrt{\left(\pp{z}{x}\right)^2+\left(\pp{z}{y}\right)^2+1}\,dA\\[2mm]
&=\int_0^{2\pi}\int_0^1(1+r\cos\theta)\sqrt{1+0+1}\,r\,dr\,d\theta\\[2mm]
&=\sqrt{2}\int_0^{2\pi}\int_0^1(r+r^2\cos\theta)\,dr\,d\theta\\[2mm]
&=\sqrt{2}\int_0^{2\pi}\left(\frac{1}{2}+\frac{1}{3}\cos\theta\right)\,d\theta=\sqrt{2}\left[\frac{\theta}{2}+\frac{\sin\theta}{3}\right]_0^{2\pi}=\sqrt{2}\pi.
\end{align*}}
\end{example}
\end{frame}












\begin{frame}[t]{Surface Integrals of Graphs Two-Variable Functions}
\small
\begin{example}

Putting all of these together, we get

\[
\iint_Sz\,dS=\frac{3\pi}{2}+0+\sqrt{2}\pi=\left(\frac{3}{2}+\sqrt{2}\right)\pi.
\]
\end{example}
\end{frame}











\begin{frame}[t]{Oriented Surfaces}
\small
In an earlier activity, we saw an example of a nonorientable surface, the M\"{o}bius strip. 
\vspace{-7mm}
\begin{center}
\includegraphics[scale=0.5]{mobius.png}
\end{center}

Our next goal is to define surface integrals of vector fields, but  to do so we need a rigorous definition of what ``direction'' on a surface really is. With the M\"{o}bius strip, because it has only one side, the concept of up and down cannot be well-defined.
\end{frame}











\begin{frame}[t]{Oriented Surfaces}
\small
In an earlier activity, we saw an example of a nonorientable surface, the M\"{o}bius strip. 
\vspace{-7mm}
\begin{center}
\includegraphics[scale=0.5]{mobius.png}
\end{center}

\textbf{Note:} From now on, we only consider \textbf{orientable} (two-sided) surfaces.
\end{frame}












\begin{frame}[t]{Oriented Surfaces}
\small
Begin with a surface $S$ that has a tangent plane at every point $(x,y,z)$ except maybe at the boundary points.\pause

\lspace
\begin{definition}
If it is possible to choose a normal vector $\mathbf{n}$ at every point $(x,y,z)$ so that $\mathbf{n}$ varies continuously over $S$, then the surface is said to be \textbf{orientable} and is called an \textbf{oriented surface}. The choice of $\mathbf{n}$ provides us with an \textbf{orientation}.
\end{definition}

\lspace
\textbf{Note:} There are two possible orientations for any orientable surface since, at each point, there are two possible normal vectors, $\mathbf{n}_1$ and $\mathbf{n}_2=-\mathbf{n}_1$.
\end{frame}











\begin{frame}[t]{Oriented Surfaces}
\small
If a surface is given by $z=g(x,y)$, then the surface has a natural orientation given by the unit normal vector

\[
\mathbf{n}=\frac{-\pp{g}{x}\mathbf{i}-\pp{g}{y}\mathbf{j}+\mathbf{k}}{\sqrt{\left(\pp{z}{x}\right)^2+\left(\pp{z}{y}\right)^2+1}}.
\]\pause 

Since the $\mathbf{k}$ component function is positive this gives the \textit{upward} orientation of the surface.\pause

\lspace
If $S$ is a smooth orientable surface given by the vector function $\mathbf{r}(u,v)$, then it is automatically supplied with the orientation of the unit normal vector

\[
\mathbf{n}=\frac{\mathbf{r}_u\times\mathbf{r}_v}{|\mathbf{r}_u\times\mathbf{r}_v|}.
\]
\end{frame}










\begin{frame}[t]{Oriented Surfaces}
\small
For example, we previously found that for a sphere of radius $a$,

\[
\mathbf{r}(\phi,\theta)=a\sin\phi\cos\theta\mathbf{i}+a\sin\phi\sin\theta\mathbf{j}+a\cos\phi\mathbf{k},\text{ and}
\]

\[
|\mathbf{r}_\phi\times\mathbf{r}_\theta|=a^2\sin\phi,
\]
so
\[
\mathbf{n}=\frac{\mathbf{r}_\phi\times\mathbf{r}_\theta}{|\mathbf{r}_\phi\times\mathbf{r}_\theta|}=\sin\phi\cos\theta\mathbf{i}+\sin\phi\sin\theta\mathbf{j}+\cos\phi\mathbf{k}=\frac{1}{a}\mathbf{r}(\phi,\theta).
\]\pause 

Notice that $\mathbf{n}$ points in the same direction as the position vector, i.e. outward from the sphere. The opposite orientation would have occurred if we had switched the order of the vectors in the cross product since $\mathbf{r}_\theta\times\mathbf{r}_\phi=-\mathbf{r}_\phi\times\mathbf{r}_\theta$.
\end{frame}








\begin{frame}[t]{Oriented Surfaces}
\small
For a \textbf{closed surface}, i.e. a surface that is the boundary of a solid region $E$, the positive orientation points outward and the negative orientation points inward.

\begin{center}
\includegraphics[scale=0.5]{orientation.png}
\end{center}
\end{frame}









\begin{frame}{Surface Integrals of Vector Fields}
\small

Suppose that $S$ is an orientable surface with unit normal vector $\mathbf{n}$ and imagine a fluid with density $\rho(x,y,z)$ and velocity field $\mathbf{v}(x,y,z)$ flowing through $S$. \pause You can imagine $S$ as a surface that doesn't impede the flow of the fluid, like a fishing net across a stream. The rate of flow (mass per unit time) per unit area is $\rho\mathbf{v}$. \pause If we divide $S$ into small patches $S_{ij}$ then $S_{ij}$ is nearly planar and we can approximate the mass of fluid per unit time crossing $S_{ij}$ in the direction of $\mathbf{n}$ by the quantity

\[
(\rho\mathbf{v}\cdotr\mathbf{n})\cdot A(S_{ij}).
\]\pause 

Summing all of these up and taking the limit, we get

\[
\iint_S\rho\mathbf{v}\cdotr\mathbf{n}\,dS=\iint_S\rho(x,y,z)\mathbf{v}(x,y,z)\cdotr\mathbf{n}(x,y,z)\,dS
\]
\end{frame}










\begin{frame}{Surface Integrals of Vector Fields}
\small
\[
\iint_S\rho\mathbf{v}\cdotr\mathbf{n}\,dS=\iint_S\rho(x,y,z)\mathbf{v}(x,y,z)\cdotr\mathbf{n}(x,y,z)\,dS
\]


Since $\rho\mathbf{v}$ is a vector field, we can write $\mathbf{F}=\rho\mathbf{v}$ and the integral on the left becomes

\[
\iint_S\mathbf{F\cdotr n}\,dS.
\]
\end{frame}










\begin{frame}{Surface Integrals of Vector Fields}
\small
\begin{definition}
If $\mathbf{F}$ is a continuous vector field defined on an oriented surface $S$ with unit normal vector $\mathbf{n}$, then the \textbf{surface integral of $\mathbf{F}$ over $S$}, i.e. the \textbf{flux},  is 

\[
\boxed{\text{flux}=\iint_S\mathbf{F}\cdotr\,d\mathbf{S}=\iint_S\mathbf{F}\cdotr\mathbf{n}\,dS}
\]\pause
\end{definition}

\textbf{Note:} Flux is the amount of stuff passing through or acting on a surface at any single moment in time. \pause The flux is greatest when our surface is perpendicular to the flow of the vector field, i.e. the normal vector to the surface is parallel to the flow of the vector field. \pause Notice that this behavior is captured by the dot product in the integral. The flux is the least when the surface is parallel to the flow of the vector field, i.e. the normal vector is orthogonal to flow.
\end{frame}











\begin{frame}{Surface Integrals of Vector Fields}
\small
If $S$ is described by the vector function $\mbf{r}(u,v)=x(u,v)\mbf{i}+y(u,v)\mbf{j}+z(u,v)\mbf{j}$,

\begin{align*}
\iint_S \mathbf{F}\cdotr\,d\mathbf{S}&=\iint_S\mathbf{F}\cdotr\mathbf{n}\,dS=\iint_S\mathbf{F}\cdotr\frac{\mathbf{r}_u\times\mathbf{r}_v}{|\mathbf{r}_u\times\mathbf{r}_v|}\,dS\\[2mm]
&=\iint_D\left[\mathbf{F}(\mathbf{r}(u,v))\cdotr\frac{\mathbf{r}_u\times\mathbf{r}_v}{|\mathbf{r}_u\times\mathbf{r}_v|}\right]|\mathbf{r}_u\times\mathbf{r}_v|\,dA\\[2mm]
&=\int_D\mathbf{F}(\mathbf{r}(u,v))\cdotr(\mathbf{r}_u\times\mathbf{r}_v)\,dA.
\end{align*}\pause
Thus,

\[
\boxed{\text{flux}=\iint_S\mathbf{F}\cdotr\,d\mathbf{S}=\iint_S\mathbf{F}\cdotr\mathbf{n}\,dS=\int_D\mathbf{F}(\mathbf{r}(u,v))\cdotr(\mathbf{r}_u\times\mathbf{r}_v)\,dA}
\]

where $D$ is the parameter domain.
\end{frame}








\begin{frame}[t]{Surface Integrals of Vector Fields}
\small
\begin{example}
Find the flux of the vector field $\mathbf{F}(x,y,z)=z\mathbf{i}+y\mathbf{j}+x\mathbf{k}$ across the unit sphere.

\begin{center}
\includegraphics[scale=0.3]{sphere_flux.png}
\end{center}
\end{example}
\end{frame}













\begin{frame}[t]{Surface Integrals of Vector Fields}
\small
\begin{example}
Find the flux of the vector field $\mathbf{F}(x,y,z)=z\mathbf{i}+y\mathbf{j}+x\mathbf{k}$ across the unit sphere.

\lspace
Naturally, we use spherical coordinate parameters again and get
\[
\mathbf{r}(\phi,\theta)=\sin\phi\cos\theta\mathbf{i}+\sin\phi\sin\theta\mathbf{j}+\cos\phi\mathbf{k} \pspace 0\leq\phi\leq\pi \pspace 0\leq\theta\leq2\pi.
\]\pause
Thus
\[
\mathbf{F}(\mathbf{r}(\phi,\theta))=\cos\phi\mathbf{i}+\sin\phi\sin\theta\mathbf{j}+\sin\phi\cos\theta\mathbf{k},\text{ so}
\]

\[
\mathbf{F}(\mathbf{r}(\phi,\theta))\cdotr(\mathbf{r}_\phi\times\mathbf{r}_\theta)=\cos\phi\sin^2\phi\cos\theta+\sin^3\phi\sin^2\theta+\sin^2\phi\cos\phi\cos\theta.
\]
\end{example}
\end{frame}














\begin{frame}[t]{Surface Integrals of Vector Fields}
\small
\begin{example}
Find the flux of the vector field $\mathbf{F}(x,y,z)=z\mathbf{i}+y\mathbf{j}+x\mathbf{k}$ across the unit sphere.

\lspace
Then
\begin{align*}
\int_S\mathbf{F}\cdotr\,d\mathbf{S}&=\iint_D\mathbf{F}\cdotr(\mathbf{r}_\phi\times\mathbf{r}_\theta)\,dA\\[2mm]
&=\int_0^{2\pi}\int_0^\pi(2\sin^2\phi\cos\phi\cos\theta+\sin^3\phi\sin^2\theta)\,d\phi\,d\theta\\[2mm]
&=2\int_0^\pi\sin^2\phi\cos\phi\,d\phi\int_0^{2\pi}\cos\theta\,d\theta+\int_0^\pi\sin^3\phi\,d\phi\int_0^{2\pi}\sin^2\theta\,d\theta\\[2mm]
&=0+\int_0^\pi\sin^3\phi\,d\phi\int_0^{2\pi}\sin^2\theta\,d\theta=\cdots=\frac{4\pi}{3}
\end{align*}

by the calculations from a previous example.
\end{example}
\end{frame}











\begin{frame}[t]{Surface Integrals of Vector Fields}
\small
\begin{example}
Find the flux of the vector field $\mathbf{F}(x,y,z)=z\mathbf{i}+y\mathbf{j}+x\mathbf{k}$ across the unit sphere.

\lspace
So, for example, if the vector field $\mathbf{F}$ was a velocity field describing the flow of a fluid with a uniform density 1, then $\frac{4\pi}{3}$ is how much fluid passes through the unit sphere at any moment in time.
\end{example}
\end{frame}










\begin{frame}{Surface Integrals of Vector Fields}
\small

\textbf{Note:} If the surface $S$ is given by the graph $z=g(x,y)$, then we can treat $x$ and $y$ as parameters, yielding 
\begin{align*}
\mathbf{F}\cdotr(\mathbf{r}_x\times\mathbf{r}_y)&=(P\mathbf{i}+Q\mathbf{j}+R\mathbf{k})\cdotr\left(-\pp{g}{x}\mathbf{i}-\pp{g}{y}\mathbf{j}+\mathbf{k}\right)\\[2mm]
&=-P\pp{g}{x}-Q\pp{g}{y}+R,
\end{align*}
so

\[
\boxed{\text{flux}=\iint_S\mathbf{F}\cdotr\,d\mathbf{S}=\iint_S\mathbf{F}\cdotr\mathbf{n}\,dS=\int_D-P\pp{g}{x}-Q\pp{g}{y}+R\,dA\hspace{5mm}\text{($*$ 16.9)}}
\]

($*$ 16.9) We will use this convenient formula in the proof of the Divergence Theorem in Section 16.9.
\end{frame}










\begin{frame}[t]{Surface Integrals of Vector Fields}
\small
\begin{example}
Evaluate $\iint_S\mathbf{F}\cdotr\,d\mathbf{S}$ where $\mathbf{F}(x,y,z)=y\mathbf{i}+x\mathbf{j}+z\mathbf{k}$ and $S$ is the boundary of the solid region $E$ enclosed by the paraboloid $z=1-x^2-y^2$ and the plane $z=0$.\pause 


\begin{center}
\includegraphics[scale=0.5]{paraboloid.png}
\end{center}
\vspace{-3mm}
$S$ consists of the parabolic cap $S_1$ and its circular base $S_2$. We will orient it in the outward direction. 
\end{example}
\end{frame}










\begin{frame}[t]{Surface Integrals of Vector Fields}
\small
\begin{example}
Evaluate $\iint_S\mathbf{F}\cdotr\,d\mathbf{S}$ where $\mathbf{F}(x,y,z)=y\mathbf{i}+x\mathbf{j}+z\mathbf{k}$ and $S$ is the boundary of the solid region $E$ enclosed by the paraboloid $z=1-x^2-y^2$ and the plane $z=0$.

\lspace
We can use the formula on the previous slides where $D$ is the unit disk. \pause We have

\[
P(x,y,z)=y\pspace Q(x,y,z)=x\pspace R(x,y,z)=z=1-x^2-y^2,\text{ and}
\]



\[
\pp{z}{x}=-2x \pspace \pp{z}{y}=-2y.
\]
\end{example}
\end{frame}













\begin{frame}[t]{Surface Integrals of Vector Fields}
\small
\begin{example}
Evaluate $\iint_S\mathbf{F}\cdotr\,d\mathbf{S}$ where $\mathbf{F}(x,y,z)=y\mathbf{i}+x\mathbf{j}+z\mathbf{k}$ and $S$ is the boundary of the solid region $E$ enclosed by the paraboloid $z=1-x^2-y^2$ and the plane $z=0$.


{\scriptsize
\begin{align*}
\iint_{S_1}\mathbf{F}\,d\mathbf{S}&=\iint_D\left(-P\pp{z}{x}-Q\pp{z}{y}+R\right)\,dA\\[2mm]
&=\iint_D-y(-2x)-x(-2y)+1-x^2-y^2\,dA\\[2mm]
&=\iint_D1+4xy-x^2-y^2\,dA=\int_0^{2\pi}\int_0^1(1+4r^2\cos\theta\sin\theta-r^2)\,r\,dr\,d\theta\\[2mm]
&=\int_0^{2\pi}\int_0^1(r-r^3+4r^3\cos\theta\sin\theta)\,dr\,d\theta=\int_0^{2\pi}\left[\frac{1}{2}r^2-\frac{1}{4}r^4+r^4\cos\theta\sin\theta\right]_{r=0}^{r=1}\,d\theta\\[2mm]
&=\int_0^{2\pi}\left(\frac{1}{4}+\cos\theta\sin\theta\right)\,d\theta=\left[\frac{1}{4}\theta-\frac{1}{4}\cos2\theta\right]_0^{2\pi}=\frac{1}{4}(2\pi)=\frac{\pi}{2}.
\end{align*}}
\end{example}
\end{frame}












\begin{frame}[t]{Surface Integrals of Vector Fields}
\small
\begin{example}
Evaluate $\iint_S\mathbf{F}\cdotr\,d\mathbf{S}$ where $\mathbf{F}(x,y,z)=y\mathbf{i}+x\mathbf{j}+z\mathbf{k}$ and $S$ is the boundary of the solid region $E$ enclosed by the paraboloid $z=1-x^2-y^2$ and the plane $z=0$.

\lspace
The disk $S_2$ is oriented downward, so its unit vector is $\mathbf{n}=-\mathbf{k}$:

\begin{align*}
\iint_{S_2}\mathbf{F}\,d\mathbf{S}&=\iint_{S_2}\mathbf{F}\cdotr(-\mathbf{k})\,dS=\iint_D\<y,x,z>\cdotr\<0,0,-1>\,dA\\[2mm]
&=\iint_D(-z)\,dA=\iint_D0\,dA=0
\end{align*}\pause

since $z=0$ on $S_2$.

\lspace
Thus the total flux is $\frac{\pi}{2}$.
\end{example}
\end{frame}















\begin{frame}{Electrostatics and Heat Flow}
\small
It is important to keep in mind that this theory has been developed for general vector fields in $\MB{R}^3$, so it can be applied in many ways that don't involve fluids or gases. \pause For example, if $\mathbf{E}$ is an electric field, the surface integral

\[
\iint_S\mathbf{E}\cdotr\,d\mathbf{S}
\]

is called the \textbf{electric flux} of $\mathbf{E}$ through the surface $S$. \pause \textbf{Gauss's Law} says that the total net charge enclosed by a closed surface $S$ is 

\[
Q=\varepsilon_0\iint_S\mathbf{E}\cdotr\,d\mathbf{S}
\]

where $\varepsilon_0$ is a constant called the \textit{permittivity of free space} that depends on the units used. (In the SI system, $\varepsilon_0\approx8.8542\times10^{-12}\,\text{C}^2/\text{N$\cdot$m$^2$}$.)
\end{frame}









\begin{frame}{Electrostatics and Heat Flow}
\small
Suppose the temperature at a point $(x,y,z)$ in a body is $u(x,y,z)$. Then the \textbf{heat flow} is defined as the vector field

\[
\mathbf{F}=-K\nabla u
\]

where $K$ is an experimentally determined constant called the \textbf{conductivity} of the substance. \pause The rate of heat flow across the surface $S$ in the body is

\[
\iint_S\mathbf{F}\cdotr\,d\mathbf{S}=-K\iint_S\nabla u\cdotr\,d\mathbf{S}.
\]
\end{frame}












\begin{frame}[t]{Electrostatics and Heat Flow}
\footnotesize
\begin{example}
The temperature $u$ in a metal ball is proportional to the square of the distance from the center of the ball. Find the rate of heat flow across a sphere $S$ of radius $a$ with center at the center of the ball.\pause 

\lspace
Take the center of the ball to be the origin. We have 
\[
u(x,y,z)=Cd^2=C(x^2+y^2+z^2).
\] \pause
Then the heat flow is
\[
\mathbf{F}(x,y,z)=-K\nabla u=-KC(2x\mathbf{i}+2y\mathbf{j}+2z\mathbf{k})=-2KC(x\mathbf{i}+y\mathbf{j}+z\mathbf{k})
\]

where $K$ is the conductivity of the metal. \pause At this point we could use the standard spherical parameterization and find the normal vector by computing a cross product as before, but we can already determine the outward unit normal to the sphere of radius $a$ because it is parallel to the position vector at any point:
\[
\mathbf{n}=\frac{1}{a}(x\mathbf{i}+y\mathbf{j}+z\mathbf{k}).
\] 
\end{example}
\end{frame}








\begin{frame}[t]{Electrostatics and Heat Flow}
\footnotesize
\begin{example}
The temperature $u$ in a metal ball is proportional to the square of the distance from the center of the ball. Find the rate of heat flow across a sphere $S$ of radius $a$ with center at the center of the ball.

\lspace
So 
\[
\mathbf{F}\cdotr\mathbf{n}=-\frac{2KC}{a}(x^2+y^2+z^2)=-\frac{2KC}{a}a^2=-2aKC
\]

on $S$. \pause Therefore, the rate of heat flow across $S$ is

\begin{align*}
\iint_S\mathbf{F}\cdotr\,d\mathbf{S}&=\iint_S\mathbf{F}\cdotr\mathbf{n}\,dS=-2aKC\iint_S\,dS\\[2mm]
&=-2aKC\cdot A(S)=-2aKC(4\pi a^2)\\[2mm]
&=-8KC\pi a^3.
\end{align*}
\end{example}
\end{frame}




\end{document}