\documentclass[11pt,oneside,english]{amsart}
\usepackage[T1]{fontenc}
\usepackage{geometry}
\usepackage{parskip}
\geometry{verbose,tmargin=0.65in,bmargin=0.65in,lmargin=0.75in,rmargin=0.75in,headheight=0.75cm,headsep=1cm,footskip=1cm}
\setlength{\parskip}{7mm}
\usepackage{setspace}
\onehalfspacing
\pagenumbering{gobble}


\usepackage{bbm}
\usepackage{multicol}
\usepackage{graphicx}
\usepackage{adjustbox}
\usepackage{tikz}
\usetikzlibrary{cd}
\usepackage{pgfplots}
\usepackage{ulem}
\usepackage{adjustbox}
\usepackage{bm}
\usepackage{stmaryrd}
\usepackage{cancel}
\usepackage{mathtools}
\DeclarePairedDelimiter{\ceil}{\lceil}{\rceil}
\DeclarePairedDelimiter\floor{\lfloor}{\rfloor}
\usepackage{enumitem}
\setlist[enumerate,1]{label=\textbf{\arabic*.}}
\usepackage{color, colortbl}
\definecolor{Gray}{gray}{0.9}
\usepackage{babel}
\usepackage{mdframed}
\usepackage{esint}

\theoremstyle{definition}
\newtheorem{theorem}{Theorem}
\newtheorem{corollary}{Corollary}
\newtheorem*{example}{Example}
\newtheorem*{examples}{Examples}
\newtheorem*{definition}{Definition}
\newtheorem*{note}{Nota Bene}

\newcommand{\aspace}{\hspace{7mm}\text{and}\hspace{7mm}}
\newcommand{\ospace}{\hspace{7mm}\text{or}\hspace{7mm}}
\newcommand{\pspace}{\hspace{10mm}}
\newcommand{\lhe}{\stackrel{\text{L'H}}{=}}
\newcommand{\lom}[2]{\lim_{{#1}\rightarrow{#2}}}
\newcommand{\R}{\mathbb{R}}
\newcommand{\dd}[2]{\frac{d{#1}}{d{#2}}}
\newcommand{\pp}[2]{\frac{\partial{#1}}{\partial{#2}}}
\newcommand{\DD}[2]{\frac{\Delta{#1}}{\Delta{#2}}}
\newcommand{\ovec}[1]{\overrightarrow{#1}}
\newcommand{\mbf}[1]{\mathbf{#1}}

\def\<#1>{\mathinner{\langle#1\rangle}}

\makeatletter
\g@addto@macro\normalsize{%
  \setlength\belowdisplayshortskip{5mm}
}
\makeatother



%Textbook: Essential Calculus - Early Transcendentals, 2nd edition - Stewart. ISBN: 978-1-133-11228-0


\begin{document}
\vspace*{-1cm}
\title{16.8 - Stokes' Theorem}
\maketitle



Stokes' Theorem is a higher dimensional version of Green's Theorem. Recall that Green's Theorem describes a profound relationship between a double integral over a planar region and the line integral of the boundary of that planar region. Stokes theorem describes the profound connection between a surface integral over a surface $S$ and its boundary curve, which is a space curve.

\begin{center}
\includegraphics[scale=0.5]{man_orient.png}
\end{center}

Let $S$ be an oriented surface with unit normal vector $\mathbf{n}$ as pictured. The orientation induces a \textbf{positive orientation of the boundary curve $C$} shown in the figure. If you walk in the positive direction around $C$ with your head pointing in the direction of $\mathbf{n}$, then the region will always be on your left.



\begin{theorem}[Stokes' Theorem]
Let $S$ be an oriented piecewise-smooth surface that is bounded by a simple, closed, piecewise-smooth boundary curve $C$ with positive orientation. Let $\mathbf{F}$ be a vector field whose components have continuous partial derivatives on an open region in $\R^3$ that contains $S$. Then

\[
\int_C\mathbf{F}\cdot\,d\mathbf{r}=\iint_S\text{curl }\mathbf{F}\cdot\,d\mathbf{S}.
\]
\end{theorem}

\begin{note}
Since

\[
\int_C\mathbf{F}\cdot\,d\mathbf{r}=\int_C\mathbf{F}\cdot\mathbf{T}\,ds\aspace \iint_S\text{curl }\mathbf{F}\cdot\,d\mathbf{S}=\iint_S\text{curl }\mathbf{F}\cdot\mathbf{n}\,dS,
\]

Stokes' Theorem says that the line integral around the boundary curve of $S$ of the tangential component of $\mathbf{F}$ is equal to the surface integral over $S$ of the normal component of the curl of $\mathbf{F}$.

In analogy with Green's theorem, Stokes' theorem says the circulation of $\mathbf{F}$ along the space curve $C$ is the same as the sum of the curl over the surface $S$ bounded by $C$.

Notice also the similarity to the FTC as with Green's Theorem: Stokes' Theorem relates an integral involving derivatives (the curl) to an integral involving only the boundary of $S$.

\textbf{Observe.} In the case where the surface $S$ is flat and lies in the $xy$-plane with upward orientation, the unit normal is $\mathbf{k}$, the surface integral becomes a double integral, and Stokes' Theorem becomes

\[
\int_C\mathbf{F}\cdot\,d\mathbf{r}=\iint_S\text{curl }\mathbf{F}\cdot\,d\mathbf{S}=\iint_S(\text{curl }\mathbf{F})\cdot\mathbf{k}\,dA=\iint_S\left(\pp{q}{x}-\pp{P}{y}\right)\,dA
\]

which is Green's Theorem.
\end{note}

\textbf{Notation.} It is common to use the symbol $\partial S$ to indicate the boundary of $S$, so Stokes' Theorem says 

\[
\int_{\partial S}\mathbf{F}\cdot\,d\mathbf{r}=\iint_S\text{curl }\mathbf{F}\cdot\,d\mathbf{S}.
\]




The proof of Stokes' Theorem is too advanced for this class, but you will see it in an undergraduate math analysis course. We can however prove it when $S$ us a graph and $\mathbf{F}$, $S$, and $C$ are well-behaved.

\begin{proof}

Suppose $S$ is a surface given by $z=g(x,y)$ where $(x,y)\in D$, $g$ has continuous second-order partial derivatives and $D$ is a simple plane region whose boundary curve $C_1$ corresponds to $C$. If the orientation of $S$ is upward, then the positive orientation of $C$ corresponds to the positive orientation of $C_1$. 

\begin{center}
\includegraphics[scale=0.5]{stokes1.png}
\end{center}

Let $\mathbf{F}=P\mathbf{i}+Q\mathbf{j}+R\mathbf{k}$ where the partial derivatives are continuous. Since $S$ is a graph, we can apply a formula for the surface integral from the last section with $\mathbf{F}$ replaced by curl $\mathbf{F}$:

\[
\iint_S\text{curl }\mathbf{F}\cdot\,d\mathbf{S}=\iint_D\left[-\left(\pp{R}{y}-\pp{Q}{z}\right)\pp{z}{x}-\left(\pp{P}{z}-\pp{R}{x}\right)\pp{z}{y}+\left(\pp{Q}{x}-\pp{P}{y}\right)\right]\,dA
\]
 
where the partial derivatives of $P,Q,$ and $R$ are evaluated at $(x,y,g(x,y))$. Let $x=x(t)$, $y=y(t)$, $a\leq t\leq b$ be a parameterization of $C_1$. Then the corresponding parameterization of $C$ is

\[
x=x(t)\pspace y=y(t)\pspace z=g(x(t),y(t))\pspace a\leq t\leq b.
\]

Now, by the Chain Rule, we have 

\begin{align*}
\int_C\mathbf{F}\cdot\,d\mathbf{R}&=\int_a^b\mathbf{F}\cdot\mathbf{r}'(t)\,dt\\[2mm]
&=\int_a^b\left(P\dd{x}{t}+Q\dd{y}{t}+R\dd{z}{t}\right)\,dt\\[2mm]
&=\int_a^b\left[P\dd{x}{t}+Q\dd{y}{t}+R\left(\pp{z}{x}\dd{x}{t}+\pp{z}{y}\dd{y}{t}\right)\right]\,dt\\[2mm]
&=\int_a^b\left[\left(P+R\pp{z}{x}\right)\dd{x}{t}+\left(Q+R\pp{z}{t}\right)\dd{y}{t}\right]\,dt\\[2mm]
&=\int_{C_1}\left(P+R\pp{z}{x}\right)\,dx+\left(Q+R\pp{z}{y}\right)\,dy\\[2mm]
&=\iint_D\left[\pp{}{x}\left(Q+R\pp{z}{y}\right)-\pp{}{y}\left(P+R\pp{z}{x}\right)\right]\,dA.
\end{align*}

The last step was done using Green's Theorem. Now, using the Chain Rule again, remembering that $P$, $Q$, and $R$ are functions of $x,y,z$ and that $z$ is a function of $x,y$, we get

\begin{align*}
\int_C\mathbf{F}\cdot\,d\mathbf{r}&=\iint_D\left[\left(\pp{Q}{x}+\pp{Q}{z}\pp{z}{x}+\pp{R}{x}\pp{z}{y}+\cancel{\pp{R}{z}\pp{z}{x}\pp{z}{y}}+\cancel{R\pp{^2z}{x\,\partial y}}\right)\right.\\[2mm]
&-\left.\left(\pp{P}{y}+\pp{P}{z}\pp{z}{y}+\pp{R}{y}\pp{z}{x}+\cancel{\pp{R}{z}\pp{z}{y}\pp{z}{x}}+\cancel{R\pp{^2z}{y\,\partial x}}\right)\right]\,dA\\[2mm]
&=\iint_D\left[\pp{Q}{x}+\pp{Q}{z}\pp{z}{x}+\pp{R}{x}\pp{z}{y}-\pp{P}{y}-\pp{P}{z}\pp{z}{y}-\pp{R}{y}\pp{z}{x}\right]\,dA\\[2mm]
&=\iint_D\left[-\pp{R}{y}\pp{z}{x}+\pp{Q}{z}\pp{z}{x}-\pp{P}{z}\pp{z}{y}+\pp{R}{x}\pp{z}{y}+\pp{Q}{x}-\pp{P}{y}\right]\,dA\\[2mm]
&=\iint_D\left[-\left(\pp{R}{y}-\pp{Q}{z}\right)\pp{z}{x}-\left(\pp{P}{z}-\pp{R}{x}\right)\pp{z}{y}+\left(\pp{Q}{x}-\pp{P}{y}\right)\right]\,dA\\[2mm]
&=\iint_S\text{curl }\mathbf{F}\cdot\,d\mathbf{S}.
\end{align*}
\end{proof}

\pagebreak


\begin{example}
In this example we will use a line integral to evaluate an equivalent surface integral.

Use Stokes' Theorem to compute the integral $\iint_S\text{curl }\mathbf{F}\cdot\,d\mathbf{S}$ where $\mathbf{F}(x,y,z)=xz\mathbf{i}+yz\mathbf{j}+xy\mathbf{k}$ and $S$ is the part of the sphere $x^2+y^2+z^2=4$ that lies inside the cylinder $x^2+y^2=1$ and above the $xy$-plane.

\begin{center}
\includegraphics[scale=0.4]{hemisphere.png}
\end{center}

First we need to find the boundary curve. To find it, we need to know when the two surfaces intersect. The equations of our two surfaces yield $4-z^2=1$, so $z=\sqrt{3}$ since $z\geq 0$. Then the boundary $C$ of our region is the circle given by $x^2+y^2=1$, $z=\sqrt{3}$. A vector equation of this circle is

\[
\mathbf{r}(t)=\cos t\mathbf{i}+\sin t\mathbf{j}+\sqrt{3}\mathbf{k}\pspace 0\leq t\leq 2\pi,
\]
\[
\text{so }\mathbf{r}'(t)=-\sin t\mathbf{i}+\cos t\mathbf{j}.
\]

Thus, $\mathbf{F}(\mathbf{r}(t))=\sqrt{3}\cos t\mathbf{i}+\sqrt{3}\sin t\mathbf{j}+\cos t\sin t\mathbf{k}$. By Stokes' theorem, we have

\begin{align*}
\iint_S\text{curl }\mathbf{F}\cdot\,d\mathbf{S}&=\int_C\mathbf{F}\cdot\,d\mathbf{r}\\[2mm]
&=\int_0^{2\pi}\mathbf{F}(\mathbf{r}(t))\cdot\mathbf{r}'(t)\,dt\\[2mm]
&=\int_0^{2\pi}(-\sqrt{3}\cos t\sin t+\sqrt{3}\sin t\cos t)\,dt\\[2mm]
&=\sqrt{3}\int_0^{2\pi}0\,dt\\[2mm]
&=0.
\end{align*}
\end{example}


\begin{note}
Notice that in this example we were able to compute the surface integral of that surface by using \textit{only} values on the boundary curve. This means that for any other oriented surface with the same boundary curve, the value of the surface integral is the same! In other words...
\end{note}

\begin{corollary}
The conclusion of Stokes' Theorem is independent of (appropriate) surface. More specifically, if $S_1$ and $S_2$ are oriented surfaces with the same oriented boundary curve $C$ and both satisfy the hypotheses of Stokes' Theorem, then

\[
\iint_{S_1}\text{curl }\mathbf{F}\cdot\,d\mathbf{S}=\int_C\mathbf{F}\cdot\,d\mathbf{r}=\iint_{S_2}\text{curl }\mathbf{F}\cdot\,d\mathbf{S}.
\]
\end{corollary}








\begin{example}
In this example, we'll evaluate a line integral by evaluating an equivalent surface integral, which we'll evaluate by evaluating a double integral.

Evaluate $\int_C\mathbf{F}\cdot\,d\mathbf{r}$ where $\mathbf{F}(x,y,z)=-y^2\mathbf{i}+x\mathbf{j}+z^2\mathbf{k}$ and $C$ is the curve of intersection of the plane $y+z=2$ and the cylinder $x^2+y^2=1$. (Orient $C$ to be counterclockwise from above.)

The curve $C$ is an ellipse.

\begin{center}
\includegraphics[scale=0.5]{ellipse.png}
\end{center}

We could compute this particular integral directly, but it is easier to use Stokes' theorem. First we need to compute the curl:

\[
\text{curl }\mathbf{F}=\begin{vmatrix}\mathbf{i}&\mathbf{j}&\mathbf{k}\\ \pp{}{x} &\pp{}{y}&\pp{}{z} \\ -y^2 & x & z^2\end{vmatrix}=(1+2y)\mathbf{k}.
\]

Now, we have our choice of whatever surface we would like to use to integrate, but the natural and easiest choice is the ellipse embedded in the plane $y+z=2$ that is bounded by $C$. The projection of this ellipse onto the $xy$-plane is the circle $x^2+y^2=1$. Setting $z=g(x,y)=2-y$, we have

\[
\mathbf{r}(x,y)=\<x,y,2-y>\aspace \mathbf{r}_u\times\mathbf{r}_v=\<0,1,1>
\]

\begin{align*}
\int_C\mathbf{F}\cdot\,d\mathbf{r}&=\iint_S\text{curl }\mathbf{F}\cdot\,d\mathbf{S}\\[2mm]
&=\iint_S(\text{curl }\mathbf{F})\cdot\mathbf{n}\,dS\\[2mm]
&=\iint_S(\text{curl }\mathbf{F})\cdot(\mathbf{r}_x\times\mathbf{r}_y)\,dS\\[2mm]
&=\iint_S\<0,0,1+2y>\cdot\<0,1,1>\,dS\\[2mm]
&=\iint_D(1+2y)\,dA\\[2mm]
&=\int_0^{2\pi}\int_0^1(1+2r\sin\theta)\,r\,dr\,d\theta\\[2mm]
&=\int_0^{2\pi}\left[\frac{1}{2}r^2+2\frac{r^2}{3}\sin\theta\right]_0^1\,d\theta\\[2mm]
&=\int_0^{2\pi}\left(\frac{1}{2}+\frac{2}{3}\sin\theta\right)\,d\theta\\[2mm]
&=\frac{1}{2}(2\pi)+0\\[2mm]
&=\pi.
\end{align*}
\end{example}


%\section*{Circulation, Clarified}
%
%
%
%
%Let $C$ be an oriented closed curve and $\mathbf{v}$ represent a velocity field in fluid flow. Note that $\mathbf{v}\cdot\mathbf{T}$ is the component of $\mathbf{v}$ in the direction of $\mathbf{T}$. Since the dot product is a measure of how aligned two vectors are, the closer the direction of $\mathbf{v}$ is to the direction of $\mathbf{T}$, the larger the value of the dot product. 
%
%\begin{definition}
%The measure of the tendency of a fluid to move around a curve $C$ is called the \textbf{circulation} of $\mathbf{v}$ around $C$, and it is the number
%
%\[
%\int_C\mathbf{v}\cdot\,d\mathbf{r}
%\]
%
%\begin{center}
%\includegraphics[scale=0.5]{circulation1.png}
%\end{center}
%\end{definition}
%
%Let $P_0(x_0,y_0,z_0)$ be a point in the fluid and let $S_a$ be a (really) small disk with radius $a$ and center $P_0$. Then 
%
%\[
%(\text{curl }\mathbf{F})(P)\approx(\text{curl }\mathbf{F})(P_0)
%\]
%
%for all points $P$ on $S_a$ because curl $\mathbf{F}$ is continuous. Thus by Stokes' Theorem, we get the following approximation to the circulation around the boundary circle $\partial S_a$:
%
%\begin{align*}
%\int_{\partial S_a}\mathbf{v}\cdot\,d\mathbf{r}&=\iint_{S_a}\text{curl }\mathbf{v}\cdot\,d\mathbf{S}\\[2mm]
%&=\iint_{S_a}\text{curl }\mathbf{v}\cdot\mathbf{n}\,dS\\[2mm]
%&\approx\iint_{S_a}\text{curl }\mathbf{v}(P_0)\cdot\mathbf{n}(P_0)\,dS\\[2mm]
%&=\text{curl }\mathbf{v}(P_0)\cdot\mathbf{n}(P_0)\iint_{S_a}\,dS\\[2mm]
%&=\text{curl }\mathbf{v}(P_0)\cdot\mathbf{n}(P_0)\pi a^2.
%\end{align*}
%
%This approximation makes sense if you look at the pieces: the dot product that appears gives the amount of ``microscopic'' circulation at $P_0$ and that value gets multiplied by the area of the disk to give an approximation of the circulation around that disk. Moreover, this approximation gets better as $a\rightarrow0$, thus,
%
%\[
%\text{curl }\mathbf{v}(P_0)\cdot\mathbf{n}(P_0)=\lom{a}{0}\frac{1}{\pi a^2}\int_{\partial S_a}\mathbf{v}\cdot\,d\mathbf{r}.
%\]
%
%
%
%\section*{Proving a Previous Theorem}
%
%Now we use Stokes' Theorem to almost immediately prove the previous theorem which stated: If curl $\mathbf{F}=\mathbf{0}$ on a simply connected region like $\R^3$, then $\mathbf{F}$ is a conservative vector field.
%
%\begin{proof}
%From our previous work, we know that if $\int_C\mathbf{F}\cdot\,d\mathbf{r}=0$ for every closed path $C$, then $\mathbf{F}$ is conservative. Given $C$, suppose we can find an orientiable surface $S$ whose boundary is $C$. (This can be done, but the proof requires advanced techniques.) Then Stokes' Theorem gives
%
%\[
%\int_C\mathbf{F}\cdot\,d\mathbf{r}=\iint_S\text{curl }\mathbf{F}\cdot\,d\mathbf{S}=\iint_S\mathbf{0}\cdot\,d\mathbf{S}=0.
%\]
%
%A curve that is not simple can be broken into a finite union of simple curves and the same argument can be applied to all of them before summing all of the integrals.
%\end{proof}












\end{document}