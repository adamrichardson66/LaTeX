\documentclass[11pt,english,
handout
]{beamer}

%Preamble  
\input{/Users/Adam/Desktop/LBCC/MATH80/MATH80_Lesson_Plans/MATH80_Slides_Preamble.tex}

%Textbook: Essential Calculus - Early Transcendentals, 2nd edition - Stewart. ISBN: 978-1-133-11228-0



\begin{document}

%Slide titles are all contained in this file..
\ExecuteMetaData[/Users/Adam/Desktop/LBCC/MATH80/MATH80_Lesson_Plans/MATH80_Slide_Titles.tex]{1608}

%Global Title Slide Format is contained in the following file.
\input{/Users/Adam/Desktop/LBCC/MATH80/MATH80_Lesson_Plans/MATH80_Title_Slide_Format.tex}
\makebeamertitle












\begin{frame}[t]{Stokes' Theorem}
\small
Put succinctly, Stokes' Theorem is a generalization of Green's Theorem to nonplanar surfaces, but it is more accurate to say that Green's Theorem is a special case of Stokes' Theorem, as we will see. \pause

\lspace
Recall the first vector form of Green's Theorem:
\[
\int_C\mathbf{F}\cdotr\,d\mathbf{r}=\iint_D(\text{curl }\mathbf{F})\cdotr\mathbf{k}\,dA
\]

where $D$ is a planar region and $C$ is its boundary.
\end{frame}




















\begin{frame}[t]{Stokes' Theorem}
\small
Let $S$ be an oriented surface with unit normal vector $\mathbf{n}$ as pictured. \visible<2->{The orientation induces a \textbf{positive orientation of the boundary curve $C$} shown in the figure.} \visible<3->{If you walk in the positive direction around $C$ with your head pointing in the direction of $\mathbf{n}$, then the region will always be on your left.}

\visible<1->{\begin{center}
\includegraphics[scale=0.3]{man_orient.png}
\end{center}}
\end{frame}









\begin{frame}[t]{Stokes' Theorem}
\small

\fbox{\parbox{\textwidth}{\begin{theorem}[Stokes' Theorem]
Let $S$ be an oriented piecewise-smooth surface that is bounded by a simple, closed, piecewise-smooth boundary curve $C$ with positive orientation. Let $\mathbf{F}$ be a vector field whose components have continuous partial derivatives on an open region in $\MB{R}^3$ that contains $S$. Then

\[
\int_C\mathbf{F}\cdotr\,d\mathbf{r}=\iint_S\textnormal{curl }\mathbf{F}\cdotr\,d\mathbf{S}.
\]
\end{theorem}}}\pause 


%Since
%
%\[
%\int_C\mathbf{F}\cdotr\,d\mathbf{r}=\int_C\mathbf{F}\cdotr\mathbf{T}\,ds\aspace \iint_S\text{curl }\mathbf{F}\cdotr\,d\mathbf{S}=\iint_S\text{curl }\mathbf{F}\cdotr\mathbf{n}\,dS,
%\]
%
%Stokes' Theorem says that the line integral around the boundary curve of $S$ of the tangential component of $\mathbf{F}$ is equal to the surface integral over $S$ of the normal component of the curl of $\mathbf{F}$.

\lspace
In analogy with Green's theorem, Stokes' theorem says the circulation of $\mathbf{F}$ along the space curve $C$ is the same as the sum of the curl over the surface $S$ bounded by $C$.
\end{frame}











\begin{frame}[t]{Stokes' Theorem}
\small

\textbf{Observe:} In the case where the surface $S$ is flat and lies in the $xy$-plane with upward orientation, the unit normal is $\mathbf{k}$, the surface integral becomes a double integral, and Stokes' Theorem becomes

\[
\int_C\mathbf{F}\cdotr\,d\mathbf{r}=\iint_S\text{curl }\mathbf{F}\cdotr\,d\mathbf{S}=\iint_S(\text{curl }\mathbf{F})\cdotr\mathbf{k}\,dA=\iint_S\left(\pp{Q}{x}-\pp{P}{y}\right)\,dA
\]

which is Green's Theorem!

\lspace

\textbf{Notation:} As before, It is common to use the symbol $\partial S$ to indicate the boundary of $S$, so Stokes' Theorem says 

\[
\int_{\partial S}\mathbf{F}\cdotr\,d\mathbf{r}=\iint_S\text{curl }\mathbf{F}\cdotr\,d\mathbf{S}.
\]

\end{frame}









\begin{frame}[t]{Stokes' Theorem}
\small
\begin{proofs}
We proceed by proving a special case of Stokes' theorem where $S$ is a surface given by $z=g(x,y)$. \visible<2->{Suppose $S$ is a surface given by $z=g(x,y)$ where $(x,y)\in D$, $g$ is $C^1$ and $D$ is a simple planar region whose boundary curve $C_1$ corresponds to $C$.} \visible<3->{If the orientation of $S$ is upward, then the positive orientation of $C$ corresponds to the positive orientation of $C_1$.} 

\visible<2->{
\begin{center}
\includegraphics[scale=0.32]{stokes1.png}
\end{center}}
\end{proofs}
\end{frame}




\begin{frame}[t]{Stokes' Theorem}
\small
\begin{proofs}
Let $\mathbf{F}=P\mathbf{i}+Q\mathbf{j}+R\mathbf{k}$ where the partial derivatives are continuous. \pause Since $S$ is a graph, we can apply a formula for the surface integral from the last section with $\mathbf{F}$ replaced by curl $\mathbf{F}$:

{\footnotesize
\[
\iint_S\text{curl }\mathbf{F}\cdotr\,d\mathbf{S}=\iint_D\left[-\left(\pp{R}{y}-\pp{Q}{z}\right)\pp{z}{x}-\left(\pp{P}{z}-\pp{R}{x}\right)\pp{z}{y}+\left(\pp{Q}{x}-\pp{P}{y}\right)\right]\,dA
\]
}
 
where the partial derivatives of $P,Q,$ and $R$ are evaluated at $(x,y,g(x,y))$. \pause Let $x=x(t)$, $y=y(t)$, $a\leq t\leq b$ be a parameterization of $C_1$. Then the corresponding parameterization of $C$ is

\[
x=x(t)\pspace y=y(t)\pspace z=g(x(t),y(t))\pspace a\leq t\leq b.
\]
\end{proofs}
\end{frame}







\begin{frame}[t]{Stokes' Theorem}
\small
\begin{proofs}
By the Chain Rule, we have 
{\footnotesize
\begin{align*}
\int_C\mathbf{F}\cdotr\,d\mathbf{r}&=\int_a^b\mathbf{F}\cdotr\mathbf{r}'(t)\,dt=\int_a^b\left(P\dd{x}{t}+Q\dd{y}{t}+R\dd{z}{t}\right)\,dt\\[2mm]
&=\int_a^b\left[P\dd{x}{t}+Q\dd{y}{t}+R\left(\pp{z}{x}\dd{x}{t}+\pp{z}{y}\dd{y}{t}\right)\right]\,dt\\[2mm]
&=\int_a^b\left[\left(P+R\pp{z}{x}\right)\dd{x}{t}+\left(Q+R\pp{z}{t}\right)\dd{y}{t}\right]\,dt\\[2mm]
&=\int_{C_1}\left(P+R\pp{z}{x}\right)\,dx+\left(Q+R\pp{z}{y}\right)\,dy\\[2mm]
&=\iint_D\left[\pp{}{x}\left(Q+R\pp{z}{y}\right)-\pp{}{y}\left(P+R\pp{z}{x}\right)\right]\,dA.
\end{align*}}

The last step employs Green's Theorem.
\end{proofs}
\end{frame}










\begin{frame}[t]{Stokes' Theorem}
\small
\begin{proofs}
Using the Chain Rule again, remembering that $P$, $Q$, and $R$ are functions of $x,y,z$ and that $z$ is a function of $x,y$, we get
{\scriptsize
\begin{align*}
\int_C\mathbf{F}\cdotr\,d\mathbf{r}&=\iint_D\left[\left(\pp{Q}{x}+\pp{Q}{z}\pp{z}{x}+\pp{R}{x}\pp{z}{y}+\cancel{\pp{R}{z}\pp{z}{x}\pp{z}{y}}+\cancel{R\pp{^2z}{x\,\partial y}}\right)\right.\\[2mm]
&-\left.\left(\pp{P}{y}+\pp{P}{z}\pp{z}{y}+\pp{R}{y}\pp{z}{x}+\cancel{\pp{R}{z}\pp{z}{y}\pp{z}{x}}+\cancel{R\pp{^2z}{y\,\partial x}}\right)\right]\,dA\\[2mm]
&=\iint_D\left[\pp{Q}{x}+\pp{Q}{z}\pp{z}{x}+\pp{R}{x}\pp{z}{y}-\pp{P}{y}-\pp{P}{z}\pp{z}{y}-\pp{R}{y}\pp{z}{x}\right]\,dA\\[2mm]
&=\iint_D\left[-\pp{R}{y}\pp{z}{x}+\pp{Q}{z}\pp{z}{x}-\pp{P}{z}\pp{z}{y}+\pp{R}{x}\pp{z}{y}+\pp{Q}{x}-\pp{P}{y}\right]\,dA\\[2mm]
&=\iint_D\left[-\left(\pp{R}{y}-\pp{Q}{z}\right)\pp{z}{x}-\left(\pp{P}{z}-\pp{R}{x}\right)\pp{z}{y}+\left(\pp{Q}{x}-\pp{P}{y}\right)\right]\,dA\\[2mm]
&=\iint_S\text{curl }\mathbf{F}\cdotr\,d\mathbf{S}.
\end{align*}}
\end{proofs}
\end{frame}






\begin{frame}[t]{Stokes' Theorem}
\small

One of the more astounding consequences of Stokes' Theorem is the following corollary.

\lspace
\fbox{\parbox{\textwidth}{
\begin{corollary}
The conclusion of Stokes' Theorem is independent of (appropriate) surface. More specifically, if $S_1$ and $S_2$ are oriented surfaces with the same oriented boundary curve $C$ and both satisfy the hypotheses of Stokes' Theorem, then

\[
\iint_{S_1}\text{curl }\mathbf{F}\cdotr\,d\mathbf{S}=\int_C\mathbf{F}\cdotr\,d\mathbf{r}=\iint_{S_2}\text{curl }\mathbf{F}\cdotr\,d\mathbf{S}.
\]
\end{corollary}}} 


\lspace
In other words, \textit{any} (appropriate) surface can be used to determine the circulation of a force field around a closed curve. We will see how this can be helpful with a couple examples.
\end{frame}






\begin{frame}[t]{Stokes' Theorem}
\small
Here we will use a line integral to evaluate an equivalent surface integral.\pause 

\begin{example}
Use Stokes' Theorem to compute the integral $\iint_S\text{curl }\mathbf{F}\cdotr\,d\mathbf{S}$ where $\mathbf{F}(x,y,z)=xz\mathbf{i}+yz\mathbf{j}+xy\mathbf{k}$ and $S$ is the part of the sphere $x^2+y^2+z^2=4$ that lies inside the cylinder $x^2+y^2=1$ and above the $xy$-plane.

\begin{center}
\includegraphics[scale=0.3]{hemisphere.png}
\end{center}
\end{example}
\end{frame}









\begin{frame}[t]{Stokes' Theorem}
\small
\begin{example}
Use Stokes' Theorem to compute the integral $\iint_S\text{curl }\mathbf{F}\cdotr\,d\mathbf{S}$ where $\mathbf{F}(x,y,z)=xz\mathbf{i}+yz\mathbf{j}+xy\mathbf{k}$ and $S$ is the part of the sphere $x^2+y^2+z^2=4$ that lies inside the cylinder $x^2+y^2=1$ and above the $xy$-plane.

\lspace
First we need to find the boundary curve, i.e. where the two surfaces intersect. \pause The equations of our two surfaces yield $4-z^2=1$, so $z=\sqrt{3}$ since $z\geq 0$. Then the boundary $C$ of our region is the circle given by $x^2+y^2=1$, $z=\sqrt{3}$. \pause A vector equation of this circle is
\[
\mathbf{r}(t)=\cos t\mathbf{i}+\sin t\mathbf{j}+\sqrt{3}\mathbf{k}\pspace 0\leq t\leq 2\pi,\text{ so}
\]
\[
\mathbf{r}'(t)=-\sin t\mathbf{i}+\cos t\mathbf{j}.
\]

\vspace{3mm}
Then $\mathbf{F}(\mathbf{r}(t))=\sqrt{3}\cos t\mathbf{i}+\sqrt{3}\sin t\mathbf{j}+\cos t\sin t\mathbf{k}$. 
\end{example}
\end{frame}









\begin{frame}[t]{Stokes' Theorem}
\small
\begin{example}
Use Stokes' Theorem to compute the integral $\iint_S\text{curl }\mathbf{F}\cdotr\,d\mathbf{S}$ where $\mathbf{F}(x,y,z)=xz\mathbf{i}+yz\mathbf{j}+xy\mathbf{k}$ and $S$ is the part of the sphere $x^2+y^2+z^2=4$ that lies inside the cylinder $x^2+y^2=1$ and above the $xy$-plane.

\lspace
By Stokes' theorem, we have
\begin{align*}
\iint_S\text{curl }\mathbf{F}\cdotr\,d\mathbf{S}&=\int_C\mathbf{F}\cdotr\,d\mathbf{r}=\int_0^{2\pi}\mathbf{F}(\mathbf{r}(t))\cdotr\mathbf{r}'(t)\,dt\\[2mm]
&=\int_0^{2\pi}(-\sqrt{3}\cos t\sin t+\sqrt{3}\sin t\cos t)\,dt\\[2mm]
&=\sqrt{3}\int_0^{2\pi}0\,dt=0.
\end{align*}
\end{example}
\end{frame}









\begin{frame}[t]{Stokes' Theorem}
\small
\begin{example}
Use Stokes' Theorem to compute the integral $\iint_S\text{curl }\mathbf{F}\cdotr\,d\mathbf{S}$ where $\mathbf{F}(x,y,z)=xz\mathbf{i}+yz\mathbf{j}+xy\mathbf{k}$ and $S$ is the part of the sphere $x^2+y^2+z^2=4$ that lies inside the cylinder $x^2+y^2=1$ and above the $xy$-plane.

\[
\iint_S\text{curl }\mathbf{F}\cdotr\,d\mathbf{S}=\sqrt{3}\int_0^{2\pi}0\,dt=0
\]

\lspace
Notice that in this example we were able to compute the surface integral of that surface by using \textit{only} values on the boundary curve. Hence why this is often called a higher dimensional version of FTC2.
\end{example}
\end{frame}








\begin{frame}[t]{Stokes' Theorem}
\small
In this example, we'll evaluate a line integral by evaluating an equivalent surface integral, which we'll evaluate by evaluating a double integral.\pause 
\begin{example}
Evaluate $\int_C\mathbf{F}\cdotr\,d\mathbf{r}$ where $\mathbf{F}(x,y,z)=-y^2\mathbf{i}+x\mathbf{j}+z^2\mathbf{k}$ and $C$ is the curve of intersection of the plane $y+z=2$ and the cylinder $x^2+y^2=1$. (Orient $C$ to be counterclockwise from above.)

\begin{center}
\includegraphics[scale=0.28]{ellipse.png}
\end{center}
\end{example}
\end{frame}








\begin{frame}[t]{Stokes' Theorem}
\small
\begin{example}
Evaluate $\int_C\mathbf{F}\cdotr\,d\mathbf{r}$ where $\mathbf{F}(x,y,z)=-y^2\mathbf{i}+x\mathbf{j}+z^2\mathbf{k}$ and $C$ is the curve of intersection of the plane $y+z=2$ and the cylinder $x^2+y^2=1$. (Orient $C$ to be counterclockwise from above.)

\lspace
The curve $C$ is an ellipse. We could compute this particular integral directly, but it is easier to use Stokes' theorem. \pause First we need to compute the curl:

\[
\text{curl }\mathbf{F}=\begin{vmatrix}\mathbf{i}&\mathbf{j}&\mathbf{k}\\[2mm] \pp{}{x} &\pp{}{y}&\pp{}{z} \\[2mm] -y^2 & x & z^2\end{vmatrix}=(1+2y)\mathbf{k}.
\]
\end{example}
\end{frame}











\begin{frame}[t]{Stokes' Theorem}
\small
\begin{example}
Evaluate $\int_C\mathbf{F}\cdotr\,d\mathbf{r}$ where $\mathbf{F}(x,y,z)=-y^2\mathbf{i}+x\mathbf{j}+z^2\mathbf{k}$ and $C$ is the curve of intersection of the plane $y+z=2$ and the cylinder $x^2+y^2=1$. (Orient $C$ to be counterclockwise from above.)

\lspace
Now, we have our choice of whatever surface we would like to use to integrate! \pause The natural and easiest choice is the ellipse embedded in the plane $y+z=2$ that is bounded by $C$. The projection of this ellipse onto the $xy$-plane is the circle $x^2+y^2=1$. \pause Setting $z=g(x,y)=2-y$, we have

\[
\mathbf{r}(x,y)=\<x,y,2-y>\aspace \mathbf{r}_u\times\mathbf{r}_v=\<0,1,1>.
\]
\end{example}
\end{frame}








\begin{frame}[t]{Stokes' Theorem}
\small
\begin{example}
Evaluate $\int_C\mathbf{F}\cdotr\,d\mathbf{r}$ where $\mathbf{F}(x,y,z)=-y^2\mathbf{i}+x\mathbf{j}+z^2\mathbf{k}$ and $C$ is the curve of intersection of the plane $y+z=2$ and the cylinder $x^2+y^2=1$. (Orient $C$ to be counterclockwise from above.)

{\footnotesize
\begin{align*}
\int_C\mathbf{F}\cdotr\,d\mathbf{r}&=\iint_S\text{curl }\mathbf{F}\cdotr\,d\mathbf{S}=\iint_S(\text{curl }\mathbf{F})\cdotr\mathbf{n}\,dS\\[2mm]
&=\iint_S(\text{curl }\mathbf{F})\cdotr(\mathbf{r}_x\times\mathbf{r}_y)\,dS=\iint_S\<0,0,1+2y>\cdotr\<0,1,1>\,dS\\[2mm]
&=\iint_D(1+2y)\,dA=\int_0^{2\pi}\int_0^1(1+2r\sin\theta)\,r\,dr\,d\theta\\[2mm]
&=\int_0^{2\pi}\left[\frac{1}{2}r^2+2\frac{r^2}{3}\sin\theta\right]_0^1\,d\theta=\int_0^{2\pi}\left(\frac{1}{2}+\frac{2}{3}\sin\theta\right)\,d\theta\\[2mm]
&=\frac{1}{2}(2\pi)+0=\pi.
\end{align*}}
\end{example}
\end{frame}

%\section*{Circulation, Clarified}
%
%
%
%
%Let $C$ be an oriented closed curve and $\mathbf{v}$ represent a velocity field in fluid flow. Note that $\mathbf{v}\cdotr\mathbf{T}$ is the component of $\mathbf{v}$ in the direction of $\mathbf{T}$. Since the dot product is a measure of how aligned two vectors are, the closer the direction of $\mathbf{v}$ is to the direction of $\mathbf{T}$, the larger the value of the dot product. 
%
%\begin{definition}
%The measure of the tendency of a fluid to move around a curve $C$ is called the \textbf{circulation} of $\mathbf{v}$ around $C$, and it is the number
%
%\[
%\int_C\mathbf{v}\cdotr\,d\mathbf{r}
%\]
%
%\begin{center}
%\includegraphics[scale=0.5]{circulation1.png}
%\end{center}
%\end{definition}
%
%Let $P_0(x_0,y_0,z_0)$ be a point in the fluid and let $S_a$ be a (really) small disk with radius $a$ and center $P_0$. Then 
%
%\[
%(\text{curl }\mathbf{F})(P)\approx(\text{curl }\mathbf{F})(P_0)
%\]
%
%for all points $P$ on $S_a$ because curl $\mathbf{F}$ is continuous. Thus by Stokes' Theorem, we get the following approximation to the circulation around the boundary circle $\partial S_a$:
%
%\begin{align*}
%\int_{\partial S_a}\mathbf{v}\cdotr\,d\mathbf{r}&=\iint_{S_a}\text{curl }\mathbf{v}\cdotr\,d\mathbf{S}\\[2mm]
%&=\iint_{S_a}\text{curl }\mathbf{v}\cdotr\mathbf{n}\,dS\\[2mm]
%&\approx\iint_{S_a}\text{curl }\mathbf{v}(P_0)\cdotr\mathbf{n}(P_0)\,dS\\[2mm]
%&=\text{curl }\mathbf{v}(P_0)\cdotr\mathbf{n}(P_0)\iint_{S_a}\,dS\\[2mm]
%&=\text{curl }\mathbf{v}(P_0)\cdotr\mathbf{n}(P_0)\pi a^2.
%\end{align*}
%
%This approximation makes sense if you look at the pieces: the dot product that appears gives the amount of ``microscopic'' circulation at $P_0$ and that value gets multiplied by the area of the disk to give an approximation of the circulation around that disk. Moreover, this approximation gets better as $a\rightarrow0$, thus,
%
%\[
%\text{curl }\mathbf{v}(P_0)\cdotr\mathbf{n}(P_0)=\lom{a}{0}\frac{1}{\pi a^2}\int_{\partial S_a}\mathbf{v}\cdotr\,d\mathbf{r}.
%\]
%
%
%
%\section*{Proving a Previous Theorem}
%
%Now we use Stokes' Theorem to almost immediately prove the previous theorem which stated: If curl $\mathbf{F}=\mathbf{0}$ on a simply connected region like $\MB{R}^3$, then $\mathbf{F}$ is a conservative vector field.
%
%\begin{proof}
%From our previous work, we know that if $\int_C\mathbf{F}\cdotr\,d\mathbf{r}=0$ for every closed path $C$, then $\mathbf{F}$ is conservative. Given $C$, suppose we can find an orientiable surface $S$ whose boundary is $C$. (This can be done, but the proof requires advanced techniques.) Then Stokes' Theorem gives
%
%\[
%\int_C\mathbf{F}\cdotr\,d\mathbf{r}=\iint_S\text{curl }\mathbf{F}\cdotr\,d\mathbf{S}=\iint_S\mathbf{0}\cdotr\,d\mathbf{S}=0.
%\]
%
%A curve that is not simple can be broken into a finite union of simple curves and the same argument can be applied to all of them before summing all of the integrals.
%\end{proof}












\end{document}