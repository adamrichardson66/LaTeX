\documentclass[11pt,oneside,english]{amsart}
\usepackage[T1]{fontenc}
\usepackage{geometry}
\usepackage{parskip}
\geometry{verbose,tmargin=0.65in,bmargin=0.65in,lmargin=0.75in,rmargin=0.75in,headheight=0.75cm,headsep=1cm,footskip=1cm}
\setlength{\parskip}{7mm}
\usepackage{setspace}
\onehalfspacing
\pagenumbering{gobble}


\usepackage{bbm}
\usepackage{multicol}
\usepackage{graphicx}
\usepackage{adjustbox}
\usepackage{tikz}
\usetikzlibrary{cd}
\usepackage{pgfplots}
\usepackage{ulem}
\usepackage{adjustbox}
\usepackage{bm}
\usepackage{stmaryrd}
\usepackage{cancel}
\usepackage{mathtools}
\DeclarePairedDelimiter{\ceil}{\lceil}{\rceil}
\DeclarePairedDelimiter\floor{\lfloor}{\rfloor}
\usepackage{enumitem}
\setlist[enumerate,1]{label=\textbf{\arabic*.}}
\usepackage{color, colortbl}
\definecolor{Gray}{gray}{0.9}
\usepackage{babel}
\usepackage{mdframed}
\usepackage{esint}

\theoremstyle{definition}
\newtheorem{theorem}{Theorem}
\newtheorem{corollary}{Corollary}
\newtheorem*{example}{Example}
\newtheorem*{examples}{Examples}
\newtheorem*{definition}{Definition}
\newtheorem*{note}{Nota Bene}

\newcommand{\aspace}{\hspace{7mm}\text{and}\hspace{7mm}}
\newcommand{\ospace}{\hspace{7mm}\text{or}\hspace{7mm}}
\newcommand{\pspace}{\hspace{10mm}}
\newcommand{\lhe}{\stackrel{\text{L'H}}{=}}
\newcommand{\lom}[2]{\lim_{{#1}\rightarrow{#2}}}
\newcommand{\R}{\mathbb{R}}
\newcommand{\dd}[2]{\frac{d{#1}}{d{#2}}}
\newcommand{\pp}[2]{\frac{\partial{#1}}{\partial{#2}}}
\newcommand{\DD}[2]{\frac{\Delta{#1}}{\Delta{#2}}}
\newcommand{\ovec}[1]{\overrightarrow{#1}}
\newcommand{\mbf}[1]{\mathbf{#1}}

\def\<#1>{\mathinner{\langle#1\rangle}}

\makeatletter
\g@addto@macro\normalsize{%
  \setlength\belowdisplayshortskip{5mm}
}
\makeatother



%Textbook: Essential Calculus - Early Transcendentals, 2nd edition - Stewart. ISBN: 978-1-133-11228-0


\begin{document}
\vspace*{-1cm}
\title{16.9 - The Divergence Theorem}
\maketitle



Recall that in section 16.5, we rewrote Green's Theorem in a vector version:

\[
\int_{\partial D}\mathbf{F}\cdot\mathbf{n}\,ds=\iint_D\text{div }\mathbf{F}(x,y)\,dA
\]

where $\partial D$ is the positively oriented boundary curve of the plane region $D$. We hope to extend this theorem to vector fields on $\R^3$. We make a guess that perhaps the higher dimensional version of Green's Theorem is

\[
\iint_S\mathbf{F}\cdot\mathbf{n}\,dS=\iiint_E\text{div }\mathbf{F}(x,y,z)\,dV
\]

where $S$ is the boundary \textit{surface} of the \textit{solid region} $E$. This turns out to be true...

\textbf{Recall.}

\begin{definition}
A solid region $E$ is said to be of \textbf{type 1} if it lies between the graphs of two continuous functions of $x$ and $y$, i.e.

\[
E=\{(x,y,z)\mid(x,y)\in D,u_1(x,y)\leq z \leq u_2(x,y)\}
\]

where $D$ is the projection of $E$ onto the $xy$-plane.
\end{definition}

\begin{definition}
A solid region $E$ is said to be of \textbf{type 2} if it lies between the graphs of two continuous functions of $y$ and $z$, i.e.

\[
E=\{(x,y,z)\mid(y,z)\in D,u_1(y,z)\leq x \leq u_2(y,z)\}
\]

where $D$ is the projection of $E$ onto the $yz$-plane.
\end{definition}

\begin{definition}
A solid region $E$ is said to be of \textbf{type 3} if it lies between the graphs of two continuous functions of $x$ and $z$, i.e.

\[
E=\{(x,y,z)\mid(x,z)\in D,u_1(x,z)\leq x \leq u_2(x,z)\}
\]

where $D$ is the projection of $E$ onto the $yz$-plane.
\end{definition}

\begin{definition}
A region $E$ in $\R^3$ is called a \textbf{simple solid region} if it is simultaneously of type 1, 2, and 3. For example, regions bounded by ellipsoids or rectangular boxes are simple solid regions. The boundary of such regions are closed surfaces.
\end{definition}

\begin{theorem}[The Divergence Theorem]
Let $E$ be a simple solid region and let $S$ be the boundary surface of $E$, given with positive (outward) orientation. Let $\mathbf{F}$ be a vector field whose component functions have continuous partial derivatives on an open region that contains $E$. Then

\[
\iint_S\mathbf{F}\cdot\mathbf{n}\,dS=\iiint_E\text{div }\mathbf{F}(x,y,z)\,dV
\]

In other words, under the given conditions, the flux of $\mathbf{F}$ across the boundary surface of $E$ is equal to the triple integral of the divergence of $\mathbf{F}$. Recall that the divergence of a vector field at a point is a measure of the tendency of a particle to flow away from that point under the influence of that vector field.
\end{theorem}

The Divergence Theorem is like Stokes' Theorem in that it allows us to possibly simplify our work by integrating over a region in space instead of a surface in space, and vice versa. Keep in mind it is a tool that is handy sometimes and not others.

\begin{proof}
Let $\mathbf{F}=P\mathbf{i}+Q\mathbf{j}+R\mathbf{k}$. Then

\[
\text{div }\mathbf{F}=\pp{P}{x}+\pp{Q}{y}+\pp{R}{z},\text{ so}
\]

\[
\iiint_E\text{div }\mathbf{F}\,dv=\iiint_E\pp{P}{x}\,dV+\iiint_E\pp{Q}{y}\,dV+\iiint_E\pp{R}{z}\,dV.
\]

If $\mathbf{n}$ is the unit outward normal of $S$, then the surface integral on the left side of the Divergence Theorem is

\begin{align*}
\iint_S\mathbf{F}\cdot\,d\mathbf{S}&=\iint_S\mathbf{F}\cdot\mathbf{n}\,dS\\[2mm]
&=\iint_S(P\mathbf{i}+Q\mathbf{j}+R\mathbf{k})\cdot\mathbf{n}\,dS\\[2mm]
&=\iint_SP\mathbf{i}\cdot\mathbf{n}\,dS+\iint_SQ\mathbf{j}\cdot\mathbf{n}\,dS+\iint_SR\mathbf{k}\cdot\mathbf{n}\,dS.
\end{align*}

Therefore, it suffices to prove the following three equations:

\begin{align*}
\iint_SP\mathbf{i}\cdot\mathbf{n}\,dS&=\iiint_E\pp{P}{x}\,dV\\[2mm]
\iint_SQ\mathbf{i}\cdot\mathbf{n}\,dS&=\iiint_E\pp{Q}{y}\,dV\\[2mm]
\iint_SR\mathbf{i}\cdot\mathbf{n}\,dS&=\iiint_E\pp{R}{z}\,dV\\[2mm]
\end{align*}

We'll prove the last equation and leave the rest as exercises. To prove the last equation, we use the fact that $E$ is a type 1 region:

\[
E=\{(x,y,z)\mid (x,y)\in D,u_1(x,y)\leq z\leq u_2(x,y)\}
\]

where $D$ is the projection of $E$ onto the $xy$-plane. Then we have

\[
\iiint_E\pp{R}{z}\,dV=\iint\left[\int_{u_1(x,y)}^{u_2(x,y)}\pp{R}{z}(x,y,z)\,dz\right]\,dA
\]

so by FTC2,

\[
\iiint_E\pp{R}{z}\,dV=\iint_D\left[R(x,y,u_2(x,y))-R(x,y,u_1(x,y))\right]\,dA.\hspace{10mm}(*)
\]

\begin{center}
\includegraphics[scale=0.5]{divergence_proof.png}
\end{center}


Now, the boundary surface $S$ consists of three pieces: the bottom surface $S_1$, the top surface $S_2$, and a possibly vertical surface $S_3$ which lies above the boundary curve $\partial D$ of $D$. Note: it might happen that $S_3$ doesn't appear, as with a sphere. Notice that on $S_3$, we have

\[
\iint_{S_3}R\mathbf{k}\cdot\mathbf{n}\,dS=\iint_{S_3}0\,dS=0.
\]

This means that regardless of whether there is a vertical surface in our boundary surface, we can write

\[
\iint_SR\mathbf{k}\cdot\mathbf{n}\,dS=\iint_{S_1}R\mathbf{k}\cdot\mathbf{n}\,dS+\iint_{S_2}R\mathbf{k}\cdot\mathbf{n}\,dS
\]

Now, $S_2$ is a surface given by the equation $z=u_2(x,y)$ where $(x,y)\in D$ and the outward normal vector $\mathbf{n}$ points upward, so by a previous formula,



\[
\iint_{S_2}R\mathbf{k}\cdot\mathbf{n}\,dS=\\int_DR(x,y,u_2(x,y))\,dA.
\]

We can use the same formula on $S_1$, but in this case, the outward normal vector points down, so we multiply by -1:

\[
\iint_{S_1}R\mathbf{k}\cdot\mathbf{n}\,dS=-\iint_DR(x,y,u_1(x,y))\,dA
\]

Notice that these expressions appear in our starred equation $(*)$. Thus, we have

\[
\iiint_E\pp{R}{z}\,dV=\iint_D\left[R(x,y,u_2(x,y))-R(x,y,u_1(x,y))\right]\,dA=\iint_SR\mathbf{k}\cdot\mathbf{n}\,dS
\]

which is what we were trying to prove. The other two equations are solved similarly, and combining them all yields

\[
\iint_S\mathbf{F}\cdot\mathbf{n}\,dS=\iiint_E\text{div }\mathbf{F}(x,y,z)\,dV.
\]
\end{proof}

Now we have a new tool, the Divergence Theorem. We can now wield it to make our lives easier.

\begin{example}
Find the flux of the vector field $\mathbf{F}(x,y,z)=z\mathbf{i}+y\mathbf{j}+x\mathbf{k}$ over the unit sphere.

First we compute the divergence of $\mathbf{F}$:

\[
\text{div }\mathbf{F}=\pp{}{x}(z)+\pp{}{y}{y}+\pp{}{z}(x)=1.
\]

The unit sphere $S$ is the boundary of the unit ball $B$ given by $x^2+y^2+z\leq 1$ so the divergence theorem gives the flux as

\[
\iint_S\mathbf{F}\cdot\,d\mathbf{S}=\iiint_B\text{div }\mathbf{F}\,dV=\iiint_B1\,dV=V(B)=\frac{4}{3}\pi(1)^2=\frac{4}{3}\pi.
\]
\end{example}

\vfill
\pagebreak

\begin{example}
Evaluate $\iint_S\mathbf{F}\cdot\,d\mathbf{S}$ where

\[
\mathbf{F}(x,y,z)=xy\mathbf{i}+(y^2+e^{xz^2})\mathbf{j}+\sin(xy)\mathbf{k}
\]

and $S$ is the surface of the region $E$ bounded by the parabolic cylinder $z=1-x^2$ and the planes $z=0$, $y=0$, and $y+z=2$.

\begin{center}
\includegraphics[scale=0.5]{ex2.png}
\end{center}

In this case, it would be extremely difficult to evaluate the given surface integral directly because we would have to evaluate 4 of them. Additionally, the divergence of $\mathbf{F}$ is much less messy than $\mathbf{F}$ itself:

\[
\text{div }\mathbf{F}=\pp{}{x}(xy)+\pp{}{y}(y^2+e^{xz^2})+\pp{}{z}(\sin(xy))=y+2y=3y.
\]

In other words, it'll be easier to evaluate a triple integral here. We now need to decide which type of region to view our region as, type 1, 2, or 3. In this case it is easiest to view the region as a type 3 region:

\[
E=\{(x,y,z)\mid-1\leq x\leq 1,0\leq z\leq 1-x^2,0\leq y,\leq 2-z\}
\]

Then we have

\begin{align*}
\iint_S\mathbf{F}\cdot\,d\mathbf{S}&=\iiint_E\text{div }\mathbf{F}\,dV\\[2mm]
&=\iiint_E3y\,dV\\[2mm]
&=3\int_{-1}^1\int_0^{1-x^2}\int_0^{2-z}y\,dy\,dz\,dx\\[2mm]
&=3\int_{-1}^1\int_0^{1-x^2}\frac{(2-z)^2}{2}\,dz\,dx\\[2mm]
&=\frac{3}{2}\int_{-1}^1\left[-\frac{(2-z)^3}{3}\right]_0^{1-x^2}\,dx\\[2mm]
&=\frac{1}{2}\int_{-1}^1(x^2+1)^3-8\,dx\\[2mm]
&=-\int_0^1(x^6+3x^4+3x^2-7)\,dx\\[2mm]
&=\frac{184}{35}.
\end{align*}
\end{example}


\begin{note}
We proved the Divergence Theorem only for simple solid regions, but it can also be proved for regions that are finite unions of simple solid regions. For example, consider the region $E$ that lies between the closed surfaces $S_1$ and $S_2$ where $S_1$ lies inside $S_2$. Let $\mathbf{n}_1$ and $\mathbf{n}_2$ outward normals of $S_1$ and $S_2$. Then the boundary surface of $E$ is $S=S_1\cup S_2$ and its normal $\mathbf{n}$ is given by $\mathbf{n}=-\mathbf{n}_1$ on $S_1$ and $\mathbf{n}=\mathbf{n}_2$ on $S_2$. Applying the Divergence Theorem, we get

\begin{align*}
\iiint_E\text{div }\mathbf{F}\,dV&=\iint_S\mathbf{F}\cdot\,d\mathbf{S}\\[2mm]
&=\iint_S\mathbf{F}\cdot\mathbf{n}\,dS\\[2mm]
&=\iint_{S_1}\mathbf{F}\cdot(-\mathbf{n}_1)\,dS+\iint_{S_2}\mathbf{F}\cdot\mathbf{n}_2\,dS\\[2mm]
&=-\iint_{S_1}\mathbf{F}\cdot\,d\mathbf{S}+\iint_{S_2}\mathbf{F}\cdot\,d\mathbf{S}.
\end{align*}
\end{note}

\pagebreak

\begin{example}
Back in section 16.1, we saw the example of the electric field

\[
\mathbf{E}(\mathbf{x})=\frac{\varepsilon Q}{|\mathbf{x}|^3}\mathbf{x}.
\]

Where the electric charge $Q$ is located at the origin and $\mathbf{x}=\<x,y,z>$ is a position vector. Use the Divergence Theorem to show that the electric flux of $\mathbf{E}$ through \textit{any} closed surface $S_2$ that encloses the origin is

\[
\iint_{S_2}\mathbf{E}\cdot\,d\mathbf{S}=4\pi\varepsilon Q.
\]

This is an interesting result because it says that the flux is independent of surface enclosing the electric charge. It also makes the question a bit tricky at first because we aren't given any specific surface over which to integrate.

Let $S_1$ be a tiny sphere of radius $a$ where $a$ is chosen such that $S_1\subseteq S_2$. The result about finite unions we saw above can help us here. Up there we had

\[
\iiint_E\text{div }\mathbf{F}\,dV=-\iint_{S_1}\mathbf{F}\cdot\,d\mathbf{S}+\iint_{S_2}\mathbf{F}\cdot\,d\mathbf{S}.
\]

Thus,

\begin{align*}
\iiint_E\text{div }\mathbf{E}\,dV&=-\iint_{S_1}\mathbf{E}\cdot\,d\mathbf{S}+\iint_{S_2}\mathbf{E}\cdot\,d\mathbf{S}\\[2mm]
\\
\iint_{S_2}\mathbf{E}\cdot\,d\mathbf{S}&=\iiint_E\text{div }\mathbf{E}\,dV+\iint_{S_1}\mathbf{E}\cdot\,d\mathbf{S}\\[2mm]
\iint_{S_2}\mathbf{E}\cdot\,d\mathbf{S}&=0+\iint_{S_1}\mathbf{E}\cdot\,d\mathbf{S}\\[2mm]
\iint_{S_2}\mathbf{E}\cdot\,d\mathbf{S}&=\iint_{S_1}\mathbf{E}\cdot\,d\mathbf{S}=\iint_{S_1}\mathbf{E}\cdot\mathbf{n}\,dS.
\end{align*}

(We will verify that the middle integral is 0 in activity.) The beauty here is that we can compute the surface integral over the sphere $S_1$ which is a nice surface. The normal vector at $\mathbf{x}$ is $\frac{\mathbf{x}}{|\mathbf{x}|}$. Therefore

\[
\mathbf{E}\cdot\mathbf{n}=\frac{\varepsilon Q}{|\mathbf{x}|^3}\mathbf{x}\cdot\frac{\mathbf{x}}{|\mathbf{x}|}=\frac{\varepsilon Q}{|\mathbf{x}|^4}\mathbf{x}\cdot\mathbf{x}=\frac{\varepsilon Q}{|\mathbf{x}|^2}=\frac{\varepsilon Q}{a^2}
\]

since the vector equation of $S_1$ is $|\mathbf{x}|=a$. Thus,

\[
\iint_{S_2}\mathbf{E}\cdot\,d\mathbf{S}=\iint_{S_1}\mathbf{E}\cdot\mathbf{n}\,dS=\frac{\varepsilon Q}{a^2}\iint_{S_1}\,dS=\frac{\varepsilon Q}{a^2}A(S_1)=\frac{\varepsilon Q}{a^2}4\pi a^2=4\pi\varepsilon Q.
\]

Thus, the flux through \textit{any} closed surface that contains the origin is $4\pi\varepsilon Q$. (This is a special case of Gauss's Law).
\end{example}


\section*{Divergence, Clarified}

Let $\mathbf{v}(x,y,z)$ be the velocity field of a fluid with constant density $\rho$. Then $\mathbf{F}=\rho \mathbf{v}$ is the rate of flow per unit area. If $P_0(x_0,y_0,z_0)$ is a point in the fluid and $B_a$ is a ball with center $P_0$ and a very small radius $a$, then $\text{div }\mathbf{F}(P)\approx\text{div }\mathbf{F}(P_0)$ for all points $P$ in $B_a$ since div $\mathbf{F}$ is continuous. We approximate the flux over the boundary sphere $\partial B_a$:

\[
\iint_{\partial B_a}\mathbf{F}\cdot\,d\mathbf{S}=\iiint_{B_a}\text{div }\mathbf{F}\,dV\approx\iiint_{B_a}\text{div }\mathbf{F}(P_0)\,dV=\text{div }\mathbf{F}(P_0)\iiint_{B_a}\,dV=\text{div }\mathbf{F}(P_0)V(B_a).
\]

This approximation gets better as $a\rightarrow0$, and we get that

\[
\text{div }\mathbf{F}(P_0)=\lom{a}{0}\frac{1}{V(B_a)}\iint_{S_a}\mathbf{F}\cdot\,d\mathbf{S}.
\]



This result says that the divergence of $\mathbf{F}$ at $P_0$ is the net rate of outward flux per unit volume at $P_0$, which makes sense. If $\text{div }\mathbf{F}(P)>0$, the net flow is outward near $P$ and $P$ is called a \textbf{source}. If $\text{div }\mathbf{F}(P)<0$, the net flow is inward near $P$ and $P$ is called a \textbf{sink}.

\begin{center}
\includegraphics[scale=0.5]{sink_source.png}
\end{center}

The vector field $\mathbf{F}=x^2\mathbf{i}+y^2\mathbf{j}$ is pictured above. $\text{div }\mathbf{F}=2x+2y$ which is positive whenever $y>-x$. So the points above the line $y=-x$ are sources and the points below are sinks. More intuitively, the vectors that end near $P_1$ are shorter than the vectors that start at $P_1$, so $P_1$ is a source. For $P_2$, the longer vectors are heading into $P_2$ and the shorter vectors are leaving $P_2$, so it is a sink.










\end{document}