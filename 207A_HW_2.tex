\documentclass[11pt,oneside,english]{amsart}
\usepackage[T1]{fontenc}
\usepackage{geometry}
\usepackage{parskip}
\geometry{verbose,tmargin=0.65in,bmargin=0.65in,lmargin=0.75in,rmargin=0.75in,headheight=0.75cm,headsep=1cm,footskip=1cm}
\setlength{\parskip}{7mm}
\usepackage{setspace}
\onehalfspacing
\pagenumbering{gobble}



\usepackage{comment}
\usepackage{bbm}
\usepackage{multicol}
\usepackage{graphicx}
\usepackage{adjustbox}
\usepackage{amssymb}
\usepackage{tikz}
\usetikzlibrary{cd}
\usepackage{pgfplots}
\usepackage{ulem}
\usepackage{adjustbox}
\usepackage{bm}
\usepackage{stmaryrd}
\usepackage{cancel}
\usepackage{mathtools}
\DeclarePairedDelimiter{\ceil}{\lceil}{\rceil}
\DeclarePairedDelimiter\floor{\lfloor}{\rfloor}
\usepackage{enumitem}
\setlist[enumerate,1]{label=\textbf{\arabic*.}}
\usepackage{color, colortbl}
\definecolor{Gray}{gray}{0.9}
\usepackage{babel}
\usepackage{mdframed}
\usepackage{esint}
\usepackage[yyyymmdd]{datetime}
\renewcommand{\dateseparator}{--}

\theoremstyle{definition}
\newtheorem{theorem}{Theorem}
\newtheorem{corollary}{Corollary}
\newtheorem*{example}{Example}
\newtheorem*{examples}{Examples}
\newtheorem*{definition}{Definition}
\newtheorem*{note}{Nota Bene}

\newcommand{\aspace}{\hspace{7mm}\text{and}\hspace{7mm}}
\newcommand{\ospace}{\hspace{7mm}\text{or}\hspace{7mm}}
\newcommand{\pspace}{\hspace{10mm}}
\newcommand{\lhe}{\stackrel{\text{L'H}}{=}}
\newcommand{\lom}[2]{\lim_{{#1}\rightarrow{#2}}}
\newcommand{\R}{\mathbb{R}}
\newcommand{\dd}[2]{\frac{d{#1}}{d{#2}}}
\newcommand{\pp}[2]{\frac{\partial{#1}}{\partial{#2}}}
\newcommand{\DD}[2]{\frac{\Delta{#1}}{\Delta{#2}}}
\newcommand{\ovec}[1]{\overrightarrow{#1}}
\newcommand{\mbf}[1]{\mathbf{#1}}
\newcommand{\MC}[1]{\mathcal{#1}}
\newcommand{\ve}{\varepsilon}

\def\<#1>{\mathinner{\langle#1\rangle}}

\makeatletter
\g@addto@macro\normalsize{%
  \setlength\belowdisplayshortskip{5mm}
}
\makeatother




\begin{document}

\rightline{Adam D. Richardson}
\rightline{207A - ODE}
\rightline{Chen, Weitao}
\rightline{HW 2}
\rightline{\today}



\vspace{1cm}
\begin{enumerate}

%\begin{comment}




\item[\textbf{3.1.13.}] Consider the integral equation 

\[
y(t)=e^{it}+\alpha \int_t^\infty\sin (t-s)\frac{y(s)}{s^2}\,ds
\]

of exercise 4. Define the successive approximations


\[
\begin{cases}\phi_0(t)=0\\
\phi_n(t)=e^{it}+\alpha \int_t^\infty\sin(t-s)\frac{\phi_{n-1}(s)}{s^2}\,ds\pspace(1\leq t\leq \infty)\end{cases}
\]

\begin{enumerate}
\item Show by induction that

\[
\left|\phi_n(t)-\phi_{n-1}(t)\right|\leq\frac{|\alpha|^{n-1}}{(n-1)!t^{n-1}}\pspace(1\leq t<\infty;\,n=1,2,\ldots)
\]


\begin{proof}
Define $r_n(t)=|\phi_n(t)-\phi_{n-1}(t)|$. For the base case $n=1$ we have

\begin{align*}
r_1(t)&=|\phi_1(t)-\phi_0(t)|\\[2mm]
&=\left|e^{it}+\alpha \int_t^\infty\sin(t-s)\frac{\phi_0(s)}{s^2}\,ds-0\right|\\[2mm]
&=\left|e^{it}+\alpha \int_t^\infty0\,ds\right|\\[2mm]
&=\left|e^{it}\right|\\[2mm]
&\leq 1\\[2mm]
&=\frac{|\alpha|^{1-1}}{(1-1)!t^{1-1}}.
\end{align*}

Now, suppose the inequality holds for some $n=p-1$. Then it suffices to show that the inequality holds for some $p>1$. Applying the induction hypothesis, we have

\begin{align*}
r_p(t)&=|\phi_p(t)-\phi_{p-1}(t)|\\[2mm]
&=\left|\alpha\int_t^\infty\frac{\sin(t-s)}{s^2}(\phi_{p-1}(s)-\phi_{p-2}(s))\,ds\right|\\[2mm]
&\leq|\alpha|\int_t^\infty\frac{|\sin(t-s)|}{s^2}|\phi_{p-1}-\phi_{p-2}|\,ds\\[2mm]
&=|\alpha|\int_t^\infty\frac{|\sin(t-s)|}{s^2}r_{p-1}(s)\,ds\\[2mm]
&\leq|\alpha|\int_t^\infty\frac{|\sin(t-s)|}{s^2}\cdot\frac{|\alpha|^{p-2}}{(p-2)!s^{p-2}}\,ds\\[2mm]
&=\frac{|\alpha|^{p-1}}{(p-2)!}\int_t^\infty\frac{|\sin(t-s)|}{s^p}\,ds.
\end{align*}

Note that $|\sin(t-s)|\leq1$ and thus $\frac{|\sin(t-s)|}{s^p}\leq\frac{1}{s^p}$. By the comparison test for integrals, $\int_t^\infty \frac{|\sin(t-s)|}{s^p}\,ds\leq\int_t^\infty\frac{1}{s^p}\,ds$, and the integral on the right hand side converges to $\frac{1}{(p-1)t^{p-1}}$. Therefore, we have

\[
r_p(t)\leq\frac{|\alpha|^{p-1}}{(p-2)!}\int_t^\infty\frac{1}{s^p}\,ds=\frac{|\alpha|^{p-1}}{(p-2)!}\cdot\frac{1}{(p-1)t^{p-1}}=\frac{|\alpha|^{p-1}}{(p-1)!t^{p-1}}.
\]

Thus, the statement is proven for all $n\geq 1$.
\end{proof}

\item Show that the limit function satisfies the given integral equation.

\begin{proof}
We show the limit function satisfies the integral equation by showing $\lom{n}{\infty}\phi_n(t)=\phi(t)$. Observe that

\begin{align*}
\left|\phi(t)-\phi_n(t)\right|&=\left|e^{it}+\alpha \int_t^\infty\sin(t-s)\frac{\phi_n(s)}{s^2}\,ds-e^{it}-\alpha \int_t^\infty\sin(t-s)\frac{\phi_{n-1}(s)}{s^2}\,ds\right|\\[2mm]
&=\left|\alpha\int_t^\infty\frac{\sin(t-s)}{s^2}(\phi_n(t)-\phi_n(t))\,ds\right|\\[2mm]
&\leq|\alpha|\int_t^\infty\frac{|\sin(t-s)|}{s^2}|\phi_n(t)-\phi_{n-1}(t)|\,ds\\[2mm]
&\leq|\alpha|\int_t^\infty\frac{|\sin(t-s)|}{s^2}r_n(t)\,ds\\[2mm]
&\leq|\alpha|\int_t^\infty\frac{|\sin(t-s)|}{s^2}\cdot\frac{|\alpha|^{n-1}}{(n-1)!t^{n-1}}\,ds\\[2mm]
&\leq\frac{|\alpha|^n}{(n-1)!t^{n-1}}\int_t^\infty\frac{1}{s^2}\,ds\\[2mm]
&=\frac{|\alpha|^n}{(n-1)!t^n}.\\[2mm]
\end{align*}

This sequence goes to 0 as $n\rightarrow \infty$ and thus the limit function satisfies the given integral equation.
\end{proof}

\item Using $|\phi_n(t)|\leq|\phi_1(t)-\phi_0(t)|+\cdots+|\phi_n(t)-\phi_{n-1}(t)|$ and the above estimate for $|\phi_n(t)-\phi_{n-1}(t)|$, show that the limit function satisfies the estimate $|\phi(t)|\leq e^{|\alpha|}$ for $1\leq t<\infty$. 

\begin{proof}
By the definition of the limit function $\phi(t)$, and the suggestions above, we have

\begin{align*}
\phi(t)&=\phi_0(t)+\sum_{n=0}^\infty\left(\phi_{n+1}(t)-\phi_n(t)\right)\\[2mm]
|\phi(t)|&\leq|\phi_0(t)|+\sum_{n=0}^\infty\left|\phi_{n+1}(t)-\phi_n(t)\right|\\[2mm]
&\leq0+\sum_{n=0}^\infty\frac{|\alpha|^n}{n!t^n}\\[2mm]
&=e^\frac{|\alpha|}{t}\\[2mm]
&\leq e^{|\alpha|}.
\end{align*}
\end{proof}
\end{enumerate}



\pagebreak

\item[\textbf{3.2.3.}] Given the system

\begin{align*}
y_1'&=y_1^2+y_2^2+1\\[2mm]
y_2'&=y_1^2-y_2^2-1
\end{align*}

Let $\displaystyle y=\begin{pmatrix}y_1\\y_2\end{pmatrix}$ and let $B$ be the ``box'' $\{(t,\mathbf{y})\mid |t|\leq 1,\,|\mathbf{y}|\leq 2\}$. Determine the bounds $M$ and $K$ in (3.19) for $\mathbf{f}$ and $\pp{\mathbf{f}}{y_j}$ for this case. Determine $\alpha$ of Theorem 3.3. Compute the first three successive approximations of the solution $\phi(t)$ satisfying the initial condition $\phi(0)=\mathbf{0}$, $\displaystyle \phi=\begin{pmatrix}\phi_1\\ \phi_2\end{pmatrix}$.


First, we have $\mathbf{f}(t,\mathbf{y})=\left(y_1^2+y_2^2+1,y_1^2-y_2^2-1\right)$. Thus,

\begin{align*}
|\mathbf{f}(t,\mathbf{y})|&=\left|(y_1^2+y_2^2+1,y_1^2-y_2^2-1)\right|\\[2mm]
&=|y_1^2+y_2^2+1|+|y_1^2-y_2^2-1|\\[2mm]
&\leq2|y_1^2+y_2^2+1|\\[2mm]
&\leq2|4+4+1|\\[2mm]
&=10.
\end{align*}

and we may choose $M=10$. Next, we have

\[
\pp{\mathbf{f}}{y_1}=(2y_1,2y_1)\aspace\pp{\mathbf{f}}{y_2}=(2y_2,-2y_2)\text{, so}
\]


\[
\left|\pp{\mathbf{f}}{y_1}\right|=\left|\pp{\mathbf{f}}{y_2}\right|=2|y_1|+2|y_2|=2(|y_1|+|y_2|)\leq2(2)=4,
\]

and we may choose $K=4$. Since $\alpha=\min\left\{a,\frac{b}{M}\right\}$, in this case we have $\alpha=\min\left\{1,\frac{2}{10}\right\}=\frac{1}{5}$. 

Now we compute the first three successive approximations. Since $\phi(0)=\mathbf{0}$, $\mathbf{f}(t,\mathbf{0})=(1,-1)$, and we have

\[
\phi_1(t)=\mathbf{0}+\int_0^t\mathbf{f}(s,\mathbf{0}))\,ds=\int_0^t(1,-1)\,ds=(t,-t).
\]

Then $\mathbf{f}(t,(t,-t))=(2t^2+1,-1)$, so

\begin{align*}
\phi_2(t)&=\mathbf{0}+\int_0^t(2s^2+1,-1)\,ds\\[2mm]
&=\left.\left(\frac{2}{3}s^3+s,-s\right)\right|_0^t\\[2mm]
&=\left(\frac{2}{3}t^3+t,-t\right).\\[2mm]
\end{align*}

Then $\mathbf{f}\left(t,\left(\frac{2}{3}t^3+t,-t\right)\right)=\left(\frac{4}{9}s^6+\frac{4}{3}s^4+s^2+1,\frac{4}{9}s^6+\frac{4}{3}s^4-1\right)$, so

\begin{align*}
\phi_3(t)&=\mathbf{0}+\int_0^t\left(\frac{4}{9}s^6+\frac{4}{3}s^4+s^2+1,\frac{4}{9}s^6+\frac{4}{3}s^4-1\right)\,ds\\[2mm]
&=\left.\left(\frac{4}{63}s^7+\frac{4}{15}s^5+\frac{1}{3}s^3+s,\frac{4}{63}s^7+\frac{4}{15}s^5-s\right)\right|_0^t\\[2mm]
&=\left(\frac{4}{63}t^7+\frac{4}{15}t^5+\frac{1}{3}t^3+t,\frac{4}{63}t^7+\frac{4}{15}t^5-t\right).
\end{align*}

Altogether, we have 

\begin{align*}
\phi_1(t)&=(t,-t),\\[2mm]
\phi_2(t)&=\left(\frac{2}{3}t^3+t,-t\right),\\[2mm]
\phi_3(t)&=\left(\frac{4}{63}t^7+\frac{4}{15}t^5+\frac{1}{3}t^3+t,\frac{4}{63}t^7+\frac{4}{15}t^5-t\right).
\end{align*}

\pagebreak

\item[\textbf{3.4.10.}] Let $f(t,y)$ be a continuous scalar function on the whole $(t,y)$-plane. Suppose that $\pp{f}{y}$ is also continuous and for any $\alpha>0$ suppose

\[
\left|\pp{f}{y}(t,y)\right|\leq K=K(\alpha)\pspace(|t|\leq\alpha,\,|y|<\infty).
\]

Show that given any $(t_0,y_0)$ the equation $y'=f(t,y)$ has a unique solution $\phi(t)$ on $-\infty<t<\infty$, such that $\phi(t_0)=y_0$. [\textit{Hint:} By exercise 9, $|\phi(t)|$ is bounded on every interval $|t-t_0|\leq\alpha$; now apply the corollary to Theorem 3.6.]

\begin{proof}
First, let $f(t,y)$ be a continuous scalar function on the entire $(t,y)$-plane. Suppose that $\pp{f}{y}$ is continuous and that for all $\alpha>0$, $\left|\pp{f}{y}\right|\leq K=K(\alpha)$. Utilizing the hint, by exercise 9, we have that the solution $\phi$ of $y'=f(t,y)$ is bounded (i.e. finite) on every interval $|t-t_0|\leq\alpha$. In particular, $M(\alpha)=\sup_{[t_0-\alpha,t_0+\alpha]}\phi(t)<\infty$ for all $\alpha>0$. By the corollary to Theorem 3.6, $\phi$ can be continued \textit{uniquely} in both directions to the interval $(t_0-\alpha,t_0+\alpha)$. Since this is true for all $\alpha>0$, and $\bigcup_{\alpha>0}(t_0-\alpha,t_0+\alpha)=(-\infty,\infty)$, $\phi$  can be extended to $-\infty<t<\infty$. By the uniqueness guaranteed by the corollary, $\phi$ is a unique solution on $-\infty<t<\infty$.
\end{proof}

\pagebreak





\item[\textbf{M5.}] Prove \textbf{the Osgood uniquness theorem:} Suppose that the function $f(t,y)$ satisfies the condition $|f(t,y_2)-f(t,y_1)|\leq h(|y_2-y_1|)$ for every pair of points $(t,y_1)$, $(t,y_2)$ in a region $D$. Here we assume that the function $h(u)$ is continuous on $0<u<\alpha$ for some $\alpha>0$, that $h(u)>0$, and that

\[
\lom{\ve}{0^+}\int_{\ve}^\alpha \frac{1}{h(u)}\,du=\infty.
\]

Then through each point $(t_0,y_0)$ in $D$ there is at most one solution of the equation $y'=f(t,y)$.

\begin{proof}
Utilizing the hint, Suppose $\phi_1$ and $\phi_2$ are solutions to the IVP with $\phi_1(t_0)=\phi_2(t_0)=y_0$. Define $\psi(t)=\phi_2(t)-\phi_1(t)$, and suppose by way of contradiction that $\psi(t)\not\equiv0$. Since $\phi_1$ and $\phi_2$ are solutions, by hypothesis we have

\vspace{-10mm}
\begin{align*}
|\psi'(t)|&=|\phi_2'(t)-\phi_1'(t)|\\[2mm]
&=|f(t,\phi_2(t))-f(t,\phi_1(t))|\\[2mm]
&\leq h(|\phi_2(t)-\phi_1(t)|)\\[2mm]
&=h(|\psi(t)|)\\[2mm]
&<2h(|\psi(t)|).
\end{align*}

Next, suppose $\psi(t_1)\neq0$ for some $t_1>t_0$ and let $u(t)$ be the solution of $u'=2h(u)$ satisfying the initial condition $u(t_1)=|\psi(t_1)|$. Then we have

\[
|\psi'(t_1)|<2h(|\psi(t_1)|)=u'(t_1).
\]

Additionally, $u'=2h(u)$ is a separabale ODE, and thus,

\vspace{-8mm}
\begin{align*}
\dd{u}{t}&=2h(u)\\[2mm]
\frac{1}{h(u)}\,du&=2\,dt\\[2mm]
\int_{u(t)}^{u(t_1)}\frac{1}{h(u)}\,du&=\int_t^{t_1}2\,dt\\[2mm]
\int_{u(t)}^{u(t_1)}\frac{1}{h(u)}\,du&=2(t_1-t).\\[2mm]
\end{align*}

Note that the integral on the left goes to infinity as $u(t)\rightarrow0^+$ by hypothesis. The right hand side can only satisfy this limit if $t\rightarrow-\infty$. Also note that $2(t_1-t)$ is strictly increasing in $t$. It follows that $u(t)\rightarrow0^+$ as $t\rightarrow-\infty$ and $u(t)>0$ on $(-\infty,t_1]$.

Now, since $|\psi(t_1)|=u(t_1)$, and $|\psi'(t_1)|<u'(t_1)$, it follows from the differentiability of both functions that $\psi(t)>u(t)$ on some interval to the left of $t_1$. Since $u(t)>0$ on $(-\infty,t_1]$, and $\psi(t_0)=0$, yet $u(t)<\psi(t)$ on some interval to the left of $t_1$, there must exist some $t_2\in(t_0,t_1)$ such that $\psi(t_2)=u(t_2)$. Again, since $|\psi(t_2)|=u(t_2)$, and $|\psi'(t_2)|<u'(t_2)$, it follows from the differentiability of both functions that $\psi(t)>u(t)$ on some interval to the left of $t_2$. We can continue this process via an inductive argument which ultimately reveals that $\psi(t_0)=u(t_0)=0$, a contradiction, since $u(t)>0$ on $(-\infty, t_1]$.

Therefore, our original assumption that $\psi(t)\not\equiv0$ is incorrect, and it must be the case that $\phi_1=\phi_2$, i.e. the solution is unique.
\end{proof}



\end{enumerate}

\end{document}