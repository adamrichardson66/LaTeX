\documentclass[11pt,oneside,english]{amsart}
\usepackage[T1]{fontenc}
\usepackage{geometry}
\usepackage{parskip}
\geometry{verbose,tmargin=0.65in,bmargin=0.65in,lmargin=0.75in,rmargin=0.75in,headheight=0.75cm,headsep=1cm,footskip=1cm}
\setlength{\parskip}{7mm}
\usepackage{setspace}
\onehalfspacing
\pagenumbering{gobble}



\usepackage{comment}
\usepackage{bbm}
\usepackage{multicol}
\usepackage{graphicx}
\usepackage{adjustbox}
\usepackage{amssymb}
\usepackage{tikz}
\usetikzlibrary{cd}
\usepackage{pgfplots}
\usepackage{ulem}
\usepackage{adjustbox}
\usepackage{bm}
\usepackage{stmaryrd}
\usepackage{cancel}
\usepackage{mathtools}
\DeclarePairedDelimiter{\ceil}{\lceil}{\rceil}
\DeclarePairedDelimiter\floor{\lfloor}{\rfloor}
\usepackage{enumitem}
\setlist[enumerate,1]{label=\textbf{\arabic*.}}
\usepackage{color, colortbl}
\definecolor{Gray}{gray}{0.9}
\usepackage{babel}
\usepackage{mdframed}
\usepackage{esint}
\usepackage[yyyymmdd]{datetime}
\renewcommand{\dateseparator}{--}

\theoremstyle{definition}
\newtheorem{theorem}{Theorem}
\newtheorem{corollary}{Corollary}
\newtheorem*{example}{Example}
\newtheorem*{examples}{Examples}
\newtheorem*{definition}{Definition}
\newtheorem*{note}{Nota Bene}

\newcommand{\aspace}{\hspace{7mm}\text{and}\hspace{7mm}}
\newcommand{\ospace}{\hspace{7mm}\text{or}\hspace{7mm}}
\newcommand{\pspace}{\hspace{10mm}}
\newcommand{\lhe}{\stackrel{\text{L'H}}{=}}
\newcommand{\lom}[2]{\lim_{{#1}\rightarrow{#2}}}
\newcommand{\R}{\mathbb{R}}
\newcommand{\dd}[2]{\frac{d{#1}}{d{#2}}}
\newcommand{\pp}[2]{\frac{\partial{#1}}{\partial{#2}}}
\newcommand{\DD}[2]{\frac{\Delta{#1}}{\Delta{#2}}}
\newcommand{\ovec}[1]{\overrightarrow{#1}}
\newcommand{\mbf}[1]{\mathbf{#1}}
\newcommand{\MC}[1]{\mathcal{#1}}
\newcommand{\ve}{\varepsilon}

\def\<#1>{\mathinner{\langle#1\rangle}}

\makeatletter
\g@addto@macro\normalsize{%
  \setlength\belowdisplayshortskip{5mm}
}
\makeatother




\begin{document}

\rightline{Adam D. Richardson}
\rightline{207A - ODE}
\rightline{Chen, Weitao}
\rightline{HW 3}
\rightline{\today}



\vspace{1cm}
\begin{enumerate}

%\begin{comment}




\item[\textbf{4.2.4(e).}] Given the following scalar differential equation, decide whether the solution given is asymptotically stable, stable but not asymptotically stable, or unstable: $y'=-y$, $\phi(t)=e^{-t}$.


The equilibrium solution to this ODE is $y_0\equiv 0$. Let $\ve>0$ and choose $\delta=\ve$. Then whenever $|\phi(t_0)-y_0|=|e^{-t_0}-0|=e^{-t_0}<\delta$, and $t\geq t_0$, we have

\[
|\phi(t)-y_0|=e^{-t}\leq e^{-t_0}<\delta=\ve,
\]

so the solution is stable. To test if it is asymptotically stable, we find that

\[
\lom{t}{\infty}\phi(t)=\lom{t}{\infty}e^{-t}=0,
\]

so the solution is asymptotically stable by definition.

\vfill
\pagebreak

\item[\textbf{4.3.7.}] Determine if the zero solution of the following system is stable, asymptotically stable, or unstable: $u''+2ku'+\alpha^2 u=0$ ($k>0$, $\alpha^2>0$ constants).

First write $u=u_1$ and $u'=u_2$. Then we have the system

\begin{multicols}{2}
\begin{align*}
u_1'&=u_2\\[2mm]
u_2'&=-\alpha^2u_1-2ku_2
\end{align*}


\[\begin{pmatrix}
0 & 1\\
-\alpha^2 & -2k
\end{pmatrix}\]

\end{multicols}
The characteristic equation of this matrix is $\lambda^2 +2k\lambda+\alpha^2=0$, yielding

\[
\lambda=\frac{-2k\pm\sqrt{4k^2-4\alpha^2}}{2}=-k\pm\sqrt{k^2-\alpha^2}.
\]

If $k\geq \sqrt{\alpha}$, then both eigenvalues will be real and negative so by Theorem 4.1 the zero solution is asymptotically stable. If $0<k<\sqrt{\alpha}$, then both eigenvalues will be complex and have negative real part, making the zero solution asymptotically stable again by Theorem 4.1.



\vfill
\pagebreak



\item[\textbf{4.3.15.}] Prove the following result:

Let all eigenvalues of $A$ have real parts negative, and let $B(t)$ be continuous for $0\leq t<\infty$ and such that $\int_0^\infty |B(s)|\,ds<\infty$. Then the zero solution of (4.5) is asymptotically stable. [Hint: start with (4.6) and take norms; then use the Gronwall inequality.]

\begin{proof}

Taking the hint, write $\psi(t,t_0,y_0)=\exp((t-t_0)A)y_0+\int_{t_0}^t\exp((t-s)A)B(s)\psi(s,t_0,y_0)\,ds$. Taking the norm of both sides and applying the triangle inequality, we have

\begin{align*}
\left|\psi(t,t_0,y_0)\right|&=\left|\exp((t-t_0)A)y_0+\int_{t_0}^t\exp((t-s)A)B(s)\psi(s,t_0,y_0)\,ds\right|\\[2mm]
&\leq \left|\exp((t-t_0)A)y_0\right|+\int_{t_0}^t\left|\exp((t-s)A)B(s)\psi(s,t_0,y_0)\right|\,ds.\\[2mm]
\end{align*}

\vspace{-10mm}
Using the inequality (4.4) on page 152, we have 

\[
|\psi(t,t_0,y_0)|\leq Ke^{-\sigma(t-t_0)}|\psi(t_0,t_0,y_0)|+K\int_{t_0}^te^{-(t-s)\sigma}|B(s)||\psi(s,t_0,y_0)|\,ds.
\]

for the appropriate $K$ and $\sigma$ as defined there. Multiplying both sides by $e^{\sigma t}$ yields

\[
e^{\sigma t}|\psi(t,t_0,y_0)|\leq Ke^{\sigma t_0}|\psi(t_0,t_0,y_0)|+K\int_{t_0}^te^{\sigma s}|B(s)||\psi(s,t_0,y_0)|\,ds.
\]

\vspace{-5mm}
Then by applying Gronwall's inequality on the function $e^{\sigma t}|\psi(t,t_0,y_0)|$, we have

\vspace{-5mm}
\begin{align*}
e^{\sigma t}|\psi(t,t_0,y_0)|&\leq K|\psi(t_0,t_0,y_0)|e^{\int_{t_0}^t|B(s)|\,ds}\\[2mm]
|\psi(t,t_0,y_0)|&\leq e^{-\sigma t}K|\psi(t_0,t_0,y_0)|\int_{t_0}^t|B(s)|\,ds.
\end{align*}

Since $\int_0^\infty |B(s)|\,ds<\infty$, the right hand side goes to 0 as $t\rightarrow \infty$. Consequently, $\psi(t,t_0,y_0)$ will approach 0 as $t\rightarrow\infty$. The rest of the argument is identical to that of the proof of Theorem 4.2 in the textbook --- and does not require the hypothesis that $\int_{t_0}^t|B(s)|\,ds<\infty$ --- which yields that the zero solution is asymptotically stable.


\end{proof}












\end{enumerate}



\end{document}