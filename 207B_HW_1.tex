\documentclass[11pt,oneside,english]{amsart}
\usepackage[T1]{fontenc}
\usepackage{geometry}
\usepackage{parskip}
\geometry{verbose,tmargin=0.65in,bmargin=0.65in,lmargin=0.75in,rmargin=0.75in,headheight=0.75cm,headsep=1cm,footskip=1cm}
\setlength{\parskip}{7mm}
\usepackage{setspace}
\onehalfspacing
\pagenumbering{gobble}



\usepackage{bbm}
\usepackage{multicol}
\usepackage{graphicx}
\usepackage{adjustbox}
\usepackage{amssymb}
\usepackage{tikz}
\usepackage{pgfplots}
\usepackage{pgffor}
\usetikzlibrary{cd}
\usepackage{ulem}
\usepackage{adjustbox}
\usepackage{bm}
\usepackage{stmaryrd}
\usepackage{cancel}
\usepackage{mathtools}
\DeclarePairedDelimiter{\ceil}{\lceil}{\rceil}
\DeclarePairedDelimiter\floor{\lfloor}{\rfloor}
\usepackage{enumitem}
\setlist[enumerate,1]{label=\textbf{\arabic*.}}
\usepackage{color, colortbl}
\definecolor{Gray}{gray}{0.9}
\usepackage{babel}
\usepackage{mdframed}
\usepackage{esint}
\usepackage[yyyymmdd]{datetime}
\renewcommand{\dateseparator}{--}
\usepackage{url}
\usepackage[unicode=true,pdfusetitle,
 bookmarks=true,bookmarksnumbered=false,bookmarksopen=false,
 breaklinks=false,pdfborder={0 0 1},backref=false,colorlinks=true]
 {hyperref}
\hypersetup{urlcolor=blue}

\theoremstyle{definition}
\newtheorem{theorem}{Theorem}
\newtheorem*{theorem*}{Theorem}
\newtheorem*{proposition*}{Proposition}
\newtheorem{corollary}{Corollary}
\newtheorem*{example}{Example}
\newtheorem*{examples}{Examples}
\newtheorem*{definition}{Definition}
\newtheorem*{note}{Nota Bene}

\newcommand{\aspace}{\hspace{7mm}\text{and}\hspace{7mm}}
\newcommand{\ospace}{\hspace{7mm}\text{or}\hspace{7mm}}
\newcommand{\pspace}{\hspace{10mm}}
\newcommand{\lhe}{\stackrel{\text{L'H}}{=}}
\newcommand{\lom}[2]{\lim_{{#1}\rightarrow{#2}}}
\newcommand{\R}{\mathbb{R}}
\newcommand{\ve}{\varepsilon}
\newcommand{\dd}[2]{\frac{d{#1}}{d{#2}}}
\newcommand{\pp}[2]{\frac{\partial{#1}}{\partial{#2}}}
\newcommand{\DD}[2]{\frac{\Delta{#1}}{\Delta{#2}}}
\newcommand{\ovec}[1]{\overrightarrow{#1}}
\newcommand{\MC}[1]{\mathcal{#1}}
\usepackage{bbm}


\def\<#1>{\mathinner{\langle#1\rangle}}

\makeatletter
\g@addto@macro\normalsize{%
  \setlength\belowdisplayshortskip{5mm}
}
\makeatother

\def\Xint#1{\mathchoice
{\XXint\displaystyle\textstyle{#1}}%
{\XXint\textstyle\scriptstyle{#1}}%
{\XXint\scriptstyle\scriptscriptstyle{#1}}%
{\XXint\scriptscriptstyle\scriptscriptstyle{#1}}%
\!\int}
\def\XXint#1#2#3{{\setbox0=\hbox{$#1{#2#3}{\int}$ }
\vcenter{\hbox{$#2#3$ }}\kern-.6\wd0}}
\def\ddashint{\Xint=}
\def\dashint{\Xint-}




\begin{document}

\rightline{Adam D. Richardson}
\rightline{207B - PDE}
\rightline{Moradifam, Amir}
\rightline{HW 1}
\rightline{\today}


Exercises are from Chapter 2 of Evans' textbook \textit{Partial Differential Equations}, 1st ed. 

\vspace{5mm}
\begin{enumerate}

\setcounter{enumi}{2}

\item Modify the proof of the mean value formulas to show for $n\geq3$ that

\[
u(0)=\dashint_{\partial B(0,r)}g\,dS+\frac{1}{n(n-2)\alpha(n)}\int_{B(0,r)}\left(\frac{1}{|x|^{n-2}}-\frac{1}{r^{n-2}}\right)\,f\,dx,
\]

provided
\[
\begin{cases}-\Delta u=f & \text{in }B^\circ(0,r)\\ \hspace{6.25mm}u=g & \text{on }\partial B(0,r).\end{cases}
\]

\begin{proof}
Considering Evans' proof, define

\[
\phi(r):=\dashint_{\partial B(0,r)} u(y)\,dS(y)=\dashint_{\partial B(0,1)}u(rz)\,dS(z).
\]

Then since $-\Delta u=f$ in $B^\circ(0,r)$, we find that

\[
\phi'(r)=\frac{r}{n}\dashint_{B(0,r)}\Delta u(y)\,dy=-\frac{r}{n}\dashint_{B(0,r)}f(y)\,dy
\]

Now, we can use FTC to yield

\begin{align*}
\int_0^r\phi'(t)\,dt&=-\int_0^r\left[\frac{t}{n}\dashint_{B(0,t)}f\,dy\right]\,dt\\[2mm]
\phi(r)-\phi(0)&=-\int_0^r\left[\frac{1}{n\alpha(n)t^{n-1}}\int_{B(0,t)}f\,dy\right]\,dt\\[2mm]
\dashint_{\partial B(0,r)}u(y)\,dS-u(0)&=-\frac{1}{n\alpha(n)}\int_0^r\left[\frac{1}{t^{n-1}}\int_{B(0,t)}f\,dy\right]\,dt\\[2mm]
u(0)&=\dashint_{\partial B(0,r)}g\,dS+\frac{1}{n\alpha(n)}\int_0^r\left[\frac{1}{t^{n-1}}\int_{B(0,t)}f\,dy\right]\,dt.
\end{align*}

Since we have some improper integrals, let $\ve>0$. In order to use integration by parts, let $\displaystyle u(t)=\int_{B(0,t)}f\,dy$ and $\displaystyle v(t)=\frac{1}{t^{n-1}}$. Then $\displaystyle du=\int_{\partial B(0,t)}f\,dy\,dt$ and $\displaystyle v=\frac{1}{2-n}\frac{1}{t^{n-2}}$ and we have

\begin{align*}
u(0)&=\dashint_{\partial B(0,r)}g\,dS+\frac{1}{n\alpha(n)}\left(\left[\frac{1}{2-n}\frac{1}{t^{n-2}}\int_{B(0,t)}f\,dy\right]_\ve^r-\int_\ve^r\left[\frac{1}{2-n}\frac{1}{t^{n-2}}\int_{\partial B(0,t)}f\,dS\right]\,dt\right)\\[2mm]
&=\dashint_{\partial B(0,r)}g\,dS+\frac{1}{n(n-2)\alpha(n)}\left(\int_\ve^r\left[\frac{1}{t^{n-2}}\int_{\partial B(0,t)}f\,dS\right]\,dt-\left[\frac{1}{t^{n-2}}\int_{B(0,t)}f\,dy\right]_\ve^r\right)\\[2mm]
&=\dashint_{\partial B(0,r)}g\,dS+\frac{1}{n(n-2)\alpha(n)}\left(\int_\ve^r\left[\frac{1}{t^{n-2}}\int_{\partial B(0,t)}f\,dS\right]\,dt-\frac{1}{r^{n-2}}\int_{B(0,r)}f\,dy+\frac{1}{\ve^{n-2}}\int_{B(0,\ve)}f\,dy\right)\\[2mm]
\end{align*}

Considering the last integral in the equation above, since $f$ is bounded, we have

\begin{align*}
\frac{1}{\ve^{n-2}}\int_{B(0,\ve)}f\,dy&\leq\frac{1}{\ve^{n-2}}\int_{B(0,\ve)}\max_{B(0,\ve)} f\,dy\\[2mm]
&=\frac{\max_{B(0,\ve)}f}{\ve^{n-2}}\int_{B(0,\ve)}1\,dy\\[2mm]
&=\frac{\max_{B(0,\ve)}f}{\ve^{n-2}}\cdot\alpha(n)\ve^n\\[2mm]
&=\max_{B(0,\ve)}f\cdot\alpha(n)\cdot\ve^2\\[2mm]
&=K\ve^2.
\end{align*}

$K\ve^2\rightarrow0$ as $\ve\rightarrow 0$ so the integral does as well. Next,

\[
\int_\ve^r\left[\frac{1}{t^{n-2}}\int_{\partial B(0,t)}f(y)\,dS(y)\right]\,dt=\int_{B(\ve,r)}\frac{1}{|y|^{n-2}}f\,dy.
\]

Since this integral is continuous in $\ve$, as $\ve\rightarrow 0$, this integral becomes $\displaystyle \int_{B(0,r)}\frac{1}{|y|^{n-2}}f\,dy$. Combining these two results, we have

\begin{align*}
u(0)&=\dashint_{\partial B(0,r)}g\,dS+\frac{1}{n(n-2)\alpha(n)}\left(\int_\ve^r\left[\frac{1}{t^{n-2}}\int_{\partial B(0,t)}f\,dS\right]\,dt-\frac{1}{r^{n-2}}\int_{B(0,r)}f\,dy+\frac{1}{\ve^{n-2}}\int_{B(0,\ve)}f\,dy\right)\\[2mm]
u(0)&=\dashint_{\partial B(0,r)}g\,dS+\frac{1}{n(n-2)\alpha(n)}\left(\int_{B(0,r)}\frac{1}{|y|^{n-2}}f\,dy-\frac{1}{r^{n-2}}\int_{B(0,r)}f\,dy+0\right)\\[2mm]
u(0)&=\dashint_{\partial B(0,r)}g\,dS+\frac{1}{n(n-2)\alpha(n)}\left(\int_{B(0,r)}\frac{1}{|y|^{n-2}}f\,dy-\int_{B(0,r)}\frac{1}{r^{n-2}} f\,dy+0\right)\\[2mm]
u(0)&=\dashint_{\partial B(0,r)}g\,dS+\frac{1}{n(n-2)\alpha(n)}\int_{B(0,r)}\left(\frac{1}{|y|^{n-2}}-\frac{1}{r^{n-2}}\right) f\,dy,
\end{align*}

so

\[
u(0)=\dashint_{\partial B(0,r)}g\,dS+\frac{1}{n(n-2)\alpha(n)}\int_{B(0,r)}\left(\frac{1}{|x|^{n-2}}-\frac{1}{r^{n-2}}\right) f\,dx\\[2mm]
\]

as required.
\end{proof}

\vfill
\pagebreak




\setcounter{enumi}{4}

\item Prove that there exists a constant $C$, depending only on $n$, such that

\[
\max_{B(0,1)}|u|\leq C\left(\max_{\partial B(0,1)}|g|+\max_{B(0,1)}|f|\right).
\]

whenever $u$ is a smooth solution of

\[
\begin{cases}-\Delta u=f & \text{in }B^\circ(0,1)\\ \hspace{6.25mm}u=g & \text{on }\partial B(0,1).\end{cases}
\]

[Hint: $-\Delta(u+\frac{|x|^2}{2n}\lambda)\leq0$ for $\lambda:=\max_{B(0,1)}|f|$.]
\begin{proof}
Utilizing the hint, define $v(x)=u(x)+\frac{|x|^2}{2n}\lambda$ where $\lambda$ is as defined above. First, we have

\[
-\Delta v=-\Delta u-\Delta\left(\frac{|x|^2}{2n}\lambda\right)=f-\lambda\leq0,
\]

so $v$ is subharmonic. By Problem 4 (see below), subharmonic functions attain their maximum on the boundary, so we have

\[
\max_{B(0,1)}|u(x)|\leq\max_{B(0,1)}|v(x)|=\max_{\partial B(0,1)}|v(x)|\leq\max_{\partial B(0,1)}|u(x)|+\max_{\partial B(0,1)}\frac{|x|^2}{2n}\lambda=\max_{\partial B(0,1)}|g|+\frac{1}{2n}\max_{B(0,1)}|f|.
\]

Thus there exists a constant $C$ such that 

\[
\max_{B(0,1)}|u|\leq C\left(\max_{\partial B(0,1)}|g|+\max_{B(0,1)}|f|\right).
\]
\end{proof}

\pagebreak

\item Use Poisson's formula for the ball to prove

\[
r^{n-2}\frac{r-|x|}{(r+|x|)^{n-1}}u(0)\leq u(x)\leq r^{n-2}\frac{r+|x|}{(r-|x|)^{n-1}}u(0)
\]

whenever $u$ is positive and harmonic in $B^\circ(0,r)$. This is an explicit form of Harnack's inequality.

\begin{proof}

Recall that Poisson's Formula for the ball is

\begin{align*}
u(x)&=\frac{r^2-|x|^2}{n\alpha(n)r}\int_{\partial B(0,r)}\frac{g(y)}{|x-y|^n}\,dS(y)\\[2mm]
&=r^{n-2}\frac{r^2-|x|^2}{n\alpha(n)r^{n-1}}\int_{\partial B(0,r)}\frac{g(y)}{|x-y|^n}\,dS(y)\\[2mm]
&=r^{n-2}(r^2-|x|^2)\dashint_{\partial B(0,r)}\frac{g(y)}{|x-y|^n}\,dS(y).\\[2mm]
\end{align*}

Note also that

\begin{align*}
u(0)&=\frac{r}{n\alpha(n)}\int_{\partial B(0,r)}\frac{g(y)}{|y|^n}\,dS(y)\\[2mm]
&=\frac{r}{n\alpha(n)}\int_{\partial B(0,r)}\frac{g(y)}{r^n}\,dS(y)\\[2mm]
&=\frac{1}{n\alpha(n)r^{n-1}}\int_{\partial B(0,r)}g(y)\,dS(y)\\[2mm]
&=\dashint_{\partial B(0,r)}g(y)\,dS(y).
\end{align*}


Now, since $y\in\partial B(0,r)$ and $x\in B^\circ(0,r)$, we have $r-|x|\leq |x-y|\leq r+|x|$. Whence

\pagebreak

\begin{multicols}{2}
\begin{align*}
u(x)&=r^{n-2}(r^2-|x|^2)\dashint_{\partial B(0,r)}\frac{g(y)}{|x-y|^n}\,dS(y)\\[2mm]
&\geq r^{n-2}(r^2-|x|^2)\dashint_{\partial B(0,r)}\frac{g(y)}{(r+|x|)^n}\,dS(y)\\[2mm]
&= r^{n-2}\frac{r^2-|x|^2}{(r+|x|)^n}\dashint_{\partial B(0,r)}g(y)\,dS(y)\\[2mm]
&= r^{n-2}\frac{(r-|x|)(r+|x|)}{(r+|x|)^n}u(0)\\[2mm]
&= r^{n-2}\frac{r-|x|}{(r+|x|)^{n-1}}u(0), \text{ and}\\[2mm]
\end{align*}

\begin{align*}
u(x)&=r^{n-2}(r^2-|x|^2)\dashint_{\partial B(0,r)}\frac{g(y)}{|x-y|^n}\,dS(y)\\[2mm]
&\leq r^{n-2}(r^2-|x|^2)\dashint_{\partial B(0,r)}\frac{g(y)}{(r-|x|)^n}\,dS(y)\\[2mm]
&= r^{n-2}\frac{r^2-|x|^2}{(r-|x|)^n}\dashint_{\partial B(0,r)}g(y)\,dS(y)\\[2mm]
&= r^{n-2}\frac{(r-|x|)(r+|x|)}{(r-|x|)^n}u(0)\\[2mm]
&= r^{n-2}\frac{r+|x|}{(r-|x|)^{n-1}}u(0),\\[2mm]
\end{align*}
\end{multicols}
therefore

\[
r^{n-2}\frac{r-|x|}{(r+|x|)^{n-1}}u(0)\leq u(x)\leq r^{n-2}\frac{r+|x|}{(r-|x|)^{n-1}}u(0).
\]
\end{proof}



\vfill
\pagebreak

\setcounter{enumi}{8}

\item Let $U^+$ denote the open half-ball $\{x\in\R^n\mid |x|<1,\, x_n>0\}$. Assume $u\in C^2(\bar{U}^+)$ is harmonic in $U^+$, with $u=0$ on $\partial U^+\cap\{x_n=0\}$. Set 

\[
v(x):=\begin{cases} u(x) & \text{if } x_n\geq0\\ -u(x_1,\ldots,x_{n-1},-x_n) & \text{if } x_n<0\end{cases}
\]

for $x\in U=B^\circ(0,1)$. Prove $v$ is harmonic in $U$.


\begin{proof}
First, if $x\in U^+$, then $v(x)=u(x)$ which is harmonic. Now, let $x\in U$ with $x_n<0$. Then by definition of $v$, $v_{x_i}=-u_{x_i}$ and $v_{x_ix_i}=-u_{x_ix_i}$ for $1\leq i\leq n-1$, but $v_{x_n}$ and $v_{x_nx_n}$ must be treated separately. We have

\[
v_{x_n}=\pp{}{x_n}(-u(x_1,\ldots,x_{n-1},-x_n))=-\pp{u}{(-x_n)}\pp{(-x_n)}{x_n}=-\pp{u}{(-x_n)}\cdot(-1)=\pp{u}{(-x_n)}=u_{-x_n}.
\]

Since we may write $x_i=-(-x_i)$ for any $1\leq i\leq n$, we have $u_{x_i}=-u_{-x_i}$ and $u_{x_ix_i}=u_{(-x_i)(-x_i)}$ so


\begin{align*}
v_{x_nx_n}&=\pp{}{x_n}(u_{-x_n})\\[2mm]
&=\pp{}{x_n}\left(\pp{u}{(-x_n)}\right)\\[2mm]
&=\pp{}{x_n}\left(\pp{u}{x_n}\pp{x_n}{(-x_n)}\right)\\[2mm]
&=\pp{}{x_n}\left(-\pp{u}{(-x_n)}\pp{x_n}{(-x_n)}\right)\\[2mm]
&=-\pp{^2u}{(-x_n)(-x_n)}\\[2mm]
&=-u_{(-x_n)(-x_n)}.
\end{align*}

We proceed by showing that $v\in C^1(U)$ and then $v\in C^2(U)$. Let $x_0=(x_1,x_2,\ldots,x_{n-1},0)$. Write $\lom{x}{x_0^\pm}$ to indicate $x_i\rightarrow x_i$ for $1\leq i\leq n-1$ and  $x_n\rightarrow 0^\pm$ respectively. We have


\begin{align*}
\lom{x}{x_0^-}Dv(x)&=\lom{x}{x_0^-}(v_{x_1}(x),v_{x_2}(x),\ldots,v_{x_{n-1}}(x),v_{x_n}(x))\\[2mm]
&=\lom{x}{x_0^-}(-u_{x_1}(x),-u_{x_2}(x),\ldots,-u_{x_{n-1}}(x),u_{-x_n}(x))\\[2mm]
&=\lom{x}{x_0^-}(u_{-x_1}(x),u_{-x_2}(x),\ldots,u_{-x_{n-1}}(x),u_{-x_n}(x))\\[2mm]
&=\lom{x}{x_0^+}(u_{x_1}(x),u_{x_2}(x),\ldots,u_{x_{n-1}}(x),u_{x_n}(x))\\[2mm]
&=(u_{x_1}(x_0),u_{x_2}(x_0),\ldots,u_{x_{n-1}}(x_0),u_{x_n}(x_0))\\[2mm]
&=Du(x_0)\\[2mm]
&=Dv(x_0),\text{ and}
\end{align*}

\begin{align*}
\lom{x}{x_0^+}Dv(x)&=\lom{x}{x_0^+}(v_{x_1}(x),\ldots,v_{x_n}(x))\\[2mm]
&=\lom{x}{x_0^+}(u_{x_1}(x),\ldots,u_{x_n}(x))\\[2mm]
&=(u_{x_1}(x_0),\ldots,u_{x_n}(x_0))\\[2mm]
&=Du(x_0)\\[2mm]
&=Dv(x_0).
\end{align*}

Thus, $v\in C^1(U)$. Similarly,

\begin{align*}
\lom{x}{x_0^+}D^2v(x)&=\lom{x}{x_0^+}(v_{x_1x_1}(x),v_{x_2x_2}(x),\ldots,v_{x_{n-1}x_{n-1}}(x),v_{x_nx_n}(x))\\[2mm]
&=\lom{x}{x_0^+}(u_{x_1x_1}(x),u_{x_2x_2}(x),\ldots,u_{x_{n-1}x_{n-1}}(x),u_{x_nx_n}(x))\\[2mm]
&=\lom{x}{x_0^+}(u_{x_1x_1}(x_0),u_{x_2x_2}(x_0),\ldots,u_{x_{n-1}x_{n-1}}(x_0),u_{x_nx_n}(x_0))\\[2mm]
&=D^2u(x_0)\\[2mm]
&=D^2v(x_0),\text{ and}
\end{align*}

\begin{align*}
\lom{x}{x_0^-}D^2v(x)&=\lom{x}{x_0^-}(v_{x_1x_1}(x),v_{x_2x_2}(x),\ldots,v_{x_{n-1}x_{n-1}}(x),v_{x_nx_n}(x))\\[2mm]
&=\lom{x}{x_0^-}(-u_{x_1x_1}(x),-u_{x_2x_2}(x),\ldots,-u_{x_{n-1}x_{n-1}}(x),-u_{(-x_n)(-x_n)}(x))\\[2mm]
&=\lom{x}{x_0^-}(u_{(-x_1)(-x_1)}(x),u_{(-x_2)(-x_2)}(x),\ldots,u_{(-x_{n-1})(-x_{n-1})}(x),u_{(-x_n)(-x_n)}(x))\\[2mm]
&=\lom{x}{x_0^+}(u_{x_1x_1}(x),u_{x_2x_2}(x),\ldots,u_{x_{n-1}x_{n-1}}(x),u_{x_nx_n}(x))\\[2mm]
&=(u_{x_1x_1}(x_0),u_{x_2x_2}(x_0),\ldots, u_{x_nx_n}(x_0))\\[2mm]
&=D^2u(x_0)\\[2mm]
&=D^2v(x_0).
\end{align*}


Thus, $v\in C^2(U)$, whence $\Delta v=0$ so $v$ is harmonic in $U$.
%Since $-x_n>0$ and $u\in C^2(\bar{U}^+)$, $-u_{(-x_n)(-x_n)}=v_{x_nx_n}$ is continuous, and we have $v\in C^2(\bar{U}^+)$. Lastly, note that $(x_1,x_2,\ldots,x_{n-1},-x_n)\in U^+$ since $x_n<0$, and $u$ is harmonic there by hypothesis, i.e. $\Delta u=0$. Consequently, 
%
%\begin{align*}
%\Delta v&=\sum_{i=1}^nv_{x_ix_i}\\[2mm]
%&=v_{x_1x_1}+v_{x_2x_2}+\cdots+v_{x_{n-1}x_{n-1}}+v_{x_nx_n}\\[2mm]
%&=-u_{x_1x_1}-u_{x_2x_2}-\cdots-u_{x_{n-1}x_{n-1}}-u_{(-x_n)(-x_n)}\\[2mm]
%&=-(u_{x_1x_1}+u_{x_2x_2}+\cdots+u_{x_{n-1}x_{n-1}}+u_{(-x_n)(-x_n)})\\[2mm]
%&=-\Delta u\\[2mm]
%&=0.
%\end{align*}
%
%Thus, $v$ is harmonic in $U$.

\end{proof}

\pagebreak

\setcounter{enumi}{3}
\item We say that $v\in C^2\left(\overline{U}\right)$ is \textit{subharmonic} if $-\Delta v\leq 0\text{ in }U$.

\begin{enumerate}
\item Prove that for subharmonic $v$, 

\[
v(x)\leq \dashint_{B(x,r)} v\,dy\text{ for all }B(x,r)\subset U. 
\]

\begin{proof}
Suppose $v$ is subharmonic. Using the same technique as in Evans' proof of equality in the case of harmonic functions, define

\[
\phi(r):=\dashint_{\partial B_r(x)}v(y)\,dS_y.
\]


Employing a change of variables, we can preserve the value of the integral while shifting the region of integration to be centered at the origin and scaled, giving

\[
\phi(r):=\dashint_{\partial B_1(0)}v(x+rz)\,dS_z\pspace\text{where }y=x+rz,\text{ and so }z=\frac{y-x}{r}.
\]

\vspace{-0.5cm}
Differentiating this with respect to $r$ and applying the Divergence theorem, we find

\begin{align*}
\dd{\phi}{r}&=\dd{}{r}\dashint_{\partial B_1(0)}v(x+rz)\,dS_z\\[2mm]
&=\dashint_{\partial B_1(0)}Dv(x+rz)\cdot z\,dS_z\\[2mm]
&=\dashint_{\partial B_1(0)}Dv(x+rz)\cdot \frac{y-x}{r}\,dS_z\\[2mm]
&=\dashint_{\partial B_1(0)}Dv\cdot \mathbf{n}\,dS_z\\[2mm]
&=\dashint_{\partial B_1(0)}\pp{v}{\mathbf{n}}\,dS_z\\[2mm]
&=\dashint_{B_1(0)}\Delta v(x+rz)\,dz\\[2mm]
&=\frac{r}{n}\dashint_{B_r(x)}\Delta(y)\,dy\geq0.\\[2mm]
\end{align*}

Thus, the mean value of $v$ over the surface of any sphere inside $U$ is increasing as $r$ increases. We also know that when $r=0$, $\phi(r)=v(x)$, therefore

\[
\dashint_{B_r(x)} v\,dy\geq v(x)
\]

as required.
\end{proof}

\item Prove that, therefore, $\displaystyle \max_{\overline{U}}v=\max_{\partial U}v$.

\begin{proof}
Suppose $v$ is subharmonic. Again we will follow Evans' proof and suppose there exists a point $x_0\in U$ such that $v(x_0)=\max_{\overline{U}}v$. Let $0<r<\text{dist}(x_0,\partial U)$. Then we may employ part (a) above to yield

\[
\max_{\overline{U}}v=v(x_0)\leq \dashint_{B_r(x_0)} v\,dy\leq \max_{B_r(x_0)}v\leq\max_{\overline{U}}v.
\]

Thus $\max_{\overline{U}}v= \dashint_{B_r(x_0)} v\,dy$, which can only be true if $v(y)=\max_{\overline{U}}v$ for all $y\in B_r(x_0)$. Now, let $\Omega=\{x\in \overline{U}\mid v(x)=\max_{\overline{U}}v(x)\}$. This set is closed by the Lemma below. Thus, $\Omega=\overline{\Omega}$. Moreover, $B_r(x_0)\subset U$ as shown above so it follows that $U=\Omega=\overline{\Omega}$ since the argument can be repeated for any $x\in B_r(x_0)$. Since $U=\overline{\Omega}$, $U=\overline{U}$ and it follows that $\displaystyle \max_{\overline{U}}v=\max_{\partial U}v$.
\end{proof}

\textbf{Lemma.} Suppose $v(x)$ is continuous on $\overline{U}$. Then the set of points at which $v$ attains its maximum value is a closed set.

\begin{proof}
Suppose $v(x)$ is continuous on $\overline{U}$. By the Extreme Value theorem, $v$ attains a maximum value on $\overline{U}$, so we may let $M=\max_{\overline{U}}v(x)$, and let $\Omega=\{x\in \overline{U}\mid v(x)=M\}$. Then $\Omega^c=\{x\in \overline{U}\mid v(x)<M)\}=v^{-1}((-\infty,M))$. Since $v$ is continuous, $\Omega^c=v^{-1}((-\infty,M))$ is an open set, and so $\Omega$ is closed by definition.
\end{proof}

\end{enumerate}




\end{enumerate}

\end{document}