\documentclass[11pt,oneside,english]{amsart}
\usepackage[T1]{fontenc}
\usepackage{geometry}
\usepackage{parskip}
\geometry{verbose,tmargin=0.65in,bmargin=0.65in,lmargin=0.75in,rmargin=0.75in,headheight=0.75cm,headsep=1cm,footskip=1cm}
\setlength{\parskip}{7mm}
\usepackage{setspace}
\onehalfspacing
\pagenumbering{gobble}



\usepackage{bbm}
\usepackage{multicol}
\usepackage{graphicx}
\usepackage{adjustbox}
\usepackage{amssymb}
\usepackage{tikz}
\usetikzlibrary{cd}
\usepackage{pgfplots}
\usepackage{ulem}
\usepackage{adjustbox}
\usepackage{bm}
\usepackage{stmaryrd}
\usepackage{cancel}
\usepackage{mathtools}
\DeclarePairedDelimiter{\ceil}{\lceil}{\rceil}
\DeclarePairedDelimiter\floor{\lfloor}{\rfloor}
\usepackage{enumitem}
\setlist[enumerate,1]{label=\textbf{\arabic*.}}
\usepackage{color, colortbl}
\definecolor{Gray}{gray}{0.9}
\usepackage{babel}
\usepackage{mdframed}
\usepackage{esint}
\usepackage[yyyymmdd]{datetime}
\renewcommand{\dateseparator}{--}

\theoremstyle{definition}
\newtheorem{theorem}{Theorem}
\newtheorem{corollary}{Corollary}
\newtheorem*{example}{Example}
\newtheorem*{examples}{Examples}
\newtheorem*{definition}{Definition}
\newtheorem*{note}{Nota Bene}

\newcommand{\aspace}{\hspace{7mm}\text{and}\hspace{7mm}}
\newcommand{\ospace}{\hspace{7mm}\text{or}\hspace{7mm}}
\newcommand{\pspace}{\hspace{10mm}}
\newcommand{\lhe}{\stackrel{\text{L'H}}{=}}
\newcommand{\lom}[2]{\lim_{{#1}\rightarrow{#2}}}
\newcommand{\R}{\mathbb{R}}
\newcommand{\dd}[2]{\frac{d{#1}}{d{#2}}}
\newcommand{\pp}[2]{\frac{\partial{#1}}{\partial{#2}}}
\newcommand{\DD}[2]{\frac{\Delta{#1}}{\Delta{#2}}}
\newcommand{\ovec}[1]{\overrightarrow{#1}}
\newcommand{\mbf}[1]{\mathbf{#1}}
\newcommand{\MC}[1]{\mathcal{#1}}

\def\<#1>{\mathinner{\langle#1\rangle}}

\makeatletter
\g@addto@macro\normalsize{%
  \setlength\belowdisplayshortskip{5mm}
}
\makeatother




\begin{document}

\rightline{Adam D. Richardson}
\rightline{209A - Real Analysis}
\rightline{Zheng, Qi}
\rightline{HW 1}
\rightline{\today}


\textbf{Exercises, p. 24.}

\vspace{1cm}
\begin{enumerate}
\setcounter{enumi}{1}
\item Prove that $\mathcal{B}_{\R}$ is generated by each of the following:

\begin{enumerate}
\itemsep6mm
\setcounter{enumii}{1}
\item the closed intervals: $\mathcal{E}_2=\{[a,b]:a<b\}$

First, in his proof, Folland establishes that $\MC{M}(\MC{E}_j)\subset\MC{B}_{\R}$ for all $j$, so it suffices to show the reverse inclusion $\MC{B}_{\R}\subset\MC{M}(\MC{E}_j)$ for $j\in\{2,3,4,5,6,7,8\}$. He also establishes that $\MC{M}(\MC{E}_1)=\MC{B}_\R$, a fact which we will use below.

\begin{proof}

Given any open interval $(a,b)$ in $\R$, we can write

\[
(a,b)=\bigcup_{n=1}^\infty\left[a+\frac{1}{n},b-\frac{1}{n}\right].
\]

The elements of this union are elements of $\MC{E}_2$ and $(a,b)$ is an element of $\MC{E}_1$, so $\MC{E}_1\subset\MC{M}(\MC{E}_2)$. Thus, by Lemma 1.1,

\[
\MC{M}(\MC{E}_2)\supset\MC{M}(\MC{E}_1)=\MC{B}_\R.
\]


%Now, let $\MC{E}\subset\MC{B}_{\R}$. Then $\MC{E}$ is a countable union of open intervals in $\R$, say, 
%
%\[
%\MC{E}=\bigcup_{n=1}^\infty(a_n,b_n).
%\]
%
%Using the construction above, we can write
%
%\[
%\MC{E}=\bigcup_{n=1}^\infty(a_n,b_n)=\bigcup_{n=1}^\infty\bigcup_{k=1}^\infty\left[a_n+\frac{1}{k},b_n-\frac{1}{k}\right].
%\]
%
%Countable unions of countable unions are at most countable, and since $\MC{E}$ is a countable union of a countable union of elements in $\MC{M}(\MC{E}_2)$, $\MC{E}\in\MC{M}(\MC{E}_2)$. By Lemma 1.1, $\MC{B}_{\R}\subset\MC{M}(\MC{E}_2)$. A similar argument will be used in the following parts.
\end{proof}

\pagebreak

\item the half-open intervals: $\MC{E}_3=\{(a,b]:a<b\}$ or $\MC{E}_4=\{[a,b):a<b\}$

\begin{proof}
We will begin by proving the inclusion $\MC{B}_{\R}\subset\MC{M}(\MC{E}_3)$. Observe that, given any open interval $(a,b)$ in $\R$, we can write 

\[
(a,b)=\bigcup_{n=1}^\infty\left(a,b-\frac{1}{n}\right].
\]

The elements of this union are elements of $\MC{E}_3$ so $\MC{E}_1\subset\MC{M}(\MC{E}_3)$. Thus, by Lemma 1.1,

\[
\MC{M}(\MC{E}_3)\supset\MC{M}(\MC{E}_1)=\MC{B}_\R.
\]

Similarly, we can write

\[
(a,b)=\bigcup_{n=1}^\infty\left[a+\frac{1}{n},b\right).
\]

The elements of this union are elements of $\MC{E}_4$ so $\MC{E}_1\subset\MC{M}(\MC{E}_4)$. Thus, by Lemma 1.1,

\[
\MC{M}(\MC{E}_4)\supset\MC{M}(\MC{E}_1)=\MC{B}_\R.
\]

% Let $\MC{E}\subset\MC{B}_{\R}$. Then $\MC{E}$ is a countable union of open intervals in $\R$, say, 
%
%\[
%\MC{E}=\bigcup_{n=1}^\infty(a_n,b_n).
%\]
%
%Using the construction above, we can write
%
%\[
%\MC{E}=\bigcup_{n=1}^\infty(a_n,b_n)=\bigcup_{n=1}^\infty\bigcap_{k=1}^\infty\left(a_n,b_n+\frac{1}{k}\right].
%\]
%
%Countable unions of countable intersections are at most countable, and since $\MC{E}$ is a countable union of a countable intersection of elements in $\MC{M}(\MC{E}_3)$, $\MC{E}\in\MC{M}(\MC{E}_3)$. By Lemma 1.1 $\MC{B}_{\R}\subset\MC{M}(\MC{E}_3)$.
%
%A very similar argument works to show the inclusion $\MC{B}_{\R}\subset\MC{M}(\MC{E}_4)$, we need only replace $\MC{E}_3$ by $\MC{E}_4$ and use the construction 
%
%\[
%(a,b)=\bigcap_{n=1}^\infty\left[a-\frac{1}{n},b\right).
%\]

\end{proof}

\pagebreak

\item the open rays: $\MC{E}_5=\{(a,\infty):a\in \R\}$ or $\MC{E}_6=\{(-\infty,a):a\in \R\}$

\begin{proof}
We will begin by proving the inclusion $\MC{B}_{\R}\subset\MC{M}(\MC{E}_5)$. Observe that, given any open interval $(a,b)$ in $\R$, we can write 

\[
(a,b)=(a,\infty)\cap[b,\infty)^c.
\]

Additionally, we can write 

\[
[b,\infty)=\bigcap_{n=1}^\infty\left(b-\frac{1}{n},\infty\right).
\]

The elements of this intersection are elements of $\MC{E}_5$, thus the intersection $[b,\infty)\in\MC{M}(\MC{E}_5)$ by the definition of a $\sigma$-algebra. Moreover, $[b,\infty)^c\in\MC{M}(\MC{E}_5)$, and so it follows that  $(a,b)=(a,\infty)\cap[b,\infty)^c\in\MC{M}(\MC{E}_5)$. Therefore, by Lemma 1.1,

\[
\MC{M}(\MC{E}_5)\supset\MC{M}(\MC{E}_1)=\MC{B}_\R.
\]




%\[
%\bigcup_{n=1}^\infty\left(b-\frac{1}{n},\infty\right)^c=[b,\infty)^c\in\MC{M}(\MC{E}_5)
%\]
%
%and therefore .



%Now, let $\MC{E}\subset\MC{B}_{\R}$. Then $\MC{E}$ is a countable union of open intervals in $\R$, say, 
%
%\[
%\MC{E}=\bigcup_{n=1}^\infty(a_n,b_n).
%\]
%
%Using the constructions above we can write
%
%\begin{align*}
%\MC{E}&=\bigcup_{n=1}^\infty(a_n,b_n)\\[2mm]
%&=\bigcup_{n=1}^\infty\left((a_n,\infty)\cap[b_n,\infty)^c\right)\\[2mm]
%&=\bigcup_{n=1}^\infty\left((a_n,\infty)\cap\bigcup_{k=1}^\infty\left(b_n-\frac{1}{k},\infty\right)^c\right)\\[2mm]
%&=\bigcup_{n=1}^\infty\bigcup_{k=1}^\infty\left((a_n,\infty)\cap\left(b_n-\frac{1}{k},\infty\right)^c\right).\\[2mm]
%\end{align*}
%
%The elements in this double union are intersections of elements in $\MC{M}(\MC{E}_5)$ and thus $\MC{E}\in\MC{M}(\MC{E}_5)$. By Lemma 1.1, $\MC{B}_{\R}\subset\MC{M}(\MC{E}_5)$. 


Similarly, observe that any open set $(a,b)$ in $\R$ can be written as $(a,b)=(-\infty,a]^c\cap(-\infty,b)$, and 

\[
(-\infty,a]=\bigcap_{n=1}^\infty\left(-\infty,a+\frac{1}{n}\right).
\]

The elements of this intersection are elements of $\MC{E}_6$, thus the intersection $(-\infty,a]\in\MC{M}(\MC{E}_6)$ by the definition of a $\sigma$-algebra. Moreover, $(-\infty,a]^c\in\MC{M}(\MC{E}_6)$, and so it follows that  $(a,b)=(-\infty,a]^c\cap(-\infty,b)\in\MC{M}(\MC{E}_6)$. Therefore, by Lemma 1.1,

\[
\MC{M}(\MC{E}_6)\supset\MC{M}(\MC{E}_1)=\MC{B}_\R.
\]



\end{proof}

\item the closed rays: $\MC{E}_7=\{[a,\infty):a\in\R\}$ or $\MC{E}_8=\{(-\infty,a]:a\in\R\}$

\begin{proof}
We will begin by proving the inclusion $\MC{B}_{\R}\subset\MC{M}(\MC{E}_7)$. Observe that, given any open interval $(a,b)$ in $\R$, we can write 

\[
(a,b)=(a,\infty)\cap[b,\infty)^c.
\]

First, $[b,\infty)^c\in\MC{M}(\MC{E}_7)$ since $[b,\infty)\in\MC{E}_7$. Additionally,

\[
(a,\infty)=\bigcap_{n=1}^\infty\left[a+\frac{1}{n},\infty\right).
\]

%and each element of this intersection is an element of $\MC{E}_7$ so $(a,\infty)\in\MC{M}(\MC{E}_7)$. Now, let $\MC{E}\subset\MC{B}_{\R}$. Then $\MC{E}$ is a countable union of open intervals in $\R$, say, 
%
%\[
%\MC{E}=\bigcup_{n=1}^\infty(a_n,b_n).
%\]
%
%Using the constructions above we can write
%
%
%\begin{align*}
%\MC{E}&=\bigcup_{n=1}^\infty(a_n,b_n)\\[2mm]
%&=\bigcup_{n=1}^\infty\left((a_n,\infty)\cap[b_n,\infty)^c\right)\\[2mm]
%&=\bigcup_{n=1}^\infty\left([b_n,\infty)^c\cap\bigcap_{k=1}^\infty\left[a_n+\frac{1}{k},\infty\right)\right).\\[2mm]
%\end{align*}

The elements of this intersection are elements of $\MC{E}_7$, thus the intersection $(a,\infty)\in\MC{M}(\MC{E}_7)$ by the definition of a $\sigma$-algebra. It follows that  $(a,b)=(a,\infty)\cap[b,\infty)^c\in\MC{M}(\MC{E}_7)$. Therefore, by Lemma 1.1,

\[
\MC{M}(\MC{E}_7)\supset\MC{M}(\MC{E}_1)=\MC{B}_\R.
\]

Similarly, observe that any open set $(a,b)$ in $\R$ can be written as $(a,b)=(-\infty,a]^c\cap(-\infty,b)$, and 

\[
(-\infty,b)=\bigcup_{n=1}^\infty\left(-\infty,b-\frac{1}{n}\right].
\]

$(-\infty,a]^c\in\MC{M}(\MC{E}_8)$ since $(-\infty,a]\in\MC{E}_8$, and the elements of the union above are elements of $\MC{E}_8$, so the intersection $(-\infty,b)\in\MC{M}(\MC{E}_8)$ as well. It follows that $(a,b)=(-\infty,a]^c\cap(-\infty,b)\in\MC{M}(\MC{E}_8)$. Therefore, by Lemma 1.1,

\[
\MC{M}(\MC{E}_8)\supset\MC{M}(\MC{E}_1)=\MC{B}_\R.
\]


\end{proof}


\end{enumerate}
\end{enumerate}

\textbf{Exercises, p. 27.}


\begin{enumerate}
\setcounter{enumi}{6}
\item If $\mu_1,\ldots, \mu_n$ are measures on $(X,\MC{M})$ and $a_1,\ldots,a_n\in[0,\infty)$, then $\sum_1^na_j\mu_j$ is a measure on $(X,\MC{M})$.

\begin{proof}
Suppose $\mu_1,\ldots, \mu_n$ are measures on $(X,\MC{M})$ and let $a_1,\ldots,a_n\in[0,\infty)$. For any $E\in\MC{M}$, write 

\[
\mu(E)=\sum_{j=1}^na_j\mu_j(E).
\]

Note that $\mu:\MC{M}\rightarrow[0,\infty]$ since each $\mu_j:\MC{M}\rightarrow[0,\infty]$. We proceed to show $\mu$ is a measure by verifying the two properties in the definition of a measure found on page 24 of Folland's book. First, observe that

\[
\mu(\varnothing)=\sum_{j=1}^na_j\mu_j(\varnothing)=\sum_{j=1}^na_j\cdot0=0.
\]

Next, let $\{E_k\}_{k=1}^\infty$ be a sequence of disjoint sets in $\MC{M}$. Then, by the countable additivity of each $\mu_j$,

\begin{align*}
\mu\left(\bigcup_{k=1}^\infty E_k\right)&=\sum_{j=1}^n a_j\mu_j\left(\bigcup_{k=1}^\infty E_k\right)\\[2mm]
&=\sum_{j=1}^n a_j\left(\sum_{k=1}^\infty\mu_j(E_k)\right)\\[2mm]
&=\sum_{j=1}^n\sum_{k=1}^\infty a_j\mu_j(E_k)\\[2mm]
&=\sum_{k=1}^\infty\sum_{j=1}^n a_j\mu_j(E_k)\\[2mm]
&=\sum_{k=1}^\infty\mu(E_k).
\end{align*}

Since these two properties are satisfied by $\mu$, $\mu$ is a measure by definition.
\end{proof}

\item Prove: If $(X,\MC{M},\mu)$ is a measure space and $\{E_j\}_{j=1}^\infty\subset\MC{M}$, then $\mu(\liminf E_j)\leq\liminf \mu(E_j)$. Also, $\mu(\limsup E_j)\geq \limsup \mu(E_j)$ provided that $\mu\left(\bigcup_1^\infty E_j\right)<\infty$.

\begin{proof}
Let $(X,\MC{M},\mu)$ is a measure space and let $\{E_j\}_{j=1}^\infty\subset\MC{M}$. Recall that 

\[
\liminf E_j=\bigcup_{k=1}^\infty\bigcap_{j=k}^\infty E_j \aspace \limsup E_j = \bigcap_{k=1}^\infty\bigcup_{j=k}^\infty E_j
\]

Write $\displaystyle F_k=\bigcap_{j=k}^\infty E_j$. Then $F_{k+1}\supset F_k$ so by continuity from below, we have

\[
\mu(\liminf E_j)=\mu\left(\bigcup_{k=1}^\infty\bigcap_{j=k}^\infty E_j\right)=\mu\left(\bigcup_{k=1}^\infty F_k\right)=\lom{k}{\infty}\mu(F_k).
\]

Now, $F_k\subset E_j$ for all $k\geq j$ so by monotonicity we have $\mu(F_k)\leq\mu(E_j)$ for all $k\geq j$, whence $\mu(F_k)\leq\inf_{k\geq j} \mu(E_j)$ by the definition of infimum. Therefore,

\[
\lom{k}{\infty}\mu(F_k)\leq\lim \inf_{k\geq j}\mu(E_j)=\liminf \mu(E_j).
\]

This result combined with the previous equality above yields

\[
\mu(\liminf E_j)\leq \liminf \mu(E_j).
\]


Next, suppose that  $\displaystyle \mu\left(\bigcup_1^\infty E_j \right)<\infty$. Let $\displaystyle F_k=\bigcup_{j=k}^\infty E_j$. Then $F_{k+1}\subset F_k$ so by continuity from above, we have

\[
\mu(\limsup E_j)=\mu\left(\bigcap_{k=1}^\infty\bigcup_{j=k}^\infty E_j\right)=\mu\left(\bigcap_{k=1}^\infty F_k\right)=\lom{k}{\infty}\mu(F_k).
\]


Now, $F_k\supset E_j$ for all $k\geq j$, whence $\mu(F_k)\geq\sup_{k\geq j}\mu(E_j)$. Therefore,

\[
\lom{k}{\infty}\mu(F_k)\geq\lim \sup_{k\geq j}\mu(E_j)=\limsup\mu(E_j).
\]

This result combined with the previous equality above yields

\[
\mu(\limsup E_j)\geq \limsup\mu(E_j).
\]

\end{proof}

\item Prove: If $(X,\MC{M}, \mu)$ is a measure space and $E,F\in\MC{M}$, then $\mu(E)+\mu(F)=\mu(E\cup F)+\mu(E\cap F)$.

\begin{proof}

Let  $(X,\MC{M}, \mu)$ be a measure space and let $E,F\in\MC{M}$. We can write

\[
E\cup F=(E-F)\cup(F-E)\cup(E\cap F),\text{ whence}
\]

\[
E=(E-F)\cup(E\cap F) \aspace F=(F-E)\cup(E\cap F).
\]

By finite additivity of $\mu$,

\[
\mu(E)=\mu((E-F)\cup(E\cap F))=\mu(E-F)+\mu(E\cap F),\text{ and}
\]
\[
\mu(F)=\mu((F-E)\cup(E\cap F))=\mu(F-E)+\mu(E\cap F).
\]

Adding the corresponding sides of the equations above, we have

\[
\mu(E)+\mu(F)=\mu(E-F)+\mu(E\cap F)+\mu(F-E)+\mu(E\cap F)=\mu(E\cup F)+\mu(E\cap F)
\]
\end{proof}


\item Given a measure space $(X,\MC{M},\mu)$ and $E\in\MC{M}$, define $\mu_E(A)=\mu(A\cap E)$ for $A\in\MC{M}$. Then $\mu_E$ is a measure.

\begin{proof}
Let $(X,\MC{M},\mu)$ be a measure space and let $E\in\MC{M}$. First, 

\[
\mu_E(\varnothing)=\mu(\varnothing\cap E)=\mu(\varnothing)=0.
\]

Next, let $\{A_j\}_{j=1}^\infty$ be a sequence of disjoint sets in $\MC{M}$. Then by the countable additivity of $\mu$,

\[
\mu_E\left(\bigcup_{j=1}^\infty A_j\right)=\mu\left(E\cap\bigcup_{j=1}^\infty A_j\right)=\mu\left(\bigcup_{j=1}^\infty E\cap A_j\right)=\sum_{j=1}^\infty \mu(E\cap A_j)=\sum_{j=1}^\infty\mu_E(A_j).
\]

Since these two properties are satisfied, $\mu_E$ is a measure by definition.
\end{proof}

\setcounter{enumi}{11}

\item Let $(X,\MC{M},\mu)$ be a finite measure space.

\begin{enumerate}
\item If $E,F\in\MC{M}$ and $\mu(E\triangle F)=0$, then $\mu(E)=\mu(F)$. 

\begin{proof}
Let $E,F\in\MC{M}$ and suppose $\mu(E\triangle F)=0$. Recall from a previous problem that

\[
E=(E-F)\cup(E\cap F)\aspace F=(F-E)\cup (E\cap F),
\]

so by finite additivity of $\mu$,

\[
\mu(E)=\mu(E-F)+\mu(E\cap F)\hspace{5mm}(*)\aspace \mu(F)=\mu(F-E)+\mu(E\cap F).
\]

Subtracting the corresponding sides of these equations reveals that

\[
\mu(E)-\mu(F)=\mu(E-F)-\mu(F-E).
\]

Now, suppose by way of contradiction that $\mu(E-F)=k>0$. Then

\[
0=\mu(E\triangle F)=\mu(E-F)+\mu(F-E)=k+\mu(F-E),
\]

which can only be true if $\mu(F-E)<0$, a contradiction. Consequently, $\mu(E-F)=0$. A similar argument reveals that $\mu(F-E)=0$ as well, thus

\[
\mu(E)-\mu(F)=\mu(E-F)-\mu(F-E)=0-0=0,
\]

and therefore $\mu(E)=\mu(F)$.
\end{proof}


\item Say that $E\sim F$ if $\mu(E\triangle F)=0$; then $\sim$ is an equivalence relation on $\MC{M}$.

\begin{proof}
For any $E\in\MC{M}$, $E\sim E$ since

\[
\mu(E\triangle E)=\mu((E-E)\cup(E-E))=\mu(\varnothing\cup\varnothing)=\mu(\varnothing)=0,
\]

so $\sim$ is reflective. Next, let $E,F\in\MC{M}$ and suppose that $E\sim F$. Then $\mu(E\triangle F)=0$ by the definition of $\sim$. Thus,

\[
0=\mu(E\triangle F)=\mu((E-F)\cup(F-E))=\mu((F-E)\cup(E-F))=\mu(F\triangle E),
\]

so $\sim$ is symmetric as well. Next, let $E,F,G\in\MC{M}$. Suppose $E\sim F$ and $F\sim G$. Then $\mu(E\triangle F)=0$ and $\mu(F\triangle G)=0$ by definition of $\sim$. Now, $E\triangle G=(E-G)\cup(G-E)$. Observe that $(E-G)\subset(E\triangle F)\cup(F\triangle G)$, so by monotonicity and subadditivity,

\[
\mu(E-G)\leq\mu((E\triangle F)\cup(F\triangle G))\leq \mu(E\triangle F)+\mu(F\triangle G)=0+0=0.
\]

Consequently, $\mu(E-G)=0$. Next, observe also that $(G-E)\subset(E\triangle F)\cup(F\triangle G)$, so by monotonicity and subadditivity,

\[
\mu(G-E)\leq\mu((E\triangle F)\cup(F\triangle G))\leq \mu(E\triangle F)+\mu(F\triangle G)=0+0=0.
\]

Thus, by finite additivity, $\mu(E\triangle G)=\mu((E-G)\cup(G-E))=\mu(E-G)+\mu(G-E)=0+0=0$. Therefore, $\sim$ is transitive, and so $\sim$ is an equivalence relation since it is reflective, symmetric, and transitive.
\end{proof}

\item For $E,F\in\MC{M}$, define $\rho(E,F)=\mu(E\triangle F)$. Then $\rho(E,G)\leq\rho(E,F)+\rho(F,G)$, and hence $\rho$ defines a metric on the space $\MC{M}/\sim$ of equivalence classes.

\begin{proof}
Let $E,F,G\in\MC{M}$, and define $\rho(E,F)=\mu(E\triangle F)$. Note that the first two conditions of a metric were shown above, so it suffices to show $\rho(E,G)\leq\rho(E,F)+\rho(F,G)$ to establish that $\rho$ is a metric. Using an argument similar to that used in the proof of 12(c), we can see that $(E\triangle G)\subset((E\triangle F)\cup(F\triangle G))$, so by monotonicity and subadditivity, $\mu(E\triangle G)\leq \mu(E\triangle F)+\mu(F\triangle G)$.
\end{proof}


\end{enumerate}

\end{enumerate}






\end{document}