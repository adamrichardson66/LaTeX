\documentclass[11pt,oneside,english]{amsart}
\usepackage[T1]{fontenc}
\usepackage{geometry}
\usepackage{parskip}
\geometry{verbose,tmargin=0.65in,bmargin=0.65in,lmargin=0.75in,rmargin=0.75in,headheight=0.75cm,headsep=1cm,footskip=1cm}
\setlength{\parskip}{7mm}
\usepackage{setspace}
\onehalfspacing
\pagenumbering{gobble}



\usepackage{bbm}
\usepackage{multicol}
\usepackage{graphicx}
\usepackage{adjustbox}
\usepackage{amssymb}
\usepackage{tikz}
\usetikzlibrary{cd}
\usepackage{pgfplots}
\usepackage{ulem}
\usepackage{adjustbox}
\usepackage{bm}
\usepackage{stmaryrd}
\usepackage{cancel}
\usepackage{mathtools}
\DeclarePairedDelimiter{\ceil}{\lceil}{\rceil}
\DeclarePairedDelimiter\floor{\lfloor}{\rfloor}
\usepackage{enumitem}
\setlist[enumerate,1]{label=\textbf{\arabic*.}}
\usepackage{color, colortbl}
\definecolor{Gray}{gray}{0.9}
\usepackage{babel}
\usepackage{mdframed}
\usepackage{esint}
\usepackage[yyyymmdd]{datetime}
\renewcommand{\dateseparator}{--}

\theoremstyle{definition}
\newtheorem{theorem}{Theorem}
\newtheorem*{theorem*}{Theorem}
\newtheorem*{proposition*}{Proposition}
\newtheorem{corollary}{Corollary}
\newtheorem*{example}{Example}
\newtheorem*{examples}{Examples}
\newtheorem*{definition}{Definition}
\newtheorem*{note}{Nota Bene}

\newcommand{\aspace}{\hspace{7mm}\text{and}\hspace{7mm}}
\newcommand{\ospace}{\hspace{7mm}\text{or}\hspace{7mm}}
\newcommand{\pspace}{\hspace{10mm}}
\newcommand{\lhe}{\stackrel{\text{L'H}}{=}}
\newcommand{\lom}[2]{\lim_{{#1}\rightarrow{#2}}}
\newcommand{\R}{\mathbb{R}}
\newcommand{\ve}{\varepsilon}
\newcommand{\dd}[2]{\frac{d{#1}}{d{#2}}}
\newcommand{\pp}[2]{\frac{\partial{#1}}{\partial{#2}}}
\newcommand{\DD}[2]{\frac{\Delta{#1}}{\Delta{#2}}}
\newcommand{\ovec}[1]{\overrightarrow{#1}}
\newcommand{\MC}[1]{\mathcal{#1}}
\usepackage{bbm}


\def\<#1>{\mathinner{\langle#1\rangle}}

\makeatletter
\g@addto@macro\normalsize{%
  \setlength\belowdisplayshortskip{5mm}
}
\makeatother




\begin{document}

\rightline{Adam D. Richardson}
\rightline{209A - Real Analysis}
\rightline{Zhang, Qi}
\rightline{HW 5}
\rightline{\today}

\vspace{-5mm}
\textbf{Folland: Exercises, p. 63.} 32, 33, 37, 40, 44



\vspace{5mm}
\begin{enumerate}
\setcounter{enumi}{31}



\item Suppose $\mu(X)<\infty$. If $f$ and $g$ are complex-valued measurable functions on $X$, define

\[
\rho(f,g)=\int\frac{|f-g|}{1+|f-g|}\,d\mu.
\]

Then $\rho$ is a metric on the space of measurable functions if we identify functions that are equal a.e., and $f_n\rightarrow f$ with respect to this metric iff $f_n\rightarrow f$ in measure.

\begin{proof}
First, we show $\rho(f,g)=0$ iff $f=g$. If $\rho(f,g)=0$, then by the definition of $\rho$ and Proposition 2.16, $|f-g|=0$ a.e. so $f=g$ a.e. Conversely, if $f=g$ a.e., then 

\[
\rho(f,g)=\rho(f,f)=\int\frac{|f-f|}{1+|f-f|}\,d\mu=0. 
\]

Second, 

\[
\rho(f,g)=\int\frac{|f-g|}{1+|f-g|}\,d\mu=\int\frac{|g-f|}{1+|g-f|}\,d\mu=\rho(g,f).
\]

Third, $\rho(f,g)\geq0$ since $|f-g|\geq0$ for any $f,g$. And lastly we verify the triangle inequality. Observe that

\begin{align*}
\frac{|f-h|}{1+|f-h|}&=1-\frac{1}{1+|f-h|}\\[2mm]
&=1-\frac{1}{1+|f-g+g-h|}\\[2mm]
&\leq1-\frac{1}{1+|f-g|+|g-h|}\\[2mm]
&=\frac{1+|f-g|+|g-h|-1}{1+|f-g|+|g-h|}\\[2mm]
&=\frac{|f-g|}{1+|f-g|+|g-h|}+\frac{|g-h|}{1+|f-g|+|g-h|}\\[2mm]
&\leq\frac{|f-g|}{1+|f-g|}+\frac{|g-h|}{1+|g-h|}.\\[2mm]
\end{align*}

Consequently,

\[
\rho(f,h)=\int\frac{|f-h|}{1+|f-h|}\,d\mu\leq\int\frac{|f-g|}{1+|f-g|}\,d\mu +\int\frac{|g-h|}{1+|g-h|}\,d\mu=\rho(f,g)+\rho(g,h).
\]

Now we proceed to show that $f_n\rightarrow f$ in this metric iff $f_n\rightarrow f$ in measure. First, suppose that $f_n\rightarrow f$ in this metric. Then $\rho(f_n,f)\rightarrow 0$, so by definition for all $\ve>0$ there exists an $N$ such that for a.e. $x\in X$, when $n\geq N$, $\rho(f_n,f)<\ve$. Let $E_n=\{x\mid \rho(f_n(x),f(x))<\ve\}$. Then $E_n^c=\{x\mid \rho(f_n(x),f(x))\geq\ve\}$. Since $\mu(X)<\infty$, we can write $\mu(E_n^C)=\mu(X)-\mu(E_n)$ and since $\rho(f_n,f)\rightarrow 0$, we have $E_{n+1}\supset E_n$ and $X=\bigcup_{n=1}^\infty E_n$. By continuity from below, $\mu(E_n)\rightarrow \mu(X)$ snd so $\mu(E_n^c)=\mu(\{x\mid \rho(f_n(x),f(x))\geq\ve\})\rightarrow 0$, i.e. $f_n\rightarrow f$ in measure.

Next, suppose $f_n\rightarrow f$ in measure. Then $\mu(\{x\mid \rho(f_n(x),f(x))\geq \ve\})\rightarrow 0$. We can write

\[
X=\{x\mid \rho(f_n(x),f(x))\geq \ve\}\cup \{x\mid \rho(f_n(x),f(x))< \ve\}.
\]

We have

\[
\mu(X)=\mu(\{x\mid \rho(f_n(x),f(x))\geq \ve\})+ \mu(\{x\mid \rho(f_n(x),f(x))< \ve\}),
\]

and since $\mu(X)<\infty$, $\mu(\{x\mid \rho(f_n(x),f(x))\geq \ve\})\rightarrow 0$ implies that $\mu(\{x\mid \rho(f_n(x),f(x))< \ve\})\rightarrow \mu(X)$. But this implies that 

\[
X=\bigcup_{n=1}^\infty \{x\mid \rho(f_n(x),f(x))< \ve\},
\]

so for a.e. $x\in X$, $\rho(f_n(x),f(x))\rightarrow 0$, i.e. $f_n\rightarrow f$ in the metric $\rho$.
\end{proof}


\item If $f_n\geq 0$ and $f_n\rightarrow f$ in measure, then $\int f\leq \liminf \int f_n$.

\begin{proof}
Suppose $f_n\geq0$ and $f_n\rightarrow f$ in measure. By Proposition 2.30 (p.61) there exists a subsequence $\{f_{n_k}\}$ which converges to $f$ a.e. Consequently, $\liminf f_n(x)\leq f(x)$ a.e. Now we proceed to show that this inequality must actually be an equality. Suppose by way of contradiction that there exists a set $E$ of positive measure such that $\liminf f(x)<f(x)$ for all $x\in E$. Let $E_k=\{x\mid \liminf f_n \leq f-\frac{1}{k}\}$. Then we may write $E=\bigcup_{k=1}^\infty E_k$. Since $\mu(E)>0$, there must exist some $\bar{k}$ such that $\mu(E_{\bar{k}})=\delta>0$. However, $f_n\rightarrow f$ in measure by hypothesis, i.e. for any $\delta>0$ there exists an $N$ such that if $n\geq N$, then $\mu(\{x\mid |f_n(x)-f(x)|\geq\frac{1}{k}\})<\delta$, which is a contradiction. Thus, $\liminf f_n=f$ a.e., and by Fatou's lemma we have

\[
\int f=\int\liminf f_n\leq\liminf \int f_n.
\]
\end{proof}


\setcounter{enumi}{36}

\item Suppose $f_n$ and $f$ are measurable complex-valued functions and $\phi:\mathbb{C}\rightarrow\mathbb{C}$.

\begin{enumerate}
\item If $\phi$ is continuous and $f_n\rightarrow f$ a.e. then $\phi\circ f_n\rightarrow \phi \circ f$ a.e.
\end{enumerate}


\setcounter{enumi}{39}

\item In Egoroff's theorem, the hypothesis ``$\mu(X)<\infty$'' can be replaced by ``$|f_n|\leq g$ for all $n$, where $g\in L^1(\mu)$.''

\begin{proof}
The condition that $\mu(X)<\infty$ in Egoroff's theorem is included so that we may invoke continuity from above and have that, in the language of Folland's proof,  $\mu(E_n(k))\rightarrow 0$. Instead suppose that $|f_n|\leq g$ for all $n$, where $g\in L^1(\mu)$. Then by the DCT, $f_n\rightarrow f$ in $L^1(\mu)$, i.e. $\int |f_n-f|\rightarrow 0$. First we prove a famous Lemma for $p=1$:

\textbf{Lemma.} (Chebyshev's Inequality) Let $f$ be a measurable function defined on $X$. Then for any real number $\ve>0$,

\[
\mu (\{x\in X\mid |f(x)|\geq \ve\})\leq \frac{1}{\ve^p}\int _{|f|\geq \ve}|f|^p\,d\mu.
\]

\begin{proof}
Observe that

\begin{align*}
\mu (\{x\in X\mid |f(x)|\geq \ve\})&=\int_X\chi_{\{x\in X\mid |f(x)|\geq \ve\}}\,d\mu\\[2mm]
&=\int_{\{x\in X\mid |f(x)|\geq \ve\}} 1\,d\mu\\[2mm]
&=\int_{\{x\in X\mid |f(x)|\geq \ve\}}\frac{\ve}{\ve}\,d\mu\\[2mm]
&=\frac{1}{\ve}\int_{\{x\in X\mid |f(x)|\geq \ve\}}\ve\,d\mu\\[2mm]
&\leq\frac{1}{\ve}\int_{\{x\in X\mid |f(x)|\geq \ve\}}|f(x)|\,d\mu\\[2mm]
&=\frac{1}{\ve}\int_{|f|\geq \ve}|f|\,d\mu.\\[2mm]
\end{align*}
\end{proof}

Returning to our original proof, by hypothesis, $\int|f_n-f|\rightarrow 0$, and by Chebyshev's inequality,

\[
\mu(E_n(k))=\mu\left(\left\{x\,\Big|\, |f_n(x)-f(x)|\geq\frac{1}{k}\right\}\right)=\leq k\int_{|f_n-f|\geq\frac{1}{k}}|f_n-f|\,d\mu\leq k\int_X|f_n-f|\,d\mu\rightarrow 0.
\]

Consequently, $\mu(E_n(k))\rightarrow 0$ as required, and the rest of the proof of Egoroff's theorem holds.
\end{proof}

\setcounter{enumi}{43}

\item (\textbf{Lusin's Theorem}) If $f:[a,b]\rightarrow \mathbb{C}$ is Lebesgue measurable and $\ve>0$, there is a compact set $E\subset[a,b]$ such that $\mu(E^c)<\ve$ and $f\mid_E$ is continuous.

\begin{proof}
Suppose $f:[a,b]\rightarrow \mathbb{C}$ is Lebesgue measurable. By theorem 2.26, there exists a sequence of (integrable) simple functions $\{\phi_k\}$ such that $\phi_k\rightarrow f$ a.e. $x\in [a,b]$ and for sufficiently large $k$, we have $\int|f-\phi_k|<\frac{1}{k}$. By Egoroff's theorem, there exists a set $E\subset[a,b]$ such that $\mu(E)<\frac{1}{k}$ and $\phi_k\rightarrow f$ uniformly on $E^c$. By a result from advanced calculus, it follows that $f\mid_{E^c}$ is continuous. $E^c$ is certainly bounded so it suffices to show that it is closed.
\end{proof}



\end{enumerate}


\end{document}