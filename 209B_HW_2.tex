\documentclass[11pt,oneside,english]{amsart}
\usepackage[T1]{fontenc}
\usepackage{geometry}
\usepackage{parskip}
\geometry{verbose,tmargin=0.65in,bmargin=0.65in,lmargin=0.75in,rmargin=0.75in,headheight=0.75cm,headsep=1cm,footskip=1cm}
\setlength{\parskip}{7mm}
\usepackage{setspace}
\onehalfspacing
\pagenumbering{gobble}



\usepackage{bbm}
\usepackage{multicol}
\usepackage{graphicx}
\usepackage{adjustbox}
\usepackage{amssymb}
\usepackage{tikz}
\usepackage{pgfplots}
\usepackage{pgffor}
\usetikzlibrary{cd}
\usepackage{ulem}
\usepackage{adjustbox}
\usepackage{bm}
\usepackage{stmaryrd}
\usepackage{cancel}
\usepackage{mathtools}
\DeclarePairedDelimiter{\ceil}{\lceil}{\rceil}
\DeclarePairedDelimiter\floor{\lfloor}{\rfloor}
\usepackage{enumitem}
\setlist[enumerate,1]{label=\textbf{\arabic*.}}
\usepackage{color, colortbl}
\definecolor{Gray}{gray}{0.9}
\usepackage{babel}
\usepackage{mdframed}
\usepackage{esint}
\usepackage[yyyymmdd]{datetime}
\renewcommand{\dateseparator}{--}
\usepackage{url}
\usepackage[unicode=true,pdfusetitle,
 bookmarks=true,bookmarksnumbered=false,bookmarksopen=false,
 breaklinks=false,pdfborder={0 0 1},backref=false,colorlinks=true]
 {hyperref}
\hypersetup{urlcolor=blue}

\theoremstyle{definition}
\newtheorem{theorem}{Theorem}
\newtheorem*{theorem*}{Theorem}
\newtheorem*{proposition*}{Proposition}
\newtheorem{corollary}{Corollary}
\newtheorem*{example}{Example}
\newtheorem*{examples}{Examples}
\newtheorem*{definition}{Definition}
\newtheorem*{note}{Nota Bene}

\newcommand{\aspace}{\hspace{7mm}\text{and}\hspace{7mm}}
\newcommand{\ospace}{\hspace{7mm}\text{or}\hspace{7mm}}
\newcommand{\pspace}{\hspace{10mm}}
\newcommand{\lhe}{\stackrel{\text{L'H}}{=}}
\newcommand{\lom}[2]{\lim_{{#1}\rightarrow{#2}}}
\newcommand{\R}{\mathbb{R}}
\newcommand{\ve}{\varepsilon}
\newcommand{\dd}[2]{\frac{d{#1}}{d{#2}}}
\newcommand{\pp}[2]{\frac{\partial{#1}}{\partial{#2}}}
\newcommand{\DD}[2]{\frac{\Delta{#1}}{\Delta{#2}}}
\newcommand{\ovec}[1]{\overrightarrow{#1}}
\newcommand{\MC}[1]{\mathcal{#1}}
\usepackage{bbm}


\def\<#1>{\mathinner{\langle#1\rangle}}

\makeatletter
\g@addto@macro\normalsize{%
  \setlength\belowdisplayshortskip{5mm}
}
\makeatother




\begin{document}

\rightline{Adam D. Richardson}
\rightline{209B - Functional Analysis}
\rightline{Baez, John}
\rightline{HW 2}
\rightline{\today}



\vspace{5mm}
\begin{enumerate}


\item Prove \textbf{Lemma 3.7}: Suppose that $\mu$ and $\nu$ are finite measures on a measurable space $(X,\MC{M})$.  Then either $\nu \perp \mu$ or there exists $\ve > 0$ and a measurable set $E \subseteq X$ such that $\mu(E) > 0$ and $\nu \geq \ve \mu$ on $E$ (that is, every measurable set $S \subseteq E$ has $\nu(S) \geq \ve \mu(S)$).

\begin{proof}
Let $\mu$ and $\nu$ be finite measures on a measurable space $(X,\MC{M})$. Consider the measure $\nu-\frac{1}{n}\mu$, and let $X=P_n\cup N_n$ be a Hahn decomposition associated with $\nu-\frac{1}{n}\mu$. Let

\[
P=\bigcup_{n=1}^\infty P_n\aspace N=\bigcap_{n=1}^\infty N_n=X\setminus P.
\]

Since $N_n$ is a $(\nu-\frac{1}{n}\mu)$-negative set for each $n$, $N$ is a $(\nu-\frac{1}{n}\mu)$-negative set as well since it is contained in each $N_n$. In particular,

\begin{align*}
\nu(N)-\frac{1}{n}\mu(N)\leq 0\\[2mm]
0\leq\nu(N)\leq\frac{1}{n}\mu(N)
\end{align*}

since $\nu:\MC{M}\rightarrow[0,\infty)$. Since this is true for all $n$, we have that $\nu(N)=0$. Now, if $\mu(P)=0$, then $\nu\perp\mu$ by definition. If $\mu(P)>0$, then it must be that $\mu(P_{\bar{n}})>0$ for some ${\bar{n}}$ so $P_{\bar{n}}$ is $(\nu-\frac{1}{{\bar{n}}}\mu)$-positive. In other words,

\begin{align*}
\nu(P_{\bar{n}})-\frac{1}{{\bar{n}}}\mu(P_{\bar{n}})&\geq0\\[2mm]
\nu(P_{\bar{n}})&\geq\frac{1}{{\bar{n}}}\mu(P_{\bar{n}}).
\end{align*}

Let $E=P_{\bar{n}}$ and $\ve=\frac{1}{{\bar{n}}}$ and we have $\nu\geq\ve\mu$ on $E$ as required.
\end{proof}

\pagebreak

\item Now, suppose that $\mu$ and $\nu$ are finite measures on a measurable space $(X,\MC{M})$. Let 

\[
\MC{F}=\left\{f:X\rightarrow[0,\infty] \,\Big|\,f\text{ is measurable and }\int_Ef\,d\mu\leq \nu(E)\text{ for all } E\in\MC{M}\right\},
\]

and let $\displaystyle a=\sup\left\{\int f\,d\mu\mid f\in \MC{F}\right\}$. Note that since $a$ is the supremum of a possibly unbounded set of nonnegative numbers, we start out only knowing that $a\in[0,\infty]$. Show that $a<\infty$.

\begin{proof}
First, note that $\MC{F}$ is nonempty since $f\equiv 0\in\MC{F}$, so it makes sense to seek a supremum. Recall that $\nu$ is finite, i.e. $\nu(X)<\infty$. Since $X\in\MC{M}$, by the definition of $\MC{F}$, we have that $a\leq\nu(X)<\infty$
\end{proof}

\item Show that there is a sequence $f_n\in\MC{F}$ such that $\int f_n\,d\mu\rightarrow a$.

\begin{proof}
By the characterization of supremum, given any $n>0$, there exists an integral $\int f_n\,d\mu\in\left\{\int f\,d\mu\mid f\in \MC{F}\right\}$ such that

\begin{align*}
a-\frac{1}{n}&<\int f_n\,d\mu\\[2mm]
a-\int f_n\,d\mu&<\frac{1}{n}\\[2mm]
\left|a-\int f_n\,d\mu\right|&<\frac{1}{n}.\\[2mm]
\end{align*}

Since this is true for all $n$, we have $\int f_n\,\mu\rightarrow a$, and $f_n\in\MC{F}$ for each $n$, so $\{f_n\}\subseteq\MC{F}$.
\end{proof}

\item Choose such a sequence $f_n$ and let $g_n=\max(f_1,\ldots,f_n)$. Show that $g_n\in\MC{F}$%, and $\int g_n\,d\mu\rightarrow a$.

\begin{proof}
Define $h=\max_{f,g\in\MC{F}}(f,g)$ and $A=\{x\mid f(x)>g(x)\}$. Then for any $E\in\MC{M}$, we can decompose $E$ as follows

\[
\int_E h\,d\mu=\int_{E\cap A}f\,d\mu+\int_{E\setminus A}g\,d\mu\leq\nu(E\cap A)+\nu(E\setminus A)=\nu(E).
\]

Thus, $\max(f,g)\in\MC{F}$ by definition, whence it follows that $g_n=\max(f_1,\ldots,f_n)\in\MC{F}$ where $f_1,\ldots,f_n\in\MC{F}$.
\end{proof}

\item[\textbf{5, 6.}] Let $f=\sup_n f_n$. Show that $g_n\rightarrow f$ pointwise and $f:X\rightarrow [0,\infty]$ is measurable. Also show that $g_{n+1}\geq g_n$.

\begin{proof}
First, $f$ is measurable by Proposition 2.7 since $f_n$ is measurable for each $n$. Next, since $\{f_1,\ldots,f_n\}\subseteq\{f_1,\ldots,f_n,f_{n+1}\}$, it follows that $g_{n+1}=\max(f_1,\ldots,f_n,f_{n+1})\geq \max(f_1,\ldots,f_n)=g_n$ so $g_n $ is an increasing sequence. Additionally, $\max_{1\leq i\leq n}(f_1(x),\ldots,f_n(x))\leq \sup_nf_n(x)$, so $g_n$ is increasing and bounded above. Consequently, $g_n(x)\rightarrow f(x)$ pointwise. 

As a result, the Monotone Convergence Theorem yields

\[
\int f\,d\mu=\int\left(\lom{n}{\infty} g_n\right)\,d\mu=\lom{n}{\infty}\int g_n\,d\mu=a<\infty.
\]

By Proposition 2.20 on page 52, $f<\infty$ a.e.

\end{proof}

\setcounter{enumi}{6}
\item Use the Monotone Convergence Theorem to show $f\in\MC{F}$.

\begin{proof}
In order to show $f\in\MC{F}$, it suffices to show that $\int_Ef\,d\mu\leq\nu(E)$ for all $E\in\MC{M}$. Let $E\in\MC{M}$. Then since $g_n$ is measurable, $g_n\leq g_{n+1}$, and $g_n\in\MC{F}$, by the Monotone Convergence Theorem, we have

\[
\int_Ef\,d\mu=\int_E\lom{n}{\infty}g_n\,d\mu=\lom{n}{\infty}\int_Eg_n\,d\mu\leq\lom{n}{\infty}\nu(E)=\nu(E).
\]

Therefore, $f\in\MC{F}$ by definition.
\end{proof}


\item Let $\lambda$ be the signed measure $\lambda=\nu-f\mu$. Using Problem 7, show that $\lambda(E)\geq0$ for every measurable set $E\subseteq X$, i.e. $\lambda$ is positive.

\begin{proof}
Let $E\in\MC{M}$. Since $f\in\MC{F}$, we have

\vspace{-5mm}
\begin{align*}
\lambda(E)&=\nu(E)-(f\mu)(E)\\[2mm]
&\geq\int_Ef\,d\mu-\int_Ef\,d\mu\\[2mm]
&=\int_Ef-f\,d\mu\\[2mm]
&=\int_E0\,d\mu\\[2mm]
&=0.
\end{align*}
\end{proof}

\item Show that $\lambda\perp\mu$.

\begin{proof}
Suppose $\lambda\not\perp\mu$. Then by Problem 1 (Lemma 3.7), there exists an $E\in\MC{M}$ and an $\ve>0$ such that $\mu(E)>0$ and $\lambda\geq\ve\mu$ on $E$. Since $\lambda$ is positive, we have

\begin{align*}
d\lambda&\geq\ve \,d\mu\\[2mm]
d\nu-f\,d\mu&\geq \ve\,d\mu\\[2mm]
d\nu&\geq f\,d\mu+\ve\,d\mu\\[2mm]
\int_E d\nu&\geq \int_E f\,d\mu+\int_E\ve\,d\mu\\[2mm]
\nu(E)&\geq \int_Ef+\ve\,d\mu.\\[2mm]
\end{align*}

\vspace{-5mm}
Thus, $(f+\ve)\in\MC{F}$ and we have
\[
\int_E f+\ve\,d\mu=a+\ve\mu(E)> a,
\]

a contradiction. Therefore, $\lambda\perp\mu$.
\end{proof}

\pagebreak

\item Show that $\rho=f\mu$ is a finite measure with $\rho\ll\mu$.

\begin{proof}
First, since $\mu$ is a finite measure by hypothesis, and $f<\infty$ a.e. by Problems 5, 6, $\rho=f\mu<\infty$. Next, let $E\in\MC{M}$ and suppose $\mu(E)=0$. Then $\rho(E)=f\mu(E)=f\cdot0=0$ as well since $f<\infty$ a.e. Since $E$ was chosen arbitrarily, $\rho\ll\mu$. 
\end{proof}

\item It is thus shown that $\nu=\lambda+\rho$, $\lambda\perp\mu$, and $\rho\ll\mu$. Suppose we also have that $\nu=\lambda'+\rho'$, $\lambda'\perp\mu$, and $\rho'\ll\mu$. Show that $\lambda=\lambda'$ and conclude also that $\rho=\rho'$.

\begin{proof}
First we need two lemmas.

\textbf{Lemma 1.} (Exercise 3.2.9) Suppose $\{\nu_j\}$ is a sequence of positive measures. If $\nu_j\perp\mu$ for all $j$, then $\nu=\sum_{j=1}^\infty\nu_j\perp\mu$; and if $\nu_j\ll\mu$ for all $j$, then $\nu=\sum_{j=1}^\infty\ll\mu$.

\begin{proof}
Let $\{\nu_j\}$ be a sequence of positive measures and suppose first that $\nu_j\perp\mu$ for all $j$. Then by definition, for each $j$,  there exists an $E_j\in\MC{M}$ such that $\nu_j(E_j)=0$ and  $\mu(E_j^c)=0$ as well. Define $E=\bigcup_{j=1}^\infty E_j$ so that $E\cup E^c=X$ and $E\cap E^c=\varnothing$. First, 

\[
\nu_j(E)=\nu_j\left(\bigcup_{j=1}^\infty E_j\right)\leq\sum_{j=1}^\infty \nu_j(E_j)=\sum_{j=1}^\infty 0=0,
\]

so $\displaystyle \nu(E)=\sum_{j=1}^\infty\nu_j(E)=0$. Next, since $\mu$ is finite,

\[
\mu(E^c)=\mu\left(\left(\bigcup_{j=1}^\infty E_j\right)^c\right)=\mu\left(\bigcap_{j=1}^\infty E_j^c\right)\leq\mu(E_{\bar{j}}^c)=0,
\]

for some $\bar{j}$, so $\nu\perp\mu$. 

Next, suppose that $\nu_j\ll\mu$ for all $j$. Then $\nu_j(E)=0$ iff $\mu(E)=0$ for all $j$ and for all $E\in\MC{M}$. Let $E\in\MC{M}$ and suppose that $\mu(E)=0$. Then $\nu_j(E)=0$ and

\[
\nu(E)=\sum_{j=1}^\infty\nu_j(E)=\sum_{j=1}^\infty 0=0,\text{ so }\nu\ll\mu.
\]
\end{proof}

\textbf{Lemma 2.} Let $\mu$ and $\nu$ be positive measures and let $(X,\MC{M})$ be a measurable space. If $\nu\perp\mu$ and $\nu\ll\mu$, then $\nu\equiv0\equiv\mu$.

\begin{proof}	
Suppose $\nu\perp\mu$ and $\nu\ll\mu$. Then there exists an $E\in\MC{M}$ such that $\nu(E)=0$ and $\mu(E^c)=0$. Additionally, for any $E\in\MC{M}$, $\nu(E)=0$ iff $\mu(E)=0$. Since $\nu(E)=0$, it follows that $\mu(E)=0$ and $\mu(E^c)=0$ so that $\mu(X)=0$. Consequently, $\mu\equiv0$ by monotonicity. Since $\nu\ll\mu$, $\nu\equiv0$ as well since they must agree on all null sets.
\end{proof}

Now, we have
\begin{align*}
\lambda+\rho&=\lambda'+\rho'\\[2mm]
\lambda +f\mu&=\lambda'+f'\mu\\[2mm]
(f-f')\mu&=\lambda'-\lambda.
\end{align*}

By Lemma 1 above, and since $\lambda\perp\mu$ and $\lambda'\perp\mu$, we have $(\lambda'-\lambda)\perp\mu$. By the same lemma, we have $(f-f')\mu\ll\mu$. Since $(\lambda'-\lambda)\perp\mu$, $(f-f')\mu\ll\mu$, and $(f-f')\mu=\lambda'-\lambda$, by Lemma 2 $(f-f')\mu=\lambda'-\lambda=0$. Consequently, $\lambda=\lambda'$, and by Proposition 2.23(b), $0=(f-f')\mu=\int_Ef-f'\,d\mu$ iff $f=f'$ $\mu$-a.e. Thus $\rho=f\mu=f'\mu=\rho'$.
\end{proof}

\item Suppose that we also have $\rho=g\mu$ for some other measurable function $g:X\rightarrow[0,\infty]$. Show that $f=g$ $\mu$-a.e.

\begin{proof}
We already have that $\rho=f\mu$, and since $f\in\MC{F}$, $f$ is $\mu$-integrable and $f:X\rightarrow[0,\infty]$. Now suppose there is another $\mu$-integrable function $g:X\rightarrow[0,\infty]$ where $\rho=g\mu$. Then $d\rho=f\,d\mu=g\,d\mu$ so by definition, for all $E\in\MC{M}$, $\int_Ef\,d\mu=\int_Eg\,d\mu$. By Proposition 2.23(b) it follows that $f=g$ $\mu$-a.e.
\end{proof}

\item[\textbf{EC.}] \href{http://mathshistory.st-andrews.ac.uk/Biographies/Nikodym.html}{Otton Marcin Nikodym} was born on 1887-08-13 in Zablotow, Galicia, Austria-Hungary (Ukraine) and died on 1974-05-04 in Utica, New York. His wife's name was Stanis\l{}awa.
\end{enumerate}



\end{document}