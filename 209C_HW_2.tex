\documentclass[11pt,oneside,english]{amsart}
\usepackage[T1]{fontenc}
\usepackage{geometry}
\usepackage{parskip}
\geometry{verbose,tmargin=0.65in,bmargin=0.65in,lmargin=0.75in,rmargin=0.75in,headheight=0.75cm,headsep=1cm,footskip=1cm}
\setlength{\parskip}{7mm}
\usepackage{setspace}
\onehalfspacing
\pagenumbering{gobble}

\usepackage{bbm}
\usepackage{multicol}
\usepackage{graphicx}
\usepackage{adjustbox}
\usepackage{amssymb}
\usepackage{tikz}
\usepackage{pgfplots}
\usepackage{pgffor}
\usetikzlibrary{cd}
\usepackage{ulem}
\usepackage{adjustbox}
\usepackage{bm}
\usepackage{stmaryrd}
\usepackage{cancel}
\usepackage{mathtools}
\DeclarePairedDelimiter{\ceil}{\lceil}{\rceil}
\DeclarePairedDelimiter\floor{\lfloor}{\rfloor}
\usepackage[shortlabels]{enumitem}
\setlist[enumerate,1]{label=\textbf{\arabic*.}}
\usepackage{color, colortbl}
\definecolor{Gray}{gray}{0.9}
\usepackage{babel}
\usepackage{mdframed}
\usepackage{esint}
\usepackage[yyyymmdd]{datetime}
\renewcommand{\dateseparator}{--}
\usepackage{url}
\usepackage[unicode=true,pdfusetitle,
 bookmarks=true,bookmarksnumbered=false,bookmarksopen=false,
 breaklinks=false,pdfborder={0 0 1},backref=false,colorlinks=true]
 {hyperref}
\hypersetup{urlcolor=blue}


\theoremstyle{definition}
\newtheorem{theorem}{Theorem}
\newtheorem*{theorem*}{Theorem}
\newtheorem*{proposition*}{Proposition}
\newtheorem{corollary}{Corollary}
\newtheorem*{lemma}{Lemma}
\newtheorem*{example}{Example}
\newtheorem*{examples}{Examples}
\newtheorem*{definition}{Definition}
\newtheorem*{note}{Nota Bene}

\newcommand{\aspace}{\hspace{7mm}\text{and}\hspace{7mm}}
\newcommand{\ospace}{\hspace{7mm}\text{or}\hspace{7mm}}
\newcommand{\pspace}{\hspace{10mm}}
\newcommand{\lhe}{\stackrel{\text{L'H}}{=}}
\newcommand{\lom}[2]{\lim_{{#1}\rightarrow{#2}}}
\newcommand{\ve}{\varepsilon}
\newcommand{\dd}[2]{\frac{d{#1}}{d{#2}}}
\newcommand{\pp}[2]{\frac{\partial{#1}}{\partial{#2}}}
\newcommand{\DD}[2]{\frac{\Delta{#1}}{\Delta{#2}}}
\newcommand{\ovec}[1]{\overrightarrow{#1}}
\newcommand{\MC}[1]{\mathcal{#1}}
\newcommand{\MB}[1]{\mathbb{#1}}
\renewcommand{\vec}[1]{\underline{#1}}



\def\<#1>{\mathinner{\langle#1\rangle}}

\makeatletter
\g@addto@macro\normalsize{%
  \setlength\belowdisplayshortskip{5mm}
}
\makeatother




\begin{document}

\rightline{Adam D. Richardson}
\rightline{209C - Real Analysis}
\rightline{Zhang, Zhenghe}
\rightline{HW 2}
\rightline{\today}



\vspace{5mm}
\begin{enumerate}
\itemsep7mm



\item  Let $\Lambda: X\to Y$ be a linear transformation between two normed vector spaces $X$ and $Y$ and let $\|\Lambda\|$ be the operator norm of $\Lambda$. Show that 
\begin{enumerate}
\item $\|\Lambda(x)\|_Y\le \|\Lambda\|\cdot\|x\|_X$ for all $x\in X$. 

\begin{proof}
If $x=0$, then the inequality holds trivially since $\|\cdot\|$ is a norm. Recall that $\|\Lambda\|=\sup_{x\neq 0}\frac{\|\Lambda x\|_Y}{\|x\|_X}$. Consequently, for any nonzero $x\in X$,
\[
\|\Lambda\|\geq\frac{\|\Lambda (x)\|_Y}{\|x\|_X}\pspace\text{whence}\pspace\|\Lambda\|\cdot\|x\|\geq \|\Lambda(x)\|
\]
\end{proof}

\item Suppose there is a $M>0$ such that $\|\Lambda(x)\|\le M\|x\|,\ \forall x\in X.$ Show that $\Lambda$ is bounded and $\|\Lambda\|\le M$.

\begin{proof}
Suppose there exists an $M>0$ such that $\|\Lambda(x)\|\leq M\|x\|$ for all $x\in X$. Then for nonzero $x$, $M$ is an upper bound of $\frac{\|\Lambda(x)\|}{\|x\|}$, and since $\|\Lambda\|=\sup_{x\neq0}\frac{\|\Lambda(x)\|}{\|x\|}$, $\Lambda$ is bounded and it follows that $\|\Lambda\|\leq M$.
\end{proof}
\end{enumerate}

Note: This problem tells us that $\|\Lambda\|=\inf\{M\}$ where $M$ is as in part (b).

\pagebreak

\item \begin{enumerate}\itemsep7mm \item Show that $f:(a,b)\to\MB{R}$ is convex function if and only if 
\[
\frac{f(y)-f(x)}{y-x}\le \frac{f(z)-f(y)}{z-y},\ \forall a<x<y<z<b.
\]

\begin{proof}
Let $a<x<y<z<b$. Then we can write $y=tx+(1-t)z$ for some $t\in(0,1)$. Thus,
\begin{align*}
\frac{f(y)-f(x)}{y-x}&\leq\frac{f(z)-f(y)}{z-y}\\[2mm]
\frac{f(tx+(1-t)z)-f(x)}{tx+(1-t)z-x}&\leq\frac{f(z)-f(tx+(1-t)z)}{z-[tx+(1-t)z]}\\[2mm]
\frac{f(tx+(1-t)z)-f(x)}{(1-t)\cancel{(z-x)}}&\leq\frac{f(z)-f(tx+(1-t)z)}{t\cancel{(z-x)}}\\[2mm]
tf(tx+(1-t)z)-tf(x)&\leq(1-t)f(z)-(1-t)f(tx+(1-t)z)\\[2mm]
\cancel{tf(tx+(1-t)z)}-tf(x)&\leq f(z)-tf(z)-f(tx+(1-t)z)+\cancel{tf(tx+(1-t)z)}\\[2mm]
f(tx+(1-t)z)&\leq tf(x)+f(z)-tf(z)\\[2mm]
f(tx+(1-t)z)&\leq tf(x)+(1-t)f(z).
\end{align*}

Since $x,y,z\in(a,b)$ are arbitrary, the inequality above holds if and only if $f$ is convex for all $a<x<y<z<b$.
\end{proof}

\item Assume $f$ is differentiable on $(a,b)$. Show that the condition above is equivalent to 
\[
f'(x)\le f'(y),\ \forall a<x<y<b.
\] 
Thus, if $f''(x)$ exists for all $x\in(a,b)$, convexity of $f$ is equivalent to
\[
f''(x)\ge 0,\ \forall x\in (a, b).
\]

\begin{proof}
Let $a<x<y<z<b$ and suppose $f$ is differentiable on $(a,b)$. For the forward direction, if $f$ is convex, then from part (a) above we have
\begin{align*}
\frac{f(y)-f(x)}{y-x}&\leq\frac{f(z)-f(y)}{z-y},\text{ so}\\[2mm]
f'(x)=\lom{y}{x}\frac{f(y)-f(x)}{y-x}&\leq\lom{z}{y}\frac{f(z)-f(y)}{z-y}=f'(y).
\end{align*}

Conversely, suppose $f'(x)\leq f'(y)$, i.e. $f'$ is increasing. By the Mean Value Theorem, there exist $c_1,c_2\in(a,b)$ such that
\[
f'(c_1)=\frac{f(y)-f(x)}{y-x}\aspace f'(c_2)=\frac{f(z)-f(y)}{z-y}.
\]
But $f'$ is increasing so $f'(c_1)\leq f'(c_2)$ whence
\[
\frac{f(y)-f(x)}{y-x}\leq \frac{f(z)-f(y)}{z-y}
\]
which implies that $f$ is convex by part (a) again. Thus, if $f$ is convex, then $f'$ is increasing, so $f''(x)\geq 0$ by the first derivative test.
\end{proof}

\item Show that $f(x)=x^p:[0,\infty)\to\MB{R}$, $p\ge 1$, is convex. Then show that
\[
(a+b)^p\le 2^{p-1}\cdot (a^p+b^p),\ \forall a,b\ge 0.
\]
\end{enumerate}

\begin{proof}
$f''(x)=p(p-1)x^{p-2}\geq0$ for all $x\in[0,\infty)$ so $f$ is convex by part (b) above. Thus, for $a,b\in [0,\infty)$, we have
\[
(ta+(1-t)b)^p\leq ta^p+(1-t)b^p\pspace\text{for all }t\in[0,1].
\]
In particular, this is true for $t=\frac{1}{2}$. When $t=\frac{1}{2}$, we have
\begin{align*}
\left(\frac{a}{2}+\frac{b}{2}\right)^p&\leq\frac{a^p}{2}+\frac{b^p}{2}\\[2mm]
\left(\frac{a+b}{2}\right)^p&\leq\frac{a^p+b^p}{2}\\[2mm]
\frac{(a+b)^p}{2^p}&\leq\frac{a^p+b^p}{2}\\[2mm]
(a+b)^p&\le 2^{p-1}(a^p+b^p)
\end{align*}
\end{proof}

\item Let $(X,\MC{M}, \mu)$ be a measure space.
\begin{enumerate}
\itemsep7mm
\item Show that $L^1(X,d\mu)$ is a normed vector space with the $L^1$ norm
\[
\|f\|_1=\int_X|f|d\mu.
\]

\begin{proof}
First note that $L^1(X,d\mu)$ is a vector space since linear combinations of $L^1$ functions are still $L^1$. We proceed with showing that $\|\cdot\|_1$ is a norm. Let $f,g\in L^1(X,d\mu)$ and let $\lambda\in \MB{C}$. Then since $|\cdot|$ is a norm on $\MB{C}$, by the triangle inequality and linearity of the Lebesgue integral, we have
\[
\|f+g\|_1=\int_X|f+g|\,d\mu\leq\int_X|f|+|g|\,d\mu=\int_X|f|\,d\mu+\int_X|g|\,d\mu=\|f\|_1+\|g\|_1\text{ and}
\]
\[
\|\lambda f\|_1=\int_X|\lambda f|\,d\mu=\int_X|\lambda|\,|f|\,d\mu=|\lambda|\int_X|f|\,d\mu=|\lambda|\,\|f\|_1.
\]
Next, $\|f\|_1=0$ if and only if $\int_X|f|\,d\mu=0$ and from a result in 209A (cf. Folland p. 54, Theorem 2.23(b)) this is the case if and only if $f=0$ a.e. Since functions that agree a.e. are identified in $L^p$ space for any $p\in[1,\infty]$, it follows that $\|\cdot\|_1$ is indeed a norm.
\end{proof}

\item Show that $L^\infty(X,d\mu)$ is a normed vector space with the $L^\infty$ norm 
\[
\|f\|_\infty=\inf\big\{M\ge 0: |f(x)|\le M \text{ for $\mu$-a.e. }x\in X\big\}=\sup_{\text{a.e. }x\in X}|f(x)|=\text{ess sup}\,|f(x)|
\]
\begin{proof}
Let $f,g\in L^\infty(X,d\mu)$ and let $\lambda\in \MB{C}$. Then $\|f\|_\infty<\infty$ and $\|g\|_\infty<\infty$. By properties of suprema, and since $|\cdot|$ is a norm on $\MB{C}$, we have
\[
\|f+g\|_\infty=\text{ess sup}|f(x)+g(x)|\leq\text{ess sup}|f(x)|+\text{ess sup}|g(x)|=\|f\|_\infty+\|g\|_\infty<\infty,
\]
so $f+g\in L^\infty(X,d\mu)$. Moreover we have also shown that $\|\cdot\|_\infty$ obeys the triangle inequality. Second, again by properties of suprema, we have
\[
\|\lambda f\|_\infty=\text{ess sup}\,|\lambda f(x)|=\text{ess sup}\,(|\lambda|\,|f(x)|)=|\lambda|\,\text{ess sup}\,|f(x)|=|\lambda|\,\|f\|_\infty<\infty,
\]
so $\lambda f\in L^\infty(X,d\mu)$. Thus, $L^\infty(X,d\mu)$ is a vector space. Moreover, we have shown absolute homogeneity of $\|\cdot\|_\infty$, so all that remains is to show that $\|f\|_\infty=0$ if and only if $f=0$ a.e. If $f=0$, then $\|0\|_\infty=\text{ess sup}\,|0|=0$. If $\|f\|_\infty=0$, then $\text{ess sup}\,|f(x)|=0$. Since $|f(x)|\geq0$ for all $x\in X$, and $\text{ess sup}|f(x)|=\inf\big\{M\ge 0: |f(x)|\le M \text{ for $\mu$-a.e. }x\in X\big\}$, we have $|f(x)|=0$, whence $f=0$ for a.e. $x\in X$. Therefore, $\|\cdot\|_\infty$ is a norm by definition, and we have shown that $L^\infty(X,d\mu)$ is a normed vector space.
\end{proof}

\end{enumerate}

\item Let $(X,\MC{M},\mu)$ be a measure space. Suppose $\mu(X)<\infty$. Show that for any $1\leq s<t\leq \infty$, there is a $C=C(s,t)>0$ such that
\[
\|f\|_s<C\|f\|_t.
\]
Note: In this case, we say that the $L^t$ norm is stronger than the $L^s$ norm. Clearly, it implies that $L^t(X,d\mu)\subset L^s(X,d\mu)$. Notice that convergence in $L^t$ norm implies convergence in $L^s$ norm.

\begin{proof}
We need to consider two cases here.


\textit{Case 1:} $1\leq s< t<\infty$. Let $F=f^s$. Write $p=\frac{t}{s}$ and $q=\frac{t}{t-s}$. Then
\[
\frac{1}{p}+\frac{1}{q}=\frac{s}{t}+\frac{t-s}{t}=\frac{s+t-s}{t}=1,
\]
so we may apply H\"older's inequality to the product $F\cdot\chi_X$, yielding
\begin{align*}
\|F\cdot\chi_X\|_1=\|f^s\cdot1\|_1=\int_X|f|^s\,d\mu&\leq\|F\|_p\|\chi_X\|_q\\[2mm]
&=\left(\int_X|F|^p\,d\mu\right)^{1/p}\left(\int_X\chi_X^q\,d\mu\right)^{1/q}\\[2mm]
&=\left(\int_X|f|^{sp}\,d\mu\right)^{1/p}\left(\int_X1^q\,d\mu\right)^{1/q}\\[2mm]
&=\left(\int_X|f|^t\,d\mu\right)^{s/t}\cdot\mu(X)^{(t-s)/t},\text{ thus,}\\[5mm]
\int_X|f|^s\,d\mu&\leq\left(\int_X|f|^t\,d\mu\right)^{s/t}\cdot\mu(X)^{(t-s)/t}\\[5mm]
\left(\int_X|f|^s\,d\mu\right)^{1/s}&\leq \left(\int_X|f|^t\,d\mu\right)^{1/t}\cdot\mu(X)^{\frac{t-s}{st}}\\[2mm]
\|f\|_s&\leq\mu(X)^{\frac{t-s}{st}}\|f\|_t\\[2mm]
\|f\|_s&\leq C\|f\|_t,
\end{align*}
where $C=C(s,t)=\mu(X)^{\frac{t-s}{st}}$. Note that this only holds if $X$ is $\mu$-finite.

\textit{Case 2:} $1\leq s<t=\infty$. In this case, consider the function $g=\frac{f}{\|f\|_\infty}$. Then $g(x)\leq 1$ for all $x\in X$, and moreover
\[
\|g\|_s=\left\|\frac{f}{\|f\|_\infty}\right\|_s\leq\|1\|_s=1.
\]
From this, we see that
\begin{align*}
\left\|\frac{f}{\|f\|_\infty}\right\|_s&\leq1\\[2mm]
\left(\int_X\left|\frac{f}{\|f\|_\infty}\right|^s\,d\mu\right)^{1/s}&\leq 1\\[2mm]
\frac{1}{\|f\|_\infty}\left(\int_X|f|^s\,d\mu\right)^{1/s}&\leq1\\[2mm]
\left(\int_X|f|^s\,d\mu\right)^{1/s}&\leq\|f\|_\infty\\[2mm]
\|f\|_s&\leq\|f\|_\infty.
\end{align*}
In this case we have that $C=C(s,t)=1$.

Note that this implies $L^t(X,d\mu)\subset L^s(X,d\mu)$: Indeed, if $f\in L^t(X,d\mu)$, then $\|f\|_t<\infty$, whence
\[
\|f\|_s\leq C\|f\|_t<\infty
\]
so $f\in L^s(X,d\mu)$ as well. Additionally, if a sequence of functions $\{f_n\}$ converges in the $L^t$ norm, it converges in the $L^s$ norm as well since
\[
\|f_n-f\|_s\leq C\|f_n-f\|_t\to0.
\]
\end{proof}

\item Show that the metric of $L^\infty(X,d\mu)$ induced by the $L^\infty$ norm is complete. [Hint: Combine part (b) of problem 2 and problem 3 to get that $L^\infty(X,d\mu)$ is a Banach space.]

\begin{proof}
First, in problem 3 we showed that $L^\infty(X,d\mu)$ was a normed vector space, so it suffices to show that a Cauchy sequence in $L^\infty$ converges to some function $f\in L^\infty(X,d\mu)$. To that end, let $\{f_n\}$ be a Cauchy sequence in $L^\infty(X,d\mu)$ and let $\ve>0$. Then there exists an $N\in\MB{Z}^+$ such that whenever $m,n\geq N$, we have $\|f_n-f_m\|_\infty<\ve$. Consequently, for each $x\in X$, the sequence $\{f_n(x)\}$ is a Cauchy sequence in $\MB{C}$. Moreover, since $\MB{C}$ is complete in the standard metric, the sequence $\{f_n(x)\}$ converges to it's pointwise limit $f(x)\in\MB{C}$. In particular, $|f(x)|<\infty$ for any $x\in X$. As a result, this limit function $f$ is in $L^\infty(X,d\mu)$: observe that
\begin{align*}
\|f\|_\infty&=\text{ess sup}\,|f(x)|\\[2mm]
&\leq\sup|f(x)|\\[2mm]
&=\sup\left|\lom{n}{\infty}f_n(x)\right|\\[2mm]
&=\limsup_{n\to\infty}|f_n(x)|<\infty
\end{align*}
since each $f_n\in L^\infty$. We now claim that $\{f_n\}$ converges to $f$ in $L^\infty$. We have
\[
\|f_n-f\|_\infty=\left\|f_n-\lom{m}{\infty}f_m\right\|_\infty=\lom{m}{\infty}\|f_n-f_m\|_\infty<\lom{m}{\infty}\ve=\ve.
\]
Since $\ve$ was chosen arbitrarily, this is true for all positive numbers and it follows that $\|f_N-f\|\to0$. Therefore, $L^\infty(X,d\mu)$ is a Banach space.
\end{proof}

\pagebreak

\item Prove Theorem 2.2 on our lecture notes. Specifically, let $X$ and $Y$ be normed vector spaces. Let $\MC{B}(X,Y)$ denote the set of all bounded linear transformations from $X$ to $Y$. Let $\MC{B}(X,Y)$ be equipped with the operator norm, i.e. the norm of $\Lambda\in\MC{B}(X,Y)$ is defined as
\[
\|\Lambda\|=\sup\left\{\|\Lambda(x)\|_Y:\ x\in X\,\|x\|_X\leq 1\right\}.
\]
Show the following:
\begin{enumerate}
\item $\MC{B}(X,Y)$ is a vector space.
\begin{proof}
Since linear combinations of linear transformations are linear, given any linear transformations $S,T\in\MC{B}(X,Y)$ and any scalar $\lambda\in \MB{C}$, $S+T$ and $\lambda S$ are linear so it suffices to show they are bounded. We have
\[
\|S+T\|=\sup_{\|x\|_X\leq1}\|S(x)+T(x)\|_Y\leq\sup_{\|x\|_X\leq1}\|S(x)\|_Y+\sup_{\|x\|_X\leq1}\|T(x)\|_Y=\|S\|+\|T\|<\infty,\text{ and}
\]
\[
\|\lambda S\|=\sup_{\|x\|_X\leq1}\|\lambda S(x)\|_Y=|\lambda|\sup_{\|x\|_X\leq1}\|S(x)\|_Y=|\lambda|\,\|S\|<\infty.
\]
Therefore, $S+T$ and $\lambda S$ are bounded, i.e. in $\MC{B}(X,Y)$, and so $\MC{B}(X,Y)$ is a vector space.
\end{proof}

\item The operator norm is indeed a norm on $\MC{B}(X,Y)$.
\begin{proof}
The result in part (a) above proves that $\|\cdot\|$ obeys the triangle inequality and absolute homogeneity, and all that remains is to show that $\|S\|=0$ if and only if $S\equiv0$. But $\|S\|=0$ if and only if $\sup_{\|x\|_X\leq1}\|S(x)\|_Y=0$ if and only if $\|S(x)\|_Y=0$ for all $x\in X$ such that $\|x\|_X\leq 1$, i.e. every $x$ in the closed unit ball in $X$. Since every point in $X$ can be written as a linear combination of vectors on the unit sphere $\{x\in X:\,\|x\|_X=1\}$ which is contained in the closed unit ball, it must be the case that $S(x)=0$ for all $x\in X$. Therefore, $\|\cdot\|=0$ if and only if $S\equiv 0$, and we have that $\|\cdot\|$ is a norm on $\MC{B}(X,Y)$.
\end{proof}

\item If $Y$ is a Banach space, then the metric of $\MC{B}(X,Y)$ induced by the operator norm is complete.
\begin{proof}
Suppose $Y$ is a Banach space. We have already seen above that $\MC{B}(X,Y)$ is a normed vector space, so it suffices to show that any Cauchy sequence of transformtations in $\MC{B}(X,Y)$ converges to some transformation in $\MC{B}(X,Y)$. To that end, let $\{S_n\}$ be a Cauchy sequence in $\MC{B}(X,Y)$ and let $\ve>0$. Then there exists an $N\in\MB{Z}^+$ such that whenever $m,n\geq N$, we have $\|S_n-S_m\|<\ve$. Consequently, for each $x\in X$, the sequence $\{S_n(x)\}$ is a Cauchy sequence in $\MB{C}$. Since $Y$ is a Banach space, it is complete, so the sequence $\{S_n(x)\}$ converges to it's pointwise limit $S(x)\in Y$. In particular, $\|S(x)\|_Y<\infty$ for any $x\in X$ so $S$ is bounded. Additionally, for any $x,y\in X$ and $\lambda\in \MB{C}$, by the linearity of $S_n$, we have
\[
S(x+y)=\lom{n}{\infty}S_n(x+y)=\lom{n}{\infty}\left(S_n(x)+S_n(y)\right)=\lom{n}{\infty}S_n(x)+\lom{n}{\infty}S_n(y)=S(x)+S(y),\text{ and}
\]
\[
S(\lambda x)=\lom{n}{\infty}S_n(\lambda x)=\lom{n}{\infty}\lambda S_n(x)=\lambda \lom{n}{\infty}S_n(x)=\lambda S(x).
\]
Therefore $S$ is in $\MC{B}(X,Y)$. We now claim that $\{S_n\}$ converges to $S$ in the operator norm. For sufficiently large $m,n$, we have
\[
\|S_n-S\|=\left\|S_n-\lom{m}{\infty}S_m\right\|=\lom{m}{\infty}\|S_n-S_m\|<\lom{m}{\infty}\ve=\ve.
\]
Therefore, the metric of $\MC{B}(X,Y)$ induced by the operator norm is complete, i.e. $\MC{B}(X,Y)$ is a Banach space. Since the dual space of a normed vector space is defined as the space of all bounded linear functionals, e.g. $\MC{B}(X,\MB{R})$, we have shown that the dual space of any normed vector space is Banach.
\end{proof}
\end{enumerate}

\item Let $\mu$ be the counting measure on $\MB{Z}$. In other words, $\mu(\{n\})=1$ for any $n\in\MB{Z}$. Note in this case every set in $\MB{Z}$ is measurable, i.e. the underlying $\sigma$-algebra is the power set $2^{\MB{Z}}$ of $\MB{Z}$. In particular, all maps $a:\MB{Z}\to\MB{C}$ are measurable. Note a map $a:\MB{Z}\to\MB{C}$ may be viewed as a sequence $\{a_n\}_{n\in\MB{Z}}$ of complex numbers.
\begin{enumerate}
\item Let $a:\MB{Z}\to[0,\infty)$ which is always integrable. Try to write its integral in terms of the sequence $\{a_n\}$ and in a form with which you are familiar.

If $\mu$ is the counting measure, then
\[
\int_\MB{Z}a(n)\,d\mu=\sum_{n=-\infty}^\infty\int_{\{n\}}a_n\,d\mu=\sum_{n=-\infty}^\infty a_n.
\]
\item Use the formula from part (a) to explain when a general $a:\MB{Z}\to\MB{C}$ is integrable.

A general function $a:\MB{Z}\to\MB{C}$ will be integrable if and only if the series above converges absolutely, i.e.
\[
\sum_{n=-\infty}^\infty |a_n|<\infty.
\]
\item Use the formula from part (a) to define the spaces $L^p(\MB{Z},d\mu)$ for $1\leq p\leq \infty$. In fact, the more standard notation for $L^p$ spaces in this case is $\ell^p(\MB{Z})$.

The space $L^p(\MB{Z},d\mu)=\ell^p(\MB{Z})$ where $\mu$ is the counting measure is the space of all two-sided infinite sequences $a:\MB{Z}\to\MB{C}$ such that 
\[
\left(\sum_{n=-\infty}^\infty |a_n|^p\right)^{1/p}<\infty.
\]
\end{enumerate}

Note: almost all the results for integration on a general measure space, such as all the convergence theorems, hold true in this special case. Also, all the results we proved in Chapter two hold true for the spaces $\ell^p(\MB{Z})$. For instance, they are Banach spaces; the dual space of $\ell^p(\MB{Z})$ is $\ell^q(\MB{Z})$ where $1\leq p\leq \infty$. I do encourage that you try to translate all the results for general measure spaces to this special case via the formula from part (a).

\item Consider the following problems which are related to problem 4 above. 
\begin{enumerate}
\item Show that $\ell^2(\MB{Z})\not\subset\ell^1(\MB{Z})$ which implies that in general $L^2(X,d\mu)\not\subset L^1(X,d\mu)$.

\begin{proof}
As a counterexample, consider the harmonic sequence $h(n)=\left(1,\frac{1}{2},\frac{1}{3},\frac{1}{4},\ldots\right)$ which can be equivalently indexed by the integers or the positive integers and realized as a two-sided infinite sequence. Then $h\in\ell^2(\MB{Z})$ since
\[
\|h\|_{\ell^2}=\left(\sum_{n=-\infty}^\infty\left|\frac{1}{n}\right|^2\right)^{1/2}=\left(\sum_{m=1}^\infty\frac{1}{m^2}\right)^{1/2}=\frac{\pi}{\sqrt{6}}<\infty.
\]
However, $h\not\in\ell^1(\MB{Z})$ since the harmonic series diverges. Consequently, $\ell^2(\MB{Z})\not\subset\ell^1(\MB{Z})$ and so in general $L^2(X,d\mu)\not\subset L^1(X,d\mu)$.
\end{proof}

\item In fact, show that $\ell^s(\MB{Z})\subset \ell^t(\MB{Z})$ for any $1\leq s<t\leq \infty$.

\begin{proof}
Let $a=(a_n)$ be a sequence in $\ell^{s}(\MB{Z})$. Suppose first that $t<\infty$. Then it suffices to show that if $\|a\|_s=\left(\sum_{n=-\infty}^\infty|a_n|^s\right)^{1/s}<\infty$, then $\|a\|_t=\left(\sum_{n=-\infty}^\infty|a_n|^t\right)^{1/t}<\infty$. Equivalently, if $\sum_{m=0}^\infty|a_m|^s<\infty$ then $\sum_{m=0}^\infty|a_m|^t<\infty$. Suppose $\sum_{m=0}^\infty|a_m|^s<\infty$. By the contrapositive of the test for divergence, $\lom{m}{\infty}|a_m|^s=\left|\lom{m}{\infty}a_m\right|^s=0$. But this means that there exists some $N\in\MB{N}_0$ such that whenever $m\geq N$, $|a_m|<1$. If $|a_m|<1$, then $|a_m|^t\leq |a_m|^s$ so by the Dominated Convergence Theorem, $\sum_{m=0}^\infty |a_m|^t<\infty$. It follows that $\|a\|_t=\left(\sum_{n=-\infty}^\infty |a_n|^t\right)^{1/t}<\infty$ and so $\ell^s(\MB{Z})\subset \ell^t(\MB{Z})$ if $t<\infty$.

Next suppose $t=\infty$ and that $a\in\ell^s(\MB{Z})$. Then, after reindexing, we have $\sum_{m=0}^\infty|a_m|^s<\infty$. Again by the contrapositive of the test for divergence, it follows that $\lom{n}{\infty}|a_m|=0$ and since this limit exists, $\limsup_{m\to\infty}|a_m|=0$. Moreover $|a_m|<\infty$ for every $m$, so we have  sequence of finite numbers that converges to 0. Consequently, $\text{ess sup}\,|a_m|=\sup_{m\geq0}|a_m|<\infty$ so $a\in\ell^\infty(\MB{Z})$. Putting these two cases together, we have shown that $\ell^s(\MB{Z})\subset \ell^t(\MB{Z})$ for any $1\leq s<t\leq \infty$
\end{proof}

\end{enumerate}

Thus in problem 4, the condition $\mu(X)<\infty$ may not be removed in obtaining $L^t(X,d\mu)\subset L^s(X,d\mu)$ for $1\leq s<t\leq \infty$.
	
	
\end{enumerate}

\end{document}