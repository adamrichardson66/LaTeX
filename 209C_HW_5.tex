\documentclass[11pt,oneside,english]{amsart}
\usepackage[T1]{fontenc}
\usepackage{geometry}
\usepackage{parskip}
\geometry{verbose,tmargin=0.65in,bmargin=0.65in,lmargin=0.75in,rmargin=0.75in,headheight=0.75cm,headsep=1cm,footskip=1cm}
\setlength{\parskip}{7mm}
\usepackage{setspace}
\onehalfspacing
\pagenumbering{gobble}

\usepackage{bbm}
\usepackage{multicol}
\usepackage{graphicx}
\usepackage{adjustbox}
\usepackage{amssymb}
\usepackage{tikz}
\usepackage{pgfplots}
\usepackage{pgffor}
\usetikzlibrary{cd}
\usepackage{ulem}
\usepackage{adjustbox}
\usepackage{bm}
\usepackage{stmaryrd}
\usepackage{cancel}
\usepackage{mathtools}
\DeclarePairedDelimiter{\ceil}{\lceil}{\rceil}
\DeclarePairedDelimiter\floor{\lfloor}{\rfloor}
\usepackage[shortlabels]{enumitem}
\setlist[enumerate,1]{label=\textbf{\arabic*.}}
\usepackage{color, colortbl}
\definecolor{Gray}{gray}{0.9}
\usepackage{babel}
\usepackage{mdframed}
\usepackage{esint}
\usepackage[yyyymmdd]{datetime}
\renewcommand{\dateseparator}{--}
\usepackage{url}
\usepackage[unicode=true,pdfusetitle,
 bookmarks=true,bookmarksnumbered=false,bookmarksopen=false,
 breaklinks=false,pdfborder={0 0 1},backref=false,colorlinks=true]
 {hyperref}
\hypersetup{urlcolor=blue}





\theoremstyle{definition}
\newtheorem{theorem}{Theorem}
\newtheorem*{theorem*}{Theorem}
\newtheorem*{proposition*}{Proposition}
\newtheorem{corollary}{Corollary}
\newtheorem*{lemma}{Lemma}
\newtheorem*{example}{Example}
\newtheorem*{examples}{Examples}
\newtheorem*{definition}{Definition}
\newtheorem*{note}{Nota Bene}

\newcommand{\aspace}{\hspace{7mm}\text{and}\hspace{7mm}}
\newcommand{\ospace}{\hspace{7mm}\text{or}\hspace{7mm}}
\newcommand{\pspace}{\hspace{10mm}}
\newcommand{\lhe}{\stackrel{\text{L'H}}{=}}
\newcommand{\lom}[2]{\lim_{{#1}\rightarrow{#2}}}
\newcommand{\ve}{\varepsilon}
\newcommand{\ess}{\text{ess sup}}
\newcommand{\dd}[2]{\frac{d{#1}}{d{#2}}}
\newcommand{\pp}[2]{\frac{\partial{#1}}{\partial{#2}}}
\newcommand{\DD}[2]{\frac{\Delta{#1}}{\Delta{#2}}}
\newcommand{\ovec}[1]{\overrightarrow{#1}}
\newcommand{\MC}[1]{\mathcal{#1}}
\newcommand{\MB}[1]{\mathbb{#1}}
\newcommand{\mbf}[1]{\,\mathbf{#1}}
\renewcommand{\vec}[1]{\underline{#1}}



\def\<#1>{\mathinner{\langle#1\rangle}}

\makeatletter
\g@addto@macro\normalsize{%
  \setlength\belowdisplayshortskip{5mm}
}
\makeatother




\begin{document}

\rightline{Adam D. Richardson}
\rightline{209C - Real Analysis}
\rightline{Zhang, Zhenghe}
\rightline{HW 5}
\rightline{\today}



\vspace{5mm}
\begin{enumerate}
\itemsep5mm




\item Show the Riemann-Lebesgue Lemma for $L^1(\MB{T},dx)$: the Fourier transformation $\hat f$ of $f\in L^1(\MB{T},dx)$ lies in $C^0(\MB{Z})$. In other words,
\[
\lom{n}{\pm\infty}|\hat f(n)|=0.
\]
[Hint: the same process at the top of Chap 3, page 24 of my lecture notes yields that trigonometric polynomials are dense in $L^1(\MB{T})$. Use this result.]

\begin{proof}
First we show that this holds for simple functions. Consider the Fourier transformation of a characteristic function $\chi_{(a,b)}$. We have
\begin{align*}
\hat \chi_{(a,b)}(n)&=\int_\MB{T}\chi_{(a,b)}(x)e^{-2\pi inx}\,dx\\[2mm]
&=\int_{(a,b)}e^{-2\pi inx}\,dx\\[2mm]
&=\left.-\frac{1}{2\pi in}e^{-2\pi inx}\right|_a^b\\[2mm]
&=\frac{e^{-2\pi ina}-e^{-2\pi inb}}{2\pi i n}.
\end{align*}
Thus,
\[
\lom{n}{\pm\infty}|\hat \chi_{(a,b)}(n)|=\lom{n}{\pm\infty}\left|\frac{e^{-2\pi ina}-e^{-2\pi inb}}{2\pi i n}\right|\leq\lom{n}{\pm\infty}\frac{1}{\pi i n}=0.
\]
Since simple functions are linear combinations of characteristic functions, by properties of limits this result holds for simple functions as well. Since the space of simple functions is dense in $L^1(\MB{T})$, given $\ve>0$ and any $f\in L^1(\MB{T})$, there is a simple function $s$ such that $\|f-s\|_1<\frac{\ve}{2}$. By our result above $\lom{n}{\pm\infty}|\hat s(n)|=0$, so there exists an $N\in \MB{Z}^+$ such that for all $n\geq N$, $|\hat s(n)|<\frac{\ve}{2}$. Consequently, for all $n\geq N$,
\begin{align*}
|\hat f(n)|&=|\hat f(n)-\hat s(n)+\hat s(n)|\\[2mm]
&\leq|\hat f(n)-\hat s(n)|+|\hat s(n)|\\[2mm]
&<\int_\MB{T}|f(x)-s(x)|\,|e^{-2\pi inx}|\,dx+\frac{\ve}{2}\\[2mm]
&\leq\int_\MB{T}|f(x)-s(x)|\,dx+\frac{\ve}{2}\\[2mm]
&=\|f-s\|_1+\frac{\ve}{2}<\frac{\ve}{2}+\frac{\ve}{2}=\ve.
\end{align*}
\end{proof}


\item Show that the natural topology of a locally convex space is Hausdorff.

\begin{proof}
Let $(X,\{\rho_\alpha\}_{\alpha\in A})$ be a locally convex space with the natural topology $\MC{T}$. By Lemma 5.1, this topology is generated by $\{B_\alpha(x,r),\,\alpha\in A,\,x\in X,\,r>0\}$. Choose $x,y\in X$ such that $x\neq y$. Since $(X,\{\rho_\alpha\}_{\alpha\in A})$ is locally convex, each $\rho_\alpha$ separates points, i.e. $\rho_\alpha(x)=0$ for all $\alpha\in A$ iff $x=0$. Since $x\neq y$, $\rho_\alpha(x-y)\neq0$ for all $\alpha\in A$. This combined with the result of Lemma 5.1 implies that  there exists an $r>0$ such that $B_\alpha(x,r)\cap B_\alpha(y,r)=\varnothing$ for all $\alpha$. (In particular, we could let $r=\frac{|\rho_\alpha(x)-\rho_\alpha(y)|}{2}$ for all $\alpha\in A$.) Since these open balls of radius $r$ are basis elements, they belong to $\MC{T}$, and so $(X,\{\rho_\alpha\}_{\alpha\in A})$ is Hasudorff by definition.
\end{proof}

\item Carry out the details of step 4 of the proof of Theorem 5.3. It's located at Chapter 5, p. 15 of my lecture notes. [Hint: here you may try to follow the proof of Theorem 3.11. It's actually easier than that.]

\begin{proof}
In Step 4, the construction of the standard mollifier, $\phi$, is detailed, as well as the scaled standard mollifier $\phi_\lambda$. Let $g\in C_c^\infty(\MB{R}^d)$. We need to show first that $g_\lambda=g*\phi_\lambda$ is smooth with compact support, and to show second that $g_\lambda$ converges to $g$ uniformly in $L^\infty$ norm as $\lambda\to0$.

First, $g_\lambda$ has compact support since both $g$ and $\phi_\lambda$ have compact support. To show smoothness, we examine the partial derivatives. Let $x\in \MB{R}^d$ be arbitrary by fixed, and let $h\in \MB{R}$. Then by the DCT,
\begin{align*}
\pp{}{x_i}\left(g_\lambda(x)\right)&=\lom{h}{0}\frac{g_\lambda(x+he_i)-g_\lambda(x)}{h}\\[2mm]
&=\lom{h}{0}\frac{1}{h}\left[\int_{\MB{R}^d} g(x+he_i-y)\phi_\lambda(y)\,dy-\int_{\MB{R}^d} g(x-y)\phi(y)\,dy\right]\\[2mm]
&=\lom{h}{0}\frac{1}{h}\int_{\MB{R}^d} \left[g(x+he_i-y)\,dy-g(x-y)\right]\phi_\lambda(y)\,dy\\[2mm]
&=\int_{\MB{R}^d} \lom{h}{0}\left[\frac{g(x+he_i-y)\,dy-g(x-y)}{h}\right]\phi_\lambda(y)\,dy\\[2mm]
&=\int_{\MB{R}^d} \pp{}{x_i}g(x-y)\phi_\lambda(y)\,dy\\[2mm]
&=\left(\pp{g}{x_i}\right)*\phi_\lambda(x).
\end{align*}
Since $\pp{g}{x_i}$ exists and $\phi_\lambda$ is smooth, their convolution is smooth as well, whence the $i$th partial derivative of $g_\lambda$ is smooth. Since this is true for all $1\leq i\leq d$, $g_\lambda$ is smooth.

Next, we need to show that $\|g_\lambda-g\|_\infty\to0$ as $\lambda\to0$. We have

\begin{align*}
\|g_\lambda-g\|_\infty&=\ess_{x\in\MB{R}^d}|g_\lambda(x)-g(x)|\\[2mm]
&=\ess_{x\in\MB{R}^d}\left|\int_{B(0,\lambda)}g(x-y)-g(x)\phi_\lambda(y)\,dy-\int_{B(0,\lambda)}g(x)\phi_\lambda(y)\,dy\right|\\[2mm]
&=\ess_{x\in\MB{R}^d}\left|\int_{B(0,\lambda)}[g(x-y)-g(x)]\phi_\lambda(y)\,dy\right|\\[2mm]
&\leq\ess_{x\in\MB{R}^d}\int_{B(0,\lambda)}|g(x-y)-g(x)|\,|\phi_\lambda(y)|\,dy.
\end{align*}

Since $g$ is continuous, for any $\ve>0$, there exists a $\lambda>0$ sufficiently small such that whenever $|x-y|<\lambda$, $|g(x-y)-g(x)|<\ve$. Thus,

\begin{align*}
\ess_{x\in\MB{R}^d}\int_{B(0,\lambda)}|g(x-y)-g(x)|\,|\phi_\lambda(y)|\,dy&<\ve \cdot \ess_{x\in\MB{R}^d}\int_{B(0,\lambda)}|\phi_\lambda(y)|\,dy\\[2mm]
&<\ve \cdot1=\ve.
\end{align*}
Thus, $\|g_\lambda-g\|_\infty\to0$ as $\lambda\to0$.

\end{proof}


\item Show that $\phi(x)=e^{-|x|^2}:\MB{R}^d\to\MB{R}^d$ is in the Schwartz space $\MC{S}(\MB{R}^d)$. Note here $|x|^2=\sum_{j=1}^dx_j^2$. This gives an example of a function $\phi\in\MC{S}$ that is not in $C_c(\MB{R}^d)$.

\begin{proof}
Notice that for any $x_j$,
\[
\pp{\phi}{x_j}=-2x_je^{-|x|^2}.
\]
With a little more exploration, for any $\beta$, $D^\beta_x\phi(x)=P(x)e^{-|x|^2}=P(x)\phi(x)$ where $P(x)$ is a polynomial. Say the polynomial has degree $n$. Since both $P(x)$ and $\phi(x)$ are continuous, so is their product, whence $\phi$ is smooth by the calculation above. Additionally, $|P(x)|\leq C|x|^n$ for some constant $C>0$ and $|x|\geq1$. Thus,
\begin{align*}
\sup_{x\in\MB{R}^d}|x^\alpha D_x^\beta\phi(x)|&=\sup_{x\in\MB{R}^d}|x^\alpha P(x)e^{-|x|^2}|\\[2mm]
&\leq\sup_{x\in\MB{R}^d}|x^\alpha| C|x|^ne^{-|x|^2}.
\end{align*}
Looking at the Taylor series for $e^x$, we can deduce the estimate $e^x\geq1+\frac{x^{n/2}}{(n/2)!}$. Consequently,
\[
e^{-|x|^2}\leq\frac{1}{1+\frac{(|x|^2)^{n/2}}{(n/2)!}}=\frac{(n/2)!}{(n/2)!+|x|^{n}}\leq\frac{(n/2)!}{|x|^{n}}.
\]
Substituting this into the inequality above, we find that
\[
\sup_{x\in\MB{R}^d}|x^\alpha D_x^\beta\phi(x)|\leq\sup_{x\in\MB{R}^d}|x^\alpha| C|x|^n\cdot\frac{(n/2)!}{|x|^{n}}\leq C(n/2)!<\infty,
\]
so $\phi\in\MC{S}(\MB{R}^d)$.
\end{proof}


\item Use Fubini to show that $\MC{F}(f*g)=\MC{F}(f)\cdot\MC{F}(g)$ for any $f,g\in \MC{S}(\MB{R}^d)$. In fact, it holds for more general $f$ and $g$. What are the weakest assumptions on $f$ and $g$ to get the formula above?

\begin{proof}
We have
\begin{align*}
\MC{F}(f*g)(x)&=\MC{F}\left(\int_{\MB{R}^d}f(x-y)g(y)\,dy\right)\\[2mm]
&=\frac{1}{(2\pi)^{d/2}}\int_{\MB{R}^d}\int_{\MB{R}^d}f(x-y)g(y)\,dy\,e^{-ix\cdot\xi}\,dx\\[2mm]
&=\frac{1}{(2\pi)^{d}}\int_{\MB{R}^d}\int_{\MB{R}^d}f(x-y)g(y)e^{-ix\cdot\xi}\,dx\,dy\\[2mm]
&=\frac{1}{(2\pi)^{d}}\int_{\MB{R}^d}\int_{\MB{R}^d}f(t)g(y)e^{-i(y+t)\cdot\xi}\,dt\,dy\\[2mm]
&=\frac{1}{(2\pi)^{d}}\int_{\MB{R}^d}\int_{\MB{R}^d}f(t)g(y)e^{-iy\cdot\xi}e^{-it\cdot\xi}\,dt\,dy\\[2mm]
&=\frac{1}{(2\pi)^{d/2}}\int_{\MB{R}^d}f(t)e^{-it\cdot\xi}\,dt\cdot\frac{1}{(2\pi)^{d/2}}\int_{\MB{R}^d}g(y)e^{-iy\cdot\xi}\,dy\\[2mm]
&=\MC{F}(f)(x)\cdot\MC{F}(g)(x)
\end{align*}

The weakest assumptions are that $f$ and $g$ be $L^1$ so that Fubini may be applied.
\end{proof}

\item Let $\phi(x)=e^{-|x|}$ on $\MB{R}$. First, compute $\MC{F}(\phi)$. Then use the Fourier transformation to show that $u=f*\phi$ solves the ODE $u-u''=f$. What assumptions are needed on $f$? For the last question, you may use Proposition 8.10 of Folland (p. 242).

\begin{proof}
Since $x\in \MB{R}$, we can split this into two integrals where $x<0$ and $x\geq 0$. We have
\begin{align*}
\MC{F}(\phi)(\xi)&=\frac{1}{\sqrt{2\pi}}\int_\MB{R}e^{-|x|}e^{-ix\xi}\,dx\\[2mm]
&=\frac{1}{\sqrt{2\pi}}\int_{-\infty}^0 e^{(1-i\xi)x}\,dx+\frac{1}{\sqrt{2\pi}}\int_0^\infty e^{-(1+i\xi)x}\,dx\\[2mm]
&=\frac{1}{\sqrt{2\pi}}\left.\frac{1}{1-i\xi}e^{(1-i\xi)x}\right|_{x\to-\infty}^{x=0}-\frac{1}{\sqrt{2\pi}}\left.\frac{1}{1+i\xi}e^{-(1+i\xi)x}\right|_{x=0}^{x\to\infty}\\[2mm]
&=\frac{1}{\sqrt{2\pi}}\frac{1}{1-i\xi}-0-\left[0-\frac{1}{\sqrt{2\pi}}\frac{1}{1+i\xi}\right]\\[2mm]
&=\frac{1}{\sqrt{2\pi}}\frac{1+i\xi+1-i\xi}{(1-i\xi)(1+i\xi)}\\[2mm]
&=\frac{1}{\sqrt{2\pi}}\frac{2}{1+\xi^2}=\sqrt{\frac{2}{\pi}}\frac{1}{1+\xi^2}.
\end{align*}
For the second part of the question, we employ a few lemmas to show that $u=f*\phi$ is a solution to the ODE $u-u''=f$. Since the Fourier transform is linear (Theorem 5.4), by Theorem 8.10 in Folland's text, Problem 5 above, and Lemma 5.5 in the class notes,
\begin{align*}
\MC{F}(u-u'')&=\MC{F}(u)-\MC{F}(u'')\\[2mm]
&=\MC{F}(f*\phi)-\MC{F}((f*\phi)'')\\[2mm]
&=\MC{F}(f*\phi)-\MC{F}(f''*\phi)\\[2mm]
&=\MC{F}(f)\MC{F}(\phi)-\MC{F}(f'')\MC{F}(\phi)\\[2mm]
&=\MC{F}(f)\MC{F}(\phi)+\xi^2\MC{F}(f)\MC{F}(\phi)\\[2mm]
&=\MC{F}(f)\MC{F}(\phi)(1+\xi^2)\\[2mm]
&=\MC{F}(f)\MC{F}(\phi)\left(\frac{1}{\MC{F}(\phi)}\right)\\[2mm]
&=\MC{F}(f).
\end{align*}
Since $\MC{F}$ is an isomorphism, by Theorem 5.4 again, it follows that $u-u''=f$ if $u=f*\phi$.

For this to hold, in particular to apply Theorem 8.10 in Folland's text, we need $f$ to be $C^2$, i.e. to have two continuous derivatives.
\end{proof}











\end{enumerate}


\end{document}