\documentclass[11pt,oneside,english]{amsart}
\usepackage[T1]{fontenc}
\usepackage{geometry}
\usepackage{parskip}
\geometry{verbose,tmargin=0.65in,bmargin=0.65in,lmargin=0.75in,rmargin=0.75in,headheight=0.75cm,headsep=1cm,footskip=1cm}
\setlength{\parskip}{7mm}
\usepackage{setspace}
\onehalfspacing
\pagenumbering{gobble}

\usepackage{bbm}
\usepackage{multicol}
\usepackage{graphicx}
\usepackage{adjustbox}
\usepackage{amssymb}
\usepackage{tikz}
\usepackage{pgfplots}
\usepackage{pgffor}
\usetikzlibrary{cd}
\usepackage{ulem}
\usepackage{adjustbox}
\usepackage{bm}
\usepackage{stmaryrd}
\usepackage{cancel}
\usepackage{mathtools}
\DeclarePairedDelimiter{\ceil}{\lceil}{\rceil}
\DeclarePairedDelimiter\floor{\lfloor}{\rfloor}
\usepackage[shortlabels]{enumitem}
\setlist[enumerate,1]{label=\textbf{\arabic*.}}
\usepackage{color, colortbl}
\definecolor{Gray}{gray}{0.9}
\usepackage{babel}
\usepackage{mdframed}
\usepackage{esint}
\usepackage[yyyymmdd]{datetime}
\renewcommand{\dateseparator}{--}
\usepackage{url}
\usepackage[unicode=true,pdfusetitle,
 bookmarks=true,bookmarksnumbered=false,bookmarksopen=false,
 breaklinks=false,pdfborder={0 0 1},backref=false,colorlinks=true]
 {hyperref}
\hypersetup{urlcolor=blue}





\theoremstyle{definition}
\newtheorem{theorem}{Theorem}
\newtheorem*{theorem*}{Theorem}
\newtheorem*{proposition*}{Proposition}
\newtheorem{corollary}{Corollary}
\newtheorem*{lemma}{Lemma}
\newtheorem*{example}{Example}
\newtheorem*{examples}{Examples}
\newtheorem*{definition}{Definition}
\newtheorem*{note}{Nota Bene}

\newcommand{\aspace}{\hspace{7mm}\text{and}\hspace{7mm}}
\newcommand{\ospace}{\hspace{7mm}\text{or}\hspace{7mm}}
\newcommand{\pspace}{\hspace{10mm}}
\newcommand{\lspace}{\vspace{5mm}}
\newcommand{\lhe}{\stackrel{\text{L'H}}{=}}
\newcommand{\lom}[2]{\lim_{{#1}\rightarrow{#2}}}
\newcommand{\ve}{\varepsilon}
\renewcommand{\Re}{\text{Re }}
\renewcommand{\Im}{\text{Im }}
\newcommand{\Log}{\text{Log }}
\newcommand{\ess}{\text{ess sup}}
\newcommand{\dd}[2]{\frac{d{#1}}{d{#2}}}
\newcommand{\pp}[2]{\frac{\partial{#1}}{\partial{#2}}}
\newcommand{\DD}[2]{\frac{\Delta{#1}}{\Delta{#2}}}
\newcommand{\ovec}[1]{\overrightarrow{#1}}
\newcommand{\MC}[1]{\mathcal{#1}}
\newcommand{\MB}[1]{\mathbb{#1}}
\newcommand{\mbf}[1]{\,\mathbf{#1}}
\renewcommand{\vec}[1]{\underline{#1}}



\def\<#1>{\mathinner{\langle#1\rangle}}

\makeatletter
\g@addto@macro\normalsize{%
  \setlength\belowdisplayshortskip{5mm}
}
\makeatother





\begin{document}

\rightline{Adam D. Richardson}
\rightline{210A - Complex Analysis}
\rightline{Wong, Bun}
\rightline{HW 1}
\rightline{\today}

\lspace

p.33: 6abcd, 7; pp. 43-44: 1,3,8,12,13,14,16,17,19,21


\textbf{p. 33}


\begin{enumerate}[leftmargin=*]
\itemsep5mm
\setcounter{enumi}{5}


\item Find the radius of convergence for the following power series:

\begin{enumerate}

\itemsep5mm
\item $\displaystyle \sum_{n=0}^\infty a^nz^n$, $a\in \MB{C}$.
\[
R=\frac{1}{\limsup|a^n|^{1/n}}=\frac{1}{\limsup|a|}=\frac{1}{|a|},\quad a\neq0.
\]
If $a=0$, then $R=\infty$.

\item $\displaystyle \sum_{n=0}^\infty a^{n^2}z^n$, $a\in \MB{C}$.
\[
R=\frac{1}{\limsup|a^{n^2}|^{1/n}}=\frac{1}{\limsup|a^n|}=\frac{1}{\limsup|a|^n}=\begin{cases}\infty & \text{if }|a|<1\\ 1 & \text{if }|a|=1\\ 0 & \text{if }|a|>1.\end{cases}
\]

\item $\displaystyle \sum_{n=0}^\infty k^{n}z^n$, $k\in\MB{Z}-\{0\}$.
\[
R=\frac{1}{\limsup|k^n|^{1/n}}=\frac{1}{\limsup|k|}=\frac{1}{|k|}.
\]

\item $\displaystyle \sum_{n=0}^\infty z^{n!}$.

For this one we can use the Ratio Test:
\[
\lim\left|\frac{a_{n+1}}{a_n}\right|=\lim\left|\frac{z^{(n+1)!}}{z^{n!}}\right|=\lim |z|^{(n+1)!-n!}=\lim |z|^{n(n!)}.
\]
This limit goes to $\infty$ if $|z|>1$, so the radius of convergence is $R=1$.
\end{enumerate}

\pagebreak
\item Show that the radius of convergence of the power series 
\[
\sum_{n=1}^\infty\frac{(-1)^n}{n}z^{n(n+1)}
\]
is 1, and discuss the convergence for $z=1, -1,$ and $i$. [Hint: The $n$th coefficient of this series is not $\frac{(-1)^n}{n}$.]

\begin{proof}
First we reexpress it in the form of a power series:
\[
\sum_{n=1}^\infty\frac{(-1)^n}{n}z^{n(n+1)}=\sum_{n=1}^\infty\frac{(-1)^n}{n}z^{n^2}\cdot z^n.
\]
Now,
\[
\limsup\left|\frac{(-1)^n}{n}z^{n^2}\right|^{1/n}=\limsup\left|\frac{(-1)}{n^{1/n}}z^{n}\right|=\limsup\frac{|z|^n}{n^{1/n}}=\frac{\limsup|z|^n}{\limsup n^{1/n}},
\]
as long as the limit in the numerator exists, i.e. is not $\infty$. The limit in the denominator is 1, and in order for the limit in the numerator to exist, $|z|\leq 1$. Thus $R=1$.
\end{proof}

If $z=1$, then the series becomes the alternating harmonic series which converges by the alternating p-test. If $z=-1$, note that $n(n+1)$ is always even since one of the factors must be even, thus $(-1)^{n(n+1)}=1$ again and the series converges by the alternating p-test. If $z=i$, note that $i^{n(n+1)}=i^{2k_n}=(-1)^{k_n}$ for some positive integer $k_n$, thus after reindexing we have
\[
\sum_{n=1}^\infty\frac{(-1)^n}{n} \ospace \sum_{n=1}^\infty\frac{(-1)^{n+1}}{n}
\]
both of which converge by the alternating p-test.

\end{enumerate}





\textbf{p. 43}


\begin{enumerate}[leftmargin=*]
\itemsep5mm
%\setcounter{enumi}{5}


\item Show that $f(z)=|z|^2=x^2+y^2$ has a derivative only at the origin.

\begin{proof}
Observe that 
\begin{align*}
f'(z)&=\lom{h}{0}\frac{|z+h|^2-|z|^2}{h}\\[2mm]
&=\lom{h}{0}\frac{(z+h)(\bar z+\bar h)-z\bar z}{h}\\[2mm]
&=\lom{h}{0}\frac{z\bar z+z\bar h+\bar z h+h\bar h-z\bar z}{h}\\[2mm]
&=\lom{h}{0}z\frac{\bar h}{h}+\bar z+\bar h.
\end{align*}

If $h\to 0$ along the real axis, then $\bar h=h$ and the limit above becomes $z+\bar z$. If $h\to0$ along the imaginary axis, then, $\bar h=-h$ so the limit above becomes $-z+\bar z$. Thus, if this limit exists, then $z+\bar z=-z+\bar z$ which implies that $z=-z$, which can only happen if $z=0$. In other words, this derivative only exists at 0.
\end{proof}

\setcounter{enumi}{2}
\item Show $\lim n^{1/n}=1$.

\begin{proof}
Let $a=n^{1/n}$. Then $\log a=\frac{1}{n}\log n$, so
\[
\lom{n}{\infty}\log a=\lom{n}{\infty}\frac{\log n}{n}=\lom{x}{\infty}\frac{\log x}{x}\lhe\lom{x}{\infty}\frac{1}{x}=0.
\]
Thus,
\[
\lom{n}{\infty}n^{1/n}=\lom{n}{\infty}e^{\log a}=e^{\lim \log a}=e^0=1.
\]
\end{proof}

\setcounter{enumi}{7}
\item Define $\tan z=\frac{\sin z}{\cos z}$. Where is this function defined and analytic?

Since $\sin z$ and $\cos z$ are analytic, $\tan z$ will be analytic as long as $\cos z\neq 0$. $\cos z=0$ when $z=\frac{(2k+1)\pi}{2}$, so $\tan z$ is defined and analytic elsewhere.




\setcounter{enumi}{11}
\item Show that the real part of $f(z)=z^{1/2}$ is always positive.

We can write
\[
z^{1/2}=\left(|z|e^{i\theta}\right)^{1/2}=|z|^{1/2}e^{i\frac{\theta}{2}}=|z|^{1/2}\left(\cos\frac{\theta}{2}+i\sin\frac{\theta}{2}\right)
\]
where $-\pi<\theta<\pi$. Since the real part is given by $|z|^{1/2}\cos\frac{\theta}{2}$, it is always positive.


\item Let $G=\MB{C}-\{z\in\MB{R}\mid z\leq 0\}$ and let $n$ be a positive integer. Find all analytic functions $f:G\to \MB{C}$ such that $z=\left(f(x)\right)^n$ for all $z\in G$.

If we write $\Log z$ as the principal branch of the complex logarithm, we find

\begin{align*}
f(z)=z^{1/n}&=e^{\log(z^{1/n})}\\[2mm]
&=e^{\frac{1}{n}\log z}\\[2mm]
&=e^{\frac{1}{n}\Log z+ \frac{1}{n}i2\pi k}\\[2mm]
&=e^{\frac{1}{n}\Log z}e^{\frac{1}{n}i2\pi k}.
\end{align*}

The factor $e^{\frac{1}{n}i2\pi k}$ represents the $n$th roots of unity so our function $f$ is a constant multiple of $e^{\frac{1}{n}\Log z}$.

\item Suppose $f:G\to \MB{C}$ is analytic and $G$ is connected. Show that if $f(z)$ is real for all $z\in G$ then $f$ is constant.

\begin{proof}
Suppose $f(z)=u+iv$ is real and analytic. Then $v(x,y)=0$ for all $(x,y)\in G$, and $f$ satisfies the C-R equations, i.e.
\[
u_x=v+y=(0)_y=0 \aspace u_y=-v_x=-(0)_x=0.
\]
Thus, $u'(z)=f'(z)=0$ so $f$ is constant.
\end{proof}

\setcounter{enumi}{15}

\item Find an open connected set $G\subset \MB{C}$ and two continuous functions $f$ and $g$ defined on $G$ such that $f(z)^2=g(z)^2=1-z^2$ for all $z\in G$. Can you make $G$ maximal? Are $f$ and $g$ analytic?

Note that we can write

\[
f(z)=(1-z^2)^{1/2}=e^{\frac{1}{2}\log(1-z^2)}
\]

where log is taken on the principal branch. This function is defined as long as $|1-z^2|>0$, i.e. $z\neq \pm1$. Taking $G=\{z:|z|<1\}$ fits the bill, and we can let $g(z)=-f(z)$. This is not maximal though, and the maximal domain would be $\MB{C}-\{\pm 1\}$. If our domain must be simply connected then the maximal set would be $\MB{C}-\{z: |\Re z|\geq1,\,\Im z=0\}$. $f$ and $g$ are analytic on any of these domains.


\item Give the principal branch of $\sqrt{1-z}$.

We can write $\sqrt{1-z}=e^{\frac{1}{2}\log(1-z)}$. This is defined as long as $|1-z|>0$, i.e. $z\neq1$, so the principal branch is the complex logarithm defined on the slit plane $\MB{C}-\{z\in\MB{R}:z\geq 0\}$.

\setcounter{enumi}{18}
\item Let $G$ be a region and define $G^*=\{z:\bar z\in G\}$. If $f:G\to\MB{C}$ is analytic, prove that $f^*:G^*\to \MB{C}$, defined by $f^*(z)=\overline{f(\bar z)}$, is also analytic.

\begin{proof}
Suppose $f$ is analytic and write $f=u+iv$. Then
\[
\overline{f(\bar z)}=\overline{u(\bar z)+iv(\bar z)}=u(\bar z)-iv(\bar z)=u(x,-y)-iv(x,-y),
\]
and since $u$, $v$ satisfy the C-R equations for all $(x,y)\in G$, we have

\[
\pp{}{x}u(\bar z)=u_x(x,-y)=v_y(x,-y)=\pp{}{y}v(\bar z), \text{and}
\]
\[
\pp{}{y}u(\bar z)=-u_y(x,-y)=v_x(x,-y)=\pp{}{x}v(\bar z).
\]
Thus, the component functions $u(\bar z)$ and $v(\bar z)$ satisfy the C-R equations, from which it follows that $f^*$ is analytic.
\end{proof}


\setcounter{enumi}{20}
\item Prove there is no branch of the logarithm defined on $G=\MB{C}-\{0\}$. [Hint: suppose such a branch exists and compare this with the principal branch.]

\begin{proof}
Suppose to the contrary that there does exist a branch of the logarithm defined on $G=\MB{C}-\{0\}$. Let $H=\MB{C}-\{z\in\MB{R}:z\geq 0\}$. Consider the principal branch of the logarithm on $H$:
\[
\Log z=\ln|z|+i\arg z
\]
Suppose that $f$ is a branch of the logarithm defined on $G$. Since $H\subset G$, $f$ is also a branch of the logarithm defiend on $H$, thus $f(z)$ differs from the principal branch by a term of $2\pi i k$ for some $k$. In other words,
\[
f(z)=\ln|z|+i\arg z+2\pi i k.
\]
Since $f$ is analytic on $G$, it is continuous there, so it is continuous at $z=-1$. But by taking limits from either side of the branch cut, $\lom{z}{-1}f(z)\neq f(-1)$.
%\[
%f(-1)=\ln|-1|+i\arg (-1)+2\pi i k=0+i\pi +2\pi ik =0+i(\pi+2\pi k)=0+i(2k+1)\pi.
%\]

\end{proof}

\end{enumerate}




\end{document}