\documentclass[11pt,oneside,english]{amsart}
\usepackage[T1]{fontenc}
\usepackage{geometry}
\usepackage{parskip}
\geometry{verbose,tmargin=0.65in,bmargin=0.65in,lmargin=0.75in,rmargin=0.75in,headheight=0.75cm,headsep=1cm,footskip=1cm}
\setlength{\parskip}{7mm}
\usepackage{setspace}
\onehalfspacing
\pagenumbering{gobble}

\usepackage{bbm}
\usepackage{multicol}
\usepackage{graphicx}
\usepackage{adjustbox}
\usepackage{amssymb}
\usepackage{tikz}
\usepackage{pgfplots}
\usepackage{pgffor}
\usetikzlibrary{cd}
\usepackage{ulem}
\usepackage{adjustbox}
\usepackage{bm}
\usepackage{stmaryrd}
\usepackage{cancel}
\usepackage{mathtools}
\DeclarePairedDelimiter{\ceil}{\lceil}{\rceil}
\DeclarePairedDelimiter\floor{\lfloor}{\rfloor}
\usepackage[shortlabels]{enumitem}
\setlist[enumerate,1]{label=\textbf{\arabic*.}}
\usepackage{color, colortbl}
\definecolor{Gray}{gray}{0.9}
\usepackage{babel}
\usepackage{mdframed}
\usepackage{esint}
\usepackage[yyyymmdd]{datetime}
\renewcommand{\dateseparator}{--}
\usepackage{url}
\usepackage[unicode=true,pdfusetitle,
 bookmarks=true,bookmarksnumbered=false,bookmarksopen=false,
 breaklinks=false,pdfborder={0 0 1},backref=false,colorlinks=true]
 {hyperref}
\hypersetup{urlcolor=blue}





\theoremstyle{definition}
\newtheorem{theorem}{Theorem}
\newtheorem*{theorem*}{Theorem}
\newtheorem*{proposition*}{Proposition}
\newtheorem{corollary}{Corollary}
\newtheorem*{lemma}{Lemma}
\newtheorem*{example}{Example}
\newtheorem*{examples}{Examples}
\newtheorem*{definition}{Definition}
\newtheorem*{note}{Nota Bene}

\newcommand{\aspace}{\hspace{7mm}\text{and}\hspace{7mm}}
\newcommand{\ospace}{\hspace{7mm}\text{or}\hspace{7mm}}
\newcommand{\pspace}{\hspace{10mm}}
\newcommand{\lspace}{\vspace{5mm}}
\newcommand{\lhe}{\stackrel{\text{L'H}}{=}}
\newcommand{\lom}[2]{\lim_{{#1}\rightarrow{#2}}}
\newcommand{\ve}{\varepsilon}
\renewcommand{\Re}{\text{Re }}
\renewcommand{\Im}{\text{Im }}
\newcommand{\Log}{\text{Log }}
\newcommand{\ess}{\text{ess sup}}
\newcommand{\dd}[2]{\frac{d{#1}}{d{#2}}}
\newcommand{\pp}[2]{\frac{\partial{#1}}{\partial{#2}}}
\newcommand{\DD}[2]{\frac{\Delta{#1}}{\Delta{#2}}}
\newcommand{\ovec}[1]{\overrightarrow{#1}}
\newcommand{\MC}[1]{\mathcal{#1}}
\newcommand{\MB}[1]{\mathbb{#1}}
\newcommand{\mbf}[1]{\,\mathbf{#1}}
\renewcommand{\vec}[1]{\underline{#1}}



\def\<#1>{\mathinner{\langle#1\rangle}}

\makeatletter
\g@addto@macro\normalsize{%
  \setlength\belowdisplayshortskip{5mm}
}
\makeatother





\begin{document}

\rightline{Adam D. Richardson}
\rightline{210A - Complex Analysis}
\rightline{Wong, Bun}
\rightline{HW 4}
\rightline{\today}

\lspace



\textbf{p. 80:} 1, 3, 6, 8, 9, 10


\begin{enumerate}[leftmargin=*]
\itemsep5mm


\item Let $f$ be an entire function and suppose there is a constant $M$, and $R>0$, and an integer $n\geq 1$ such that $|f(z)|\leq M|z|^n$ for $|z|>R$. Show that $f$ is a polynomial of degree $\leq n$.

\begin{proof}
If $f$ is entire, then it has a power series expansion
\[
f(z)=\sum_{k=0}^\infty=a_0+a_1z^1+\cdots+a_nz^n+\cdots
\]
Since $|f(z)|\leq M|z|^n$ for $|z|>R$, it must be the case that $a_k=0$ for $k\geq n+1$, otherwise there exists a $z\in \MB{C}$ and a $k\geq n+1$ such that $|f(z)|>M|z|^n$ due to the higher order terms after $a_nz^n$. 

More rigorously, we know that
\[
a_k=\frac{f^{(k)(0)}}{k!}
\]
and we must show that these are all 0 for $k\geq n+1$. Let $|z|=r>R$ and let $\gamma(t)=re^{it}$. Then by a previous theorem, 
\[
a_k=\frac{f^{(k)}}{k!}=\frac{1}{2\pi i}\int_\gamma\frac{f(z)}{z^{n+1}}\,dz.
\]
Taking the absolute value,

\begin{align*}
|a_k|&=\left|\frac{f^{(k)}}{k!}\right|\\[2mm]
&=\left|\frac{1}{2\pi i}\int_\gamma\frac{f(z)}{z^{k+1}}\,dz\right|\\[2mm]
&\leq\frac{1}{2\pi}\int_\gamma\frac{|f(z)|}{|z|^{k+1}}\,dz\\[2mm]
&\leq\frac{1}{2\pi}\int_\gamma\frac{M|z|^n}{|r|^{k+1}}\,dz\\[2mm]
&=\frac{M}{2\pi}\int_\gamma |r|^{n-k-1}\,dz\\[2mm]
&=\frac{M|r|^{n-k-1}}{2\pi}\int_\gamma \,dz\\[2mm]
&=M|r|^{n-k}\\[2mm]
&=\frac{M}{r^{k-n}}\to0.
\end{align*}



Thus,
\[
f(z)=\sum_{k=0}^n=a_0+a_1z^1+\cdots+a_nz^n,
\]
i.e. $f$ is a polynomial of degree $\leq n$.
\end{proof}

\setcounter{enumi}{2}

\item Find all entire functions $f$ such that $f(x)=e^x$ for $x\in\MB{R}$.

First, let $f(z)=e^z$. It is entire and $f(x)=e^x$ for $x\in \MB{R}$. Now suppose there exists another entire function, $g(z)$ such that $g(x)=e^x$ for $x\in\MB{R}$. Now consider $Z=\{z\in\MB{C}:f(z)=g(z)\}$. Then $0$ is a limit point of $Z$ and so by Corollary 3.8, $f(z)=g(z)$ for all $z\in \MB{C}$. Thus, $f(z)=e^z$ is the unique (set) entire function such that $f(x)=e^x$ for all $x\in \MB{R}$.



\setcounter{enumi}{5}

\item Let $G$ be a region and suppose that $f:G\to\MB{C}$ is analytic and $a\in G$ such that $|f(a)|\leq |f(z)|$ for all $z\in G$. Show that either $f(a)=0$ or $f$ is constant.

\begin{proof}
Suppose $f$ is analytic in $G$ and there exists an $a\in G$ such that $|f(a)|\leq |f(z)|$ for all $z\in G$. Consider $\bar B(a,r)$ and let $\gamma(t)=a+re^{it}$. Suppose also that $f(a)\neq0$ for all $a\in G$. Define the function
\[
g(z)=\frac{1}{f(z)},
\]
on $G$, and note that it is analytic there by implication of our assumption that $f(a)\neq 0$. Moreover, 
\[
\left|\frac{1}{f(z)}\right|\leq\left|\frac{1}{f(a)}\right|
\]
since $|f(a)|\leq |f(z)|$, so by the Maximum Modulus Theorem, $g(z)$ is constant on $G$. This in turn means that $f(z)$ must be constant on $G$.
\end{proof}


\setcounter{enumi}{7}

\item Let $G$ be a region and let $f$ and $g$ be analytic functions on $G$ such that $f(z)g(z)=0$ for all $z\in G$. Show that either $f\equiv 0$ or $g\equiv 0$.

\begin{proof}
First we prove a lemma:

\textit{Lemma.} If $f(z)=k$ a constant on $B(a,R)\subset G$, then $f(z)=k$ for all $z\in G$.

\begin{proof}
Suppose $f(z)=k$ in $B(a,R)$, and write $g(z)=f(z)-k$. Then $g$ is analytic in $G$, and $g^{(n)}(a)=f^{(n)}(a)=0$ for all $n>0$, and for $n=0$, we have $g(a)=f(a)-k=k-k=0$. Thus, $g(z)=f(z)-k\equiv 0$ on $G$, i.e. $f\equiv k$ on $G$.
\end{proof}

Now suppose that $f(z)g(z)=0$ for all $z\in G$ and suppose as well that $g\not\equiv 0$ on $G$. Then there exists an $a\in G$ such that $g(a)\neq 0$. Since $g$ is analytic, there exists a radius $R$ such that $g(z)\neq 0$ for all $z\in B(a,R)$. Consequently, $f\equiv 0$ on $B(a,R)$ so that $f(z)g(z)=0$ holds there. Since $f\equiv 0$ on $B(a,R)$, by the lemma above,we have that $f\equiv 0$ on $G$. 

Supposing instead that $f\not\equiv0$ in the beginning, and using a similar argument, we have that $g\equiv 0$. Thus, $f\equiv 0$ or $g\equiv 0$.
\end{proof}

\item Let $U:\MB{C}\to\MB{R}$ be a harmonic function such that $U(z)\geq 0$ for all $z\in \MB{C}$; prove that $U$ is constant.

(ask)

\item Show that if $f$ and $g$ are analytic functions on a region $G$ such that $\bar fg$ is analytic, then either $f$ is constant or $g\equiv 0$.
\end{enumerate}

\end{document}