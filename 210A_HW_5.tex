\documentclass[11pt,oneside,english]{amsart}
\usepackage[T1]{fontenc}
\usepackage{geometry}
\usepackage{parskip}
\geometry{verbose,tmargin=0.65in,bmargin=0.65in,lmargin=0.75in,rmargin=0.75in,headheight=0.75cm,headsep=1cm,footskip=1cm}
\setlength{\parskip}{7mm}
\usepackage{setspace}
\onehalfspacing
\pagenumbering{gobble}

\usepackage{bbm}
\usepackage{multicol}
\usepackage{graphicx}
\usepackage{adjustbox}
\usepackage{amssymb}
\usepackage{tikz}
\usepackage{pgfplots}
\usepackage{pgffor}
\usetikzlibrary{cd}
\usepackage{ulem}
\usepackage{adjustbox}
\usepackage{bm}
\usepackage{stmaryrd}
\usepackage{cancel}
\usepackage{mathtools}
\DeclarePairedDelimiter{\ceil}{\lceil}{\rceil}
\DeclarePairedDelimiter\floor{\lfloor}{\rfloor}
\usepackage[shortlabels]{enumitem}
\setlist[enumerate,1]{label=\textbf{\arabic*.}}
\usepackage{color, colortbl}
\definecolor{Gray}{gray}{0.9}
\usepackage{babel}
\usepackage{mdframed}
\usepackage{esint}
\usepackage[yyyymmdd]{datetime}
\renewcommand{\dateseparator}{--}
\usepackage{url}
\usepackage[unicode=true,pdfusetitle,
 bookmarks=true,bookmarksnumbered=false,bookmarksopen=false,
 breaklinks=false,pdfborder={0 0 1},backref=false,colorlinks=true]
 {hyperref}
\hypersetup{urlcolor=blue}





\theoremstyle{definition}
\newtheorem{theorem}{Theorem}
\newtheorem*{theorem*}{Theorem}
\newtheorem*{proposition*}{Proposition}
\newtheorem{corollary}{Corollary}
\newtheorem*{lemma}{Lemma}
\newtheorem*{example}{Example}
\newtheorem*{examples}{Examples}
\newtheorem*{definition}{Definition}
\newtheorem*{note}{Nota Bene}

\newcommand{\aspace}{\hspace{7mm}\text{and}\hspace{7mm}}
\newcommand{\ospace}{\hspace{7mm}\text{or}\hspace{7mm}}
\newcommand{\pspace}{\hspace{10mm}}
\newcommand{\lspace}{\vspace{5mm}}
\newcommand{\lhe}{\stackrel{\text{L'H}}{=}}
\newcommand{\lom}[2]{\lim_{{#1}\rightarrow{#2}}}
\newcommand{\ve}{\varepsilon}
\renewcommand{\Re}{\text{Re }}
\renewcommand{\Im}{\text{Im }}
\newcommand{\Log}{\text{Log }}
\newcommand{\ess}{\text{ess sup}}
\newcommand{\dd}[2]{\frac{d{#1}}{d{#2}}}
\newcommand{\pp}[2]{\frac{\partial{#1}}{\partial{#2}}}
\newcommand{\DD}[2]{\frac{\Delta{#1}}{\Delta{#2}}}
\newcommand{\ovec}[1]{\overrightarrow{#1}}
\newcommand{\MC}[1]{\mathcal{#1}}
\newcommand{\MB}[1]{\mathbb{#1}}
\newcommand{\mbf}[1]{\,\mathbf{#1}}
\renewcommand{\vec}[1]{\underline{#1}}



\def\<#1>{\mathinner{\langle#1\rangle}}

\makeatletter
\g@addto@macro\normalsize{%
  \setlength\belowdisplayshortskip{5mm}
}
\makeatother





\begin{document}

\rightline{Adam D. Richardson}
\rightline{210A - Complex Analysis}
\rightline{Wong, Bun}
\rightline{HW 5}
\rightline{\today}

\lspace



\textbf{p. 87:} 4, 5, 6, 7, 8, 9, \textbf{p. 96:} 8, 10, 11

\lspace

\textbf{p. 87:} 4, 5, 6, 7, 8, 9

\begin{enumerate}[leftmargin=*]
\itemsep5mm

\setcounter{enumi}{3}

\item Show that Cauchy's Integral Formula follows from Cauchy's Theorem.

\begin{proof}
Suppose that Cauchy's Theorem holds. Let $G\subset\MB{C}$ be open and $f:G\to\MB{C}$ be analytic. Let $\gamma$ be a closed rectifiable curve in $G$ such that $n(\gamma;w)=0$ for all $w\in\MB{C}-G$. Let $a\in G-\{\gamma\}$. To proceed, let us define
\[
g(z)=\begin{cases}\frac{f(z)-f(a)}{z-a} & \text{if }z\neq a\\ f'(z) & \text{if } z=a.
\end{cases}
\]
$g$ is analytic in $G$, so by Cauchy's Theorem, we have
\begin{align*}
\int_\gamma g(z)\,dz&=0\\[2mm]
\int_\gamma \frac{f(z)-f(a)}{z-a}\,dz&=0\\[2mm]
\int_\gamma\frac{f(z)}{z-a}\,dz-\int_\gamma\frac{f(a)}{z-a}\,dz&=0\\[2mm]
\int_\gamma\frac{f(z)}{z-a}\,dz&=f(a)\int_\gamma\frac{1}{z-a}\,dz\\[2mm]
\int_\gamma\frac{f(z)}{z-a}\,dz&=f(a)\cdot n(\gamma; a)\cdot 2\pi i\\[2mm]
\frac{1}{2\pi i}\int_\gamma\frac{f(z)}{z-a}\,dz&=f(a)\cdot n(\gamma; a),
\end{align*}
which is the statement of Cauchy's Integral Formula.
\end{proof}

\pagebreak


\item Let $\gamma$ be a closed rectifiable curve in $\MB{C}$ and $a\not\in\{\gamma\}$. Show that for $n\geq 2$, $\displaystyle \int_\gamma \frac{1}{(z-a)^n}\,dz=0$.

\begin{proof}
Here we employ Corollary 5.9. In that statement, let $f(z)=1$, and $k\geq 1$. Then $f^{(k)}(a)=0$ for all $k\geq 1$, so
\begin{align*}
f^{(k)}(a)n(\gamma;a)&=\frac{k!}{2\pi i}\int_\gamma \frac{f(z)}{(z-a)^{k+1}}\,dz\\[2mm]
0\cdot n(\gamma;a)&=\frac{k!}{2\pi i}\int_\gamma \frac{f(z)}{(z-a)^{k+1}}\,dz\\[2mm]
0&=\int_\gamma \frac{f(z)}{(z-a)^{k+1}}\,dz\\[2mm]
0&=\int_\gamma \frac{f(z)}{(z-a)^{n}}\,dz\\[2mm]
\end{align*}
for $n\geq 2$.
\end{proof}




\item Let $f$ be analytic on $D=B(0;1)$ and suppose $|f(z)|\leq 1$ for $|z|<1$. Show $|f'(0)|\leq 1$.

\begin{proof}
Here we can just apply Cauchy's Estimate with $k=1$, $M=1$, and $R=1$:
\begin{align*}
|f^{(k)}(a)|&\leq \frac{k!M}{R}\\[2mm]
|f'(0)|&\leq\frac{1!\cdot1}{1}\\[2mm]
|f'(0)|&\leq 1.
\end{align*}
\end{proof}


\pagebreak


\item Let $\gamma(t)=1+e^{it}$ for $0\leq t\leq 2\pi$. Find $\displaystyle \int_\gamma\left(\frac{z}{z-1}\right)^n\,dz$ for all $n\in\MB{Z}^+$.

Here we can apply Corollary 5.9 since we can choose a region $G\supset \{\gamma\}$ such that $n(\gamma;w)=0$ for all $w\in \MB{C}-G$. Let $f(z)=z^n$. Then $f^{(k)}(z)=(n-k+1)!z^{n-k}$. Let $k=n-1$, and we have by Corollary 5.9,
\begin{align*}
f^{(k)}(a)n(\gamma;a)&=\frac{k!}{2\pi i}\int_\gamma \frac{f(z)}{(z-a)^{k+1}}\,dz\\[2mm]
f^{(n-1)}(1)n(\gamma;1)&=\frac{(n-1)!}{2\pi i}\int_\gamma \frac{z^n}{(z-1)^{n}}\,dz\\[2mm]
n!\cdot1&=\frac{(n-1)!}{2\pi i}\int_\gamma \left(\frac{z}{z-1}\right)^n\,dz\\[2mm]
\frac{2\pi i n!}{(n-1)!}&=\int_\gamma \left(\frac{z}{z-1}\right)^n\,dz\\[2mm]
2\pi i n&=\int_\gamma \left(\frac{z}{z-1}\right)^n\,dz.
\end{align*}




\item Let $G$ be a region and suppose $f_n:G\to\MB{C}$ is analytic for each $n\geq 1$. Suppose that $\{f_n\}$ converges uniformly to a function $f:G\to\MB{C}$. Show that $f$ is analytic.

\begin{proof}
Let $\ve>0$ be given. Since $f_n$ is analytic in $G$ for all $n$, $f_n$ is continuous in $G$ for all $n$, so by a theorem in real analysis, $f$ is continuous as well. Next, by Cauchy's Theorem, given any triangular path $T$ in $G$, we have
\[
\int_T f_n=0.
\]
Observe that
\[
\left|\int_Tf\right|\leq\int_T|f|=\int_T|f-f_n+f_n|\leq \int_T|f-f_n|+\int_T|f_n|=\int_T|f-f_n|+0<\int_T\ve=\ve\ell(T)
\]
where $\ell(T)$ is the length of $T$. Since this is true for any $\ve>0$ and $\ell(T)<\infty$, we have $\left|\int_Tf\right|=0$, and so $\int_Tf=0$ for any triangular path $T$ in $G$. By Morera's theorem, $f$ must be analytic.
\end{proof}


\pagebreak


\item Show that if $f:\MB{C}\to\MB{C}$ is a continuous function such that $f$ is analytic off $[-1,1]$, then $f$ is an entire function.


\begin{proof}
Here we use Morera's theorem again, but we have several cases based on where our triangular path $T$ is located.

\textit{Case 1:} $T$ does not intersect $[-1,1]$, and $[-1,1]$ is not surrounded by $T$. In this case, we can find an open neighborhood $G\supset \{T\}$ where $f$ is analytic. Then by Cauchy's Theorem, we have $\displaystyle \int_Tf=0$ since $T$ is closed, rectifiable, and $n(T;w)=0$ for all $w\in \MB{C}-G$.

\textit{Case 2:}  $T$ does not intersect $[-1,1]$, and $[-1,1]$ is surrounded by $T$. I am not sure what to do here.

\textit{Case 3:} $T$ intersects $[-1,1]$ at exactly 1 point. WOLOG, we may assume that $T$ intersects $[-1,1]$ at the top vertex of $T$ (the procedure is equivalent if this is not the case). Let $T_\ve$ be $T$ translated downward by $-i\ve$. Then for all $\ve >0$, $\int_{T_\ve}f=0$ by Cauchy's Theorem. Since $f$ is continuous,
\[
\int_Tf=\lom{\ve}{0}\int_{T_\ve}f=\lom{\ve}{0}0=0
\]
as well.

\textit{Case 4:} An edge of $T$ coincides with $[-1,1]$ for more than one point. In this case, we can use the same technique as in Case 3.

\textit{Case 5:} $T$ intersects $[-1,1]$ at two points. In this case, $[-1,1]$ partitions $T$ into a triangle and a trapezoid which can then also be partitioned into two triangles. These three triangles take on the same form as those covered in the previous cases.

\textit{Case 6:} $T$ contains parts of $[-1,1]$. This can also be dealt with by earlier cases and proper partitions

Since the integral of $f$ over every triangular path $T$ in $\MB{C}$ is 0, we have that $f$ is analytic in $\MB{C}$ by Morera's theorem and so it is entire.
\end{proof}
\end{enumerate}

\pagebreak



\textbf{p. 96:} 8, 10, 11



\begin{enumerate}[leftmargin=*]
\itemsep5mm

\setcounter{enumi}{7}

\item Let $G=\MB{C}-\{a,b\}$, $a\neq b$, and let $\gamma$ be the curve in the figure below.

\begin{center}
\includegraphics[scale=0.2]{path2.jpeg}
\end{center}

\begin{enumerate}
\itemsep5mm
\item Show that $n(\gamma;a)=n(\gamma;b)=0$

\begin{proof}
Demarcating the intersection points pictured, we can decompose this path into 6 paths. Focusing on $a$ first, let $\gamma_1$ and $\gamma_2$ be as pictured. Clearly, $\gamma_1\sim C_1$ and $\gamma_2\sim C_2$ where $C_1(t)=a+re^{it}$, $0\leq t\leq \pi$ and $C_2(t)=a+re^{it}$, $\pi\leq t\leq 2\pi$. Then
\[
\int_{C_1}\frac{1}{z-a}\,dz=\int_0^\pi\frac{1}{\cancel{re^{it}}}\cdot i\cancel{re^{it}}\,dt=\pi i=\int_{C_2}\frac{1}{z-a}\,dz.
\]
Let $\sigma=\gamma_1-\gamma_2$. $\gamma$ is a closed curve, and we have
\[
\int_\sigma \frac{1}{z-a}\,dz=\int_{\gamma_1-\gamma_2}\frac{1}{z-a}\,dz=\int_{\gamma_1}\frac{1}{z-a}\,dz+\int_{-\gamma_2}\frac{1}{z-a}\,dz=\int_{\gamma_1}\frac{1}{z-a}\,dz-\int_{\gamma_2}\frac{1}{z-a}\,dz=\pi i-\pi i=0.
\]
Consequently, $n(\sigma;a)=0$. A similar argument reveals the same for the respective $n(\sigma;b)$. For the remaining components of out path $\gamma$, a similar argument holds as well, whence $n(\gamma;a)=n(\gamma;b)=0$.
\end{proof}

\item Convince yourself that $\gamma$ is not homotopic to zero.

(done)
\end{enumerate}

\pagebreak

\setcounter{enumi}{9}

\item Find all possible values of $\displaystyle \int_\gamma \frac{dz}{1+z^2}$ where $\gamma$ is any closed rectifiable curve in $\MB{C}$ not passing through $\pm i$.

Here we have a few cases: (i) $\gamma$ surrounds $i$, (ii) $\gamma$ surrounds $-i$, (iii) $\gamma$ surrounds both $\pm i$, (iv) $\gamma$ does not surround either $\pm i$. Since $\gamma$ is a closed rectifiable curve, it is homotopic to a circle, so we can apply Exercise 9(c) from HW3. In that problem, partial fraction decomposition yields
\[
\int_\gamma\frac{1}{z^2+1}\,dz=\frac{1}{2i}\int_\gamma\frac{1}{z-i}\,dz-\frac{1}{2i}\int_\gamma \frac{1}{z+i}\,dz=\pi -\pi,
\]
but since $\gamma$ may not be a circle here, Cauchy's integral formula gives us that
\[
\int_\gamma\frac{1}{z^2+1}\,dz=n(\gamma; i)\pi -n(\gamma;-i)\pi.
\]


In case (i), the integral will be $n(\gamma;i)\pi$, in case (ii) the integral will be $n(\gamma;-i)\pi$, in case (iii) the integral will be

\item Evaluate $\displaystyle \int_\gamma \frac{e^z-e^{-z}}{z^4}\,dz$ where $\gamma$ is one of the curves depicted below. (Justify your answer.)

\begin{center}
\includegraphics[scale=0.5]{path3.png}
\end{center}

There are a few ways to do this problem. One way involves decomposing each path into a union of paths that are homotopic to either a circle or 0, then applying the result of Exercise 9(a) in HW3. That exercise yielded that, for the unit circle centered at 0, we have

\[
\int_\gamma \frac{e^z-e^{-z}}{z^m}\,dz=\begin{cases}0 & \text{if $m$ is odd} \\ \frac{4\pi i}{(m-1)!} & \text{if $m$ is even,}\end{cases}
\]
i.e.
\[
\int_\gamma \frac{e^z-e^{-z}}{z^4}\,dz=\begin{cases}0 & \text{if $m$ is odd} \\ \frac{2\pi i}{3} & \text{if $m$ is even.}\end{cases}
\]
The path in (a) can be decomposed into a path homotopic to 0 and another that is homotopic to a circle. Thus, the integral is a sum of the integrals over these two paths, one of which is 0, and the other of which is $\frac{2\pi}{3}n(\gamma_0;0)=\frac{2\pi}{3}$.

The path in part (b) is homotopic to a circle traversed twice, so we have that the integral is $2\cdot\frac{2\pi}{3}=\frac{4\pi}{3}$.

The path in part (c) can be decomposed ito three path,s two of which are homotopic to 0, and one which is homotopic to the unit circle centered at 0, yielding that the integral is $\frac{2\pi}{3}$.

If this type of reasoning is not rigorous enough, we turn to Corollary 5.9 on p. 86 which yields
\[
\int_\gamma\frac{f(z)}{(z-a)^{k+1}}\,dz=2\pi i\cdot \frac{f^{(k)}(a)}{k!}\cdot n(\gamma;a).
\]
Letting $f(z)=e^z-e^{-z}$ and $k=3$ gives $f^{(3)}(0)=e^0+e^{-0}=2$, whence
\[
\int_\gamma\frac{e^z-e^{-z}}{z^{4}}\,dz=2\pi i\cdot \frac{2}{3!}\cdot n(\gamma;0)=\frac{2\pi i}{3}\cdot n(\gamma;0).
\]
Deducing the winding numbers from inspection as before yields the same results given above.
\end{enumerate}





\end{document}