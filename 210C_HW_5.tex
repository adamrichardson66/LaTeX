\documentclass[11pt,oneside,english,reqno]{amsart}
\usepackage[T1]{fontenc}
\usepackage{geometry}
\usepackage{parskip}
\geometry{verbose,tmargin=0.65in,bmargin=0.65in,lmargin=0.75in,rmargin=0.75in,headheight=0.75cm,headsep=1cm,footskip=1cm}
\setlength{\parskip}{7mm}
\usepackage{setspace}
\onehalfspacing
\pagenumbering{gobble}

\usepackage{bbm}
\usepackage{multicol}
\usepackage{graphicx}
\usepackage{adjustbox}
\usepackage{amssymb}
\usepackage{tikz}
\usepackage{pgfplots}
\usepackage{pgffor}
\usetikzlibrary{cd}
\usepackage{ulem}
\usepackage{adjustbox}
\usepackage{bm}
\usepackage{stmaryrd}
\usepackage{cancel}
\usepackage{mathtools}
\DeclarePairedDelimiter{\ceil}{\lceil}{\rceil}
\DeclarePairedDelimiter\floor{\lfloor}{\rfloor}
\usepackage[shortlabels]{enumitem}
\setlist[enumerate,1]{label=\textbf{\arabic*.}}
\usepackage{color, colortbl}
\definecolor{Gray}{gray}{0.9}
\usepackage{babel}
\usepackage{mdframed}
\usepackage{esint}
\usepackage[yyyymmdd]{datetime}
\renewcommand{\dateseparator}{--}
\usepackage{url}
\usepackage[unicode=true,pdfusetitle,
 bookmarks=true,bookmarksnumbered=false,bookmarksopen=false,
 breaklinks=false,pdfborder={0 0 1},backref=false,colorlinks=true]
 {hyperref}
\hypersetup{urlcolor=blue}





\theoremstyle{definition}
\newtheorem{theorem}{Theorem}
\newtheorem*{theorem*}{Theorem}
\newtheorem*{proposition*}{Proposition}
\newtheorem{corollary}{Corollary}
\newtheorem*{lemma}{Lemma}
\newtheorem*{example}{Example}
\newtheorem*{examples}{Examples}
\newtheorem*{definition}{Definition}
\newtheorem*{note}{Nota Bene}

\newcommand{\aspace}{\hspace{7mm}\text{and}\hspace{7mm}}
\newcommand{\ospace}{\hspace{7mm}\text{or}\hspace{7mm}}
\newcommand{\pspace}{\hspace{10mm}}
\newcommand{\lspace}{\vspace{5mm}}
\newcommand{\lhe}{\stackrel{\text{L'H}}{=}}
\newcommand{\lom}[2]{\lim_{{#1}\rightarrow{#2}}}
\newcommand{\ve}{\varepsilon}
\renewcommand{\Re}{\text{Re }}
\renewcommand{\Im}{\text{Im }}
\newcommand{\Log}{\text{Log }}
\newcommand{\ess}{\text{ess sup}}
\newcommand{\dd}[2]{\frac{d{#1}}{d{#2}}}
\newcommand{\pp}[2]{\frac{\partial{#1}}{\partial{#2}}}
\newcommand{\DD}[2]{\frac{\Delta{#1}}{\Delta{#2}}}
\newcommand{\ovec}[1]{\overrightarrow{#1}}
\newcommand{\MC}[1]{\mathcal{#1}}
\newcommand{\MB}[1]{\mathbb{#1}}
\newcommand{\mbf}[1]{\,\mathbf{#1}}
\renewcommand{\vec}[1]{\underline{#1}}
\newcommand{\Res}{\text{Res}}
\newcommand{\im}{\text{im\,}}
\newcommand{\Hom}{\text{Hom}}
\newcommand{\coker}{\text{coker\,}}
\newcommand{\ev}{\text{ev}}


\def\<#1>{\mathinner{\langle#1\rangle}}

\makeatletter
\g@addto@macro\normalsize{%
  \setlength\belowdisplayshortskip{5mm}
}
\makeatother





\begin{document}

\rightline{Adam D. Richardson}
\rightline{210C - Riemann Surfaces}
\rightline{Wong, Bun}
\rightline{HW 5}
\rightline{\today}

\lspace




\begin{enumerate}[leftmargin=*]
\itemsep5mm

\item (p. 117) Let $p_1,\ldots,p_d$ be  $d$ distinct points on a compact Riemann surface $X$. Then $\Res_{p_j}f\in T_{p_j}(X)$ for $1\leq j\leq d$ are the residues of a meromorphic function $f$ if and only if
\[
\sum_{j=1}^d\<\Res_{p_j}f,\theta( p_j)>=0
\]
for all holomorphic 1-forms $\theta$.

\begin{proof}
Firrst, suppose that $f$ is a meromorphic function on $X$ with simple poles at $p_1,\ldots, p_d$, and let $\theta$ be a holomorphic 1-form on $X$. Then there exists a holomorphic function $g$ on $X$ such that $\theta=g\,dz$ by definition. Now, $f\theta$ is a meromorphic 1-form, and


\begin{align*}
\<\Res_{p_j}f,\theta(p_j)>&=\iint_X\Res_{p_j}f\wedge\theta(p_j)\\
&=\iint_Xa_{-1}^j\pp{}{z_j}\wedge g_j(p_j)\,dz_j\\
&=a_{-1}^j\iint_X\left.\pp{g_j}{z_j}\right|_{p_j}\,dz_j\\
&=a_{-1}^j\cdot g_j(p_j)\\
&=\Res_{p_j}(f\,dz_j)\cdot g_j(p_j)\\
&=\Res_{p_j}(f\cdot g_j(p_j)\,dz_j)\\
&=\Res_{p_j}(f\theta)\\
\end{align*}
Stokes' theorem implies that $\sum_{j=1}^d\Res_{p_j}(f\theta)=0$ so we have that $\sum_{j=1}^d\<\Res_{p_j}f,\theta(p_j)>=0$ as well. 

Conversely, if $\sum_{j=1}^d\<\Res_{p_j}f,\theta( p_j)>=0$ for all holomorphic 1-forms $\theta$, then the set $T=\{\Res_{p_j}f\}_{j=1}^d\subseteq \ker \uline{A}$, where
\[
\uline{A}:\bigoplus TX_{p_i}\to H^{0,1}.
\]
By the exactness of the sequence on p. 116, $T$ is in the image of the residue map $R:H^0(D)\to\bigoplus_jTX_{p_j}$, therefore there exists a meromorphic function with residues given by $T$.
\end{proof}

\item Use the Riemann-Roch formula to construct a meromorphic function on $\Sigma_0$ (compact Riemann surface of genus $g=0$) with a simple pole at one point.

\begin{proof}
Let $D=\{p\}$. By the Riemann-Roch formula, we have
\[
h^0(D)-h^1(D)=d-g+1=d+1\pspace \iff \pspace h^0(D)=h^1(D)+d+1\geq1.
\]
The space of holomorphic functions (functions with not poles) has dimension 1, so it must be the case that $h^(0)\geq2$. Thus there exists at least one meromorphic function on $\Sigma_0$ with a simple pole at $p$.
\end{proof}

\pagebreak

\item Use the Riemann-Roch formula to construct a meromorphic function on $\Sigma_1$ with simple poles at two distinct points. 

Let $D=\{p_1,p_2\}$. By the argument above, $h^0(D)\geq2$ so there are two cases: there exists a meromorphic function with a simple pole at $p_1$ or $p_2$ but not both, or there exists a meromorphic function with simple poles at $p_1$ and $p_2$. If there exists a meromorphic function with a simple pole at $p_1$ or $p_2$ but not both, then $\Sigma_1\cong S^2$, a contradiction, so we must have the second case.
%If $d=0$, then $H^0(D)$ would be nonempty but also any $f\in H^0(D)$ would be constant, meaning the poles in $D$ would not be isolated, which is a contradiction. Therefore, $d\geq 1$, so there exists a meromorphic function on $\Sigma_1$ with simple poles at $p_1,p_2$.

\end{enumerate}
\end{document}
