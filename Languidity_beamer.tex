\documentclass[11pt,english,
handout
]{beamer}
\usepackage[T1]{fontenc}
\usepackage[latin9]{inputenc}

\setcounter{secnumdepth}{3}
\setcounter{tocdepth}{3}
\usepackage{babel}
  

\setlength{\parskip}{\medskipamount}
\setlength{\parindent}{0pt}

\usepackage{comment}
\usepackage{bbm}
\usepackage{multicol}
%\usepackage{graphicx}
%\usepackage{adjustbox}
%\usepackage{amssymb}
\usepackage{tikz,tikz-3dplot}
\usetikzlibrary{arrows}
\usetikzlibrary{3d}
\usetikzlibrary{lindenmayersystems}
\pgfdeclarelindenmayersystem{cantor set}{
  \rule{F -> FfF}
  \rule{f -> fff}
}

\usepackage{pgfplots}
\usepackage{pgffor}
%\usetikzlibrary{cd}
\usepackage{ulem}
\usepackage[font=tiny]{caption}
\usepackage{subcaption}
%\usepackage{adjustbox}
\usepackage{bm}
%\usepackage{stmaryrd}
\usepackage{cancel}
\usepackage{mathtools}
\usepackage{commath}
%\DeclarePairedDelimiter{\ceil}{\lceil}{\rceil}
%\DeclarePairedDelimiter{\floor}{\lfloor}{\rfloor}
%\usepackage[shortlabels]{enumitem}
%\setlist[enumerate,1]{label=\textbf{\arabic*.}}
%\usepackage{color, colortbl}
%\definecolor{Gray}{gray}{0.9}
%\usepackage{babel}
\usepackage{mdframed}

%\usepackage{esint}
%\usepackage[yyyymmdd]{datetime}
%\renewcommand{\dateseparator}{--}
%\usepackage{url}
%\usepackage[unicode=true,pdfusetitle,
% bookmarks=true,bookmarksnumbered=false,bookmarksopen=false,
% breaklinks=false,pdfborder={0 0 1},backref=false,colorlinks=true]
% {hyperref}
%\hypersetup{urlcolor=blue}
\hypersetup{colorlinks,linkcolor=,urlcolor=blue}
\usepackage{pdfpages}

\usepackage{amsthm}
\theoremstyle{definition}
%\newtheorem{theorem}{Theorem}
%\newtheorem*{theorem*}{Theorem}
\newtheorem{conjecture}{Conjecture}
\newtheorem{proposition}{Proposition}
%\newtheorem{lemma}{Lemma}
\newmdtheoremenv{conj}{Theorem}
%\newtheorem{corollary}{Corollary}
%\newtheorem*{lemma}{Lemma}
\newtheorem*{highlight}{Note}
\newtheorem*{highlight2}{Lemma}
%\newtheorem*{example}{Example}
%\newtheorem*{examples}{Examples}
%\newtheorem*{definition}{Definition}
%\newtheorem*{note}{Note}

\newcommand{\aspace}{\hspace{7mm}\text{and}\hspace{7mm}}
\newcommand{\ospace}{\hspace{7mm}\text{or}\hspace{7mm}}
\newcommand{\pspace}{\hspace{10mm}}
\newcommand{\lspace}{\vspace{5mm}}
\newcommand{\lhe}{\stackrel{\text{L'H}}{=}}
\newcommand{\lom}[2]{\lim_{{#1}\rightarrow{#2}}}
\newcommand{\ve}{\varepsilon}
\newcommand{\dd}[2]{\frac{d{#1}}{d{#2}}}
\newcommand{\pp}[2]{\frac{\partial{#1}}{\partial{#2}}}
\newcommand{\DD}[2]{\frac{\Delta{#1}}{\Delta{#2}}}
\newcommand{\ovec}[1]{\overrightarrow{#1}}
\newcommand{\MC}[1]{\mathcal{#1}}
\newcommand{\MB}[1]{\mathbb{#1}}
\renewcommand{\vec}[1]{\underline{#1}}
\renewcommand{\Re}{\text{Re}}
\renewcommand{\Im}{\text{Im}}
\newcommand{\mbf}[1]{\mathbf{#1}}
\renewcommand{\qedsymbol}{\textcolor{black}{\openbox}}
\newcommand{\res}{\text{res}}
\DeclareMathOperator{\sgn}{sgn}
\DeclarePairedDelimiter{\ceil}{\lceil}{\rceil}



 \newcommand\makebeamertitle{\frame{\maketitle}}%



\makeatletter
 \AtBeginDocument{%
   \let\origtableofcontents=\tableofcontents
   \def\tableofcontents{\@ifnextchar[{\origtableofcontents}{\gobbletableofcontents}}
   \def\gobbletableofcontents#1{\origtableofcontents}
 }
\makeatother




\numberwithin{equation}{section}
\numberwithin{figure}{section}
%  \theoremstyle{plain}
%  \newtheorem*{thm*}{\protect\theoremname}
%  \theoremstyle{remark}
%  \newtheorem*{claim*}{\protect\claimname}
%  \theoremstyle{plain}
%  \newtheorem*{prop*}{\protect\propositionname}




\makeatletter
\newenvironment<>{proofs}[1][\proofname]{%
    \par
    \def\insertproofname{#1\@addpunct{.}}%
    \usebeamertemplate{proof begin}#2}
  {\usebeamertemplate{proof end}}
\makeatother



%Footer color controls, opacity in particular.
\definecolor{beamer@ColorIPN}{RGB}{45,108,192}
\setbeamercolor{myfootlinetext}{fg=beamer@ColorIPN!70}


%Footer information controls.
\usenavigationsymbolstemplate{}
\usepackage{textcomp}
\useoutertheme{infolines}
\setbeamertemplate{footline}
{
  \leavevmode%
  \hbox{%
\begin{beamercolorbox}[wd=.25\paperwidth,ht=2.25ex,dp=1ex,center]{myfootlinetext}%
    \usebeamerfont{author in head/foot}\insertauthor%Copyright symbol and author here.
  \end{beamercolorbox}%
  \begin{beamercolorbox}[wd=.55\paperwidth,ht=2.25ex,dp=1ex,center]{myfootlinetext}%
    \usebeamerfont{title in head/foot}{Oscillatory Behavior of RFDs Associated with Space-Filling Curves} %Title of document.
  \end{beamercolorbox}%
  \begin{beamercolorbox}[wd=.20\paperwidth,ht=2.25ex,dp=1ex,right]{myfootlinetext}%
    %\usebeamerfont{date in head/foot}\insertshortdate{}\hspace*{2em} %Date
    \insertframenumber{} / \inserttotalframenumber{}\hspace*{2ex} %Frame counter
  \end{beamercolorbox}}
}





\usefonttheme[onlymath]{serif}
\definecolor{textblue}{RGB}{52, 57, 176}

%\setbeameroption{show notes}




\AtBeginDocument{
  \def\labelitemi{\(\Rrightarrow\)}
}





\definecolor{UCRblue}{RGB}{45,108,192}
\definecolor{UCRgray}{RGB}{57,59,65}
\definecolor{UCRgold}{RGB}{241,171,0}
\definecolor{UCRgoldfade}{RGB}{255,244,218}
\setbeamercolor{title}{fg=UCRblue}
\setbeamercolor{frametitle}{fg=UCRblue}
\setbeamercolor{structure}{fg=UCRblue}
\setbeamercolor{block title example}{fg=UCRgold}






\usebackgroundtemplate{%
\begin{tikzpicture}
\node[anchor=south east, opacity=0.07] at (1,0) {\includegraphics[scale=0.4]{ucr_seal_RGB_blue}};
\end{tikzpicture}
}
   
   
   
   


\newlength{\bibitemsep}\setlength{\bibitemsep}{.2\baselineskip plus .05\baselineskip minus .05\baselineskip}
\newlength{\bibparskip}\setlength{\bibparskip}{0pt}
\let\oldthebibliography\thebibliography
\renewcommand\thebibliography[1]{%
  \oldthebibliography{#1}%
  \setlength{\parskip}{\bibitemsep}%
  \setlength{\itemsep}{\bibparskip}%
}











 
   
  
\begin{document}

\title{The Oscillatory Behavior of Relative Fractal Drums Associated with a Class of Space-Filling Curves}

\author{Adam D. Richardson}
%\institute{University of California, Riverside \\ AMS Western Sectional Meeting 2021}
%\institute{University of California, Riverside \\ FRG Seminar}
\institute{University of California, Riverside \\ MPDS Seminar}
%\institute{University of California, Riverside \\ Oral Examination}
\makebeamertitle
















\begin{frame}{Overview}
\footnotesize
\begin{multicols}{2}
\begin{itemize}
\itemsep5mm
\item Defining a Class of Space-Filling Curves
\item Theory of Complex Dimensions in $\MB{R}$
\item Theory of Complex Dimensions in $\MB{R}^N$
\item A Theorem about the Complex Dimensions of a Class of Space-Filling Curves
\item A Model of the Hilbert RFD
\item Geometric Oscillation of Points
\item Geometric Oscillation in the Volume of the Tubular Neighborhood
\item Languidity
\item Recovery of a Fractal Tube Formula from a General Theorem
\end{itemize}
\end{multicols}
\end{frame}
















\begin{frame}{Plane-Filling Curves}
\small

\begin{itemize}
\itemsep5mm
\item In 1890 Giuseppe Peano (1858 -- 1932) constructed the first plane-filling curve by proving that there exists a continuous, surjective mapping from the unit interval to the unit square. Peano gave no geometric interpretation of the curve in his paper.

\item In 1891 David Hilbert (1862 -- 1943) produced a geometric interpretation, and gave a generalized generating principle for such curves after recognizing that Peano's construction yielded a triadic correspondence that mapped adjacent subintervals of $I=[0,1]$ onto adjacent subsquares of $I^2=[0,1]^2$ in an ordered traverse.
\end{itemize}

\end{frame}
















\begin{frame}{Defining a Class of Plane-Filling Curves}
\small
\begin{definition}
A \textbf{space-filling function} is a function that maps a $1$-dimensional interval onto an $N$-dimensional space for $N\geq0$. %A \textbf{plane-filling function} is a function that maps a 1-dimensional interval onto an $2$-dimensional area.
\end{definition}

\lspace
\begin{definition}
A \textbf{space-filling curve} is the image of a space-filling function. 
\end{definition}

\lspace
\textbf{Note:} We will restrict ourselves to a specific class of plane-filling curves ($N=2$) as described below.
\end{frame}











%
%
%\begin{frame}{Defining a Class of Plane-Filling Curves}
%\small
%\begin{itemize}
%\itemsep5mm
%\item We will be considering sequences $f_n:[0,1]\to[0,1]^2$ where the image of the limit function $f$ is the entire unit square.\pause
%\item Our curves are constructed as follows: 
%
%\vspace{3mm}
%First, let $\lambda\in \MB{N}$ with $\lambda\neq1$ and partition the unit square into $\lambda^{2n}$ subsquares 
%
%of side length $\left(\frac{1}{\lambda}\right)^n$ for each $n\in \MB{N}$.
%\end{itemize}
%
%\end{frame}





\begin{frame}{Defining a Class of Plane-Filling Curves}
\begin{minipage}[t][3cm]{\textwidth}
\footnotesize
\begin{itemize}
\itemsep3mm
\item We will be considering sequences $f_n:[0,1]\to[0,1]^2$ where the image of the limit function $f$ is the entire unit square.
\item Our curves are constructed as follows: 

\vspace{2mm}
Let $\lambda\in \MB{N}$ with $\lambda\neq1$ and partition the unit square into $\lambda^{2n}$ subsquares of

side length $\left(\frac{1}{\lambda}\right)^n$ for each $n\in \MB{N}$.
\end{itemize}
\end{minipage}
\begin{minipage}[t][8cm]{\textwidth}
\begin{center}
\visible<1->{\begin{figure}
\includegraphics[scale=0.3]{hilbert_grid.png}
\caption{The first three partitions of $I^2=[0,1]^2$ with $\lambda=2$}
\end{figure}}
\end{center}
\end{minipage}
\end{frame}













\begin{frame}{Defining a Class of Plane-Filling Curves}
\begin{minipage}[t][3cm]{\textwidth}
\small
\begin{itemize}
\itemsep4mm
\visible<1->{\item In each generation determined by $n$, we connect the centers of every subsquare by a polygonal path which does not intersect itself. This polygonal path is an $\mathbf{n}$\textbf{th approximation} of the curve.}
\visible<1->{\item The plane-filling curve itself is the \textit{limit} of these polygonal paths.}
\end{itemize}
\end{minipage}
\begin{minipage}[t][8cm]{\textwidth}
\begin{center}
\visible<1->{\begin{figure}
\includegraphics[scale=0.3]{hilbert1.png}
\caption{The first three approximations of the Hilbert curve}
\end{figure}}
\end{center}
\end{minipage}
\end{frame}




















\begin{frame}{Defining a Class of Plane-Filling Curves}
\begin{minipage}[t][3cm]{\textwidth}
\small
\begin{itemize}
\itemsep4mm
\item This produces a sequence of continuous functions that is uniformly convergent, which yields an existent, continuous limit curve. 
\visible<1->{\item This construction does not yield an injective map, but it does ensure that, given any point in the unit interval, its images under successive approximations always lie within the previous subsquares.}
\end{itemize}
\end{minipage}
\begin{minipage}[t][8cm]{\textwidth}
\begin{center}
\visible<1->{\begin{figure}
\includegraphics[scale=0.3]{hilbert1.png}
\caption{The first three approximations of the Hilbert curve}
\end{figure}}
\end{center}
\end{minipage}

\end{frame}

















\begin{frame}{Defining a Class of Plane-Filling Curves}
\begin{minipage}[t][3cm]{\textwidth}
\small
\begin{itemize}
\itemsep3mm
\item The plane-filling property follows from the surjectivity of the map $f$.
\visible<1->{\item The surjectivity of the map arises from density and compactness: every point in $I^2$ lies in the closure of the image of $f$, and this image is both dense and compact, hence equal to its closure.}
\end{itemize}
\end{minipage}
\begin{minipage}[t][8cm]{\textwidth}
\begin{center}
\visible<1->{\begin{figure}
\includegraphics[scale=0.3]{hilbert1.png}
\caption{The first three approximations of the Hilbert curve}
\end{figure}}
\end{center}
\end{minipage}

\end{frame}


















\begin{frame}{What is a Fractal?}
\small
``A fractal is by definition a set for which the Hausdorff-Besicovitch dimension strictly exceeds the topological dimension.'' 
\begin{flushright}
\vspace{-2mm}
--- Benoit Mandelbr\"{o}t

\textit{The Fractal Geometry of Nature}, 1982
\end{flushright}

``[The definition's] generality was to prove excessive: not only awkward but genuinely inappropriate. [...] This definition left out many `borderline fractals', yet it took care, more or less, of the frontier `against' Euclid. But the frontier `against' true geometric chaos was left wide open! I know definitions matter little, but this one can still be improved upon.''
\begin{flushright}
--- Benoit Mandelbr\"{o}t

\textit{The Beauty of Fractals} (Peitgen \& Richter), 1986
\end{flushright}
\end{frame}
















\begin{frame}{What is a Fractal?}
\footnotesize

``My personal feeling is that the definition of a `fractal' should be regarded in the same way as a biologist regards the definition of `life'. There is no hard and fast definition, but just a list of properties characteristic of a living thing, such as the ability to reproduce or to move or to exist to some extent independently of the environment.''
\begin{flushright}
\vspace{-2mm}
--- Kenneth Falconer

\textit{Fractal Geometry: Mathematical Foundations and Applications}, 1990
\end{flushright}

\lspace
\begin{highlight}
These `borderline fractals' are a bit more than borderline. Examples include the Cantor function (the Devil's Staircase) and \textit{space-filling curves}. Plane-filling curves fill the square (the plane) so their Hausdorff dimension and topological dimension are both 2, classifying them as \textit{not} fractal under Mandelbr\"{o}t's definition.
%Previous definitions are not entirely accurate since they \textit{exclude} objects that are widely considered to be fractals, like the Cantor function (the Devil's Staircase) and \textit{space-filling curves} in particular.
\end{highlight}


%\vspace{2mm}
%``A set is called fractal if it has at least one nonreal complex dimension.''
%\begin{flushright}
%--- Michel L. Lapidus, Machiel v. Frankenhuijsen
%
%\textit{Fractal Geometry, Complex Dimensions and Zeta Functions}, 2006
%\end{flushright}
\end{frame}



















\begin{frame}{The Theory of Complex Dimensions in $\MB{R}$}
\small
\begin{itemize}
\itemsep3mm
\item Can one hear the shape of a \textit{fractal} drum?
\item Fractal Geometry, Complex Dimensions and Zeta Functions (FGCD)
\item A fractal drum in $\MB{R}^1$: the Cantor String ($CS$)
\end{itemize}

\begin{center}
\begin{tikzpicture}[scale=0.9]
\draw[thick] (0,-0.1) -- (0,0.1) node[above]{$0$};
\draw[thick] (10,-0.1) -- (10,0.1) node[above]{$1$};
\foreach \order in {0,...,6}
    \draw[thick, yshift=-\order*7mm]  l-system[l-system={cantor set, axiom=F, order=\order, step=100mm/(3^\order)}];
\end{tikzpicture}
\end{center}
\end{frame}

















\begin{frame}{The Theory of Complex Dimensions in $\MB{R}$}
\small
\begin{itemize}
\itemsep3mm
\item Can one hear the shape of a \textit{fractal} drum?
\item Fractal Geometry, Complex Dimensions and Zeta Functions (FGCD)
\item A fractal drum in $\MB{R}^1$: the Cantor String ($CS$)
\item The geometric zeta function of a fractal string $\MC{L}=\{l_j\}_{j=1}^\infty$:
\[
\zeta_\MC{L}(s)=\sum_{j=1}^\infty l_j^s
\]\vspace*{-\baselineskip}
\item The set of \textbf{complex dimensions} of $CS$ is the set of singularities of the meromorphic continuation of this zeta function.
\item Nonreal complex dimensions indicate the presence of geometric oscillations in $\MC{L}$ and the tubular neighborhood of $\MC{L}$.
\end{itemize}

\end{frame}



























\begin{frame}{The Theory of Complex Dimensions in $\MB{R}$}
\begin{minipage}{0.6\textwidth}
\scriptsize
\begin{itemize}
\itemsep5mm
\item The geometric zeta function of the Cantor string is
\[
\zeta_{CS}(s)=\sum_{n=0}^\infty2^n\cdot3^{-(n+1)s}=\frac{3^{-s}}{1-2\cdot3^{-s}}.
\]
\item Thus the set of complex dimensions of the Cantor string is

\[
\MC{D}_{CS}=\{D+in\mbf{p}\}
\]

\vspace{3mm}
\linespread{1.6}\selectfont
where $D=\log_32$, the Minkowski dimension of $CS$, $n\in\MB{Z}$, and $\mbf{p}=\frac{2\pi}{\log3}$, the \textit{oscillatory period} of $CS$.
\end{itemize}
\end{minipage}\hspace{5mm}
\begin{minipage}{0.34\textwidth}
\begin{figure}
\includegraphics[scale=0.33]{CS_poles_trans.png}
\caption{The complex dimensions of $CS$ [FGCD]}
\end{figure}
\end{minipage}
\end{frame}












\begin{frame}{The Theory of Complex Dimensions in $\MB{R}$}

\begin{minipage}[t][4.3cm]{\textwidth}
\begin{figure}
\includegraphics[scale=0.5]{CS_tubular_nbd.png}
\caption{The inner $\ve$-tubular neighborhood of the Cantor string [FGCD]}
\end{figure}
\end{minipage}

\begin{minipage}[t][5cm]{\textwidth}
\footnotesize
\begin{itemize}
\itemsep3mm
\item Through direct computation, one finds the inner $\ve$-tubular neighborhood of $CS$ is
\[
V_{CS}(\ve)=(2\ve)^{1-D}\left(\left(\frac{1}{2}\right)^{\{-\log_3(2\ve)\}}+\left(\frac{3}{2}\right)^{\{-\log_3(2\ve)\}}\right)-2\ve
\]
\item This function is multiplicatively periodic.
\end{itemize}
\end{minipage}
\end{frame}























\begin{frame}{The Theory of Complex Dimensions in $\MB{R}$}

\begin{minipage}[t][4.3cm]{\textwidth}
\begin{figure}
\includegraphics[scale=0.5]{CS_tubular_nbd.png}
\caption{The inner $\ve$-tubular neighborhood of the Cantor string [FGCD]}
\end{figure}
\end{minipage}

\begin{minipage}[t][5cm]{\textwidth}
\footnotesize
\begin{itemize}
\itemsep3mm
\item Computing its Fourier series we have
\[
V_{CS}(\ve)=\frac{1}{2\log 3}\sum_{n=-\infty}^\infty\frac{(2\ve)^{1-D-in\mbf{p}}}{(D+in\mbf{p})(1-D-in\mbf{p})}-2\ve
\]
so we can see that there are oscillations in the volume of the tubular neighborhood.
\end{itemize}
\end{minipage}
\end{frame}





















\begin{frame}{The Theory of Complex Dimensions in $\MB{R}$}

\begin{minipage}[t][4.3cm]{\textwidth}
\begin{figure}
\includegraphics[scale=0.5]{CS_tubular_nbd.png}
\caption{The inner $\ve$-tubular neighborhood of the Cantor string [FGCD]}
\end{figure}
\end{minipage}

\begin{minipage}[t][5cm]{\textwidth}
\footnotesize
\begin{itemize}
\itemsep3mm
\item Moreover, we can write this formula in terms of the complex dimensions:

\[
V_{CS}(\ve)=\frac{1}{2\log 3}\sum_{\omega\in\MC{D}_{CS}}\frac{(2\ve)^{1-\omega}}{\omega(1-\omega)}-2\ve
\]
\end{itemize}
\end{minipage}
\end{frame}
































\begin{frame}{The Theory of Complex Dimensions in $\MB{R}^N$}
\small
\begin{itemize}
\itemsep3mm
\item Theory extended to higher dimensional objects
\item Fractal Zeta Functions and Fractal Drums (FZF)
\item A fractal drum in $\MB{R}^2$: the (complement of the) Sierpi\'nski gasket ($SG$) in the unit triangle
\end{itemize}
\vspace*{-6mm}
\begin{figure}
\includegraphics[scale=0.09]{sierpinski_gasket.png}
\caption{The Sierpi\'{n}ski gasket}
\end{figure}


\end{frame}





















\begin{frame}{The Theory of Complex Dimensions in $\MB{R}^N$}
\small
\begin{itemize}
\itemsep3mm
\item Theory extended to higher dimensional objects
\item Fractal Zeta Functions and Fractal Drums (FZF)
\item A fractal drum in $\MB{R}^2$: the (complement of the) Sierpi\'nski gasket ($SG$) in the unit triangle
\item The distance zeta function of a bounded set $A\subseteq \MB{R}^N$:
\[
\zeta_{A_\delta}(s)=\int_{A_\delta}d(x,A)^{s-N}\,\dif{x}
\]\vspace*{-\baselineskip}
\item The set of \textbf{complex dimensions} is the set of singularities of the meromorphic continuation of this zeta function.
\item Nonreal complex dimensions indicate the presence of geometric oscillations in $A$ and the tubular neighborhood of $A$.
\end{itemize}

\end{frame}



















\begin{frame}{Some Basic Definitions}
\footnotesize

\begin{definition}
For a point $x\in\MB{R}^N$ and a set $A\subseteq \MB{R}^N$ we define $d(x,A)$ as
\[
d(x,A)\coloneqq\inf_{y\in A}|x-y|
\]
where $|\cdot|$ is the standard Euclidean distance.
\end{definition}

\lspace
\begin{definition}
Given a set $A\subseteq\MB{R}^N$ and $t>0$, define the \textbf{tubular neighborhood of $A$} as
\[
A_t\coloneqq\{x\in\MB{R}^N\mid d(x,A)<t\}.
\]
\end{definition}


\begin{definition}
Given a set $A\subseteq \MB{R}^N$ and a real number $k\in \MB{R}$, we define
\[
kA\coloneqq\{kx\in\MB{R}^N\mid x\in A\}=\{(kx_1,kx_2,\ldots, kx_N)\mid (x_1,x_2,\ldots, x_N)\in A\}.
\]
\end{definition}
\end{frame}



















\begin{frame}{Some Important Definitions}
\lspace
\footnotesize
\begin{definition}
Let $\Omega\subseteq\MB{R}^N$ be an open set of finite $N$-dimensional Lebesgue measure. Let $A\subseteq\MB{R}^N$ such that $\Omega\subseteq A_\delta$ for some $\delta>0$. The \textbf{relative distance zeta function of $A$}, $\zeta_{A,\Omega}$, is defined as
\[
\zeta_{A,\Omega}(s)\coloneqq\int_\Omega d(x,A)^{s-N}\dif{x} \quad \iff \quad \zeta_{A,\Omega}(s;\delta)\coloneqq\int_{\Omega\cap A_\delta}d(x,A)^{s-N}\dif{x}
\]
for all $s\in\MB{C}$ such that $\Re(s)>\overline{\dim}_B(A,\Omega)$. Moreover, this function can often be meromorphically continued to all of $\MB{C}$. We also denote by $D(\zeta_{A,\Omega})$ the abscissa of convergence of $\zeta_{A,\Omega}$.
\end{definition}

\lspace
\begin{definition}
The pair $(A,\Omega)$ in the previous definition is called a \textbf{relative fractal drum (RFD)}.  In practice, $A$ is the fractal of interest and $\Omega$ is an open set whose closure contains $A$.
\end{definition}
\end{frame}




















\begin{frame}{Some Important Definitions}
\footnotesize
\begin{definition}
The set of \textbf{complex dimensions} of $A$ is the set of singularities of the meromorphic continuation of $\zeta_{A,\Omega}$ and is denoted $\MC{D}(\zeta_{A,\Omega})$.
\end{definition}

\lspace
\begin{definition}[M.L. Lapidus]
A set is defined as \textbf{\uline{fractal}} if and only if the meromorphic continuation of its associated zeta function has at least one nonreal complex dimension.
\end{definition}

\lspace
\begin{highlight}
This definition of fractality is the most accurate to date, and correctly classifies fractals and nonfractals by the existence of nonreal complex dimensions. The results of the research herein further validate the soundness of this definition, and emphasize the importance of having a proper definition of fractality.
\end{highlight}
\end{frame}















\begin{frame}%{A Theorem for a Class of Plane-Filling Curves}
\small
\begin{conj}[A.D. Richardson, 2021]
Let $\Lambda$ be a plane-filling curve constructed as described above with scalar $\lambda\in\MB{N}$, $\lambda\neq1$. Let $(\Lambda, \Omega)$ be the associated RFD, constructed as described below using the appropriate fundamental unit. Then a relative distance zeta function for $(\Lambda,\Omega)$ is

\[
\zeta_{\Lambda,\Omega}(s)=\frac{\lambda^2-1}{(s-2)(s-1)(\lambda^s-1)}.
\]

\lspace
Consequently, the set of relative complex dimensions of any curve of this type is

\[
\MC{D}(\zeta_{\Lambda,\Omega},\MB{C})=\left\{0+\frac{2\pi}{\log \lambda}i\mathbbm{Z}\right\}\cup\{1,2\},
\]

\lspace
and therefore plane-filling curves of this type are fractals by definition.
\end{conj}
 
\end{frame}







































\begin{frame}{A Model of the Hilbert RFD}
\begin{minipage}{0.6\textwidth}
\tiny
\begin{align*}
\zeta_{H,\Omega_0}(s)&=\int_{\Omega_0}d(\vec x, H)^{s-3}\,\dif{\vec x}\\[2mm]
&=\int_0^1\int_0^{\frac{1}{2}}\int_{y}^{1-y}z^{s-3}\,\dif{z}\,\dif{y}\,\dif{x}\\[2mm]
&=\frac{1}{s-2}\int_0^{\frac{1}{2}}(1-y)^{s-2}-y^{s-2}\,\dif{y}\\[2mm]
&=\frac{1}{(s-2)(s-1)}\left[-(1-y)^{s-1}-y^{s-1}\right]_0^{\frac{1}{2}}\\[2mm]
&=\frac{1}{(s-2)(s-1)}\left[-\left(\frac{1}{2}\right)^{s-1}-\left(\frac{1}{2}\right)^{s-1}+1\right]\\[2mm]
&=\frac{1}{(s-2)(s-1)}[-2\cdot2^{1-s}+1]\\[2mm]
&=\frac{(1-2^{2-s})}{(s-2)(s-1)}.
\end{align*}
\end{minipage}\hspace{6mm}%
\begin{minipage}{0.31\textwidth}
\centering
\begin{figure}
%Generalized fundamental unit
\tdplotsetmaincoords{80}{130}
\begin{tikzpicture}[scale=2,tdplot_main_coords]
\coordinate (O) at (0,0,0);

%-----rectangle-----
\begin{scope}[canvas is yx plane at z=0,transform shape]
\draw[dotted, fill=blue!20,fill opacity=0.3] (0,1,0) rectangle (1/2,0,0);
\end{scope}

%------axes-----
\draw[->] (-0.3,0,0) -- (1.6,0,0) node[anchor=north east]{\tiny$x$};
\draw[->] (0,0,0) -- (0,1,0) node[anchor=north west]{\tiny$y$};
\draw[dotted] (0,-0.9,0) -- (O);
\draw (0,-0.84,0) -- (0,-1.35,0);
\draw [->] (0,0,1) -- (0,0,1.25) node[anchor=south]{\tiny$z$};
\draw [dotted] (O) -- (0,0,1);

%-----figure-----
\draw [thick](O) --  (0,0.5,0.5);
\draw[->]   (0,5/8,1/4) node[right] {\tiny $z=y$} to [in=315,out=180] (0,1/3,1/4);

\draw [thick](1,0,0) -- (1,0.5,0.5);

\draw [thick](1,0.5,0.5) -- (0,0.5,0.5);
\draw [thick] (1,0.5,0.5) -- (1,0,1);

\draw[thick] (1,0,0) -- (1,0,1);
\draw[thick,dashed] (0,0,0) -- (0,0,1);

\draw [thick] (0,0.5,0.5) --node[right]{\tiny $z=1-y$} (0,0,1);


\draw[thick] (0,0,1) -- (1,0,1);
\draw [thick,blue](O) -- (1,0,0);

%-----ticks-----
\draw (1,0,-0.05) -- (1,0,0.05) node [label={[label distance=0.3mm]270:{\tiny$1$}}] {};
\draw (0,-0.05,1) -- (0,0.05,1) node [label={[label distance=-2mm]5:{\tiny$1$}}] {};
\draw (0,0.5,-0.05) -- (0,0.5,0.05) node [label={[label distance=0.3mm]270:{\tiny$\frac{1}{2}$}}] {};
\draw [dotted] (0,0.5,0.5) -- (0,0.5,0);
\draw [dotted] (1,0.5,0.5) -- (1,0.5,0);
\end{tikzpicture}
\centering \caption{The Fundamental Unit, $\Omega_0$, \\ for the Hilbert Curve}
\end{figure}
\end{minipage}

\end{frame}









\begin{frame}{A Model of the Hilbert RFD}

\begin{minipage}{0.6\textwidth}
\tiny
\begin{align*}
\zeta_{H,\Omega}(s)&=\frac{(1-2^{2-s})}{(s-2)(s-1)}\sum_{n=1}^\infty 2^{-sn}(4^n-1)\\[2mm]
&=\frac{(1-2^{2-s})}{(s-2)(s-1)}\sum_{n=1}^\infty \left[(2^{2-s})^n-(2^{-s})^n\right]\\[2mm]
&=\frac{(1-2^{2-s})}{(s-2)(s-1)}\left[\frac{2^{2-s}}{1-2^{2-s}}-\frac{2^{-s}}{1-2^{-s}}\right]\\[2mm]
&=\frac{\cancel{(1-2^{2-s})}}{(s-2)(s-1)}\left[\frac{2^{2-s}-\cancel{2^{-s}2^{2-s}}-2^{-s}+\cancel{2^{-s}2^{2-s}}}{\cancel{(1-2^{2-s})}(1-2^{-s})}\right]\\[2mm]
&=\frac{2^{-s}(4-1)}{(s-2)(s-1)(1-2^{-s})}\\[2mm]
&=\frac{3}{(s-2)(s-1)(2^s-1)}.
\end{align*}
\end{minipage}\hspace{4mm}%
\begin{minipage}{0.31\textwidth}
\centering
%Generalized fundamental unit
\begin{figure}
\tdplotsetmaincoords{80}{130}
\begin{tikzpicture}[scale=2,tdplot_main_coords]
\coordinate (O) at (0,0,0);

%-----rectangle-----
\begin{scope}[canvas is yx plane at z=0,transform shape]
\draw[dotted, fill=blue!20,fill opacity=0.3] (0,1,0) rectangle (1/2,0,0);
\end{scope}

%------axes-----
\draw[->] (-0.3,0,0) -- (1.6,0,0) node[anchor=north east]{\tiny$x$};
\draw[->] (0,0,0) -- (0,1,0) node[anchor=north west]{\tiny$y$};
\draw[dotted] (0,-0.9,0) -- (O);
\draw (0,-0.84,0) -- (0,-1.35,0);
\draw [->] (0,0,1) -- (0,0,1.25) node[anchor=south]{\tiny$z$};
\draw [dotted] (O) -- (0,0,1);

%-----figure-----
\draw [thick](O) -- (0,0.5,0.5);
\draw[->]   (0,5/8,1/4) node[right] {\tiny $z=y$} to [in=315,out=180] (0,1/3,1/4);

\draw [thick](1,0,0) -- (1,0.5,0.5);

\draw [thick](1,0.5,0.5) -- (0,0.5,0.5);
\draw [thick] (1,0.5,0.5) -- (1,0,1);

\draw[thick] (1,0,0) -- (1,0,1);
\draw[thick,dashed] (0,0,0) -- (0,0,1);

\draw [thick] (0,0.5,0.5) --node[right]{\tiny $z=1-y$} (0,0,1);

\draw[thick] (0,0,1) -- (1,0,1);
\draw [thick,blue](O) -- (1,0,0);

%-----ticks-----
\draw (1,0,-0.05) -- (1,0,0.05) node [label={[label distance=0.3mm]270:{\tiny$1$}}] {};
\draw (0,-0.05,1) -- (0,0.05,1) node [label={[label distance=-2mm]5:{\tiny$1$}}] {};
\draw (0,0.5,-0.05) -- (0,0.5,0.05) node [label={[label distance=0.3mm]270:{\tiny$\frac{1}{2}$}}] {};
\draw [dotted] (0,0.5,0.5) -- (0,0.5,0);
\draw [dotted] (1,0.5,0.5) -- (1,0.5,0);
\end{tikzpicture}
\centering \caption{The Fundamental Unit, $\Omega_0$, for the Hilbert Curve}
\end{figure}
\end{minipage}
\end{frame}




 
















\begin{frame}{A Model of the Hilbert RFD}
(Mathematica)
\end{frame}















\begin{frame}{Geometric Oscillations}
\small
The presence of nonreal poles above 0 indicate the presence of geometric oscillations of the 0-dimensional objects in the fractal, namely the points. We now see this oscillation in two ways: through the trajectories of consecutive images of points through approximations of the space-filling curve, and in the volume of the tubular neighborhood.

\lspace
In 1991, Hans Sagan produced the first valid ``arithmetizaton'' of the Hilbert curve, i.e. a formula that gives an approximation of the image point in $I^2$ of a point in $I$ under the Hilbert mapping using a numeric representation of the point in $I$. This arithmetization gives the location of an image to any degree of precision.
\end{frame}

















\begin{frame}{Geometric Oscillations of the Points}
\footnotesize
\lspace
The nature of the Hilbert mapping suggests that we should use quaternary numbers for our analysis. The arithmetization uses compositions of the following four similarity transformations of the complex plane:
\begin{align*}
\mathfrak{H}_0(z)&=\frac{1}{2}\bar zi & \mathfrak{H}_2(z)&=\frac{1}{2}z+\frac{1}{2}+\frac{i}{2}\\[2mm]
\mathfrak{H}_1(z)&=\frac{1}{2}z+\frac{i}{2} & \mathfrak{H}_3(z)&=-\frac{1}{2}\bar zi+1+\frac{i}{2}
\end{align*}

\begin{figure}
\centering
\includegraphics[scale=0.5]{similarity_transformations.png}
\vspace{-2mm}\caption{Applying the transformations above to (a) produces (b), and applying them to (b) produces (c).}
\end{figure}
\end{frame}








\begin{frame}{Geometric Oscillations of the Points}
\linespread{1.3}\selectfont
\footnotesize

These transformations can be used to generate intermediate approximating polygons and in the limit, they produce the Hilbert mapping and the Hilbert curve itself.



\lspace
\begin{proposition}
Let $t=0.q_1q_2q_3q_4\ldots$ be a point in $I$ written in quaternary (base-4), i.e. $q_i\in\{0,1,2,3\}$ for all $i\geq 1$. Let $f_h(t)$ be the image of a point $t\in I$ under the Hilbert mapping $f_h$. Then
\[
f_h(0.q_1q_2q_3\ldots)=\sum_{j=1}^\infty\left(\frac{1}{2}\right)^j(-1)^{e_{0_j}}\sgn(q_j) {(1-d_j)q_j-1 \choose 1-d_jq_j}
\]

\vspace{3mm}
where $e_{k_j}$ is (the number of $k$'s preceding $q_j$) $\rm{mod} \,2$ for $k=0,3$, and \\ $d_j=(e_{0_j}+e_{3_j})\mod 2$.
\end{proposition}
\end{frame}
















\begin{frame}{Geometric Oscillations of the Points}
\footnotesize

\[
f_h(0.q_1q_2q_3\ldots)=\sum_{j=1}^\infty\left(\frac{1}{2}\right)^j(-1)^{e_{0_j}}\sgn(q_j) {(1-d_j)q_j-1 \choose 1-d_jq_j}
\]

\textbf{Note:} the quantities $e_{k_j}$ and $d_j$ are accounting for cancellations that occur when some of the transformations are applied consecutively.

\lspace
Any real number $t\in I$ has a quaternary representation, and it can be approximated arbitrarily closely by quaternary rationals just as we can do with decimal rationals. Thus, using this arithmetization, we can get better and better approximations of the image point $f_h(t)\in I^2$ by taking better and better quaternary approximations of the point $t\in I$.

\lspace
The present author has coded this arithmetization in Mathematica, and we can now see the oscillations of these image points as they converge to their (unique) attractors.
\end{frame}



















\begin{frame}{Geometric Oscillations of the Points}
(Mathematica)
\end{frame}



















\begin{frame}{Geometric Oscillations in $V(t)=|\Lambda_t\cap\Omega|_3$}

\footnotesize
\begin{itemize}
\itemsep5mm

\item $V(t)$ is the volume of the intersection of the tubular neighborhood of $I^2$ and the RFD $\Omega$.
\item Since the RFD lies entirely and directly above and below the unit square,
\[
\Lambda_t\cap \Omega=[0,1]^2\times (-t,t)\cap\Omega.
\]
\item By counting prisms according to multiplicity in each generation $n$, we can write
\[
\Lambda_t\cap\Omega=[0,1]^2\times[0,t)\cap\Omega.
\]
\item Intuitively, this allows us to view $\Lambda_t\cap\Omega$ as being the part of the RFD that is submerged up to some ``water line'' $z=t$.
\end{itemize}
\end{frame}




























\begin{frame}{Decomposition of $\Omega$}

\begin{minipage}{0.6\textwidth}
\begin{figure}
\begin{tikzpicture}[scale=4.75]

\fill [blue!20] (0,0) -- (1/3,1/6) -- (0,1/6) -- cycle;
\fill [blue!20] (0+2/3,0) -- (2/3+2/9,1/9) -- (2/3+1/3-1/6,1/6) -- (0+2/3,1/6) -- cycle;
\fill [blue!20] (2/3+2/9,0) -- (2/3+2/3^2+2/3^3,1/3^3) -- (2/3+2/3^2,1/3^2) -- cycle;
\fill [blue!20] (2/3+2/9+2/27,0) -- (2/3+2/3^2+2/3^3+2/3^4,1/3^4) -- (2/3+2/3^2+2/3^3,1/3^3) -- cycle;

%-----Axes-----
\draw[->] (-0.05,0) -- (1.1,0) node[right] {\footnotesize$y$};
\draw[->] (0,-0.05) -- (0,1.1) node[above] {\footnotesize$z$};

%-----Ticks-----
\draw (-0.01,1) node[left] {\tiny $\frac{1}{\lambda^{M-2}}$} -- (0.01,1);
\draw (-0.01,1/3) node[left] {\tiny $\frac{1}{\lambda^{M-1}}$} -- (0.01,1/3);
\draw (-0.01,1/3^2) node[left] {\tiny $\frac{1}{\lambda^{M}}$} -- (0.01,1/3^2);
\draw (-0.01,1/3^3) node[left] {\tiny $\frac{1}{\lambda^{M+1}}$} -- (0.01,1/3^3);

\draw [thick,rounded corners=0.01mm](0,0) -- (2/3,1/3) -- (0,1) -- cycle;
\draw [thick,rounded corners=0.01mm](2/3,0) -- (2/3+2/3^2,1/3^2) -- (2/3,1/3) -- cycle;
\draw [thick,rounded corners=0.01mm](2/3+2/9,0) -- (2/3+2/3^2+2/3^3,1/3^3) -- (2/3+2/3^2,1/3^2) -- cycle;
\draw [thick,rounded corners=0.01mm](2/3+2/9+2/27,0) -- (2/3+2/3^2+2/3^3+2/3^4,1/3^4) -- (2/3+2/3^2+2/3^3,1/3^3) -- cycle;


\draw [dashed,thick,blue] (-0.05,1/6) --  (1.1,1/6) node[right] {\footnotesize$t$};
\end{tikzpicture}
\caption{Side view of the RFD submerged to the  ``water line'' $z=t$.}
\end{figure}

\end{minipage}\hspace{3mm}%
\begin{minipage}{0.35\textwidth}
\footnotesize
Let $\lambda\in \MB{N}$, $\lambda\neq 1$, $t>0$.

\lspace
There exists an $M\in \MB{N}$ such that 

\[
\frac{1}{\lambda^M}\leq t<\frac{1}{\lambda^{M-1}}.
\]

\lspace
In particular,

\[
M=\ceil[\Bigg]{\log_\lambda\left(\frac{1}{t}\right)}
\]

\lspace
where $\ceil{\cdot}$ indicates the ceiling function.
\end{minipage}
\end{frame}
































\begin{frame}{Decomposition of $\Omega$}

\begin{minipage}{0.6\textwidth}
\begin{figure}
\begin{tikzpicture}[scale=4.75]

\fill [blue!20] (0,0) -- (1/3,1/6) -- (0,1/6) -- cycle;
\fill [blue!20] (0+2/3,0) -- (2/3+2/9,1/9) -- (2/3+1/3-1/6,1/6) -- (0+2/3,1/6) -- cycle;
\fill [blue!20] (2/3+2/9,0) -- (2/3+2/3^2+2/3^3,1/3^3) -- (2/3+2/3^2,1/3^2) -- cycle;
\fill [blue!20] (2/3+2/9+2/27,0) -- (2/3+2/3^2+2/3^3+2/3^4,1/3^4) -- (2/3+2/3^2+2/3^3,1/3^3) -- cycle;

%-----Axes-----
\draw[->] (-0.05,0) -- (1.1,0) node[right] {\footnotesize$y$};
\draw[->] (0,-0.05) -- (0,1.1) node[above] {\footnotesize$z$};

%-----Ticks-----
\draw (-0.01,1) node[left] {\tiny $\frac{1}{\lambda^{M-2}}$} -- (0.01,1);
\draw (-0.01,1/3) node[left] {\tiny $\frac{1}{\lambda^{M-1}}$} -- (0.01,1/3);
\draw (-0.01,1/3^2) node[left] {\tiny $\frac{1}{\lambda^{M}}$} -- (0.01,1/3^2);
\draw (-0.01,1/3^3) node[left] {\tiny $\frac{1}{\lambda^{M+1}}$} -- (0.01,1/3^3);

\draw [thick,rounded corners=0.01mm](0,0) -- (2/3,1/3) -- (0,1) -- cycle;
\draw [thick,rounded corners=0.01mm](2/3,0) -- (2/3+2/3^2,1/3^2) -- (2/3,1/3) -- cycle;
\draw [thick,rounded corners=0.01mm](2/3+2/9,0) -- (2/3+2/3^2+2/3^3,1/3^3) -- (2/3+2/3^2,1/3^2) -- cycle;
\draw [thick,rounded corners=0.01mm](2/3+2/9+2/27,0) -- (2/3+2/3^2+2/3^3+2/3^4,1/3^4) -- (2/3+2/3^2+2/3^3,1/3^3) -- cycle;


\draw [dashed,thick,blue] (-0.05,1/6) --  (1.1,1/6) node[right] {\footnotesize$t$};
\end{tikzpicture}
\caption{Side view of the RFD submerged to the  ``water line'' $z=t$.}
\end{figure}

\end{minipage}\hspace{3mm}%
\begin{minipage}{0.35\textwidth}
\footnotesize
The choice of $t$ partitions $\Omega$ into three disjoint collections of prisms, $\Omega_1, \Omega_2$, and $\Omega_3$, where 

\begin{align*}
\Omega_1&=\bigcup_{n=1}^{M-2}\bigcup_{i=1}^{\lambda^{2n}-1}\Omega_i^n\\[3mm]
\Omega_2&=\bigcup_{i=1}^{\lambda^{2(M-1)}-1}\Omega^{M-1}_i\\[3mm]
\Omega_3&=\bigcup_{n=M}^\infty\bigcup_{i=1}^{\lambda^{2n}-1}\Omega_i^n
\end{align*}
\end{minipage}
\end{frame}

































\begin{frame}{Computing $|\Lambda_t\cap \Omega_1|_3$}

\begin{minipage}{0.46\textwidth}
\vspace{2mm}
\begin{figure}
\begin{tikzpicture}[scale=4.8]

\fill [blue!20] (0,0) -- (1/3,1/6) -- (0,1/6) -- cycle;

%-----Axes-----
\draw[->] (-0.05,0) -- (0.75,0) node[right] {\footnotesize$y$};
\draw[->] (0,-0.05) -- (0,1.1) node[above] {\footnotesize$z$};

%-----Ticks-----
\draw (-0.01,1) node[left] {\tiny $\frac{1}{\lambda^{M-2}}$} -- (0.01,1);
\draw (-0.01,1/3) node[left] {\tiny $\frac{1}{\lambda^{M-1}}$} -- (0.01,1/3);
\draw (-0.01,1/3^2) node[left] {\tiny $\frac{1}{\lambda^{M}}$} -- (0.01,1/3^2);
\draw (-0.01,1/3^3) node[left] {\tiny $\frac{1}{\lambda^{M+1}}$} -- (0.01,1/3^3);
\draw (2/3,-0.01) node[below] {\tiny $\frac{\lambda-1}{\lambda^{M-1}}$} -- (2/3, 0.01);

\draw [thick,rounded corners=0.01mm](0,0) -- (2/3,1/3) -- (0,1) -- cycle;
%\draw [thick,rounded corners=0.01mm](2/3,0) -- (2/3+2/3^2,1/3^2) -- (2/3,1/3) -- cycle;
%\draw [thick,rounded corners=0.01mm](2/3+2/9,0) -- (2/3+2/3^2+2/3^3,1/3^3) -- (2/3+2/3^2,1/3^2) -- cycle;
%\draw [thick,rounded corners=0.01mm](2/3+2/9+2/27,0) -- (2/3+2/3^2+2/3^3+2/3^4,1/3^4) -- (2/3+2/3^2+2/3^3,1/3^3) -- cycle;
%\draw [dotted] (2/3+2/3^2+2/3^3+2/3^4,1/3^4) -- (1,0);

\draw [dashed,thick,blue] (-0.05,1/6) --  (0.75,1/6) node[right] {\footnotesize$t$};

\draw[->]   (0.4,1/13) node[right] {\tiny $\left(t(\lambda-1),t\right)$} to [in=300,out=180] (1/3,1/6-0.01);

\end{tikzpicture}
\caption{A prism in $\Omega_1$.}
\end{figure}
\end{minipage}\hspace{4mm}
\begin{minipage}{0.49\textwidth}
\footnotesize
The cross-section of the submerged section of any prism in $\Omega_1$ is a triangle that has area
\[
A=\frac{1}{2}t\cdot t(\lambda-1)=\frac{t^2(\lambda-1)}{2}.
\]

\lspace
The base of any prism has length $\frac{1}{\lambda^n}$, thus the volume of the submerged portion is
\[
V_n=\frac{t^2(\lambda-1)}{2\lambda^n}.
\]

\lspace
There are $\lambda^{2n}-1$ prisms in any $n$th generation contained in $\Omega_1$, so we have
\[
|\Lambda_t\cap \Omega_1|_3=\sum_{n=1}^{M-2}(\lambda^{2n}-1)\cdot V_n.
\]
\end{minipage}
\end{frame}



















\begin{frame}{Computing $|\Lambda_t\cap \Omega_1|_3$}
\footnotesize
Simplifying, we have
\begin{align*}
|\Lambda_t\cap \Omega_1|_3&=\sum_{n=1}^{M-2}(\lambda^{2n}-1)\cdot V_n=\sum_{n=1}^{M-2}(\lambda^{2n}-1)\frac{t^2(\lambda-1)}{2\lambda^n}\\[2mm]
&=\frac{t^2(\lambda-1)}{2}\sum_{n=1}^{M-2}\left(\lambda^n-\frac{1}{\lambda^n}\right)\\[2mm]
&=\frac{t^2\cancel{(\lambda-1)}}{2}\left[\frac{\lambda^{M-1}+\lambda^{2-M}-\lambda-1}{\cancel{\lambda-1}}\right]\\[2mm]
&=\frac{t^2}{2}\left[\lambda^{M-1}+\lambda^{2-M}-\lambda-1\right].\quad\quad(*)
\end{align*}

\end{frame}





















\begin{frame}{Computing $|\Lambda_t\cap \Omega_2|_3$}
\begin{minipage}{0.4\textwidth}
\begin{figure}
\begin{tikzpicture}[scale=10]

\fill [blue!20] (0,0) -- (2/9,1/9) -- (1/3-1/6,1/6) -- (0,1/6) -- cycle;

%-----Axes-----
\draw[->] (-0.05,0) -- (1/3,0) node[right] {\footnotesize$y$};
\draw[->] (0,-0.05) -- (0,0.4) node[above] {\footnotesize$z$};

%-----Ticks-----

\draw (-0.01,1/3) node[left] {\tiny $\frac{1}{\lambda^{M-1}}$} -- (0.01,1/3);
\draw (-0.01,1/3^2) node[left] {\tiny $\frac{1}{\lambda^{M}}$} -- (0.01,1/3^2);
\draw (-0.01,1/3^3) node[left] {\tiny $\frac{1}{\lambda^{M+1}}$} -- (0.01,1/3^3);
\draw (2/3^2,-0.01) node[below] {\tiny $\frac{\lambda-1}{\lambda^{M}}$} -- (2/3^2, 0.01);

\draw (0.08,0.08) node {\tiny $T$};
\draw (0.1,0.14) node {\tiny $Z$};

%\draw [thick,rounded corners=0.01mm](0,0) -- (2/3,1/3) -- (0,1) -- cycle;
\draw [thick,rounded corners=0.01mm](0,0) -- (0+2/3^2,1/3^2) -- (0,1/3) -- cycle;
%\draw [thick,rounded corners=0.01mm](2/3+2/9,0) -- (2/3+2/3^2+2/3^3,1/3^3) -- (2/3+2/3^2,1/3^2) -- cycle;
%\draw [thick,rounded corners=0.01mm](2/3+2/9+2/27,0) -- (2/3+2/3^2+2/3^3+2/3^4,1/3^4) -- (2/3+2/3^2+2/3^3,1/3^3) -- cycle;
%\draw [dotted] (2/3+2/3^2+2/3^3+2/3^4,1/3^4) -- (1,0);

\draw [dashed,thick,blue] (-0.05,1/6) --  (0.3,1/6) node[right] {\footnotesize$t$};

\draw [dotted] (0,1/9) --  (2/9,1/9);
\draw [dotted] (1/6, 1/6) -- (1/6, 1/9);
%\draw [double] (1/6-0.008,5/36) -- (1/6+0.008,5/36);

\end{tikzpicture}
\caption{A prism in $\Omega_2$.}
\end{figure}
\end{minipage}\hspace{4mm}%
\begin{minipage}{0.55\textwidth}
\scriptsize
The cross-section of these prisms is a quadrilateral $Q$ which can be decomposed into a right triangle $T$ and a trapezoid $Z$:
\[
Q=T\cup Z
\]

\[
|T|_2=\frac{1}{2}\cdot\frac{1}{\lambda^M}\cdot\frac{\lambda-1}{\lambda^M}=\frac{\lambda-1}{2\lambda^{2M}}.
\]
\begin{align*}
&|Z|_2=\frac{1}{2}h(b_1+b_2)\\[2mm]
&=\frac{1}{2}\left(t-\frac{1}{\lambda^M}\right)\left[\frac{\lambda-1}{\lambda^M}+\frac{\lambda-1}{\lambda^M}-\left(t-\frac{1}{\lambda^M}\right)\right]\\[2mm]
&=\frac{1}{2}\left(t-\frac{1}{\lambda^M}\right)\left(\frac{2\lambda-1}{\lambda^M}-t\right)\\[2mm]
&=\frac{1}{2\lambda^M}\left((2\lambda-1)t-\lambda^Mt^2-\frac{2\lambda-1}{\lambda^M}+t\right).
\end{align*}
\end{minipage}
\end{frame}




























\begin{frame}{Computing $|\Lambda_t\cap \Omega_2|_3$}
\footnotesize
Putting these results together, we have
\begin{align*}
|Q|_2&=|T|_2+|Z|_2\\[2mm]
&=\frac{\lambda-1}{2\lambda^{2M}}+\frac{1}{2\lambda^M}\left((2\lambda-1)t-\lambda^Mt^2-\frac{2\lambda-1}{\lambda^M}+t\right)\\[2mm]
&=\frac{1}{2\lambda^M}\left(\frac{\lambda-1}{\lambda^M}+(2\lambda-1)t-\lambda^Mt^2-\frac{2\lambda-1}{\lambda^M}+t\right)\\[2mm]
&=\frac{1}{2\lambda^M}\left(-\frac{1}{\lambda^{M-1}}+2\lambda t-\lambda^Mt^2\right)\\[2mm]
&=\frac{t}{\lambda^{M-1}}-\frac{1}{2\lambda^{2M-1}}-\frac{t^2}{2}.
\end{align*}
\end{frame}

















\begin{frame}{Computing $|\Lambda_t\cap \Omega_2|_3$}
\footnotesize
Therefore,
\begin{align*}
|Q|_3&=\frac{1}{\lambda^{M-1}}\cdot|Q|_2=\frac{1}{\lambda^{M-1}}\left[\frac{t}{\lambda^{M-1}}-\frac{1}{2\lambda^{2M-1}}-\frac{t^2}{2}\right]\\[2mm]
&=\frac{t}{\lambda^{2M-2}}-\frac{1}{2\lambda^{3M-2}}-\frac{t^2}{2\lambda^{M-1}}.
\end{align*}

Since there are exactly $\lambda^{2(M-1)}-1$ prisms in $\Omega_2$, we have
\begin{align*}
|\Lambda_t\cap \Omega_2|_3&=\left(\lambda^{2(M-1)}-1\right)\cdot|Q|_3\\[2mm]
&=\left(\lambda^{2M-2}-1\right)\left(\frac{t}{\lambda^{2M-2}}-\frac{1}{2\lambda^{3M-2}}-\frac{t^2}{2\lambda^{M-1}}\right).\quad\quad(**)
\end{align*}
\end{frame}






























\begin{frame}{Computing $|\Lambda_t\cap \Omega_3|_3$}
\begin{itemize}
\itemsep5mm
\item The entirety of any prism in $\Omega_3$ is included in $\Lambda_t\cap\Omega$. 
\item The volume of any prism in generation $n$ is
\[
V_n=\frac{1}{2}\cdot\frac{1}{\lambda^n}\cdot\frac{\lambda-1}{\lambda^{n+1}}\cdot\frac{1}{\lambda^n}=\frac{\lambda-1}{2\lambda^{3n+1}}.
\]
\item There are $\lambda^{2n}-1$ prisms in any generation $n$, so we have
\[
|\Lambda_t\cap \Omega_3|_3=\sum_{n=M}^\infty(\lambda^{2n}-1)\cdot V_n.
\]
\end{itemize}
\end{frame}


















\begin{frame}{Computing $|\Lambda_t\cap \Omega_3|_3$}
\footnotesize
Simplifying, we have
\begin{align*}
|\Lambda_t\cap \Omega_3|_3&=\sum_{n=M}^\infty(\lambda^{2n}-1)\cdot V_n=\sum_{n=M}^\infty(\lambda^{2n}-1)\cdot \frac{\lambda-1}{2\lambda^{3n+1}}\\[2mm]
&=\frac{\lambda-1}{2\lambda}\sum_{n=M}^\infty\left(\frac{1}{\lambda^n}-\frac{1}{\lambda^{3n}}\right)\\[2mm]
&=\frac{\cancel{\lambda-1}}{2\cancel{\lambda}}\left[\frac{\lambda^{\cancel{1}-3M}\left(\lambda^{2M+1}+\lambda^{2M+2}+\lambda^{2M}-\lambda^2\right)}{\cancel{(\lambda-1)}(\lambda^2+\lambda+1)}\right]\\[2mm]
&=\frac{\lambda^{-3M}\left(\lambda^{2M+1}+\lambda^{2M+2}+\lambda^{2M}-\lambda^2\right)}{2(\lambda^2+\lambda+1)}\quad\quad(***)
\end{align*}
\end{frame}




















\begin{frame}{Computing $|\Lambda_t\cap \Omega|_3$}
\footnotesize
Now that we have all three subvolumes calculated, we combine $(*)$, $(**)$, and $(***)$:

\begin{align*}
|\Lambda_t\cap \Omega|_3&=|\Lambda_t\cap (\Omega_1\cup \Omega_2\cup \Omega_3)|_3\\[2mm]
&=|\Lambda_t\cap \Omega_1|_3+|\Lambda_t\cap\Omega_2|_3+|\Lambda_t\cap \Omega_3|_3\\[2mm]
&=\frac{t^2}{2}\left[\lambda^{M-1}+\lambda^{2-M}-\lambda-1\right]\\[2mm]
&+\left(\lambda^{2M-2}-1\right)\left(\frac{t}{\lambda^{2M-2}}-\frac{1}{2\lambda^{3M-2}}-\frac{t^2}{2\lambda^{M-1}}\right)\\[2mm]
&+\frac{\lambda^{-3M}\left(\lambda^{2M+1}+\lambda^{2M+2}+\lambda^{2M}-\lambda^2\right)}{2(\lambda^2+\lambda+1)}.
\end{align*}

\end{frame}















\begin{frame}{Computing $|\Lambda_t\cap \Omega|_3$}
\footnotesize
Simplifying, we find
\[
|\Lambda_t\cap\Omega|_3=\frac{1}{2}\left(\frac{(\lambda^{1-M})^3(1+\lambda)}{1+\lambda+\lambda^2}-2t(\lambda^{1-M})^2+(\lambda^{1-M}-1)(1+\lambda)t^2+2t\right).
\]

\lspace
Next, observe that 
\[
\lambda^M=\lambda^{\ceil{-\log_\lambda t}}=\lambda^{-\log_\lambda t+(1-\{-\log_\lambda t\})}=\lambda^{-\log_\lambda t}\cdot\lambda^{1-\{-\log_\lambda t\}}=\frac{1}{t}\cdot\lambda^{1-\{-\log_\lambda t\}}
\]
where $\{\cdot\}$ indicates the fractional part of a number. Thus,
\[
\lambda^{1-M}=t\cdot\lambda^{\{-\log_\lambda t\}}
\]
\end{frame}
















\begin{frame}{Computing $|\Lambda_t\cap \Omega|_3$}
\footnotesize
Substituting, we have
\begin{align*}
&V(t)=|\Lambda_t\cap\Omega|_3\\[2mm]
&=\frac{1}{2}\left(\frac{(\lambda^{1-M})^3(1+\lambda)}{1+\lambda+\lambda^2}-2t(\lambda^{1-M})^2+(\lambda^{1-M}-1)(1+\lambda)t^2+2t\right)\\[2mm]
&=\frac{1}{2}\left(\frac{t^3\cdot\lambda^{3\{-\log_\lambda t\}}(1+\lambda)}{1+\lambda+\lambda^2}-2t^3\cdot\lambda^{2\{-\log_\lambda t\}}+t^3\cdot\lambda^{\{-\log_\lambda t\}}(1+\lambda) \right)\\[2mm]
&\quad\quad-\frac{1}{2}\Bigg((1+\lambda)t^2-2t\Bigg)\\[2mm]
&=t-\frac{t^2}{2}(1+\lambda)+\frac{t^3}{2}(1+\lambda)\left(\frac{\lambda^{3\{-\log_\lambda t\}}}{1+\lambda+\lambda^2}-\frac{2\lambda^{2\{-\log_\lambda t\}}}{1+\lambda}+\lambda^{\{-\log_\lambda t\}} \right)
\end{align*}
\end{frame}




















\begin{frame}{Geometric Oscillations in $V(t)$}
\footnotesize

\[
V(t)=t-\frac{t^2}{2}(1+\lambda)+\frac{t^3}{2}(1+\lambda)\left(\frac{\lambda^{3\{-\log_\lambda t\}}}{1+\lambda+\lambda^2}-\frac{2\lambda^{2\{-\log_\lambda t\}}}{1+\lambda}+\lambda^{\{-\log_\lambda t\}} \right)
\]

\lspace
The expression in the large parentheses is multiplicatively periodic: it takes the same value at $t$ and $\frac{t}{\lambda}$. 

\lspace
This illustrates the oscillatory behavior within the tubular volume, and we can see this more explicitly using the Fourier transform of the map $u\mapsto b^{-\{u\}}$:
\[
b^{-\{u\}}=\frac{b-1}{b}\sum_{n\in\MB{Z}}\frac{e^{2\pi i nu}}{\log b+2\pi i n}
\]
\end{frame}


















\begin{frame}{Geometric Oscillations in $V(t)$}
\footnotesize
\[
b^{-\{u\}}=\frac{b-1}{b}\sum_{n\in\MB{Z}}\frac{e^{2\pi i nu}}{\log b+2\pi i n}
\]
\lspace
Writing $\mbf{p}=\frac{2\pi}{\log\lambda}$, we have
\begin{align*}
\lambda^{\{-\log_\lambda t\}}&=(\lambda^{-1})^{-\{-\log_\lambda t\}}=\frac{\lambda^{-1}-1}{\lambda^{-1}}\sum_{n\in\MB{Z}}\frac{e^{2\pi in(-\log_\lambda t)}}{\log\lambda^{-1}+2\pi in}\\[2mm]
&=(1-\lambda)\sum_{n\in\MB{Z}}\frac{e^{-2\pi in\left(\frac{\log t}{\log \lambda}\right)}}{-\log\lambda+2\pi in}=(1-\lambda)\sum_{n\in\MB{Z}}\frac{t^{-\frac{2\pi}{\log\lambda}in}}{-\log\lambda\left(1-\frac{2\pi}{\log\lambda} in\right)}\\[2mm]
&=\frac{\lambda-1}{\log\lambda}\sum_{n\in\MB{Z}}\frac{t^{-in\mbf{p}}}{1-in\mbf{p}}\text{, so}
\end{align*}

\[
\lambda^{2\{-\log_\lambda t\}}=\frac{\lambda^2-1}{\log\lambda}\sum_{n\in\MB{Z}}\frac{t^{-in\mbf{p}}}{2-in\mbf{p}} \aspace \lambda^{3\{-\log_\lambda t\}}=\frac{\lambda^3-1}{\log\lambda}\sum_{n\in\MB{Z}}\frac{t^{-in\mbf{p}}}{3-in\mbf{p}}.
\]
\end{frame}















\begin{frame}{Geometric Oscillations in $V(t)$}
\footnotesize
\[
V(t)=t-\frac{t^2}{2}(1+\lambda)+\frac{t^3}{2}(1+\lambda)\left(\frac{\lambda^{3\{-\log_\lambda t\}}}{1+\lambda+\lambda^2}-\frac{2\lambda^{2\{-\log_\lambda t\}}}{1+\lambda}+\lambda^{\{-\log_\lambda t\}} \right)
\]

\lspace
Isolating the oscillatory factor in the last term of $V(t)$, we can rewrite it as

\begin{align*}
&\frac{1}{1+\lambda+\lambda^2}\cdot\frac{\lambda^3-1}{\log\lambda}\sum_{n\in\MB{Z}}\frac{t^{-in\mbf{p}}}{3-in\mbf{p}}
-\frac{2}{1+\lambda}\cdot\frac{\lambda^2-1}{\log\lambda}\sum_{n\in\MB{Z}}\frac{t^{-in\mbf{p}}}{2-in\mbf{p}}
+\frac{\lambda-1}{\log\lambda}\sum_{n\in\MB{Z}}\frac{t^{-in\mbf{p}}}{1-in\mbf{p}}\\[2mm]
&=\frac{\lambda-1}{\log\lambda}\left(\sum_{n\in\MB{Z}}\frac{t^{-in\mbf{p}}}{1-in\mbf{p}}-2\sum_{n\in\MB{Z}}\frac{t^{-in\mbf{p}}}{2-in\mbf{p}}+\sum_{n\in\MB{Z}}\frac{t^{-in\mbf{p}}}{3-in\mbf{p}}\right).
\end{align*}
\end{frame}

























\begin{frame}{Geometric Oscillations in $V(t)$}
\footnotesize
Thus
\[
V(t)=t-\frac{t^2}{2}(1+\lambda)+\frac{t^3}{2}\cdot\frac{\lambda^2-1}{\log\lambda}\left(\sum_{n\in\MB{Z}}\frac{t^{-in\mbf{p}}}{1-in\mbf{p}}-2\sum_{n\in\MB{Z}}\frac{t^{-in\mbf{p}}}{2-in\mbf{p}}+\sum_{n\in\MB{Z}}\frac{t^{-in\mbf{p}}}{3-in\mbf{p}}\right)
\]

\vspace{3mm}
Noting that $\{\omega=0+in\mbf{p}\}\subset\MC{D}(\zeta_{\Lambda,\Omega},\MB{C})$, we can already see how we can rewrite the formula above using a sum over (some of) the complex dimensions of $\zeta_{\Lambda,\Omega}$ if we observe that

\[
\sum_{n\in\MB{Z}}\frac{t^{-in\mbf{p}}}{1-in\mbf{p}}=\sum_{\substack{\omega\in\MC{D}(\zeta_{\Lambda,\Omega},\MB{C}) \\ \omega\neq1,2}}\frac{t^{-\omega}}{1-\omega}.
\]
%
%\[
%V(t)=t-\frac{t^2}{2}(1+\lambda)+\frac{t^3}{2}\cdot\frac{\lambda^2-1}{\log\lambda}\left(\sum_{\substack{\omega\in\MC{D}(\zeta_{\Lambda,\Omega},\MB{C}) \\ \omega\neq1,2}}\frac{t^{-\omega}}{1-\omega}-2\sum_{\substack{\omega\in\MC{D}(\zeta_{\Lambda,\Omega},\MB{C}) \\ \omega\neq1,2}}\frac{t^{-\omega}}{2-\omega}+\sum_{\substack{\omega\in\MC{D}(\zeta_{\Lambda,\Omega},\MB{C}) \\ \omega\neq1,2}}\frac{t^{-\omega}}{3-\omega}\right).
%\]
\end{frame}





















\begin{frame}{Geometric Oscillations}
\footnotesize
\[
V(t)=t-\frac{t^2}{2}(1+\lambda)+\frac{t^3}{2}\cdot\frac{\lambda^2-1}{\log\lambda}\left(\sum_{n\in\MB{Z}}\frac{t^{-in\mbf{p}}}{1-in\mbf{p}}-2\sum_{n\in\MB{Z}}\frac{t^{-in\mbf{p}}}{2-in\mbf{p}}+\sum_{n\in\MB{Z}}\frac{t^{-in\mbf{p}}}{3-in\mbf{p}}\right)
\]

We will see this formula later when we recover the tubular volume formula from a more general result. Additionally, we have the following corollary.

\lspace
\begin{corollary}
The RFD $(\Lambda,\Omega)$ is Minkowski measurable for any $\lambda\in\MB{N}$ with $\lambda\neq 1$, and the relative Minkowski dimension of the unit square with respect to $\Omega$ is $D=2$ as expected.
\end{corollary}
\end{frame}



















\begin{frame}{Geometric Oscillations}
\footnotesize

\begin{definition}
For any $r\in \MB{R}$, the \textbf{upper $r$-dimensional Minkowski content of $A$ relative to $\Omega$} is
\[
\MC{M}^{*r}(A,\Omega)=\limsup_{t\to0^+}\frac{|A_t\cap \Omega|_N}{t^{N-r}},
\]
and the \textbf{relative upper box (or Minkowski) dimension of $A$ with respect to $\Omega$} is
\begin{align*}
\overline{\dim}_B(A,\Omega)&=\inf\{r\in\MB{R}\mid \MC{M}^{*r}(A,\Omega)=0\}\\[2mm]
&=\inf\{r\in\MB{R}\mid \MC{M}^{*r}(A,\Omega)<\infty\}\\[2mm]
&=\sup\{r\in\MB{R}\mid \MC{M}^{*r}(A,\Omega)=+\infty\}
\end{align*}
\end{definition}

\textbf{Note:} The lower $r$-dimensional Minkowski content of $A$ relative to $\Omega$, denoted $\MC{M}_*^r(A,\Omega)$, is defined similarly except the $\liminf$ is taken, and the relative lower box dimension, denoted $\uline{\dim}_B(A,\Omega)$, is defined as above except $\MC{M}_*^r(A,\Omega)$ is used.
\end{frame}


















\begin{frame}{Geometric Oscillations}
\footnotesize

\begin{definition}
We say $(A,\Omega)$ is \textbf{Minkowski nondegenerate} if $0<\MC{M}_*^D(A,\Omega)\leq \MC{M}^{*D}(A,\Omega)<\infty.$

\vspace{3mm}
It follows then that $\uline{\dim}_B(A,\Omega)=\overline{\dim}_B(A,\Omega)=\dim_B(A,\Omega)=D$, called simply the \textbf{relative Minkowski dimension of $(A,\Omega)$}.
\end{definition}

\vspace{2mm}
\begin{definition}
If $\MC{M}_*^D(A,\Omega) =\MC{M}^{*D}(A,\Omega)$, this common value is denoted $\MC{M}^D(A,\Omega)$ and is called the \textbf{relative Minkowski content of $(A,\Omega)$}.
\end{definition}

\vspace{2mm}
\begin{definition}
If $\MC{M}^D(A,\Omega)$ exists and is different from $0$ and $+\infty$, we say $(A,\Omega)$ is \textbf{Minkowski measurable}.
\end{definition}
\end{frame}


















\begin{frame}{Geometric Oscillations}
\footnotesize
\begin{proof}
\begin{align*}
\frac{|\Lambda_t\cap \Omega|_3}{t^{3-2}}&=1-\frac{t}{2}(1+\lambda)+\frac{t^2}{2}\cdot\frac{\lambda^2-1}{\log\lambda}\left(\sum_{n\in\MB{Z}}\frac{t^{-in\mbf{p}}}{1-in\mbf{p}}-2\sum_{n\in\MB{Z}}\frac{t^{-in\mbf{p}}}{2-in\mbf{p}}+\sum_{n\in\MB{Z}}\frac{t^{-in\mbf{p}}}{3-in\mbf{p}}\right).
\end{align*}

\lspace
Taking the limit as $t\to0^+$, the last two terms go to 0, and all that remains is the constant term, 1. Thus, 
\[
\MC{M}^{*2}(\Lambda,\Omega)=\MC{M}_*^{2}(\Lambda,\Omega)=\MC{M}^{2}(\Lambda,\Omega)=\lim_{t\to0^+}\frac{|\Lambda_t\cap \Omega|_3}{t^{3-2}}=1,
\]
from which the conclusion follows.
\end{proof}
\end{frame}




















\begin{frame}{Recovering $V(t)$ from Explicit Formulas}
\linespread{1.5}\selectfont
\scriptsize

\begin{theorem}[FZF 5.3.13; Exact pointwise fractal tube formula via $\zeta_{A,\Omega}$]
Let $(A,\Omega)$ be a relative fractal drum in $\MB{R}^N$ which is strongly d-languid for some $\delta>0$ and with d-languidity exponent $\kappa_d\in\MB{R}$. Furthermore, let $k>\kappa_d-1$ be a nonnegative integer and assume that $\overline{\dim}_B(A,\Omega)<N$. Then the following exact pointwise fractal tube formula, expressed in terms of the distance zeta function $\zeta_{A,\Omega}\coloneqq \zeta_{A,\Omega}(\cdot;\delta)$, holds for every $t\in(0,\min\{1,\delta,B^{-1}\})$:

\[
V_{A,\Omega}^{[k]}(t)=\sum_{\omega\in\MC{D}(\zeta_{A,\Omega},\MB{C})}\res\left(\frac{t^{N-s+k}}{(N-s)_{k+1}}\zeta_{A,\Omega}(s),\omega\right)
\]

\vspace{3mm}
Here, $B$ is the constant appearing in \textbf{L2'}, $\kappa_d$ is the d-languidity exponent occuring in the statement of hypotheses \textbf{L1} and \textbf{L2}, $V_{A,\Omega}^{[k]}(t)$ is the $k$th primitive (antiderivative) of $V_{A,\Omega}(t)$, and $(s)_k$ is the Pocchammer (or rising factorial) symbol where $(s)_k\coloneqq s(s+1)(s+2)\cdots(s+k-1)$.
\end{theorem}
\end{frame}






















































\begin{frame}{Languidity: Definition}
\scriptsize

\vspace{3mm}
An admissible relative fractal drum $(A,\Omega)$ in $\MB{R}^N$ is said to be \textbf{d-languid} if for some fixed $\delta>0$, its relative distance zeta function $\zeta_{A,\Omega}(\cdot;\delta)$ satisfies the following growth conditions:

\lspace
There exists a real constant $\kappa_d\in\MB{R}$ and a two-sided sequence $(T_n)_{n\in\MB{Z}}$ of real numbers such that $T_{-n}<0<T_n$ for $n\geq 1$, and 
\[
\lim_{n\to\infty}T_n=+\infty\aspace\lim_{n\to\infty}T_{-n}=-\infty
\]
satisfying the following two hypotheses \textbf{L1} and \textbf{L2}:

\textbf{L1}: For a real fixed constant $c>N>\overline{\dim}_B(A,\Omega)$, there exists a positive constant $C>0$ such that for all $n\in\MB{Z}$ and all $\sigma$ in the interval $(S(T_n),c)$,
\[
|\zeta_{A,\Omega}(\sigma+iT_n;\delta)|\leq C(|T_n|+1)^{\kappa_d}.
\] 
\textbf{L2}: For all $t\in\MB{R}$, $|t|\geq 1$,
\[
|\zeta_{A,\Omega}(S(t)+it;\delta)|\leq C|t|^{\kappa_d},
\]
where $C$ is a positive constant which (without loss of generality) can be chosen to be the same one as in condition \textbf{L1}.

\end{frame}





















\begin{frame}{Languidity: Definition}
\scriptsize
\linespread{1.5}\selectfont
\lspace
In addition, an admissible relative fractal drum $(A,\Omega)$ in $\MB{R}^N$ is said to be \textbf{strongly d-languid} if for some fixed $\delta>0$, its relative distance zeta function $\zeta_{A,\Omega}(\cdot;\delta)$ satisfies condition \textbf{L1} with $S(T_n)\equiv -\infty$, i.e. for every $\sigma<c$, and additionally there exists a sequence of screens $\mbf{S}_m:t\mapsto  S_m(t)+it$ for $m\geq 1$, $t\in\MB{R}$ with $\sup S_m\to-\infty$ as $m\to\infty$ and with a uniform Lipschitz bound $\sup_{m\geq 1}\|S_m\|_{\rm{Lip}}<\infty$ such that the following condition holds:

\lspace
\textbf{L2'}: There exist constants $B,C>0$ such that for all $t\in \MB{R}$ and $m\geq 1$,
\[
|\zeta_{A,\Omega}(S_m(t)+it;\delta)|\leq CB^{|S_m(t)|}(|t|+1)^{\kappa_d}.
\]
\end{frame}










\begin{frame}[t]{Languidity}
\small
\linespread{1.5}\selectfont

\textbf{Note.} Hypothesis \textbf{L1} is a polynomial growth condition along horizontal line segments in the plane, while hypotheses \textbf{L2} and \textbf{L2'} are polynomial growth conditions along the vertical direction of the screen(s). These hypotheses are necessary prerequisites for determining the pointwise and distributional fractal tubular volume formulas with and without error term.
\end{frame}























\begin{frame}[t]{Languidity}
\footnotesize

\lspace
\begin{proposition}
The RFDs constructed above are strongly d-languid.
\end{proposition}


\lspace
\begin{proofs}
Recall
\[
\zeta_{\Lambda,\Omega}(s)=\frac{\lambda^2-1}{(s-2)(s-1)(\lambda^s-1)}.
\]

\lspace
For any $s$ that is not a singularity of $\zeta_{\Lambda,\Omega}$, we have
\[
\left|\zeta_{\Lambda,\Omega}(s)\right|=\left|\frac{\lambda^2-1}{(s-2)(s-1)(\lambda^s-1)}\right|\leq\frac{\lambda^2}{|s-2||s-1||\lambda^s-1|}
\]
\end{proofs}
\end{frame}























\begin{frame}[t]{Languidity}
\footnotesize


\begin{proofs}

We'll verify \textbf{L1} first. Choose our sequence $(T_n)_{n\in\MB{Z}}$ to be given by $T_n=\frac{(2n+1)\pi}{\log\lambda}$. This satisfies all the necessary requirements of our sequence, and for any $\sigma<c$, we have
\begin{align*}
\left|\zeta_{\Lambda,\Omega}(\sigma+iT_n;\delta)\right|&\leq\frac{\lambda^2}{|\sigma-2+iT_n||\sigma-1+iT_n||\lambda^\sigma\lambda^{iT_n}-1|}\\[2mm]
&\leq\frac{\lambda^2}{|T_n||T_n||e^{(\log\lambda)\sigma}e^{i(\log\lambda)T_n}-1|}.
\end{align*}

By our choice of sequence, $e^{i(\log\lambda)T_n}=e^{i(2n+1)\pi}=-1$ for all $n\in \MB{Z}$. Consequently,
\[
|e^{(\log\lambda)\sigma}e^{i(\log\lambda)T_n}-1|=|e^{(\log\lambda)\sigma}+1|\geq1
\]
for all $\sigma<c$. \pause It is important to note that this choice of $T_n$ avoids any singularities since $T_n\neq n\mbf{p}=n\frac{2\pi}{\log\lambda}$ for any $n$. Thus,
\[
\left|\zeta_{\Lambda,\Omega}(\sigma+iT_n;\delta)\right|\leq\frac{\lambda^2}{|T_n|^2}.
\]
\end{proofs}
\end{frame}






















\begin{frame}[t]{Languidity}
\footnotesize

\begin{proofs}
For any $x\in \MB{R}$ and any $\lambda\in\MB{N}\setminus\{1\}$, if $|x|\geq 1$, then $\lambda|x|\geq |x|+1$. Consequently, $\frac{1}{|x|^2}\leq\frac{\lambda^2}{(|x|+1)^2}$. Since $\min_{n\in\MB{Z}}\{|T_n|\}=|T_0|=\frac{\pi}{\log\lambda}>1$, it follows that

\[
\left|\zeta_{\Lambda,\Omega}(\sigma+iT_n;\delta)\right|\leq\frac{\lambda^2}{|T_n|^2}\leq\frac{\lambda^5}{(|T_n|+1)^2}=\lambda^5(|T_n|+1)^{-2}.
\]

Choosing $C=\lambda^5$ and $\kappa_d=-2$, we see that we have verified \textbf{L1}.

\lspace
\lspace
To show condition \textbf{L2'}, let our sequence of screens be given by
\[
\mathbf{S}_m=\{S_m(t)+it:m\in\MB{N},\,t\in\MB{R}\}=\{-m+it:m\in\MB{N},\,t\in\MB{R}\}
\]
Then clearly $\sup S_m\to-\infty$ as $m\to\infty$ and $\sup_{m\geq1}\|S_m\|_{\rm{Lip}}=1<\infty$.
\end{proofs}
\end{frame}












\begin{frame}[t]{Languidity}
\footnotesize
\linespread{1.5}\selectfont

\begin{proofs}
\begin{align*}
|\zeta_{\Lambda,\Omega}(S_m(t)+it;\delta)|&=\left|\frac{\lambda^2-1}{(s-2)(s-1)(\lambda^s-1)}\right|_{s\in\mbf{S}_m}\\[2mm]
&\leq\frac{\lambda^2}{|-m-2+it||-m-1+it||\lambda^{-m}\lambda^{it}-1|}\\[2mm]
&\leq\frac{\lambda^2}{|m||m||\lambda^{-m}-1|}.
\end{align*}

\lspace
Since $m\geq1$, $\lambda^{-m}<\lambda^{-1}<1$. Thus $|\lambda^{-m}-1|>\lambda^{-1}$, whence $\frac{1}{|\lambda^{-m}-1|}\leq\lambda$, so
\[
|\zeta_{\Lambda,\Omega}(S_m(t)+it;\delta)|\leq\frac{\lambda^3}{|m|^2}\leq\frac{\lambda^5}{(|m|+1)^2}=\lambda^5(|m|+1)^{-2}\sim\lambda^5(|t|+1)^{-2}.
\]
\end{proofs}
\end{frame}




























\begin{frame}[t]{Languidity}
\footnotesize
\linespread{1.5}\selectfont

\begin{proof}
 
Choosing $B=1$, we have $B^{|S_m(t)|}=1$ for all $m\geq1$, and so the above calculation yields
\[
\left|\zeta_{\Lambda,\Omega}(S_m(t)+it;\delta)\right|\leq\lambda^5(|t|+1)^{-2},
\]
and we have verified condition \textbf{L2'}. Therefore, $(\Lambda,\Omega)$ is strongly languid, and the exact pointwise and distributional tube formulas hold for $V(t)$.
\end{proof}
\end{frame}




























\begin{frame}{Recovering $V(t)$ from an Explicit Formula}
\small

By the theorem above [FZF 5.3.13], we have

\[
V_{\Lambda,\Omega}^{[k]}(t)=\sum_{\omega\in\MC{D}(\zeta_{\Lambda,\Omega},\MB{C})}\res\left(\frac{t^{3-s+k}}{(3-s)_{k+1}}\zeta_{\Lambda,\Omega}(s),\omega\right).
\]
In particular, for $k=0$,
\[
V_{\Lambda,\Omega}(t)=\sum_{\omega\in\MC{D}(\zeta_{\Lambda,\Omega},\MB{C})}\res\left(\frac{t^{3-s}}{3-s}\zeta_{\Lambda,\Omega}(s),\omega\right).
\]
\end{frame}
















\begin{frame}{Recovering $V(t)$ from an Explicit Formula}
\small
Observe that when $\omega=2$, we have
\begin{align*}
\res\left(\frac{t^{3-s}}{3-s}\zeta_{\Lambda,\Omega}(s),2\right)&=\lim_{s\to2}\cancel{(s-2)}\cdot\frac{\lambda^2-1}{\cancel{(s-2)}(s-1)(\lambda^s-1)}\cdot\frac{t^{3-s}}{3-s}\\[2mm]
&=\frac{(\lambda^2-1)t}{(1)(\lambda^2-1)(1)}=t,
\end{align*}

which is the first term in our volume formula from above.

\end{frame}
























\begin{frame}{Recovering $V(t)$ from an Explicit Formula}
\small
When $\omega=1$, we have
\begin{align*}
\res\left(\frac{t^{3-s}}{3-s}\zeta_{\Lambda,\Omega}(s),1\right)&=\lim_{s\to1}\cancel{(s-1)}\cdot\frac{\lambda^2-1}{(s-2)\cancel{(s-1)}(\lambda^s-1)}\cdot\frac{t^{3-s}}{3-s}\\[2mm]
&=\frac{(\lambda^2-1)t^2}{(-1)(\lambda-1)(2)}\\[2mm]
&=\frac{(\lambda+1)\cancel{(\lambda-1)}t^2}{(-1)\cancel{(\lambda-1)}(2)}=-\frac{t^2}{2}(\lambda+1),
\end{align*}

which is the second term in our volume formula from above. 
\end{frame}





















\begin{frame}{Recovering $V(t)$ from an Explicit Formula}
\footnotesize
When $\omega=0+in\mbf{p}$, for each $n\in\MB{Z}$ we have
\begin{align*}
\res\left(\frac{t^{3-s}}{3-s}\zeta_{\Lambda,\Omega}(s),0+in\mbf{p}\right)&=\lim_{s\to in\mbf{p}}\frac{(s-in\mbf{p})(\lambda^2-1)t^{3-s}}{(s-2)(s-1)(3-s)(\lambda^s-1)}\\[2mm]
&=\lim_{s\to in\mbf{p}}\frac{t^3(\lambda^2-1)t^{-s}}{(1-s)(2-s)(3-s)}\cdot\lim_{s\to in\mbf{p}}\frac{s-in\mbf{p}}{\lambda^s-1}\\[2mm]
&=\frac{t^3(\lambda^2-1)t^{-in\mbf{p}}}{(1-in\mbf{p})(2-in\mbf{p})(3-in\mbf{p})}\cdot\lim_{s\to in\mbf{p}}\frac{1}{(\log\lambda)\lambda^s}\\[2mm]
&=\frac{t^3(\lambda^2-1)}{\log\lambda}\cdot\frac{t^{-in\mbf{p}}}{(1-in\mbf{p})(2-in\mbf{p})(3-in\mbf{p})}.
\end{align*}
\end{frame}




















\begin{frame}{Recovering $V(t)$ from an Explicit Formula}
\footnotesize
Using partial fraction decomposition, we get
\begin{align*}
\res\left(\frac{t^{3-s}}{3-s}\zeta_{\Lambda,\Omega}(s),0+in\mbf{p}\right)&=\frac{t^3(\lambda^2-1)}{\log\lambda}\cdot\frac{t^{-in\mbf{p}}}{(1-in\mbf{p})(2-in\mbf{p})(3-in\mbf{p})}\\[2mm]
&=\frac{t^3}{2}\cdot\frac{(\lambda^2-1)}{\log\lambda}\left(\frac{t^{-in\mbf{p}}}{1-in\mbf{p}}-2\cdot\frac{t^{-in\mbf{p}}}{2-in\mbf{p}}+\frac{t^{-in\mbf{p}}}{3-in\mbf{p}}\right),
\end{align*}

whence
\begin{align*}
&\sum_{n\in\MB{Z}}\res\left(\frac{t^{3-s}}{3-s}\zeta_{\Lambda,\Omega}(s),0+in\mbf{p}\right)\\[2mm]
&=\frac{t^3}{2}\cdot\frac{(\lambda^2-1)}{\log\lambda}\left(\sum_{n\in\MB{Z}}\frac{t^{-in\mbf{p}}}{1-in\mbf{p}}-2\sum_{n\in\MB{Z}}\frac{t^{-in\mbf{p}}}{2-in\mbf{p}}+\sum_{n\in\MB{Z}}\frac{t^{-in\mbf{p}}}{3-in\mbf{p}}\right)
\end{align*}
which is the third, oscillatory term in our tubular volume formula.
\end{frame}










\begin{frame}{Recovering $V(t)$ from an Explicit Formula}
\footnotesize
Behold,

\begin{align*}
V_{\Lambda,\Omega}(t)&=\sum_{\omega\in\MC{D}(\zeta_{\Lambda,\Omega},\MB{C})}\res\left(\frac{t^{3-s}}{3-s}\zeta_{\Lambda,\Omega}(s),\omega\right)\\[2mm]
&=t-\frac{t^2}{2}(1+\lambda)+\frac{t^3}{2}\cdot\frac{\lambda^2-1}{\log\lambda}\left(\sum_{n\in\MB{Z}}\frac{t^{-in\mbf{p}}}{1-in\mbf{p}}-2\sum_{n\in\MB{Z}}\frac{t^{-in\mbf{p}}}{2-in\mbf{p}}+\sum_{n\in\MB{Z}}\frac{t^{-in\mbf{p}}}{3-in\mbf{p}}\right).
\end{align*}
\end{frame}











\begin{frame}
\vspace{2cm}
\centering
{\Huge \bf Thank you!}
\vspace{2cm}
\end{frame}















\begin{frame}{References}
\scriptsize
\bibliographystyle{amsalpha}
\nocite{*}
\bibliography{Languidity_biblio}
\end{frame}



























\end{document}