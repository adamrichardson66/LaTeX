\documentclass[12pt,oneside,english]{amsart}
\usepackage[T1]{fontenc}
\usepackage{geometry}
\usepackage{parskip}
\geometry{verbose,tmargin=0.75in,bmargin=0.75in,lmargin=0.75in,rmargin=0.75in,headheight=0.75cm,headsep=1cm,footskip=1cm}
\setlength{\parskip}{\medskipamount}
\usepackage{setspace}
\onehalfspacing
\usepackage{multicol}
\usepackage{graphicx}
\usepackage{ulem}
\usepackage[export]{adjustbox}
\usepackage{enumitem}
\setlist[enumerate,1]{label=\textbf{\arabic*.}}
\makeatother
\usepackage{babel}
\usepackage{tikz, pgfplots}
\usepackage{mathtools}
\setlist[enumerate]{topsep=0.4cm}

\DeclarePairedDelimiter{\abs}{\lvert}{\rvert}


\begin{document}

\title{MATH 70 - Activity \#18 - Polar Coordinates}

\maketitle
\thispagestyle{empty}

\hrule
\medskip
\hrule

\vspace{1cm}


\begin{enumerate}[leftmargin=*]
\setlength\itemsep{1.5cm}

\item First, plot the following point given in polar coordinates, then find two other equivalent pairs of polar coordinates of this point, one with $r>0$ and one with $r<0$. Second, find the Cartesian representation of these points.

\begin{enumerate}
\item $\left(1,\frac{\pi}{4}\right)$
\item $\left(-2,\frac{3\pi}{2}\right)$
\end{enumerate}


\item Sketch the following region in the plane: $2<r<3$, \hspace{0.25cm}$\frac{5\pi}{3}\leq\theta\leq\frac{7\pi}{3}$

\item Sketch the following curves by first sketching the graph of $r$ as a function of $\theta$ in Cartesian coordinates.

\begin{enumerate}
\item $r=1+2\cos\theta$
\item $r=3\cos\theta$
\item $r=-5\sin\theta$
\item $r=1+5\sin\theta$
\item $r=2+\sin3\theta$
\end{enumerate}

\item Find the points on the given curve where the tangent line is horizontal or vertical.

\begin{enumerate}
\item $r=1+\cos\theta$
\item $r=e^\theta$
\end{enumerate}
\end{enumerate}
\end{document}