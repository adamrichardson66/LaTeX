\documentclass[12pt,oneside,english]{amsart}
\usepackage[T1]{fontenc}
\usepackage{geometry}
\usepackage{parskip}
\geometry{verbose,tmargin=0.75in,bmargin=0.75in,lmargin=0.75in,rmargin=0.75in,headheight=0.75cm,headsep=1cm,footskip=1cm}
\setlength{\parskip}{\medskipamount}
\usepackage{setspace}
\onehalfspacing
\usepackage{multicol}
\usepackage{graphicx}
\usepackage{ulem}
\usepackage[export]{adjustbox}
\usepackage{enumitem}
\setlist[enumerate,1]{label=\textbf{\arabic*.}}
\makeatother
\usepackage{babel}
\usepackage{tikz, pgfplots}
\usepackage{mathtools}

\DeclarePairedDelimiter{\abs}{\lvert}{\rvert}


\begin{document}

\title{MATH 70 - Activity \#5 - Integration Strategy Solutions}

\maketitle
\thispagestyle{empty}

\hrule

Evaluate the following integrals using any necessary technique. Make sure your submitted work is clean and clear, with evaluations of integrals elucidated as an annotated chain of \textit{true} equalities.

\medbreak
\hrule

\vspace{1cm}
\begin{enumerate}[leftmargin=*]
\setlength\itemsep{2cm}

\item $\displaystyle \int t\sin t\cos t\,dt$

First, observe that

\begin{align*}
\int t\sin t\cos t\,dt&=\int t\cdot\frac{1}{2}(2\sin t\cos t)\,dt \\
&=\frac{1}{2}\int t\sin2t\,dt.
\end{align*}

Now we use integration by parts: Let $u=t$ and $dv=\sin2t\,dt$. Then $du=dt$ and $v=-\frac{1}{2}\cos t$. Thus,

\begin{align*}
\frac{1}{2}\int t\sin2t\,dt&=\frac{1}{2}\left(-\frac{1}{2}t\cos2t-\int-\frac{1}{2}\cos2t\,dt\right) \\
&=-\frac{1}{4}t\cos2t+\frac{1}{4}\int\cos2t\,dt \\
&=-\frac{1}{4}t\cos2t+\frac{1}{8}\sin2t+C.
\end{align*}

\item $\displaystyle \int_{-1}^2|e^x-1|\,dx$

There is no "formula" that allows us to integrate a function with absolute values, we just have to go back to the definition:

\[
|e^x-1|=\begin{cases}
e^x-1 & \text{if }e^x-1\geq0 \\
-(e^x-1) & \text{if }e^x-1<0
\end{cases}
\hspace{0.5cm}=\hspace{0.5cm}\begin{cases}
e^x-1 & \text{if }x\geq0 \\
1-e^x & \text{if }x<0
\end{cases}.
\]

The second equality is derived by solving the two conditional inequalities $e^x-1\geq0$ and $e^x-1<0$ for $x$. Now we can split up our integral into two easy intergals:

\begin{align*}
\int_{-1}^2|e^x-1|\,dx&=\int_{-1}^01-e^x\,dx+\int_0^2e^x-1\,dx \\
&=\Big[x-e^x\Big]_{-1}^0+\Big[e^x-x\Big]_0^2 \\
&=(0-1)-(-1-e^{-1})+(e^2-2)-(1-0) \\
&=e^2+\frac{1}{e}-3.
\end{align*}

\item $\displaystyle \int\frac{1}{x^2\sqrt{4x+1}}\,dx$

This one is tricky. Start by using a $u$-sub: Let $u=\sqrt{4x+1}$. Then

\[
du=\frac{4}{2\sqrt{4x+1}}\,dx=\frac{2}{\sqrt{4x+1}}\,dx.
\]

Additionally, we can solve for $x$ to get $x=\frac{1}{4}(u^2-1)$. Thus,

\begin{align*}
\int\frac{1}{x^2\sqrt{4x+1}}\,dx&=\int\frac{\frac{1}{2}}{\left(\frac{1}{4}(u^2-1)\right)^2}\,du \\
&=\int\frac{\frac{1}{2}}{\frac{1}{16}(u^2-1)^2}\,du \\
&=8\int\frac{1}{(u^2-1)^2}\,du \\
&=8\int\frac{1}{(u+1)^2(u-1)^2}\,du
\end{align*}

Now we use partial fraction decomposition:

\begin{align*}
\frac{1}{(u^2-1)^2}&=\frac{A}{u+1}+\frac{B}{(u+1)^2}+\frac{C}{u-1}+\frac{D}{(u-1)^2} \\
1&=A(u+1)(u-1)^2+B(u-1)^2+C(u-1)(u+1)^2+D(u+1)^2.
\end{align*}

Setting $u=1$ yields $D=\frac{1}{4}$. Setting $u=-1$ yields $B=\frac{1}{4}$. To get the other two quantities we need to compare coefficients so we need to multiply everything out and regroup:

\begin{align*}
1&=A(u+1)(u-1)^2+B(u-1)^2+C(u-1)(u+1)^2+D(u+1)^2 \\
1&=Au^3-Au^2-Au+A+Bu^2-2Bu+B+Cu^3+Cu^2-Cu-C+Du^2+2Du+D \\
1&=(A+C)u^3+(-A+B+C)u^2+(-A-2B-C+2D)u+(A+B-C+D).
\end{align*}

Comparing coefficients we see that $A+C=0$ and $A+B-C+D=1$. Substituting $D=\frac{1}{4}$ and $B=\frac{1}{4}$ into the second equation yields $A-C=\frac{1}{2}$ so $A=\frac{1}{2}+C$. Using the first equation, we have

\begin{align*}
A+C&=0 \\
\frac{1}{2}+C+C&=0 \\
C&=-\frac{1}{4}
\end{align*}

which means that $A=\frac{1}{4}$ since $A+C=0$. Now we return to the integral.

\begin{align*}
\int\frac{1}{x^2\sqrt{4x+1}}\,dx&=8\int\frac{1}{(u+1)^2(u-1)^2}\,du \\
&=8\int\frac{\frac{1}{4}}{u+1}+\frac{\frac{1}{4}}{(u+1)^2}+\frac{-\frac{1}{4}}{u-1}+\frac{\frac{1}{4}}{(u-1)^2}\,du \\
&=2\int\frac{1}{u+1}+\frac{1}{(u+1)^2}-\frac{1}{u-1}+\frac{1}{(u-1)^2}\,du \\
&=2\ln|u+1|-\frac{2}{u+1}-2\ln|u-1|-\frac{2}{u-1}+C \\
&=2\ln(\sqrt{4x+1}+1)-\frac{2}{\sqrt{4x+1}+1}-2\ln|\sqrt{4x+1}-1|-\frac{2}{\sqrt{4x+1}-1}+C.
\end{align*}

\item $\displaystyle \int_0^1\frac{y^2}{\sqrt{y^2+1}}\,dy$

We can use trig sub for this. Let $y=\tan\theta$. Then $dy=\sec^2\theta\,d\theta$ and $\sqrt{y^2+1}=\sec\theta$. For our bounds of integration, when $y=0$, $\theta=0$ and when $y=1$, $\theta=\frac{\pi}{4}$. Thus,

\begin{align*}
\int_0^1\frac{y^2}{\sqrt{y^2+1}}\,dy&=\int_0^\frac{\pi}{4}\frac{\tan^2\theta}{\sec\theta}\sec^2\theta\,d\theta \\
&=\int_0^\frac{\pi}{4}\tan^2\theta\sec\theta\,d\theta \\
&=\int_0^\frac{\pi}{4}(\sec^2\theta-1)\sec\theta\,d\theta \\
&=\int_0^\frac{\pi}{4}\sec^3\theta-\sec\theta \,d\theta \\
&=\left[\frac{1}{2}\sec\theta\tan\theta+\frac{1}{2}\ln|\sec\theta+\tan\theta|-\ln|\sec\theta+\tan\theta|\right]_0^\frac{\pi}{4} \\
&=\left[\frac{1}{2}\sec\theta\tan\theta-\frac{1}{2}\ln|\sec\theta+\tan\theta|\right]_0^\frac{\pi}{4} \\
&=\frac{1}{2}\cdot\frac{2}{\sqrt{2}}\cdot1-\frac{1}{2}\ln\left(\frac{2}{\sqrt{2}}+1\right) \\
&=\frac{1}{\sqrt{2}}-\frac{1}{2}\ln(\sqrt{2}+1)
\end{align*}


\item $\displaystyle \int\frac{x-3}{(x^2+2x+4)^2}\,dx$

We need to use an old trick for this one: completing the square! Completing
the square in the denominator yields
\[
\int\frac{x-3}{(x^{2}+2x+4)^{2}}\,dx=\int\frac{x-3}{[(x+1)^{2}+3]^{2}}.
\]
Let $u=x+1$. Then $du=dx$ and $x=u-1$, giving us
\begin{align*}
 & =\int\frac{u-1-3}{(u^{2}+3)^{2}}\,du\\
 & =\int\frac{u}{(u^{2}+3)^{2}}\,du-4\int\frac{1}{(u^{2}+3)^{2}}\,du.
\end{align*}
Now we need to use two tools. Let's focus on the first integral. For
the first integral, let $v=u^{2}+3$. Then $dv=2u\,du$, so we have
\[
\int\frac{u}{(u^{2}+3)^{2}}\,du=\frac{1}{2}\int\frac{1}{v^{2}}\,dv=-\frac{1}{2v}.\,(*)
\]
For the second integral, we use trig substitution after realizing
that $u^{2}+3=u^{2}+(\sqrt{3})^{2}$. Let $u=\sqrt{3}\tan\theta$.
Then $du=\sqrt{3}\sec^{2}\theta\,d\theta$ and 
\[
u^{2}+3=(\sqrt{3}\tan\theta)^{2}+3=3\tan^{2}\theta+3=3(\tan^{2}\theta+1)=3\sec^{2}\theta.
\]
So our integral becomes
\begin{align*}
-4\int\frac{1}{(u^{2}+3)^{2}}\,du & =-4\int\frac{\sqrt{3}\sec^{2}\theta}{9\sec^{4}\theta}\,d\theta\\
 & =-\frac{4\sqrt{3}}{9}\int\frac{1}{\sec^{2}\theta}\,d\theta\\
 & =-\frac{4\sqrt{3}}{9}\int\cos^{2}\theta\,d\theta\\
 & =-\frac{4\sqrt{3}}{9}\int\frac{1+\cos(2\theta)}{2}\,d\theta\\
 & =-\frac{2\sqrt{3}}{9}\int1+\cos(2\theta)\,d\theta\\
 & =-\frac{2\sqrt{3}}{9}\left(\theta+\frac{1}{2}\sin2\theta\right)+C.\,(**)
\end{align*}
Combining $(*)$ and $(**)$ we are almost there:
\begin{align*}
\int\frac{x-3}{(x^{2}+2x+4)^{2}}\,dx & =-\frac{1}{2v}-\frac{2\sqrt{3}}{9}\left(\theta+\frac{1}{2}\sin2\theta\right)+C\\
 & =-\frac{1}{2(u^{2}+3)}-\frac{2\sqrt{3}}{9}\left(\theta+\frac{1}{2}\sin2\theta\right)+C\\
 & =-\frac{1}{2\left[(x+1)^{2}+3\right]}-\frac{2\sqrt{3}}{9}\left(\theta+\frac{1}{2}\underline{\sin2\theta}\right)+C.
\end{align*}
Since we chose $u=\sqrt{3}\tan\theta$, $\tan\theta=\frac{u}{\sqrt{3}}=\frac{x+1}{\sqrt{3}}$,
and thus $\theta=\arctan\left(\frac{x+1}{\sqrt{3}}\right)$, but we
still need to deal with that $\sin2\theta$. Recall the double angle
formula $\sin2\theta=2\sin\theta\cos\theta$. Substituting this in
gives us
\[
=-\frac{1}{2\left[(x+1)^{2}+3\right]}-\frac{2\sqrt{3}}{9}\left(\theta+\sin\theta\cos\theta\right)+C.
\]

\vspace{0.5cm}
\begin{multicols}{2}By constructing a right triangle, we find that 
\[
\sin\theta=\frac{x+1}{\sqrt{(x+1)^{2}+3}},\text{ and}
\]
\[
\cos\theta=\frac{\sqrt{3}}{\sqrt{(x+1)^{2}+3}},
\]
\columnbreak
\begin{center}
\begin{tikzpicture}

\node [label={[shift={(0.8,-0.1)}]$\theta$}] {};
\draw (0,0)
	-- node[below=2pt] {$\sqrt{3}$} (4,0)
	-- node[right=2pt] {$x+1$} (4,4)
	-- node[left=0.25cm] {$\sqrt{(x+1)^2+3}$} cycle;

\end{tikzpicture}
\end{center}
\end{multicols}so, behold, our final integral becomes

\begin{align*}
\int\frac{x-3}{(x^{2}+2x+4)^{2}}\,dx & =-\frac{1}{2\left[(x+1)^{2}+3\right]}-\frac{2\sqrt{3}}{9}\left(\arctan\left(\frac{x+1}{\sqrt{3}}\right)+\frac{x+1}{\sqrt{(x+1)^{2}+3}}\cdot\frac{\sqrt{3}}{\sqrt{(x+1)^{2}+3}}\right)+C\\
 & =-\frac{1}{2\left[(x+1)^{2}+3\right]}-\frac{2\sqrt{3}}{9}\arctan\left(\frac{x+1}{\sqrt{3}}\right)-\frac{2\cdot3\cdot(x+1)}{9\left[(x+1)^{2}+3\right]}+C\\
 & =-\frac{1}{2\left[x^{2}+2x+4\right]}-\frac{2\sqrt{3}}{9}\arctan\left(\frac{x+1}{\sqrt{3}}\right)-\frac{2(x+1)}{3(x^{2}+2x+4)}+C.
\end{align*}





\end{enumerate}


\end{document}