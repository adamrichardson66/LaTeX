\documentclass[11pt,oneside,english]{amsart}
\usepackage[T1]{fontenc}
\usepackage{geometry}
\usepackage{parskip}
\geometry{verbose,tmargin=0.65in,bmargin=0.65in,lmargin=0.75in,rmargin=0.75in,headheight=0.75cm,headsep=1cm,footskip=1cm}
\setlength{\parskip}{7mm}
\usepackage{setspace}
\onehalfspacing
%\pagenumbering{gobble}

\usepackage{comment}
\usepackage{bbm}
%\usepackage{multicol}
%\usepackage{graphicx}
%\usepackage{adjustbox}
\usepackage{amssymb}
\usepackage{tikz}
\usetikzlibrary{cd, quotes}
%\usepackage{pgfplots}
%\usepackage{pgffor}
\usepackage{ulem}
\usepackage{adjustbox}
\usepackage{bm}
%\usepackage{stmaryrd}
\usepackage{mathrsfs}
\usepackage{cancel}
\usepackage{mathtools}
\usepackage{commath}
\DeclarePairedDelimiter{\ceil}{\lceil}{\rceil}
\DeclarePairedDelimiter\floor{\lfloor}{\rfloor}
\usepackage[shortlabels]{enumitem}
\setlist[enumerate,1]{label=\textbf{\arabic*.}}
\usepackage{color, colortbl}
\definecolor{Gray}{gray}{0.9}
\usepackage{babel}
\usepackage{mdframed}
\usepackage{esint}
\usepackage[yyyymmdd]{datetime}
\renewcommand{\dateseparator}{--}
\usepackage{url}
\usepackage[unicode=true,pdfusetitle,
 bookmarks=true,bookmarksnumbered=false,bookmarksopen=false,
 breaklinks=false,pdfborder={0 0 1},backref=false,colorlinks=true]
 {hyperref}
\hypersetup{urlcolor=blue}

\usepackage[all]{xypic}




\theoremstyle{definition}
\newtheorem{theorem}{Theorem}
\newtheorem*{theorem*}{Theorem}
\newtheorem*{proposition*}{Proposition}
\newtheorem{corollary}{Corollary}
\newtheorem*{lemma}{Lemma}
\newtheorem*{example}{Example}
\newtheorem*{examples}{Examples}
\newtheorem*{definition}{Definition}
\newtheorem*{note}{Nota Bene}

\newcommand{\aspace}{\hspace{7mm}\text{and}\hspace{7mm}}
\newcommand{\ospace}{\hspace{7mm}\text{or}\hspace{7mm}}
\newcommand{\pspace}{\hspace{10mm}}
\newcommand{\lspace}{\vspace{5mm}}
\newcommand{\lhe}{\stackrel{\text{L'H}}{=}}
\newcommand{\lom}[2]{\lim_{{#1}\rightarrow{#2}}}
\newcommand{\ve}{\varepsilon}
\renewcommand{\Re}{\text{Re }}
\renewcommand{\Im}{\text{Im }}
\newcommand{\Log}{\text{Log }}
\newcommand{\ess}{\text{ess sup}}
\newcommand{\dd}[2]{\frac{d{#1}}{d{#2}}}
\newcommand{\pp}[2]{\frac{\partial{#1}}{\partial{#2}}}
\newcommand{\DD}[2]{\frac{\Delta{#1}}{\Delta{#2}}}
\newcommand{\ovec}[1]{\overrightarrow{#1}}
\newcommand{\MC}[1]{\mathcal{#1}}
\newcommand{\MB}[1]{\mathbb{#1}}
\newcommand{\MF}[1]{\mathfrak{#1}}
\newcommand{\MS}[1]{\mathscr{#1}}
\newcommand{\mbf}[1]{\,\mathbf{#1}}
\renewcommand{\vec}[1]{\underline{#1}}
\newcommand{\im}{\text{im\,}}
\newcommand{\Hom}{\text{Hom}}
\newcommand{\coker}{\text{coker\,}}



\def\<#1>{\mathinner{\langle#1\rangle}}

\makeatletter
\g@addto@macro\normalsize{%
  \setlength\belowdisplayshortskip{5mm}
}
\makeatother





\begin{document}

\rightline{Adam D. Richardson}
\rightline{224 - Homological Algebra}
\rightline{Grifo, Elo\'isa}
\rightline{HW 2}
\rightline{\today}

\lspace




\begin{enumerate}[leftmargin=*]
\itemsep5mm

\item Let $f$ be a map of complexes of $R$-modules. Show that if $\ker f$ and $\text{coker}\,f$ are exact complexes, then $f$ is a quasi-isomorphism. Is the converse true?

\begin{proof}
Let $F$ and $G$ be complexes of $R$-modules and let $f:F\to G$. Then for each $n$, $\ker f_n\subseteq F_n$ and $\coker f_n\subseteq G_n$. Supposing that $\ker f$ and $\coker f$ are exact complexes, first consider the short exact sequence:

\begin{center}
\begin{tikzcd}
0 \arrow[r] & \ker f_n \arrow[r] & F_n \arrow[r] & \im f_n \arrow[r] & 0
\end{tikzcd}
\end{center}
By Theorem 9.67 (Long exact sequence in homology), there are connecting homomorphisms $\partial:H_n(\im f_n)\to H_{n-1}(\ker f_n)$ such that 
\begin{center}
\begin{tikzcd}
\cdots \arrow[r] & H_{n+1}(\im f_n) \arrow[r, "\partial"] & H_n(\ker f_n) \arrow[r] & H_n(F_n) \arrow[r] & H_n(\im f_n) \arrow[r, "\partial"] & H_{n-1}(\ker f_n) \arrow[r] & \cdots
\end{tikzcd}
\end{center}
is exact. But by exactness of $\ker f$, this becomes
\begin{center}
\begin{tikzcd}
\cdots \arrow[r] & H_{n+1}(\ker f_n) \arrow[r, "\partial"] & H_n(\ker f_n) \arrow[r] & H_n(F_n) \arrow[r] & H_n(\ker f_n) \arrow[r, "\partial"] & H_{n-1}(\ker f_n) \arrow[r] & \cdots
\end{tikzcd}
\end{center}
Thus, $H_n(F_n)\cong H_n(\im f_n)$ so $F_n\to \im f_n$ is a quasi-isomorphism. Next consider the short exact sequence:
\begin{center}
\begin{tikzcd}
0 \arrow[r] & \im f_n \arrow[r] & G_n \arrow[r] & \coker f_n \arrow[r] & 0
\end{tikzcd}
\end{center}
Again by the long exact sequence in homology and exactness of $\coker f$, we get
\begin{center}
\begin{tikzcd}
\cdots \arrow[r] & H_{n+1}(\coker f_n) \arrow[r, "\partial"] & H_n(\ker f_n) \arrow[r] & H_n(G_n) \arrow[r] & H_n(\coker f_n) \arrow[r, "\partial"] & H_{n-1}(\ker f_n) \arrow[r] & \cdots
\end{tikzcd}
\end{center}
so $H_n(\im f_n)\cong H_n(G_n)$ meaning $\im f_n \to G_n$ is a quasi-isomorphism. Putting these together, we have that $F_n\to\im f_n\to G_n$ is a quasi-isomorphism, but this is just $f_n$, and so $f$ is a quasi-isomorphism.
\end{proof}

The converse is false. Consider the short exact sequences $A$ and $B$ and the quasi-isomorphism $f$ between them given below.
\begin{center}
\begin{tikzcd}
B&=&0 \arrow[r]                & 0 \arrow[r]                     & \MB{Z} \arrow[r]                        & \MB{Z} \arrow[r]           & 0                \\
A \arrow[u, "f"]&=&0 \arrow[r] \arrow[u, "0"] & \MB{Z} \arrow[r] \arrow[u, "0"] & \MB{Z} \arrow[r] \arrow[u, "\text{Id}"] & 0 \arrow[r] \arrow[u, "0"] & 0 \arrow[u, "0"]
\end{tikzcd}
\end{center}
Then $\ker f$ is 
\begin{center}
\begin{tikzcd}
0 \arrow[r] & \MB{Z} \arrow[r] & 0 \arrow[r] & 0 \arrow[r] & 0
\end{tikzcd}
\end{center}
and $\coker f$ is
\begin{center}
\begin{tikzcd}
0 \arrow[r] & 0 \arrow[r] & 0 \arrow[r] & \MB{Z} \arrow[r] & 0
\end{tikzcd}
\end{center}
which are not exact.


\pagebreak

\item A complex is \textit{contractible} if its identity map is null-homotopic. Show that every contractible complex of $R$-modules is an exact sequence.

\begin{proof}
Let $\text{Id},0:F\to F$ be maps of complexes and suppose that $F$ is contractible. Then $\text{Id}$ is homotopic to 0, so there exists a sequence of maps $h_n:F_n\to F_{n+1}$ such that $d_{n+1}h_n+h_{n-1}d_n=\text{Id}$. Thus,
\begin{align*}
d_n&=d_n\text{Id}\\[2mm]
&=d_n(d_{n+1}h_n+h_{n-1}d_n)\\[2mm]
&=d_nd_{n+1}h_n+d_nh_{n-1}d_n\\[2mm]
d_n&=d_nd_{n+1}h_n+d_n\\[2mm]
0&=d_nd_{n+1}h_n.
\end{align*}
$h_n$ cannot be 0 since otherwise we would have $0=\text{Id}$, so it must be the case that $d_nd_{n+1}=0$. In other words, $F$ is exact, since $n$ was arbitrary.
\end{proof}

\pagebreak

\item Consider a short exact sequence in $\text{Ch}(R)$, say:
\begin{center}
\begin{tikzcd}
0 \arrow[r] & A \arrow[r] & B \arrow[r] & C \arrow[r] & 0
\end{tikzcd}
\end{center}


Show that if any two of $A$, $B$, and $C$ are exact, so is the third one.

\begin{proof}
For this problem we use a diagram chase. Here is the sequence of chain complexes expanded:
\begin{center}
\begin{tikzcd}
            & \vdots \arrow[d]                                        & \vdots \arrow[d]                                       & \vdots \arrow[d]                        &   \\
0 \arrow[r] & A_{n+2} \arrow[d, "d_{n+2}"'] \arrow[r, "\alpha_{n+2}"] & B_{n+2} \arrow[d, "d_{n+2}"'] \arrow[r, "\beta_{n+2}"] & C_{n+2} \arrow[d, "d_{n+2}"'] \arrow[r] & 0 \\
0 \arrow[r] & A_{n+1} \arrow[r, "\alpha_{n+1}"] \arrow[d, "d_{n+1}"'] & B_{n+1} \arrow[d, "d_{n+1}"'] \arrow[r, "\beta_{n+1}"] & C_{n+1} \arrow[r] \arrow[d, "d_{n+1}"'] & 0 \\
0 \arrow[r] & A_n \arrow[r, "\alpha_n"] \arrow[d, "d_n"']             & B_n \arrow[r, "\beta_n"] \arrow[d, "d_n"']             & C_n \arrow[r] \arrow[d, "d_n"']         & 0 \\
0 \arrow[r] & A_{n-1} \arrow[r, "\alpha_{n-1}"] \arrow[d, "d_{n-1}"'] & B_{n-1} \arrow[r, "\beta_{n-1}"] \arrow[d, "d_{n-1}"'] & C_{n-1} \arrow[r] \arrow[d, "d_{n-1}"'] & 0 \\
0 \arrow[r] & A_{n-2} \arrow[d] \arrow[r, "\alpha_{n-2}"]             & B_{n-2} \arrow[d] \arrow[r, "\beta_{n-2}"]             & C_{n-2} \arrow[d] \arrow[r]             & 0 \\
            & \vdots                                                  & \vdots                                                 & \vdots                                  &  
\end{tikzcd}
\end{center}

{\linespread{2}
First suppose the sequence above is exact at $B$ and $C$, and let $a\in \ker d_n\subseteq A_n$. Write $b=\alpha_n(a)$, and we have $d_n(b)=d_n(\alpha_n(a))=0\in B_{n-1}$ by exactness of $B$, so there exists a $b_1\in B_{n+1}$ such that $d_{n+1}(b_1)=b=\alpha_n(a)\in\im\alpha_n$. Since these are complexes, $\beta_n(b)=\beta_n(\alpha_n(a))=0\in C_n$. Now, write $c_1=\beta_{n+1}(b_1)$, and by exactness of $C$, we have $d_{n+1}(c_1)=d_{n+1}(\beta_{n+1}(b_1))=0\in C_n$. Thus there exists a $c_2\in C_{n+2}$ such that $d_{n+2}(c_2)=c_1$. By exactness of $B$, there is a $b_2\in B_{n+2}$ such that $\beta_{n+2}(b_2)=c_2$. But then $\beta_{n+1}(d_{n+2}(b_2))=c_1=\beta_{n+1}(b_1)$ so $b_1-d_{n+2}(b_2)\in \ker \beta_{n+1}$. By exactness of $B$, there exists an $a_1\in A_{n+1}$ such that $\alpha_{n+1}(a_1)=b_1-d_{n+2}(b_2)$. But
\[
d_{n+1}(\alpha_{n+1}(a_1))=d_{n+1}(b_1-d_{n+2}(b_2))=d_{n+1}(b_1)-d_{n+1}d_{n+2}(b_2)=d_{n+1}(b_1)=b.
\]
Additionally, $\alpha_n(d_{n+1}(a_1))=b$ and since $\alpha_n$ is injective by exactness of $B$, we have $a=d_{n+1}(a_1)$, so $\ker d_n=\im d_{n+1}$ and it follows that $A$ is exact.

Next suppose that $A$ and $B$ are exact and let $c\in \ker d_n$. By exactness of $B$, there exists a $b\in B_n$ such that $\beta_n(b)=c$. Then $\beta_{n-1}(d_n(b))=d_n(\beta_n(b))=d_n(c)=0$. Then by exactness of $B$, there is an $a_1\in A_{n-1}$ such that $\alpha_{n-1}(a_1)=d_{n}(b)$. Now, $\alpha_{n-2}(d_{n-1}(a_1))=d_{n-1}(\alpha_{n-1}(a_1))=d_{n-1}(d_n(b))=0$, but since $\alpha_{n-2}$ is injective by exactness, it must be the case that $d_{n-1}(a_1)=0$. Since these are complexes, there exists an $a\in A_n$ such that $d_{n}(a)=a_1$. Now, note that $b-\alpha_n(a)\in B_n$, and
\[
d_n(b-\alpha_n(a))=d_n(b)-d_n(\alpha_n(a))=d_n(b)-a_{n-1}(d_n(a))=d_n(b)-d_n(b)=0.
\]
So by exactness of $B$ there is a $b_1\in B_{n+1}$ such that $d_{n+1}(b_1)=b-\alpha_n(a)$. Now consider $\beta_{n+1}(b_1)\in C_{n+1}$. By exactness again we have 
\[
d_{n+1}(\beta_{n+1}(b_1))=\beta_n(d_{n+1}(b_1))=\beta_n(b-\alpha_n(a))=\beta_n(b)-\beta_n(\alpha_n(a))=\beta_n(b)-0=\beta_n(b)=c.
\]
Therefore $\ker d_n=\im d_{n+1}$ and it follows that $C$ is exact.
}
\end{proof}

\pagebreak

\item We used the Snake Lemma to deduce the long exact sequence in homology. In fact, the two statements are equivalent! Assuming the statement of the long exact sequence in homology holds, but without using the Snake Lemma, prove that any commutative diagram of $R$-modules with exact rows
\begin{center}
\begin{tikzcd}
0 \arrow[r] & A_1 \arrow[r, "i"] \arrow[d, "f"'] & B_1 \arrow[r, "p"] \arrow[d, "g"'] & C_1 \arrow[r] \arrow[d, "h"'] & 0 \\
0 \arrow[r] & A_0 \arrow[r, "i'"']              & B_0 \arrow[r, "p'"']              & C_0 \arrow[r]                & 0
\end{tikzcd}
\end{center}
induces an exact sequence
\begin{center}
\begin{tikzcd}
0 \arrow[r] & \ker f \arrow[r] & \ker g \arrow[r] & \ker h \arrow[r] & \coker f \arrow[r] & \coker g \arrow[r] & \coker h \arrow[r] & 0
\end{tikzcd}
\end{center}

\begin{proof}
First note we can view the columns of the diagram as complexes $A$, $B$, and $C$ where
\begin{align*}
A&=\ker f\to A_1\to A_0\to \coker f\\[2mm]
B&=\ker g\to B_1\to B_0\to \coker g\\[2mm]
C&=\ker h\to C_1\to C_0\to \coker h\\[2mm]
\end{align*}
Then the diagram describes a short exact sequence of complexes, so Theorem 9.67 ensures that there is a connecting homomorphism $\partial :H_1(C)\to H_{0}(A)$ such that 
\begin{center}
\begin{tikzcd}
0 \arrow[r] & H_1(A) \arrow[r] & H_1(B) \arrow[r] & H_1(C) \arrow[r, "\partial"] & H_{0}(A) \arrow[r] & H_{0}(B) \arrow[r] & H_{0}(C) \arrow[r] & 0
\end{tikzcd}
\end{center}

Observe:
\begin{align*}
H_1(A)&=\frac{\ker f}{\im(\ker f)}=\frac{\ker f}{0}=\ker f & H_0(A)&=\frac{\ker(A_0\to\coker f)}{\im f}=\frac{A_0}{\im f}=\coker f \\[2mm]
H_1(B)&=\frac{\ker g}{\im(\ker g)}=\frac{\ker g}{0}=\ker g & H_0(B)&=\frac{\ker(B_0\to\coker g)}{\im g}=\frac{B_0}{\im g}=\coker g \\[2mm]
H_1(C)&=\frac{\ker h}{\im(\ker h)}=\frac{\ker h}{0}=\ker h & H_0(C)&=\frac{\ker(C_0\to\coker h)}{\im h}=\frac{C_0}{\im h}=\coker h \\[2mm]
\end{align*}
So the exact sequence above is
\begin{center}
\begin{tikzcd}
0 \arrow[r] & \ker f \arrow[r] & \ker g \arrow[r] & \ker h \arrow[r, "\partial"] & \coker f \arrow[r] & \coker g \arrow[r] & \coker h \arrow[r] & 0
\end{tikzcd}
\end{center}
\end{proof}

\pagebreak

\item (omitted)

\item Let $R=\MB{Q}[x,y,z]/(x^2,xy)$. Check that $f$ below is a map of complexes, and compute its kernel, cokernel, and homology.

\[
\xymatrix@R=2mm{D = \\ \\ \\ \\ \\ C = \ar@<1ex>[uuuuu]^-f} \xymatrix@R=2mm@C=30mm{R \ar[r]^-{\begin{pmatrix} z \\ -y \\ x \end{pmatrix}} & R^3 \ar[r]^-{\begin{pmatrix} -y & -z & 0 \\ x & 0 & -z \\ 0 & x & y \end{pmatrix}
} & R^3 \ar[r]^-{\begin{pmatrix} x & y & z\end{pmatrix}
} & R \\ &&&\\ &&&\\ &&&\\ &&&\\ 0 \ar[r]_-0 \ar[uuuuu]^-{0} & R \ar[uuuuu]^-{\begin{pmatrix} 0 \\ 0 \\ 1 \end{pmatrix}} \ar[r]_-{\begin{pmatrix} -z \\ y \end{pmatrix}}  \ar[uuuuu]^-{} & R^2 \ar[uuuuu]^-{\begin{pmatrix} 0 & 0 \\ 1 & 0 \\ 0 & 1 \end{pmatrix}} \ar[r]_-{\begin{pmatrix} y & z\end{pmatrix}} & R. \ar@{=}[uuuuu] \\
\text{{\tiny 3}} & \text{{\tiny 2}} & \text{{\tiny 1}} & \text{{\tiny 0}}}
\]

(see file \verb!Richardson_MATH224_HW2.m2!)

\item (see previous HW submission, file: \verb!Richardson_MATH224_HW1.pdf!)



\end{enumerate}
\end{document}