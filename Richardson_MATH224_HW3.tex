\documentclass[11pt,oneside,english]{amsart}
\usepackage[T1]{fontenc}
\usepackage{geometry}
\usepackage{parskip}
\geometry{verbose,tmargin=0.65in,bmargin=0.65in,lmargin=0.75in,rmargin=0.75in,headheight=0.75cm,headsep=1cm,footskip=1cm}
\setlength{\parskip}{7mm}
\usepackage{setspace}
\onehalfspacing
%\pagenumbering{gobble}

\usepackage{comment}
\usepackage{bbm}
%\usepackage{multicol}
%\usepackage{graphicx}
%\usepackage{adjustbox}
\usepackage{amssymb}
\usepackage{tikz}
\usetikzlibrary{cd, quotes}
%\usepackage{pgfplots}
%\usepackage{pgffor}
\usepackage{ulem}
\usepackage{adjustbox}
\usepackage{bm}
%\usepackage{stmaryrd}
\usepackage{mathrsfs}
\usepackage{cancel}
\usepackage{mathtools}
\usepackage{commath}
\DeclarePairedDelimiter{\ceil}{\lceil}{\rceil}
\DeclarePairedDelimiter\floor{\lfloor}{\rfloor}
\usepackage[shortlabels]{enumitem}
\setlist[enumerate,1]{label=\textbf{\arabic*.}}
\usepackage{color, colortbl}
\definecolor{Gray}{gray}{0.9}
\usepackage{babel}
\usepackage{mdframed}
\usepackage{esint}
\usepackage[yyyymmdd]{datetime}
\renewcommand{\dateseparator}{--}
\usepackage{url}
\usepackage[unicode=true,pdfusetitle,
 bookmarks=true,bookmarksnumbered=false,bookmarksopen=false,
 breaklinks=false,pdfborder={0 0 1},backref=false,colorlinks=true]
 {hyperref}
\hypersetup{urlcolor=blue}

\usepackage[all]{xypic}




\theoremstyle{definition}
\newtheorem{theorem}{Theorem}
\newtheorem*{theorem*}{Theorem}
\newtheorem*{proposition*}{Proposition}
\newtheorem{corollary}{Corollary}
\newtheorem*{lemma}{Lemma}
\newtheorem*{example}{Example}
\newtheorem*{examples}{Examples}
\newtheorem*{definition}{Definition}
\newtheorem*{note}{Nota Bene}

\newcommand{\aspace}{\hspace{7mm}\text{and}\hspace{7mm}}
\newcommand{\ospace}{\hspace{7mm}\text{or}\hspace{7mm}}
\newcommand{\pspace}{\hspace{10mm}}
\newcommand{\lspace}{\vspace{5mm}}
\newcommand{\lhe}{\stackrel{\text{L'H}}{=}}
\newcommand{\lom}[2]{\lim_{{#1}\rightarrow{#2}}}
\newcommand{\ve}{\varepsilon}
\renewcommand{\Re}{\text{Re }}
\renewcommand{\Im}{\text{Im }}
\newcommand{\Log}{\text{Log }}
\newcommand{\ess}{\text{ess sup}}
\newcommand{\dd}[2]{\frac{d{#1}}{d{#2}}}
\newcommand{\pp}[2]{\frac{\partial{#1}}{\partial{#2}}}
\newcommand{\DD}[2]{\frac{\Delta{#1}}{\Delta{#2}}}
\newcommand{\ovec}[1]{\overrightarrow{#1}}
\newcommand{\MC}[1]{\mathcal{#1}}
\newcommand{\MB}[1]{\mathbb{#1}}
\newcommand{\MF}[1]{\mathfrak{#1}}
\newcommand{\MS}[1]{\mathscr{#1}}
\newcommand{\mbf}[1]{\,\mathbf{#1}}
\renewcommand{\vec}[1]{\underline{#1}}
\newcommand{\im}{\text{im\,}}
\newcommand{\Hom}{\text{Hom}}
\newcommand{\coker}{\text{coker\,}}



\def\<#1>{\mathinner{\langle#1\rangle}}

\makeatletter
\g@addto@macro\normalsize{%
  \setlength\belowdisplayshortskip{5mm}
}
\makeatother





\begin{document}

\rightline{Adam D. Richardson}
\rightline{224 - Homological Algebra}
\rightline{Grifo, Elo\'isa}
\rightline{HW 3}
\rightline{\today}

\lspace




\begin{enumerate}[leftmargin=*]
\itemsep5mm

\item Let $R$ be a ring, $I$ and $J$ ideals in $R$, and $M$ be an $R$-module.
\begin{enumerate}
\item Show that $R/I \otimes_R R/J \cong R/(I+J)$.

\begin{proof}
First, define a map $f:R/I\times R/J\to R/(I+J)$ by $f(\bar x, \bar y)=xy+(I+J)$. This map is well defined since, if $x'\in\bar x$ and $y'\in \bar y$ as well, we have $xy-x'y'\in I+J$ by definition of an ideal, and so $f(\bar x,\bar y)=f(\bar{x'}, \bar{y'})$. It is also bilinear:
\begin{align*}
f(\overline{x_1+x_2},\bar y)&=(x_1+x_2)y+(I+J)\\[2mm]
&=x_1y+(I+J)+x_2y+(I+J)\\[2mm]
&=f(\bar{x_1},\bar y)+f(\bar{x_2},\bar y),
\end{align*}
similarly for $f(x,\overline{y_1+y_2})$. Also, for $r\in R$,
\[
f(\overline{rx},\bar y)=(rx)y+(I+J)=x(ry)+(I+J)=r(xy)+(I+J),
\]
the last two expressions being equal to $f(\bar x,\overline{ry})$ and $rf(\bar x,\bar y)$ respectively.

By definition of a tensor product, there exists an $R$-module homomorphism $\tilde f:R/I \otimes_R R/J\to R/(I+J)$ where $\bar x\otimes \bar y\mapsto xy+(I+J)$. This map is surjective: given any $\bar z\in R/(I+J)$ we have $\bar z\otimes \bar 1\mapsto z\cdot1+(I+J)=\bar z$.

To show $\tilde f$ is injective, observe the following. Let $z\in R/I\otimes R/J$. Then
\begin{align*}
z&=\sum_{i=1}^n\bar{x_i}\otimes \bar{y_i}=\sum_{i=1}^n(x_i+I)(y_i+J)=\sum_{i=1}^n\left(x_i\otimes y_i+I\otimes J\right)=\sum_{i=1}^n \bar{x_i}\otimes \bar 1\\[2mm]
&=\bar{x_1}\otimes \bar 1+\bar{x_2}\otimes \bar 1+\cdots +\bar{x_n}\otimes \bar 1=(\bar{x_1}+\bar{x_2}+\cdots+\bar{x_n})\otimes\bar 1=\bar x\otimes \bar 1.
\end{align*}
In other words, every element in $R/I\otimes R/J$ has the form $\bar x\otimes \bar 1$. Let $\bar z \otimes \bar 1\in \ker \tilde f$. Then $z\in I+J$ so $z=x+y$ for some $x\in I$ and some $y\in J$. Then
\[
\bar z\otimes \bar 1=(\bar x+\bar y)\otimes \bar 1=\bar x\otimes \bar 1+\bar y\otimes\bar 1=x+(I+J)+y+(I+J)=x+y+(I+J)=\bar 0.
\]
Thus, $\ker \tilde f=\bar 0$ and so $\tilde f$ is injective. Therefore, $\tilde f$ is an isomorphism as was to be shown.
\end{proof}

\pagebreak

\item Show that $R/I \otimes_R M \cong M/IM$.

\begin{proof}
Consider the short exact sequence:

\begin{center}
\begin{tikzcd}
0 \arrow[r] & I \arrow[r] & R \arrow[r] & R/I \arrow[r] & 0
\end{tikzcd}
\end{center}
Recall that the tensor functor is right exact. So if we apply $-\otimes_R M$ to our sequence, we get:

\begin{center}
\begin{tikzcd}
0\otimes_R M \arrow[r] & I\otimes_R M \arrow[r] & R\otimes_R M \arrow[r] & (R/I)\otimes_R M \arrow[r] & 0\otimes_R M
\end{tikzcd}
\end{center}

By Lemma 10.22, $R\otimes_R M\cong M$ via $a\otimes m\mapsto am$ so the map $I\otimes_R M\mapsto M$ is given by the same mapping, but this means that the image of the map is $IM$. Thus, by exactness, $(R/I)\otimes_R M\cong M/IM$.
\end{proof}

\item There is an $R$-module map $I \otimes_R M \longrightarrow IM$ induced by the $R$-bilinear map $(a,m) \mapsto am$. This map is always clearly surjective; must it be injective?

No. Suppose $am_1=am_2$. Then we would have injectivity if and only if $a\otimes m_1=a\otimes m_2$, i.e. $a\otimes (m_1-m_2)$. But this would only hold if $a=0$ or $m_1=m_2$ which will not always be the case. A similar argument can be made if we instead suppose $a_1m=a_2m$.
\end{enumerate}

\item Show that over a filed $k$ the Hom functor and the tensor functor are always exact.

\begin{proof}
Since $k$ is a field, $M$ is a free module. Any free module is projective, and so by Theorem 11.4 the Hom functor is exact.

Let $M$ be a $k$-module. By remark 11.38, $M$ is flat if and only if for every injective $k$-module map $i:A\to B$, the map $(1\otimes_ki):M\otimes_kA\to M\otimes_kB$ is injective. And by definition $M$ is flat if and only if $M\otimes_k-$ is an exact functor. But the map $1\otimes_ki$ is injective in this case since $k$ is a field so $M$ has a basis. 
\end{proof}

\pagebreak

\item \begin{enumerate}
\item Consider $R$-module homomorphisms $A \xrightarrow{f} B $ and $B \xrightarrow{g} C$. If
$$\xymatrix{\Hom_R(M,A) \ar[r]^{f_*} & \Hom_R(M,B) \ar[r]^-{g_*} & \Hom_R(M,C)}$$
is exact for all $M$, then $\xymatrix{A \ar[r]^-f & B \ar[r]^-g & C}$ is an exact sequence.\footnote{In particular, we are not assuming $\xymatrix{A \ar[r]^-f & B \ar[r]^-g & C}$ is a complex!}

\begin{proof}
If the first sequence of Homs is exact, then by Theorem 11.4 $M$ is projective. It follows then (by Remark 11.2) that $f_*(g)=f$. Thus, $\im f = \im f_*(g)=\ker g_*(g)=\ker g$ by exactness of the Hom sequence above.
\end{proof}

\item Suppose that $(F,G)$ is an adjoint pair of covariant functors $R\textbf{-mod} \longrightarrow R\textbf{-mod}$. Show that $G$ is left exact.\footnote{Later we will explain how it follows easily that $F$ is right exact.}

\begin{proof}
Let $\xymatrix{0 \ar[r] & A \ar[r] & B \ar[r] & C \ar[r] &0}$ be an exact sequence of $R$-modules and let $M$ be another $R$-module. The Hom functor is left exact and covariant so we have
\begin{center}
$\xymatrix{0 \ar[r] & \text{Hom}(F(M),A) \ar[r] & \text{Hom}(F(M),B) \ar[r] & \text{Hom}(F(M),C)}$
\end{center}
is also exact. Since $F$ and $G$ are adjoints, we get
\begin{center}
$\xymatrix{0 \ar[r] & \text{Hom}(M,G(A)) \ar[r] & \text{Hom}(M,G(B)) \ar[r] & \text{Hom}(M,G(C)).}$
\end{center}
By the Yoneda lemma, there is a bijection $\text{Nat}(\text{Hom}(M,-),G)\to G(M)$ which yields that 
\begin{center}
$\xymatrix{0 \ar[r] & G(A) \ar[r] & G(B) \ar[r] & G(C)}$
\end{center}
is exact, i.e. $G$ is left exact.
\end{proof}
\end{enumerate}

\pagebreak
\noindent
\fbox{\begin{minipage}{\textwidth}
Let $R$ be a domain and $M$ be an $R$-module. The {\bf torsion}\index{torsion} of $M$ is the submodule
	$$T(M) \coloneqq \lbrace m \in M \mid rm = 0 \textrm{ for some nonzero } r \in R \rbrace.$$
	The elements of $T(M)$ are called {\bf torsion elements}, and we say that $M$ {\bf is torsion} if $T(M) = M$. Finally, $M$ is {\bf torsion free} if $T(M) = 0$.
\end{minipage}} 
\item Let $R$ be a domain and $M$ be an $R$-module.
\begin{enumerate}
\item The $R$-module $M/T(M)$ is torsion free.

\begin{proof}
The set of torsion elements of $M/T(M)$ are the elements $m+T(M)$ such that there exists an $r\in R$ making $rm+T(M)=0$. But this means that $rm\in T(M)$ so $rm=0$ for some nonzero $r$. Since $R$ is a domain, we must have that $m=0$, and thus the set of torsion elements of $M/T(M)$ is 0 and so $M/T(M)$ is torsion free.
\end{proof}

\item If $f\!: M \longrightarrow N$ is an $R$-module homomorphism, $f(T(M)) \subseteq T(N)$.

\begin{proof}
Let $n\in f(T(M))$. Then there exists an $m\in T(M)$ such that $f(m)=n$. Since $m\in T(M)$, there exists a nonzero $r\in R$ such that $rm=0$. Since $f$ is a homomorphism, $0=f(0)=f(rm)=rf(m)=rn$. Thus $n\in T(N)$ by definition and so $f(T(M)) \subseteq T(N)$.
\end{proof}

\item Torsion is a left exact covariant functor $R\textbf{-mod} \longrightarrow R\textbf{-mod}$.

\begin{proof}
Let
\[
\xymatrix{0 \ar[r] & A \ar[r]^f & B \ar[r]^g & C}
\]
be an exact sequence of $R$-modules. We need to show the sequence
\[
\xymatrix{0 \ar[r] & T(A) \ar[r]^{T(f)} & T(B) \ar[r]^{T(g)} & T(C)}
\]
is exact, where $T(f)=f\mid_{T(A)}$ and $T(g)=g\mid_{T(B)}$. In particular we need to show $T(f)$ is injective, $T(g)T(f)=0$, and $\im T(f)=\ker T(g)$.

Let $m\in\ker T(f)$. Then $T(f)(m)=f(m)=0$ and since $f$ is injective by hypothesis, $m=0$ and $T(f)$ is injective. Let $m\in T(A)$. Then $T(g)T(f)(m)=g(f(m))=0$ since $\im f=\ker g$ by exactness. 

If $n\in \ker T(g)$, then $g(n)=0$ so $n\in \ker g=\im f$ by exactness. Thus, $n\in \im T(f)$. This combined with the result that $T(g)T(f)=0$ yields that  $\im T(f)=\ker T(g)$. Therefore, torsion is left exact.
\end{proof}
\end{enumerate}

\item (omitted)

\item Consider the domain $R = \mathbb{Q}[x,y,z,a,b,c]/(xb-ac,yc-bz,xc-az)$, the ideal $I = (x,a)$ in $R$, the $R$-module $N = \mathbb{Q}$, $I = (x,a)$, and consider the $2$-generated $R$-module $M = Rf + Rg$, where the generators $f, g$ satisfy the relations 
\[
yf-xg = 0 \quad bf - cg = 0 \quad cf - zg = 0.
\]

(see file \verb!Richardson_MATH224_HW3.m2!)
\begin{enumerate}
\item Are there nontrivial $R$-module homomorphisms $M \longrightarrow N$? How about $N \longrightarrow M$?

Yes, there are nontrivial $R$-module homomorphisms $M\to N$. No, there no nontrivial homomorphisms $N\to M$

\item Does $- \otimes_R M$ preserve the injectivity of the inclusion $I \subseteq R$?

Yes, $-\otimes_R M$ does preserve injectivity of the inclusion $I\subseteq R$.
\item Apply $\Hom_R(-,R)$ to the short exact sequence 
\[
\xymatrix{0 \ar[r] & I \ar[r] & R \ar[r] & R/I \ar[r] & 0}.
\]
Is $R$ an injective $R$-module?

No. Recall that a module is injective iff $\text{Hom}_R(-,I)$ is exact. Consider the ideal $I=(x,a,b)$, and we see that the homology of $\text{Hom}$ is not always 0, so $\text{Hom}$ is not exact and so $R$ is not injective.
\end{enumerate}
		
\item (omitted)

\end{enumerate}
\end{document}