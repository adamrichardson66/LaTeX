\documentclass[11pt,oneside,english]{amsart}
\usepackage[T1]{fontenc}
\usepackage{geometry}
\usepackage{parskip}
\geometry{verbose,tmargin=0.65in,bmargin=0.65in,lmargin=0.75in,rmargin=0.75in,headheight=0.75cm,headsep=1cm,footskip=1cm}
\setlength{\parskip}{7mm}
\usepackage{setspace}
\onehalfspacing
\pagenumbering{gobble}

\usepackage{bbm}
\usepackage{multicol}
\usepackage{graphicx}
\usepackage{adjustbox}
\usepackage{amssymb}
\usepackage{tikz}
\usepackage{pgfplots}
\usepackage{pgffor}
\usetikzlibrary{cd}
\usepackage{ulem}
\usepackage{adjustbox}
\usepackage{bm}
\usepackage{stmaryrd}
\usepackage{cancel}
\usepackage{mathtools}
\DeclarePairedDelimiter{\ceil}{\lceil}{\rceil}
\DeclarePairedDelimiter\floor{\lfloor}{\rfloor}
\usepackage[shortlabels]{enumitem}
\setlist[enumerate,1]{label=\textbf{\arabic*.}}
\usepackage{color, colortbl}
\definecolor{Gray}{gray}{0.9}
\usepackage{babel}
\usepackage{mdframed}
\usepackage{esint}
\usepackage[yyyymmdd]{datetime}
\renewcommand{\dateseparator}{--}
\usepackage{url}
\usepackage[unicode=true,pdfusetitle,
 bookmarks=true,bookmarksnumbered=false,bookmarksopen=false,
 breaklinks=false,pdfborder={0 0 1},backref=false,colorlinks=true]
 {hyperref}
\hypersetup{urlcolor=blue}





\theoremstyle{definition}
\newtheorem{theorem}{Theorem}
\newtheorem*{theorem*}{Theorem}
\newtheorem*{proposition*}{Proposition}
\newtheorem{corollary}{Corollary}
\newtheorem*{lemma}{Lemma}
\newtheorem*{example}{Example}
\newtheorem*{examples}{Examples}
\newtheorem*{definition}{Definition}
\newtheorem*{note}{Nota Bene}

\newcommand{\aspace}{\hspace{7mm}\text{and}\hspace{7mm}}
\newcommand{\ospace}{\hspace{7mm}\text{or}\hspace{7mm}}
\newcommand{\pspace}{\hspace{10mm}}
\newcommand{\lspace}{\vspace{5mm}}
\newcommand{\lhe}{\stackrel{\text{L'H}}{=}}
\newcommand{\lom}[2]{\lim_{{#1}\rightarrow{#2}}}
\newcommand{\ve}{\varepsilon}
\renewcommand{\Re}{\text{Re }}
\renewcommand{\Im}{\text{Im }}
\newcommand{\Log}{\text{Log }}
\newcommand{\ess}{\text{ess sup}}
\newcommand{\dd}[2]{\frac{d{#1}}{d{#2}}}
\newcommand{\pp}[2]{\frac{\partial{#1}}{\partial{#2}}}
\newcommand{\DD}[2]{\frac{\Delta{#1}}{\Delta{#2}}}
\newcommand{\ovec}[1]{\overrightarrow{#1}}
\newcommand{\MC}[1]{\mathcal{#1}}
\newcommand{\MB}[1]{\mathbb{#1}}
\newcommand{\mbf}[1]{\,\mathbf{#1}}
\renewcommand{\vec}[1]{\underline{#1}}
\newcommand{\Res}{\text{Res}}


\def\<#1>{\mathinner{\langle#1\rangle}}

\makeatletter
\g@addto@macro\normalsize{%
  \setlength\belowdisplayshortskip{5mm}
}
\makeatother





\begin{document}

\rightline{Adam D. Richardson}
\rightline{225 - Commutative Algebra}
\rightline{Grifo, Elo\'isa}
\rightline{HW 1}
\rightline{\today}

\lspace




\begin{enumerate}[leftmargin=*]
\itemsep5mm


\item \begin{enumerate}
\itemsep5mm

\item Install Macaulay2. Hardcore version: install emacs and run Macaulay2 through emacs.

(see file \verb!Richardson_MATH225_HW1_1.m2!)

\item Make an .m2 file setting up a polynomial ring $R$ over a field $k$, a nontrivial ideal $I$ in $R$, the
$R$-module $M=R/I$ and the ring $S=R/I$.

(see file \verb!Richardson_MATH225_HW1_1.m2!)

\end{enumerate}

\item Use Macaulay2 to find:

\begin{enumerate}
\itemsep5mm

\item A presentation for the $\MB{Q}$-algebra $\MB{Q}[xy, xu, yv, uv]\subseteq \MB{Q}[x, y, u, v]$.

(see file \verb!Richardson_MATH225_HW1_2.m2!)

\item A presentation for the $k$-algebra $U$, where $k = \MB{Z}/101$ and
\[
U=k\begin{bmatrix}ux & uy & uz \\ vx & vy & vz\end{bmatrix}\subseteq\frac{k[u,v,x,y,z]}{(x^3+y^3+z^3)}.
\]

(see file \verb!Richardson_MATH225_HW1_2.m2!)
\end{enumerate}

\item Let $K$ be a field, and $R := K[x^2, x^3] \subseteq S := K[x]$. Let $I =x^2R$ be the ideal generated by $x^2$ in $R$. Show that $IS\cap R\supsetneq I$, and conclude that $R$ is not a direct summand of $S$.

\begin{proof}
First, $R$ is the set of all polynomials that are $K$-linear combinations of $x^{2i+3j}$ ($i,j\geq0$), but since any $n\in\MB{Z}^+$ with $n\geq2$ can be written in the form $n=2i+3j$,  we have that $R$ consists of all polynomials of the form $a_0+a_2x^2+a_3x^3+\cdots$ where $a_k\in K$, i.e. polynomials without the linear term. For ease, we will write
\[
R=\{a_0+a_2x^2+a_3x^3+\cdots+a_nx^n\}.
\]
We also have that
\begin{align*}
I=x^2R&=\{\text{``stuff in $R$ multiplied by $x^2$''}\}\\[2mm]
&=\{b_0x^2+b_1x^4+b_2x^5+\cdots+b_nx^{n+2}\}.
\end{align*}
Notice that there are no $x^3$ terms in the elements of $I$. Then
\begin{align*}
IS=x^2RK[x]&=\{\text{``stuff in $K[x]$ multiplied by stuff in $I$ and summed''}\}\\[2mm]
&=\{c_0x^2+c_1x^3+c_2x^4+\cdots+c_nx^{n+2}\}.
\end{align*}
Thus,
\[
IS\cap R=\{d_0x^2+d_1x^3+d_2x^4+\cdots+d_nx^{n+2}\}\supsetneq I
\]
\end{proof}


\item Let $k$ be a field. Is $k[x^{42},y^{17}+y^7,z^{73}+z]\subseteq k[x,y,z]$ module-finite?

\begin{proof}
Yes, and we proceed by employing Corollary 1.35. First, $k[x,y,z]$ is certainly algebra finite over $k[x^{42},y^{17}+y^7,z^{73}+z]$ since it is a polynomial ring and one set of generators is $\{x,y,z\}$, so we need only show that it is integral over $k[x^{42},y^{17}+y^7,z^{73}+z]$. To do this we show that each generator of $k[x,y,z]$ is integral over $k[x^{42},y^{17}+y^7,z^{73}+z]$. In other words, we need to find polynomials with coefficients in $k[x^{42},y^{17}+y^7,z^{73}+z]$ that $x$, $y$, and $z$ satisfy. To that end, note that for $t=x, y, z$ respectively, 
\begin{align*}
t^{42}-x^{42}&=0,\\[2mm]
t^{17}+t^7-(y^{17}+y^7)&=0,\text{ and}\\[2mm]
t^{73}+t-(z^{73}+z)&=0
\end{align*}
are such polynomials. Thus $k[x,y,z]$ is algebra finite and integral over $k[x^{42},y^{17}+y^7,z^{73}+z]$, so by Corollary 1.35, $k[x,y,z]$ is module finite over $k[x^{42},y^{17}+y^7,z^{73}+z]$.
\end{proof}


\item Find $\overline{\MB{Z}}$, the integral closure of $\MB{Z}$ in its field of fractions $\MB{Q}$.


The integral closure of $\MB{Z}$ in $\MB{Q}$ is $\MB{Z}$ itself, i.e, $\overline{\MB{Z}}=\MB{Z}$. Clearly $\MB{Z}\subseteq\overline{\MB{Z}}$. To show the reverse inclusion, notice that for any $\frac{p}{q}\in\MB{Q}\setminus\MB{Z}$ in lowest terms, any polynomial it satisfies cannot be monic. Hence, $\overline{\MB{Z}}$ is contained in the complement of $\MB{Q}\setminus\MB{Z}$, i.e. $\overline{\MB{Z}}\subseteq\MB{Q}\setminus(\MB{Q}\setminus\MB{Z})=\MB{Z}$. Therefore $\overline{\MB{Z}}=\MB{Z}$.



\item If $R[x]$ is Noetherian, must $R$ be Noetherian?


\begin{proof}
Yes. Recall Remark 1.4 from the notes: If $R$ is Noetherian and $I$ is an ideal of $R$, then $R/I$ is Noetherian. Consider the ideal $(x)$ in $R[x]$. We have $R[x]/(x)\cong R$, so $R$ is Noetherian.
\end{proof}


\item Suppose that every ascending chain of prime ideals in $R$ stabilizes. Must $R$ be a noetherian ring? [Hint: it may be helpful to think of how we could construct rings with very few prime ideals.]

(omitted)

\item Show that $R$ is a Noetherian ring if and only if every prime ideal of $R$ is finitely generated. [Hint: Use Zorn?s Lemma to show that if $R$ is a ring that is not noetherian, the set of ideals that are not finitely generated has a maximal element; show that element is a prime ideal.]

(omitted)
\end{enumerate}


\end{document}