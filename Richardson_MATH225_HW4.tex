\documentclass[11pt,oneside,english]{amsart}
\usepackage[T1]{fontenc}
\usepackage{geometry}
\usepackage{parskip}
\geometry{verbose,tmargin=0.65in,bmargin=0.65in,lmargin=0.75in,rmargin=0.75in,headheight=0.75cm,headsep=1cm,footskip=1cm}
\setlength{\parskip}{7mm}
\usepackage{setspace}
\onehalfspacing
\pagenumbering{gobble}

\usepackage{comment}
\usepackage{bbm}
\usepackage{multicol}
\usepackage{graphicx}
\usepackage{adjustbox}
\usepackage{amssymb}
\usepackage{tikz}
\usepackage{pgfplots}
\usepackage{pgffor}
\usetikzlibrary{cd}
\usepackage{ulem}
\usepackage{adjustbox}
\usepackage{bm}
\usepackage{stmaryrd}
\usepackage{cancel}
\usepackage{mathtools}
\usepackage{commath}
\DeclarePairedDelimiter{\ceil}{\lceil}{\rceil}
\DeclarePairedDelimiter\floor{\lfloor}{\rfloor}
\usepackage[shortlabels]{enumitem}
\setlist[enumerate,1]{label=\textbf{\arabic*.}}
\usepackage{color, colortbl}
\definecolor{Gray}{gray}{0.9}
\usepackage{babel}
\usepackage{mdframed}
\usepackage{esint}
\usepackage[yyyymmdd]{datetime}
\renewcommand{\dateseparator}{--}
\usepackage{url}
\usepackage[unicode=true,pdfusetitle,
 bookmarks=true,bookmarksnumbered=false,bookmarksopen=false,
 breaklinks=false,pdfborder={0 0 1},backref=false,colorlinks=true]
 {hyperref}
\hypersetup{urlcolor=blue}





\theoremstyle{definition}
\newtheorem{theorem}{Theorem}
\newtheorem*{theorem*}{Theorem}
\newtheorem*{proposition*}{Proposition}
\newtheorem{corollary}{Corollary}
\newtheorem*{lemma}{Lemma}
\newtheorem*{example}{Example}
\newtheorem*{examples}{Examples}
\newtheorem*{definition}{Definition}
\newtheorem*{note}{Nota Bene}

\newcommand{\aspace}{\hspace{7mm}\text{and}\hspace{7mm}}
\newcommand{\ospace}{\hspace{7mm}\text{or}\hspace{7mm}}
\newcommand{\pspace}{\hspace{10mm}}
\newcommand{\lspace}{\vspace{5mm}}
\newcommand{\lhe}{\stackrel{\text{L'H}}{=}}
\newcommand{\lom}[2]{\lim_{{#1}\rightarrow{#2}}}
\newcommand{\ve}{\varepsilon}
\renewcommand{\Re}{\text{Re }}
\renewcommand{\Im}{\text{Im }}
\newcommand{\Log}{\text{Log }}
\newcommand{\ess}{\text{ess sup}}
\newcommand{\dd}[2]{\frac{d{#1}}{d{#2}}}
\newcommand{\pp}[2]{\frac{\partial{#1}}{\partial{#2}}}
\newcommand{\DD}[2]{\frac{\Delta{#1}}{\Delta{#2}}}
\newcommand{\ovec}[1]{\overrightarrow{#1}}
\newcommand{\MC}[1]{\mathcal{#1}}
\newcommand{\MB}[1]{\mathbb{#1}}
\newcommand{\MF}[1]{\mathfrak{#1}}
\newcommand{\mbf}[1]{\,\mathbf{#1}}
\renewcommand{\vec}[1]{\underline{#1}}
\newcommand{\Res}{\text{Res}}


\def\<#1>{\mathinner{\langle#1\rangle}}

\makeatletter
\g@addto@macro\normalsize{%
  \setlength\belowdisplayshortskip{5mm}
}
\makeatother





\begin{document}

\rightline{Adam D. Richardson}
\rightline{225 - Commutative Algebra}
\rightline{Grifo, Elo\'isa}
\rightline{HW 3}
\rightline{\today}

\lspace




\begin{enumerate}[leftmargin=*]
\itemsep5mm

\item Let $M$ be a Noetherian $R$-module.
\begin{enumerate}
\item Show that every surjective $R$-module homomorphism $M\to M$ is an isomorphism.
\begin{proof}
Let $\phi$ be a surjective $R$-module homomorphism $\phi: M\to M$. It suffices to show that $\phi$ must be injective. Let $x,y\in M$ and suppose $\phi(x)=\phi(y)$. Since $R$ is Noetherian, every $R$-module is finitely generated, hence there exists generators $m_1,\ldots,m_n\in M$ such that any element in $M$ can be written as an $R$-linear combination of these generators. Hence, with $r_i,s_i\in R$, we can write
\[
x=\sum_{i}^nr_im_i \aspace y=\sum_{i=1}^ns_im_i. 
\]
Since $\phi$ is a homomorphism, we have
\begin{align*}
\phi(x)-\phi(y)&=0\\[2mm]
\phi\left(\sum_{i=1}^nr_im_i\right)-\phi\left(\sum_{i=1}^ns_im_i\right)&=0\\[2mm]
\sum_{i=1}^nr_i\phi(m_i)-\sum_{i=1}^ns_i\phi(m_i)&=0\\[2mm]
\sum_{i=1}^n(r_i-s_i)\phi(m_i)&=0.
\end{align*}
This implies that $r_i=s_i$ for each $i$, and thus $x=y$. As a result, $\phi$ is injective and so $\phi$ is an isomorphism as well.
\end{proof}

\item Must an injective $R$-module homomorphism $M\to M$ be an isomorphism?

No. Consider the right shift on the space of countably infinite sequences: $r(x_1,x_2,\ldots)=(0,x_1,\ldots)$. This map is an injective homomorphism on this vector space, but not surjective since the sequence $(x_1,x_2,\ldots)$ has no preimage under $r$.
\end{enumerate}

\pagebreak
\item Show that $M$ is a Noetherian $R$-module if and only if $M$ is finitely generated and $R/\text{ann}\,M$ is a Noetherian ring. 

\begin{proof}
First suppose $M$ is a Noetherian $R$-module. Then it is finitely generated by definition, say by $\{m_1,\ldots,m_n\}$. Define the map $\phi:R\to M^n$ by $1\mapsto (m_1,m_2,\ldots,m_n)$. Then $\ker\phi$ must be $\text{ann}\,M$ by definition of an annhiliator. By the first isomorphism theorem we have $R/\ker\phi=R/\text{ann}\,M\cong\phi(R)$. Since $M$ is Noetherian, $M^n$ is Noetherian, and thus $\phi(R)$ is Noetherian as well, from which it follows that $R/\text{ann}\,M$ is Noetherian.

Conversely, suppose $M$ is finitely generated and $R/\text{ann}\,M$ is a Noetherian ring. Then $M$ must be a a Noetherian $R$-module by definition. (Incorrect)
\end{proof}

\item (omitted)

\item (omitted)

\item Show that the containment of ideals is a local property: given any ideals $I$ and $J$ in a ring $R$, 
\[
I\subseteq J \quad \iff \quad I_P\subseteq J_P
\]
for all $P\in\text{Spec}(R)$.
\begin{proof}
First suppose $I\subseteq J$. Note that
\begin{align*}
I_P=(I\setminus P)^{-1}I&=\left\{\frac{i}{w}\,\Big|\, i\in I,\,w\in I\setminus P\right\},\text{ and}\\[2mm]
J_P=(J\setminus P)^{-1}J&=\left\{\frac{j}{v}\,\Big|\, j\in J,\,v\in J\setminus P\right\}.
\end{align*}

So if $x\in I_P$, then $x=\frac{i}{w}$ for $i\in I\subseteq J$ and $w\in I\setminus P\subseteq J\setminus P$, so $x\in J_P$.

Conversely, suppose $I_P\subseteq J_P$ and let $i\in I$. Then for all $w\in I\setminus P$, $\frac{i}{w}\in I_P\subseteq J_P$. This implies $I\setminus P\subseteq J\setminus P$, and there exists a $j\in J$ such that $\frac{i}{w}=\frac{j}{w}$. But then it follows that $w(i-j)=0$, and since $w\neq0$ we must have $i=j$, i.e. $i\in J$. Thus, $I\subseteq J$, and the equivalence is shown. 
\end{proof}

\pagebreak

\item (See file \verb!Richardson_MATH225_HW4.m2!)
\begin{enumerate}
\itemsep5mm
\item Describe $\text{supp}(I/I^2)$, where $I=(xz)$ in $R=\MB{C}[x,y,z]/(xy,yz)$.

In the course notes, Proposition 4.9 states that $\text{supp}(M)=V(\text{ann}_R(M))$. Using Macauly2, we find that $\text{ann}(I/I^2)=(xz,y)$. Using the command \verb!minimalPrimes!, we find that the primes containing $(xz,y)$ are $(x,y)$, and $(y,z)$ so  $\text{supp}(I/I^2)=\{(x,y),(y,z)\}$.

\item Find all the minimal primes of $J=(ab,bc,cd,ad)$ in $k[a,b,c,d]$ over any field $k$.

Macauly2 reveals that the minimal primes of $J$ are $(a,c)$ and $(b,d)$.

\item Find all the minimal primes of the ring $S$, where
\[
S=\MB{Q}\begin{bmatrix}ux & uy & uz \\ vx & vy & vz\end{bmatrix}\subseteq\frac{\MB{Q}[u,v,x,y,z]}{(x^3+y^3+z^3)}.
\]

Macauly2 has a \verb!minimalPrimes! command but it does not accept rings as an input. However, as my colleague Rahul Rajkumar pointed out, minimal primes in a ring are minimal over the 0 ideal, so asking Macauly2 to find the minimal primes of $S$ revealed that the 0 ideal is the only minimal prime.

\item Let $I$ be the defining ideal of the curve parameterized by $(t^{13},t^{42},t^{73})$ over $\MB{Q}$. Find $\mu(I)$ and a minimal generating set for $I$.

Using a graded map, and setting $I$ to be the kernel of that map allows us to use the commands \verb!numgens! and \verb!mingens!. This yields that the number of minimal generators is 3, and a minimal generating set is $\{x^8y-z^2, y^7-x^{17}z, x^{25}-y^6z\}$.

\end{enumerate}

\pagebreak

\item Let $R=\MB{Q}[x,y,z]$ and $I=(x^3,x^2y,x^2z,xyz)$. Find a prime filtration of $R/I$ and a primary decomposition of $I$.

(see file \verb!Richardson_MATH225_HW4.m2!)

Macauly2 reveals to us that a primary decomposition of this ideal is $I=(x)\cap(x^2,y)\cap(x^2,z)\cap(x^3,y,z)$.  Also according to Macauly2, the associated primes of $I$ are $(x)$, $(x,y)$, $(x,z)$, and $(x,y,z)$.

I am not certain about this part, and it is incomplete, but I think to start the filtration, we have $(x,y)=(I:y)$, so our first module is $((y)+I)/I$.


%
%Assuming absolute convergence, we have 
%\begin{align*}
%\sum_{n=0}^\infty n^pa_n &=\sum_{n=0}^\infty a_n\\[2mm]
%0^pa_0+1^pa_1+2^pa_2+3^pa_3+\cdots &= a_0+a_1+a_2+a_3+\cdots\\[2mm]
%a_1+2^pa_2+3^pa_3+\cdots&=a_0+a_1+a_2+a_3+\cdots\\[2mm]
%a_1-a_1+2^pa_2-a_2+3^pa_3-a_3+\cdots&=a_0\\[2mm]
%a_2(2^p-1)+a_3(3^p-1)+\cdots&=a_0.
%\end{align*}
%%This is only true if $a_n=0$ for all $n$, since $n^p-1\neq0$ for any $n>1$.
%If it is not absolutely convergent, use partial sums and the same argument holds.
\end{enumerate}
\end{document}